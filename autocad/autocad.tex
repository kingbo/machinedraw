%!Tex Program=xelatex
\documentclass[12pt,a4paper,twoside,openany]{book}

%%%%%%%%%%导入相关宏包%%%%%%%%%%%%%%%%%%%%%%

\usepackage{CJKutf8,CJKnumb}
\usepackage{amsmath, amssymb,amsthm,amscd}
\usepackage{gensymb}
\usepackage{titlesec}
\usepackage{titletoc}
\usepackage{fancyhdr}
\usepackage[paperwidth=185mm,paperheight=260mm,text={148mm,210mm},
			left=21mm,vmarginratio=1:1]{geometry}
\usepackage{xcolor}
\usepackage{listings}
\usepackage{enumitem}
\usepackage[labelfont=bf,labelsep=quad]{caption}
\usepackage{multirow}
\usepackage{floatrow}
\usepackage{subfig}
\newif\ifpdf
\ifx\pdfoutput\undefined
   \pdffalse
\else
   \pdfoutput=1
   \pdftrue
\fi
\ifpdf
   \usepackage[pdftex]{graphicx}
   \usepackage[pdftex,unicode=true,colorlinks,linkcolor=red,anchorcolor=blue,citecolor=green]{hyperref}
\else
   \usepackage{graphicx}
   \usepackage[unicode={true},colorlinks,linkcolor=red,anchorcolor=blue,citecolor=green]{hyperref}
\fi
\usepackage{asymptote}
\usepackage{lscape}
\usepackage{tikz}
\usepackage{indentfirst}
\usetikzlibrary{arrows,calc,through}
\usetikzlibrary{intersections}

\setlength{\parindent}{2em}
\endinput

\makeindex%生成建立索引

\begin{document}
%%%%%%%%%% 定理类环境的定义 %%%%%%%%%% 
%% 必须在导入中文环境之后 
\newtheorem{example}{例}             % 整体编号 
\newtheorem{algorithm}{算法} 
\newtheorem{theorem}{定理}[section]  % 按 section 编号 
\newtheorem{definition}{定义} 
\newtheorem{axiom}{公理} 
\newtheorem{property}{性质} 
\newtheorem{proposition}{命题} 
\newtheorem{lemma}{引理} 
\newtheorem{corollary}{推论} 
\newtheorem{remark}{注解} 
\newtheorem{condition}{条件} 
\newtheorem{conclusion}{结论} 
\newtheorem{assumption}{假设} 
 
%%%%%%%%%% 一些重定义 %%%%%%%%%% 

%% 必须在导入中文环境之后 
\renewcommand{\contentsname}{目\quad 录}% 将 Contents 改为目录 
%\renewcommand{\abstractname}{摘\ \ 要} % 将 Abstract 改为摘要 
\renewcommand{\bibname}{参考文献}      % 将 References 改为参考文献 
\renewcommand{\indexname}{索\quad 引} %将Idex 改为索引
\renewcommand{\figurename}{图} 
\renewcommand{\tablename}{表} 
\renewcommand{\appendixname}{附\quad 录} 
\renewcommand{\proofname}{\hei 证明} 
\renewcommand{\algorithm}{\hei 算法} 
 
%%%%%%%%%% 重定义字号命令 %%%%%%%%%% 

\newcommand{\yihao}{\fontsize{26pt}{36pt}\selectfont}%一号, 1.4 倍行距 
\newcommand{\erhao}{\fontsize{22pt}{28pt}\selectfont}%二号, 1.25 倍行距 
\newcommand{\xiaoer}{\fontsize{18pt}{18pt}\selectfont}%小二, 单倍行距 
\newcommand{\sanhao}{\fontsize{16pt}{24pt}\selectfont}%三号, 1.5 倍行距
\newcommand{\xiaosan}{\fontsize{15pt}{22pt}\selectfont}%小三, 1.5 倍行距 
\newcommand{\sihao}{\fontsize{14pt}{21pt}\selectfont}%四号, 1.5 倍行距 
\newcommand{\bansi}{\fontsize{13pt}{19.5pt}\selectfont}%半四, 1.5 倍行距 
\newcommand{\xiaosi}{\fontsize{12pt}{18pt}\selectfont}% 小四, 1.5 倍行距 
\newcommand{\dawu}{\fontsize{11pt}{11pt}\selectfont} % 大五, 单倍行距 
\newcommand{\wuhao}{\fontsize{10.5pt}{10.5pt}\selectfont}%五号, 单倍行距

%%%%%%%%%%%%%重定义章节的格式%%%%%%%%%%%%%%%%%%%

\titlecontents{chapter}[1em]{\vspace{.5\baselineskip}\bfseries}
{第\CJKnumber{\thecontentslabel}章\quad}{}
{\hspace{.5em}\titlerule*[10pt]{$\cdot$}\contentspage}
\renewcommand{\chaptername}{第\,\CJKnumber{\thechapter}\,章}
\titleformat{\chapter}[hang]{\centering\huge\bfseries}{\chaptername}{1em}{}
\titlespacing{\chapter}{0pt}{*0}{*4}
\pagestyle{fancy}
\fancyhf{}
\renewcommand{\chaptermark}[1]{\markboth{第\,\thechapter\,章\ #1}{}}
\renewcommand{\sectionmark}[1]{\markright{\thesection\ #1}{}}
%%%%%%%%%%%%定义页眉而脚%%%%%%%%%%%%%%%%%%%%%%%%%%%%%%%%%

\fancyhead[ER]{\leftmark}
\fancyhead[OL]{\rightmark}
\fancyhead[EL,OR]{$\cdot$\ \thepage\ $\cdot$}
\renewcommand{\headrulewidth}{0.4pt}

%%%%%%%%%%%%%设置列表格式%%%%%%%%%%%%%%%%

\setenumerate[1]{itemsep=0pt,partopsep=0pt,parsep=\parskip,topsep=5pt}

\setitemize[1]{itemsep=0pt,partopsep=0pt,parsep=\parskip,topsep=5pt}

\setdescription{itemsep=0pt,partopsep=0pt,parsep=\parskip,topsep=5pt}

\endinput
\title{\hei{\yihao{ 工程制图与CAD讲义}}}
\author{\sanhao{金波}}
\date{\today}
\maketitle
%\CJKtilde
\frontmatter
\chapter*{前言}
\section*{本书结构}
\section*{目标读者}
\section*{命令提示说明}
书中用到的相关命令及提示都会以代码的形式贴出。对于第一次出现的命令则会根据命令的提示进行逐步讲解,其形式如下:
\begin{lstlisting}
命令:CYLINDER
\end{lstlisting}

命令提示指定圆柱体的底面中心点或者绘图选项。
\begin{lstlisting}
指定底面的中心点或 [三点(3P)/两点(2P)/切点、切点、半径(T)/椭圆(E)]:
\end{lstlisting}

接下来,根据命令提示输入底面的半径
\begin{lstlisting}
指定底面半径或 [直径(D)]: 7
\end{lstlisting}

最后,根据命令提示指定圆柱体的高度,并按回车或空格键结束命令。
\begin{lstlisting}
指定高度或 [两点(2P)/轴端点(A)]: 28
\end{lstlisting}

对于已经使用过的命令,则直接贴整个命令提示,其形式如下:

\begin{lstlisting}
命令: CYLINDER
指定底面的中心点或 [三点(3P)/两点(2P)/切点、切点、半径(T)/椭圆(E)]:
指定底面半径或 [直径(D)]: 8
指定高度或 [两点(2P)/轴端点(A)]: 2
\end{lstlisting}
\section*{关于勘误}
由于编者的时间和水平比较有限,书中难免会出现一些纰漏和错误。如果读者在阅读过程中发现任何错误,请及时与本人联系,提出修改意见和建议。本人会在本书后续的版本中加以改正。本人专门为本书设立的电子邮箱是:kingbo2001@gmail.com。本人欢迎并希望和大家一起学习和讨论AutoCAD的三维建模功能,促进大家的共同进步。
\section*{致谢}
\endinput
\tableofcontents
\mainmatter
\graphicspath{{pdf/}{png/}}
\part{小轮组三维建模}
作为本书的第一部分,我们将构建小轮组组成零件的三维模型和小轮组装配三维模型。通过完成小轮组三维建模任务,让读者对应用AutoCAD进行三维建模的特点具有初步的认识和了解,掌握和理解以下几个方面的内容:
\begin{itemize}
\item 零件图和三维模型之间的对应关系
\item AutoCAD基本的三维建模命令
\item 三视图规律
\item 相关的国家制图标准
\item AutoCAD生成基本视图的方法
\end{itemize} 
%%%%%%%%%%%%%第一章%%%%%%%%%%%%%%%%%%%%%%%
\chapter{套筒}

\endinput
\section{初识零件图}
图\ref{fig:xiaoluntaotong}被称为零件图。所谓零件图是用于表达零件的图样,它广泛运用于工程技术设计、施工或产品制造,它是制造、加工、测量、检验的依据,是工程界的共同技术语言,是表达和交流技术思想的必备工具,是工程技术部门的一项重要技术文件。掌握零件图的阅读和绘制不仅是构建零件三维模型的基础,也是从事工程设计和技术的工程技术人员所必须具备的基本能力。一张完整的零件图应包括以下几个组成部分:
\begin{itemize}
\item 一组视图
\item 完整的尺寸
\item 技术要求
\item 标题栏
\end{itemize}
\begin{figure}[htbp]
\centering
\includegraphics[scale=0.5]{xiaoluntaotong2.pdf}
\caption{零件图组成部分}\label{fig:xiaoluntaotong2}
\end{figure}
图\ref{fig:xiaoluntaotong2}清晰的标识图\ref{fig:xiaoluntaotong}所示零件图的各个组成部分。
\section{理解视图}

\subsection{标题栏}
\subsection{尺寸}

\endinput
\section{套筒三维模型构建}\label{sec:taotongjianmo}
基于上面的认识和理解,现在可以用AutoCAD来构建套筒的三维模型,具体构建方法是:
\begin{procedure}

\item 启动AutoCAD软件。

启动AutoCAD软件的方法通常有:
\begin{itemize}
\item 双击桌面图标\includegraphics[scale=0.2]{cadicon.png}。
\item 【开始】$\rightarrow$ 【所有程序】$\rightarrow$【Autodesk】$\rightarrow$【AutoCAD 2014 – 简体中文 (Simplified Chinese)】$\rightarrow$【AutoCAD 2014 – 简体中文 (Simplified Chinese)】。
\end{itemize}
AutoCAD软件启动完成后,将出现图\ref{fig:cadui}所示的软件界面。
\begin{figure}[htbp]
\centering
\includegraphics[scale=0.5]{cadui.pdf}
\caption{“AutoCAD经典”工作空间}\label{fig:cadui}
\end{figure}
AutoCAD软件的界面与Word字处理软件的界面非常相似,也是由标题栏、菜单栏、工具栏、绘图区、状态栏等要素构成。不同的是AutoCAD软件的界面在标题栏上还有菜单浏览器和快速访问工具栏;绘图区的左下方有坐标系,右上方有ViewCube工具,左边有绘图工具栏,右边有编辑工具栏;状态栏的上方有命令窗口;状态栏中有功能按钮。

\item 将视图切换为左视图。

AutoCAD启动后默认的视图方向是俯视图方向,而套筒零件的特征图位于左视图方向,为使套筒零件的三维模型与纸方向一致,需要将AutoCAD的视图方向切换为左视图方向。实现左视图切换的方法有:
\begin{itemize}
\item 键盘输入-VIME\index{-view,视图} 或-V,选择【正交】选项中的【左视】项。
\item 键盘输入-VIME,并输入left。
\item 【视图】$\rightarrow$【三维视图】$\rightarrow$【左视图】。
\item 【视图】$\triangleright$【左视】图标\includegraphics[scale=0.6]{lefttool.png}。
\end{itemize}
同理,如果要将视图切换为其它视图方向,其操作方法与切换左视图的方法是一致的,只是需要将“左视”换成其它视图方向即可。例如要将视图方向切换为俯视图方向则将上述方法中的【左视】改为【俯视】。

\begin{lstlisting}
命令: -VIEW
输入选项 [?/删除(D)/正交(O)/恢复(R)/保存(S)/设置(E)/窗口(W)]: left
\end{lstlisting}

\yaodian{结合能够显著表达物体特征的视图选择AutoCAD的三维视图方向,将有助于三维模型的构建。}

\item 构建外圆柱
在AutoCAD中,创建圆柱体需要用到圆柱体命令,通常启动【圆柱体】命令的方法有:
\begin{itemize}
\item 键盘输入CYLINDER\index{cylinder,圆柱体}或CYL。
\item 【绘图】$\rightarrow$【建模】$\rightarrow$【圆柱体】。
\item 【建模】$\triangleright$【圆柱体】图标\includegraphics[scale=0.6]{cylinder.png}。
\end{itemize}
在图\ref{fig:commandline}所示的命令行窗口中输入CYLINDER命令,并回车或按空格键结束命令。结束命令输入后,绘图区中的光标形状由带拾取框的十字光标“\includegraphics[scale=0.8]{guangbiao1} ”变成十字形光标“\includegraphics[scale=0.6]{guangbiao2}”,表示此时AutoCAD进入了绘图状态。
\begin{figure}[htbp]
\centering
\includegraphics[scale=0.35]{commandline.png}
\caption{AutoCAD命令行}\label{fig:commandline}
\end{figure}

\yaodian{AutoCAD命令行是重要的信息窗口,是正确绘图的关键。初学者应多关注命令行的提示。}
\begin{lstlisting}
命令:CYLINDER
\end{lstlisting}
接下来命令提示行中会提示,指定圆柱体的底面中心点或者绘图选项。
\begin{lstlisting}
指定底面的中心点或 [三点(3P)/两点(2P)/切点、切点、半径(T)/椭圆(E)]:
\end{lstlisting}
看到上述提示后,可以用鼠标在绘图区中任意单击一下,以完成底面中心点的指定,也可能输入其它选项来指定底面。

接下来,命令提示输入底面的半径,此时根据$\phi 14$直径尺寸计算得到半径应该是7,因此直接从键盘上输入数字7并按空格键或回车键结束半径的指定。如果要指定直径则需要在指定半径的提示下,输入选项字母D来进入指定直径状态。
\begin{lstlisting}
指定底面半径或 [直径(D)]: 7
\end{lstlisting}
最后,命令提示指定圆柱体的高度,此时从键盘上输入数字28,并按回车或空格键结束高度指定。
\begin{lstlisting}
指定高度或 [两点(2P)/轴端点(A)]: 28
\end{lstlisting}
此时,绘图区的光标切换为带拾取框的十字光标,表示AutoCAD当前处于非绘图命令状态。
\item 将视图方向切换为西南等轴测。
\begin{lstlisting}
命令: -VIEW
-VIEW输入选项 [?/删除(D)/正交(O)/恢复(R)/保存(S)/设置(E)/窗口(W)]:  swiso
\end{lstlisting}
完成视图切换后,可以看到图\ref{fig:taotong1} 所示的结果。
\begin{figure}[htbp]
\centering
\subfloat[]{\label{fig:taotong1}\includegraphics[scale=0.3]{taotong1.png}}\hspace{20pt}
\subfloat[]{\label{fig:centerselect}\includegraphics[scale=0.3]{centerselect.png}}\hspace{20pt}
\subfloat[]{\label{fig:taotong2}\includegraphics[scale=0.3]{taotong2.png}}
\caption{构建圆柱体}
\end{figure}

\item 构建内圆柱
\begin{lstlisting}
命令: CYLINDER
指定底面的中心点或 [三点(3P)/两点(2P)/切点、切点、半径(T)/椭圆(E)]:
\end{lstlisting}
当命令提示指定底面圆心时,为保证我们所绘制的$\phi 8$圆柱与$\phi 14$圆上下端面对齐并且同轴,需要应用AutoCAD的对象后捕捉方式来选取$\phi 14$圆底端面的圆心作为$\phi 8$圆柱底端面的圆心。其具体操作方法是:将鼠标移至图\ref{fig:centerselect}所示的位置,直到出现图示的圆心标记和提示,然后单击鼠标左键确定圆柱体的圆心,并按下面的提示完$\phi 8$圆柱体的构建。
\begin{lstlisting}
指定底面半径或 [直径(D)] <7.0000>: 4
指定高度或 [两点(2P)/轴端点(A)] <28.0000>:
\end{lstlisting}

\yaodian{合理使用捕捉是实现精确绘图和快速绘图的重要方法之一。}

\item 进行差集操作,制作内孔

由于前面构建的是两个实体的圆柱体,因此并没有真构成套筒零件所需要的孔,而实现孔的构建则需要从$\phi 14$的圆柱体中去除一个$\phi 8$的圆柱体,要实现这一目标,需要用到实体编辑中的【差集】命令,其启动方法有:
\begin{itemize}
\item 键盘输入SUBTRACT\index{subtract,差集}或SU。
\item 【修改】$\rightarrow$【实体编辑】$\rightarrow$【差集】。
\item 【实体编辑】$\triangleright$【差集】图标\includegraphics[scale=0.7]{subtracttool.png}。
\end{itemize}
\begin{figure}[htbp]
\centering
\subfloat[]{\label{fig:subtractselect}\includegraphics[scale=0.3]{subtractselect.png}}\hspace{40pt}
\subfloat[]{\label{fig:subtractselect1}\includegraphics[scale=0.3]{subtractselect1.png}}
\caption{差集操作}
\end{figure}
\begin{lstlisting}
命令: SUBTRACT
选择要从中减去的实体、曲面和面域...
\end{lstlisting}
输入SUBTACT命令后,提示选择要从中减去的实体,同时绘图区的鼠标形状变为矩形小方框“\includegraphics[scale=0.8]{selectobject.png}”,表示进入选择对象状态,此时按照图\ref{fig:subtractselect}所示选择$\phi 14$的圆柱体作为要从中减去的对象,并按回车或空格键结束选择。
\begin{lstlisting}
选择对象: 找到 1 个
选择对象:  选择要减去的实体、曲面和面域...
\end{lstlisting}
结束从中减去实体选择后,命令提示选择要减去的实体,此时按照图\ref{fig:subtractselect1}所示选择$\phi 8$圆柱体作为要选择的实体,并按驾车可空格键结束选择。
\begin{lstlisting}
选择对象: 找到 1 个
选择对象:
\end{lstlisting}
\item 进行倒角操作
至此,已经完成了套筒的整体部分的构建,还剩下两个倒角没有完成,在AutoCAD中完成三维立体倒角边构建的命令是“倒角边”,其启动方法有:
\begin{itemize}
\item 键盘输入CHAMFEREDGE\index{charmferedge,倒角边}。
\item 【修改】$\rightarrow$【实体编辑】$\rightarrow$【倒角边】。
\item 【实体编辑】$\triangleright$【倒角边】图标\includegraphics[scale=0.6]{chamferedge.png}。
\end{itemize}

\begin{lstlisting}
命令: CHAMFEREDGE
距离 1 = 1.0000,距离 2 = 1.0000
\end{lstlisting}
启动完倒角边命令后,命令会提示当前的倒角边距离的默认值,并提示选择要倒角的边,此时按照图\ref{fig:chamferedgeselect}所示选择要倒角的一条边,选择完成后会自动生成倒角预览。
\begin{figure}[htbp]
\centering
\subfloat[]{\label{fig:chamferedgeselect}\includegraphics[scale=0.3]{chamferedgeselect.png}}\hspace{20pt}
\subfloat[]{\label{fig:chamferedgeselect1}\includegraphics[scale=0.3]{chamferedgeselect1.png}}
\hspace{20pt}
\subfloat[]{\label{fig:chamferedgeresult}\includegraphics[scale=0.3]{chamferedgeresult.png}}
\caption{倒角边操作}
\end{figure}
\begin{lstlisting}
选择一条边或 [环(L)/距离(D)]:
\end{lstlisting}
接下来,按照图\ref{fig:chamferedgeselect1}所示,选择另一条要倒角的边。
\begin{lstlisting}
选择同一个面上的其他边或 [环(L)/距离(D)]:
\end{lstlisting}
套筒零件只有两个倒角边,因此选择倒角边至此结束,但是系统默认值并不是需要的倒距离,故需要选择距离(d)进行修改。
\begin{lstlisting}
选择同一个面上的其他边或 [环(L)/距离(D)]:d
\end{lstlisting}
最后将两个倒角距离均设置成为0.5,并用回车键接受,其结果如图\ref{fig:chamferedgeresult}所示。
\begin{lstlisting}
指定基面倒角距离或 [表达式(E)] <1.0000>: 0.5
指定其他曲面倒角距离或 [表达式(E)] <1.0000>: 0.5
按 Enter 键接受倒角或 [距离(D)]:
\end{lstlisting}
\item 进行着色
\begin{figure}[htbp]
\centering
\includegraphics[scale=0.6]{taotonglititu.png}
\caption{套筒三维模型}\label{fig:taotonglititu}
\end{figure}
最后,将完成套筒三维模型以灰度方式的视觉样式进行着色,以使得三维模型看起更加真实。启动灰度视觉样式的方法有:
\begin{itemize}
\item 键盘输入 VSCURRENT\index{vscurrent,视觉样式}或VS,并输入G选项。
\item 【视图】$\rightarrow$【视觉样式】$\rightarrow$【灰度】。
\end{itemize}
\begin{lstlisting}
命令: VSCURRENT
输入选项 [二维线框(2)/线框(W)/隐藏(H)/真实(R)/概念(C)/着色(S)/带边缘着色(E)/灰度(G)/勾画(SK)/X 射线(X)/其他(O)] <二维线框>: G
\end{lstlisting}
完成灰度视觉样式切换后,其结果如图\ref{fig:taotonglititu}所示。

\item 保存模型

将建立好的套筒三维模型保存为“小轮组套筒.dwg”,AutoCAD中保存文件的方法有:
\begin{itemize}
\item 键盘输入SAVE\index{save,保存}或SAVEAS\index{saveas,另存为}。
\item 键盘输入\fbox{Ctrl}+\fbox{S}。
\item 【文件】$\rightarrow$【保存】或【另存为】。
\item 【工具栏】$\triangleright$【保存】图标\includegraphics[scale=0.6]{savetool.png}。
\end{itemize}
调用保存命令后,会弹出图\ref{fig:saveasdialog}所示的文件保存对话框,此时在文件名处输入“小轮组套筒.dwg”,并单击保存。
\begin{figure}[htbp]
\centering
\includegraphics[scale=0.6]{saveasdialog.png}
\caption{保存文件对话框}\label{fig:saveasdialog}
\end{figure}
\end{procedure}

\endinput
\section{理解视图}
为什么图\ref{fig:xiaoluntaotong}所示的零件图能够唯一也表达\ref{fig:taotonglititu}所示的套筒零件呢,为什么需要两视图呢,一个视图能不能够表达呢,要回答这个问题,需要进一步了解一些与视图相关的知识。
\subsection{视图的概念}
\begin{figure}[htbp]
\centering
\subfloat[斜投影法]{\label{fig:xietouyinfa}
\includegraphics[scale=0.4]{xietoying.png}
}\hspace{30pt}
\subfloat[正投影法]{\label{fig:zhentouyinfa}
\includegraphics[scale=0.4]{zhengtouying.png}
}
\caption{平行投影法}\label{pingxingtouyin}
\end{figure}

 在图\ref{fig:xiaoluntaotong}所示的套筒零件图中,位于左边的图形称之为主视图,其表达式为全剖视图,关于全剖视图的概念及画法将在后面予以介绍。主视图清晰的表达了套筒零件的长度及内外结构。而位于右边的视图称之为左视图,它清楚地显示套筒零件为回转体类零件。

要理解什么是视图,首先需要了解投影的概念。投影是物体在阳光或灯光下所产生的影子。由于影子只能够表现物体轮廓而不能够表现物体的内部结果,工程实际中将物体内外空间几何元素加以抽象,并用不同的线型进行表示,实现物体内外细节的表达,从而形成的比较完备的、实用的投影方法。投影法分为中心投影法和平行投影法两类。所有投影线都互不平行且汇聚于点的投影法称为中心投影法。中心投影法主要用于绘制效果比较逼真的建筑或产品立体图。图\ref{pingxingtouyin}所示的投影法是平行投影法,从中可以看出其所有的投影线都是相互平行的,其中投影线倾斜于投影面则为斜投影,投影线垂直于投影面则为正投影法。工程中将用正投影法绘制的物体图形称为视图。

\subsection{三视图的形成}
了解完视图的概念后,让们来探讨一个视图能不能够准确地表达出物体的形状这个问题,首先让我们来看一下图\ref{fig:singleprojection}所示的投影。从图\ref{fig:singleprojection}中,我们可以看出不同的形状的物体在同一个视图投影面内具有相同的视图表示。之所以如此,主要是因为仅用一个视图只能反映物体两方向的尺寸,而空间物体需要用长宽高三个方向的尺寸帮能够将其大小形状完整清晰地表达出来。在没有尺寸标注的辅助的情况下,要解决投影只能够表达物体两个尺寸方向的问题,我们需要将物体向多个投影面进行投影,通过多个投影视图来实现物体上下、左右、前后各部分的形状和大小完整表达。在工程零件图中,通常都会使用两个或三个视图来表达一个零件。当然对于简单的轴类零件也经常使用一个视图来表达,对于特别复杂的零件还需要使用更多的辅助视图加以表达。但无论使用多少个视图,三视图是基本的零件图表达方式。下面,我们就来了解一下三视图的形成过程。

\begin{figure}[htbp]
\centering
\includegraphics[scale=0.4]{buweiyi.png}
\caption{一个投影面不能确定物体在空间中的形状和位置}\label{fig:singleprojection}
\end{figure}

\begin{figure}[htbp]
\centering
\subfloat[物体在三面投影中的投影]{\label{fig:threeviewprojection}
\includegraphics[scale=0.4]{wutisanmiantouying.png}}
\hspace{30pt}
\subfloat[三个投影面的展开]{\label{fig:threeviewzhankai}
\includegraphics[scale=0.45]{touyingzhankai.png}}


\subfloat[展开后的三视图]{\label{fig:threeview}
\includegraphics[scale=0.45]{zhankairesult.png}}
\hspace{30pt}
\subfloat[最终三视图]{\label{fig:threeviewguilu}
\includegraphics[scale=0.45]{sanshituresult.png}}
\caption{三视图的形成}
\end{figure}

为此,我们需要根据国家标准规定,选三个相互垂直的投影面构成图\ref{fig:threeviewprojection}所示的三投影面体系。在三视图投影体系中,正对观察者的投影面称为正平面,用$V$表示。水平放置的投影面称水平面,用$H$表示。侧立的投影面称为侧平面,用$W$表示。将物体放置于三视图投影体系中,将其由前向后投影所得的$V$面视图称为主视;将其由上向下投影所得的$H$面视图称为府视图;将物体由左向右投影所得的$W$面视图称为左视图。最后,按照国家标准,以图\ref{fig:threeviewzhankai}所示方式展开,即以$V$面视图为基准,$H$面绕$V$面与$H$面的交线所形成的$X$轴向下旋转$90^o$;$W$面绕$V$面与$W$面的交线所形成的$Z$轴向右旋转$90^o$,使$V$、$H$、$W$面处于同一个平面内,如图\ref{fig:threeview}所示。展开后的三视图既不需要画边框和投影轴,也不需要标视图名称,如图\ref{fig:threeviewguilu}所示。

\subsection{三视图投影规律}
既然表达一个零件需要多个视图,那么这些视图之间就需要遵循一定规律,才能够准确的表示物体各个组成部分之间的关系。而视图之间所遵循的规律则称之为对应关系。图\ref{fig:threeviewguanxi}标识了三个视图之间的对应关系和视图所能表达的方位。从图中,我们可以得出:主视图反映物体的上下和左右关系,即反映物体的长和高;俯视图反映物体的左右和前后关系,即反映物体的长和宽;左视图反映物体的上下和前后,即反映物体的高和宽。因此,三视图的投影规律为:
\begin{itemize}
\item 主、俯视图长对正;
\item 主、左视图高平齐;
\item 俯、左视图宽相等。
\end{itemize}
\begin{figure}[htbp]
\includegraphics[scale=0.6]{touyingguilu.png}
\caption{三视图投影规律}\label{fig:threeviewguanxi}
\end{figure}
三视图的投影规律不仅适用于物体整体之间的投影,也适用于空间中的点、线面。同时它也是画图和读图的基础规则。

\endinput
\section{检验结果}

在了解了相关的视图知识后,让我们来简单地检验下套筒零件的三维模型的投影是不是与零件图的视图基本一致的。为什么说是基本一致呢,主要是因为接下来的检验方法仅仅是在AutoCAD的模型空间中,用切换视图方向的方法进行观察套筒的三维模型,其表现方式类似于生活中的影子,并不符合工程图的制图规范。

\begin{procedure}
\item 将视觉样式切换为二维线框

在\ref{sec:taotongjianmo}节中,为了便于看到真实的套筒零件三维模型,我们将视觉样式设置成了灰度。在本节中为了方便观察不同方向的视图,我们需要将视觉样式设置为二维线框方式。

\begin{lstlisting}
命令: VSCURRENT
输入选项 [二维线框(2)/线框(W)/隐藏(H)/真实(R)/概念(C)/着色(S)/带边缘着色(E)/灰度(G)/勾画(SK)/X 射线(X)/其他(O)] <灰度>: 2
\end{lstlisting}
\item 观察主视图

要观察套筒的主视图,需要将三维视图切换为前视图方向,其切换方法与\ref{sec:taotongjianmo}节中切换左视图的方法基本相同,图\ref{fig:taotongfront} 所示为切换为前视图后的结果。从图中可以看出,整个图形的外形与套筒零件的主视图外形是一致的,主要区别是关于内部结构的表达。

\begin{lstlisting}
命令: -VIEW
输入选项 [?/删除(D)/正交(O)/恢复(R)/保存(S)/设置(E)/窗口(W)]: front
\end{lstlisting}
\item 观察左视图

将视图方向切换为左视图,可以得到图\ref{fig:taotongleft}所示的结果。从图中可以看出,他与套筒的左视图是基本相同。
\begin{lstlisting}
命令: -VIEW
输入选项 [?/删除(D)/正交(O)/恢复(R)/保存(S)/设置(E)/窗口(W)]: left
\end{lstlisting}
\item 观察俯视图

将视图切换为俯视图后,其结果如图\ref{fig:taotongtop}所示,可以看它与主视图方向的一图形是一致的。
\begin{lstlisting}
命令: -VIEW
输入选项 [?/删除(D)/正交(O)/恢复(R)/保存(S)/设置(E)/窗口(W)]: top
\end{lstlisting}
\begin{figure}[htbp]
\subfloat[主视图]{\label{fig:taotongfront}\includegraphics[scale=0.5]{taotongfront.png}}\hspace{15pt}
\subfloat[左视图]{\label{fig:taotongleft}\includegraphics[scale=0.5]{taotongleft.png}}\hspace{15pt}
\subfloat[俯视图]{\label{fig:taotongtop}\includegraphics[scale=0.5]{taotongfront.png}}
\end{figure}
\end{procedure}

从上面的简单检验结果来看,我们套筒零件的三维建模是正确。其实,这种简单的检验方法将在AutoCAD的建模过程中会经常用到,尽管他与实际工作图存在一定的差别,但是它的操作步骤简便,能够快速地验证结果,能够有效地辅助三维建模,比较实用。
\endinput
\section{小结}
本章通过套筒零件的三维模型的构建,介绍了利用AutoCAD的圆柱体命令构建套筒的外形实体和内孔实全,并利用并集命令实现孔洞的构建,最后运用倒角边命令来实现三维倒角,整个过程化繁为简。化繁为简一种重要的思维方法。另一个方面,我们还介绍了与套筒三维模型构建的三视图知识,掌握必要的视图知识,是准确构建三维模型的基础,需要在练习中逐步掌握和运用三视图的支应规律,分析零件的重要组成部分。
\endinput
\section*{练习题}
\begin{enumerate}
\item 
\begin{question}
构建图\ref{exerc:exercise1-4}和图\ref{exerc:exercise1-2}  所示物体的三维模型。
\begin{figure}[htbp]
\centering
\begin{floatrow}[2]
\ffigbox{\caption{ }\label{exerc:exercise1-4}}{\includegraphics[scale=0.35]{exercise1-4}}
\ffigbox{\caption{ }\label{exerc:exercise1-2}}{\includegraphics[scale=0.5]{exercise1-2}}
\end{floatrow}
\end{figure}
\end{question}

\item 
\begin{question}
构建图\ref{exerc:exercise1-3}和图\ref{exerc:exercise1-1}  所示物体的三维模型。
\begin{figure}[htbp]
\centering
\begin{floatrow}[2]
\ffigbox{\caption{ }\label{exerc:exercise1-3}}{\includegraphics[scale=0.45]{exercise1-3}}
\ffigbox{\caption{ }\label{exerc:exercise1-1}}{\includegraphics[scale=0.6]{exercise1-1}}
\end{floatrow}
\end{figure}
\end{question}

\item 
\begin{question}
构建图\ref{exerc:exercise1-5} 所示物体的三维模型。
\begin{figure}[htbp]
\centering
\includegraphics[scale=0.6]{exercise1-5}
\caption{ }\label{exerc:exercise1-5}
\end{figure}
\end{question}
\end{enumerate}
\endinput
%%%%%%%%%%%%%第二章%%%%%%%%%%%%%%%%%%%%%%%
\chapter{轮轴}
\noindent
\begin{figure}[htbp]
\centering
\includegraphics[scale=0.45]{xiaolunzhou.pdf}
\caption{轮轴零件图}\label{fig:xiaolunzhou}
\end{figure}
\endinput
\section{轴建模过程分析}\label{sec:zhoufengxi}
图\ref{fig:xiaolunzhou}所示的轴零件由一个主视图构成,根据\ref{sec:lijieshitu}节的知识可知一个视图通常是不能够唯一表达物体的。但仔细观察轴零件图的垂直方向的尺寸标注,可以发现垂直方向尺寸标注的一个共同点是均有表示直径的符号$\phi$,由此可知轴零件的各个组成部分都是圆柱体。 基于\ref{sec:taotongjianmo}节套筒零件的三维建模经验,可以采用以下两种方式进行三维建模。

\yaodian{合理清晰的尺寸标注有助于视图表达。}
\subsection{运用叠加方式构建}
从图\ref{fig:xiaolunzhou}中可以直观的看出整个轴零件由三段叠加而成,分别是直径$\phi 6$长3的圆柱体、直径$\phi 8$长35的圆柱体和直径$\phi 16$长2的圆柱体。结果如图\ref{fig:zhoufengxi1}所示。
\begin{figure}[htbp]
\centering
\includegraphics[scale=0.6]{zhoufengxi1.png}
\caption{分段建模}\label{fig:zhoufengxi1}
\end{figure}
\subsection{运用包含关系构建}
轴零件整个都是实体,因此可以将两个轴零件看作这样一种包含关系,即$\phi 16$的圆柱体包含了$\phi 8$和$\phi 6$两个圆柱体的部分实体,$\phi 8$的圆柱体包含了$\phi 6$的部分实体。因此,$\phi 8$圆柱体的长度要加上$\phi 16$圆柱体的长度,故长为37。与此类似,可得$\phi 6$圆柱体的长度则为40。结果如图\ref{fig:zhoufengxi2}所示。
\begin{figure}[htbp]
\centering
\includegraphics[scale=0.6]{zhoufengxi2.png}
\caption{按包含关系建模}\label{fig:zhoufengxi2}
\end{figure}

对于图\ref{fig:xiaolunzhou}所示的轴零件而言,用分段建模更为直观和直接,但实际中两建模方式并没有优劣之分,应当根据所需建模零件的实际情况灵活地综合运用。
\endinput
\section{轴三维模型构建}

\begin{lstlisting}
命令: -VIEW
输入选项 [?/删除(D)/正交(O)/恢复(R)/保存(S)/设置(E)/窗口(W)]: left
\end{lstlisting}

\begin{lstlisting}
命令: -VIEW
输入选项 [?/删除(D)/正交(O)/恢复(R)/保存(S)/设置(E)/窗口(W)]: swiso
\end{lstlisting}

\begin{lstlisting}
命令: CYLINDER
指定底面的中心点或 [三点(3P)/两点(2P)/切点、切点、半径(T)/椭圆(E)]:
指定底面半径或 [直径(D)]: 8
指定高度或 [两点(2P)/轴端点(A)]: 2
\end{lstlisting}

\begin{lstlisting}
命令: CYLINDER
指定底面的中心点或 [三点(3P)/两点(2P)/切点、切点、半径(T)/椭圆(E)]:
指定底面半径或 [直径(D)] <8.0000>: 4
指定高度或 [两点(2P)/轴端点(A)] <2.0000>: 35
\end{lstlisting}

\begin{lstlisting}
命令: CYLINDER
指定底面的中心点或 [三点(3P)/两点(2P)/切点、切点、半径(T)/椭圆(E)]:
指定底面半径或 [直径(D)] <4.0000>: 3
指定高度或 [两点(2P)/轴端点(A)] <35.0000>: 3
\end{lstlisting}

\begin{lstlisting}
命令:  UNION
选择对象: 指定对角点: 找到 3 个
选择对象:
\end{lstlisting}

\begin{lstlisting}
命令: CHAMFEREDGE
距离 1 = 1.0000,距离 2 = 1.0000
选择一条边或 [环(L)/距离(D)]: d
指定距离 1 或 [表达式(E)] <1.0000>: 0.5
指定距离 2 或 [表达式(E)] <1.0000>: 0.5
选择一条边或 [环(L)/距离(D)]:
选择同一个面上的其他边或 [环(L)/距离(D)]:
按 Enter 键接受倒角或 [距离(D)]:
\end{lstlisting}

\begin{lstlisting}
命令: CHAMFEREDGE
距离 1 = 0.5000,距离 2 = 0.5000
选择一条边或 [环(L)/距离(D)]:
选择同一个面上的其他边或 [环(L)/距离(D)]:
按 Enter 键接受倒角或 [距离(D)]:
\end{lstlisting}

\begin{lstlisting}
命令: VSCURRENT
输入选项 [二维线框(2)/线框(W)/隐藏(H)/真实(R)/概念(C)/着色(S)/带边缘着色(E)/灰度(G)/勾画(SK)/X 射线(X)/其他(O)] <二维线框>: g
\end{lstlisting}

\endinput
\section{轴主视图生成}

\endinput
\section{图幅与比例}
在\ref{sec:zhoushitu}节中,我们将图纸尺寸设置为ISO A4(210.00x297.00毫米),视口的显示比例设置成了5:1。之所以这样设置,其目的是确保使用的图纸和比例都符合国家制定的制图标准。因此本节将介绍图纸幅面和比例相关的国家标准。
\subsection{图纸幅面与格式}
\subsubsection{图纸幅面}
图纸幅面是指整张图纸的尺寸大小。工程中为了便于图纸的装订、保管及合理利用图纸,要求图纸幅面大小要符合表\ref{tab:tuzhifumian}的规定。
\begin{table}[htbp]
\caption{图纸幅面}\label{tab:tuzhifumian}
\begin{tabu}to \linewidth {X[cm]*5{|X[cm]}}
\tabucline -
幅面代号&A0&A1&A2&A3&A4\\
\tabucline -
$B\times L$&$841\times 1189$&$594\times 841$& $420\times 594$&$297\times 420$&$210\times 297$\\
\tabucline -
$e$&\multicolumn{2}{c|}{20}&\multicolumn{3}{c}{10}\\
\tabucline -
$c$&\multicolumn{3}{c|}{10}&\multicolumn{2}{c}{5}\\
\tabucline -
$a$&\multicolumn{5}{c}{25}\\
\tabucline -
\tabuphantomline
\end{tabu}
\end{table}
\subsubsection{图框}
\subsection{比例}
比例是图形与其实物相应要素的线性尺寸之比。线性尺寸是能够用直线表达的尺寸。例如圆弧的半径,直线的长度等。

通常图样的比例分为原值比例、放大比例和缩小比例三种。画图时应优先采用1:1的比例进行绘制,以便能够看出物体的真实大小。若无法采用1:1的比例时,则应优先选用表\ref{tab:biaozhunbili}中规定的比例系列,必要时也可采用表\ref{tab:biaozhunbili} 规定的比例系列。
\begin{table}[htbp]
\caption{标准比例系列}\label{tab:biaozhunbili}

\begin{tabu} to \linewidth {X[cm]|X[c m]|X[c m]|X[c m]}
\tabucline -
种\qquad 类&\multicolumn{3}{c}{ 比\qquad 例 } \\
\tabucline -
原始比例&\multicolumn{3}{c}{1:1}\\
\tabucline -
放大比例&
$\begin{tabu}{c}
2:1\\
2\times 10^n:1
\end{tabu}$
&
$\begin{tabu}{c}
5:1\\
5\times 10^n:1
\end{tabu}$
&$1\times 10^n$:1\\
\tabucline -
缩小比例&
$\begin{tabu}{c}
1:2\\
1:2\times 10^n
\end{tabu}$
&
$\begin{tabu}{c}
1:5\\
1:5\times 10^n
\end{tabu}$
&
$\begin{tabu}{c}
1:10\\
1:10\times 10^n
\end{tabu}$\\
\tabucline -
\tabuphantomline
\end{tabu}
\end{table}

\begin{table}[htbp]

\begin{tabu}to \linewidth {X[cm]|X[2cm]|X[cm]|X[cm]|X[2cm]}
\tabucline -
种\qquad 类&\multicolumn{4}{c}{比\qquad 例}\\
\tabucline -
放大比例&\multicolumn{2}{c|}{4:1\quad $4\times 10^n$:1}&\multicolumn{2}{c}{2.5:1\quad $2.5\times 10^n$:1}\\
\tabucline -
缩小比例&$\begin{tabu}{c}
1:3\\
1:3\times 10^n
\end{tabu}$
&
\multicolumn{2}{c|}{$\begin{tabu}{c}
1:4\\
1:4\times 10^n
\end{tabu}$}
&
$\begin{tabu}{c}
1:6\\
1:6\times 10^n
\end{tabu}$\\
\tabucline -
\tabuphantomline
\end{tabu}
\caption{比例系列}\label{tab:biaoxilei}
\end{table}
\endinput
\section{小结}
本章通过小轮组轴零件的三维建模,进一步熟悉了圆柱体的应用,首先学习了如何利用圆柱体这类简单的回转体构建轴类零件的三维模型。整个三维模型构建过程是:
\begin{enumerate}
\item 分析轴零件的组成部分
\item 分段构建轴零件实体
\item 组合各个组成部分实体,构成轴零件整体
\end{enumerate} 

其次是学习了如何利用三维模型生成平面基本的平面视图,更深入地理解了三维模型与平面图形之间的对应关系。应用三维模型构建单一基本视图的基本过程是:
\begin{enumerate}
\item 从模型空间切换至图纸空间
\item 修改页面设置
\item 进入图纸视口模型空间切换视图方向
\item 设置视口图形显示比例
\item 提出模型轮廓
\item 修改图层设置
\end{enumerate}
最后介绍图纸幅面和比例的相关国家标准。
\endinput
\section*{练习题}
%\SetupExSheets[question]{counter-within=ch}
\begin{enumerate}

\item 
\begin{question}
构建图 所示物体的三维模型。

\end{question}
\item 
\begin{question}
构建图 所示物体的三维模型。
\end{question}
\end{enumerate}
\endinput
%%%%%%%%%%%%%第三章%%%%%%%%%%%%%%%%%%%%%%%
\chapter{连接杆}
\begin{figure}[htbp]
\centering
\includegraphics[scale=0.45]{xiaolunlianjiegan.pdf}
\caption{边接杆零件图}\label{fig:xiaolunlianjiegan}
\end{figure}

本章的目标是构建图\ref{fig:xiaolunlianjiegan}所示的小轮组连接杆零件的三维模型,并在此基础之上制作连接标杆的主视图、图纸图框和标题栏。因此本章重点讲解以下内容:
\begin{itemize}
\item 连接杆的三维模型构建
\item 块的定义和保存
\item 标题栏的制作
\item 线性尺寸标注
\item 文字标注
\end{itemize}


%\section{标题栏、尺寸、文字}

\endinput
\endinput
\section{连接杆建模方法分析}\label{sec:lianjieganfenxi}
\subsection{实体建模法}
从图\ref{fig:xiaolunlianjiegan}中可以看出,连接杆由$\phi 14$长6、$\phi 14$长27、$\phi 14$长9、$\phi 18$长3和$\phi 8$长4五段组成的阶梯轴,因此可以分段构建连接杆的实体,最后进组合,其分段实体如图\ref{fig:lianjieganfenxi1}所示。我们把这种应用简单实体进行组合来构建实体模型的方法称之为实体建模法。

\begin{figure}[htbp]
\centering
\includegraphics[scale=0.6]{lianjieganfenxi1.png}
\caption{连接杆实体建模}\label{fig:lianjieganfenxi1}
\end{figure}
\subsection{旋转建模法}
图\ref{fig:xiaolunlianjiegan}的连接杆还可以采用旋转法进行建模。所谓旋转建模法是应用能够表征回转体特征的轴向截面,围绕其轴线旋转一周来构建实体模型的方法。旋转建模法只能够用于构建具有回转体结构的模型。图\ref{fig:lianjieganjiemian}为忽略倒角后的截面图,图形是上下对称的结构。从旋转建模法的对特征面定义可知,能够表征连接标杆回转体特征的轴向截面如图\ref{fig:lianjieganduichen}所示。根据前面学习的经验,可以将图\ref{fig:lianjieganduichen}所示的特征截面看作处于同一水平线上的五个不同尺寸的矩形组合而成,其结果如图\ref{fig:lianjieganfenxi2}所示。通过这样的处理,再一次将复杂的图形处理为几个简单的平面图形的组合,减小的构图的难度。
\begin{figure}[htbp]
\begin{floatrow}[3]
\ffigbox{\caption{连接杆截面}\label{fig:lianjieganjiemian}}{\includegraphics[scale=0.4]{lianjieganjiemian.png}}
\ffigbox{\caption{连接杆特征截面}\label{fig:lianjieganduichen}}{\includegraphics[scale=0.4]{lianjieganduichen.png}}
\ffigbox{\caption{特征截面组合示意}\label{fig:lianjieganfenxi2}}{\includegraphics[scale=0.4]{lianjieganfenxi2.png}}
\end{floatrow}
\end{figure}

\yaodian{分析的目的是要从复杂对象中找出构成它的简单对象及组合方式。}
\endinput
\section{连接杆三维建模}

\endinput
\section{连接杆零件图制作}

\endinput
%%%%%%%%%%%%%第四章%%%%%%%%%%%%%%%%%%%%%%%
\chapter{轮子}

\endinput
\section{轮建模分析}
\subsection{实体建模方案}
\subsubsection{方案一}
根据制作图\ref{fig:xiaoluntaotong}所示套筒三维模型的经验,可以将图\ref{fig:xiaolunlun.pdf}所示的轮零件拆分为图\ref{fig:lunfenxi1.png}所示的套筒和\ref{fig:lunfenxi2.png}所示结构的组合。\ref{fig:lunfenxi2.png}所示结构又可以以此方式进一步简单为两个套筒的组合。这种简化方案的三维建模过程简单,难点主要在于各个组成部分的定位,需要利用套筒模型的轴心线的中点进行定位,才能够获得准确的模型。故需要绘制轴心辅助线,以利于组合定位。
\begin{figure}[htbp]
\centering
\subfloat[]{\label{fig:lunfenxi1.png}\includegraphics[scale=0.4]{lunfenxi1.png}}\hspace{20pt}
\subfloat[]{\label{fig:lunfenxi2.png}\includegraphics[scale=0.25]{lunfenxi2.png}}
\caption{轮实体建模方案一}
\end{figure}
\subsubsection{方案二}

注意到图\ref{fig:xiaolunlun.pdf}中的主视图不仅具有上下对称的特点,同时还具备上下对称的特性。故可以先构一半的模型,然后利用镜像来快速构建另一半模型。因此,图\ref{fig:xiaolunlun.pdf}所示的轮零件可以图\ref{fig:lunfenxi3.png}所示$\frac{1}{2}$长的套筒,然后减去图\ref{fig:lunfenxi4.png}所示的套筒,来获得图\ref{fig:lunfenxi5.png}所示模型,最后,利用镜像构建另一半模型。这种建模方案定位比较方便,不需要绘制用于定位的辅助线,构建方式也简单直接。为便于叙述,\ref{sec:lunjianmo}节 将图\ref{fig:lunfenxi3.png}所示的套筒称为被减套筒,图\ref{fig:lunfenxi4.png}所示的套筒称为减去套筒。
\begin{figure}[htbp]
\centering
\subfloat[]{\label{fig:lunfenxi3.png}\includegraphics[scale=0.6]{lunfenxi3.png}}\hspace{20pt}
\subfloat[]{\label{fig:lunfenxi4.png}\includegraphics[scale=0.6]{lunfenxi4.png}}\hspace{20pt}
\subfloat[]{\label{fig:lunfenxi5.png}\includegraphics[scale=0.6]{lunfenxi5.png}}
\caption{轮实体建模方案二}
\end{figure}

当然,还存在其它的实体建模方案,这里就不一一赘述。读者有兴趣,可以逐尝用多种不同的构建方案进行三维模型构建,仔细体会每种组合方式的特点。
\subsection{旋转建模方案}

\begin{figure}[htbp]
\centering
\subfloat[]{\label{fig:lunfenxi6.png}\includegraphics[scale=0.7]{lunfenxi6.png}}\hspace{40pt}
\subfloat[]{\label{fig:lunfenxi7.png}\includegraphics[scale=1]{lunfenxi7.png}}
\caption{轴旋转建模方案}
\end{figure}

基于图\ref{fig:xiaolunlianjiegan}所示连接杆的三维建模经验,图\ref{fig:xiaolunlun.pdf}所示的轮零件忽略圆角后具有图\ref{fig:lunfenxi6.png}所示的截面,因此可绘制图\ref{fig:lunfenxi7.png}所示的轮廓线来构成旋转特征面,并通过旋转的方式来完成模型的构建。这种建模方式也比较简单快捷。

\yaodian{尝用不同的方式构建模型,能有效提升灵活解决问题的能力。}
\endinput
\section{轮三维建模}
\begin{procedure}
\item 切换视图方向为西南等轴测

\begin{lstlisting}
命令: -VIEW
输入选项 [?/删除(D)/正交(O)/恢复(R)/保存(S)/设置(E)/窗口(W)]: swiso
\end{lstlisting}

\item 切换坐标系为左视图方向

\begin{lstlisting}
命令: UCS
当前 UCS 名称: *世界*
指定 UCS 的原点或 [面(F)/命名(NA)/对象(OB)/上一个(P)/视图(V)/世界(W)/X/Y/Z/Z 轴(ZA)] <世界>: za
指定新原点或 [对象(O)] <0,0,0>:
在正 Z 轴范围上指定点 <0.0000,0.0000,1.0000>: -1,0,0
\end{lstlisting}

\item 构建图\ref{fig:lunfenxi3.png}所示套筒

\begin{lstlisting}
命令: CYLINDER
指定底面的中心点或 [三点(3P)/两点(2P)/切点、切点、半径(T)/椭圆(E)]:
指定底面半径或 [直径(D)]: 50
指定高度或 [两点(2P)/轴端点(A)]: 14
\end{lstlisting}

\begin{lstlisting}
命令: CYLINDER
指定底面的中心点或 [三点(3P)/两点(2P)/切点、切点、半径(T)/椭圆(E)]:
指定底面半径或 [直径(D)] <50.0000>: 7
指定高度或 [两点(2P)/轴端点(A)] <14.0000>:
\end{lstlisting}

\begin{lstlisting}
命令: SUBTRACT
选择要从中减去的实体、曲面和面域...
选择对象: 找到 1 个
选择对象:  选择要减去的实体、曲面和面域...
选择对象: 找到 1 个
选择对象:
\end{lstlisting}

\item 构建图\ref{fig:lunfenxi4.png}所示套筒

\begin{lstlisting}
命令: CYLINDER
指定底面的中心点或 [三点(3P)/两点(2P)/切点、切点、半径(T)/椭圆(E)]:
指定底面半径或 [直径(D)] <7.0000>: 35
指定高度或 [两点(2P)/轴端点(A)] <14.0000>: -6
\end{lstlisting}

\begin{lstlisting}
命令: CYLINDER
指定底面的中心点或 [三点(3P)/两点(2P)/切点、切点、半径(T)/椭圆(E)]:
指定底面半径或 [直径(D)] <35.0000>: 15
指定高度或 [两点(2P)/轴端点(A)] <-6.0000>:
\end{lstlisting}

\begin{lstlisting}
命令: SUBTRACT
选择要从中减去的实体、曲面和面域...
选择对象: 找到 1 个
选择对象:  选择要减去的实体、曲面和面域...
选择对象: 找到 1 个
选择对象:
\end{lstlisting}

\item 组合构建图\ref{fig:lunfenxi5.png}所示模型

\begin{lstlisting}
命令: SUBTRACT
选择要从中减去的实体、曲面和面域...
选择对象: 找到 1 个
选择对象:  选择要减去的实体、曲面和面域...
选择对象: 找到 1 个
选择对象:
\end{lstlisting}

\item 构建图\ref{fig:lunfenxi5.png}所示模型的镜像c

\begin{lstlisting}
命令: mirror3d
选择对象: 找到 1 个
选择对象:
指定镜像平面 (三点) 的第一个点或
  [对象(O)/最近的(L)/Z 轴(Z)/视图(V)/XY 平面(XY)/YZ 平面(YZ)/ZX 平面(ZX)/三点(3)] <三点>: xy
指定 XY 平面上的点 <0,0,0>:
是否删除源对象?[是(Y)/否(N)] <否>:
\end{lstlisting}

\item 合成轮零件三维模型

\begin{lstlisting}
命令: UNION
选择对象: 指定对角点: 找到 2 个
选择对象:
\end{lstlisting}

\item 切换视觉样式为灰度

\item 保存在轮零件三维模型

\end{procedure}
\endinput
\section{轮零件图制作}
在制作轮零件图之前,先将轮零件的三维模型另存在为“轮视图布局.dwg”,并以此副本来进行轮零件图的制作。
\subsection{制作图幅}
为方便控制视图位置,使轮零件的两个视图在图纸中分布比较均均美观,因此首先进行图幅的制作。
\begin{procedure}
\item 进行页面设置

首先点击“布局1”选项卡,从模型空间切换至图纸空间,参照第\ref{sec:lianjieganshitu}节中修改页面设置中的操作方法,将轮零件图的打印机绘图设备设置为“DWG TO PDF.pc3”,图纸幅面设置为“ISO A4(210.00x297.00毫米)”,打印方向设置为“横向”,并将打印机绘图设备特性中的“修改标准图纸尺寸(可打印区域)”设置为带装订边的图幅边尺寸。完成设置后的结果如图\ref{fig:lunshitu1} 所示。
\begin{lstlisting}
命令:pagesetup
\end{lstlisting}
\item 进行图层设置

新建“图框”图层,线型为“continuous”,线宽为0.5毫米;新建“尺寸标注”和“标题栏”图层,线型为“continuous”,线宽为默认;新建“中心线”图层,线型为“center",线宽为默认,结果如图\ref{fig:lunlayerset} 所示。
\begin{lstlisting}
命令:layer
\end{lstlisting}
\begin{figure}[htbp]
\centering
\begin{floatrow}[3]
\ffigbox{\caption{页面设置结果}\label{fig:lunshitu1}}{\includegraphics[scale=0.2]{lunshitu1}}
\ffigbox{\caption{新建图层结果}\label{fig:lunlayerset}}{\includegraphics[scale=0.25]{lunlayerset}}
\ffigbox{\caption{图框绘制结果}\label{fig:lunshitu2}}{\includegraphics[scale=0.2]{lunshitu2}}
\end{floatrow}
\end{figure}
\item 绘制图框

将当前图层设置为“图框”图层,并绘制图框矩形,结果如图\ref{fig:lunshitu2} 所示。
\begin{lstlisting}
命令: RECTANG
指定第一个角点或 [倒角(C)/标高(E)/圆角(F)/厚度(T)/宽度(W)]: 0,0
指定另一个角点或 [面积(A)/尺寸(D)/旋转(R)]: 267,200
\end{lstlisting}

\item 插入标题栏

我们在\ref{sec:lianjieganshitu}节中制作了标题栏块并保为可以共享的文件,因此在本节中可以直接插入标题块来完成标题的制作。首先先将图层切换为“标题栏”图层,然后调用AutoCAD的块插入命令,其方法有:
\begin{itemize}
\item 键盘输入insert\index{insert,块插入}或I
\item 【插入】$\rightarrow $【块】
\item 【绘图】\includegraphics[scale=0.45]{drawtools} 工具栏中的\includegraphics[scale=0.45]{blocktool}图标
\end{itemize}

\begin{figure}[htbp]
\centering
\subfloat[]{\label{fig:insertdialog}\includegraphics[scale=0.3]{insertdialog}}\hspace{20pt}
\subfloat[]{\label{fig:blockfileselect}\includegraphics[scale=0.2]{blockfileselect.png}}\hspace{20pt}
\subfloat[]{\label{fig:insertdialog2}\includegraphics[scale=0.3]{insertdialog2.png}}
\caption{调入块文件过程}
\end{figure}

块插入命令启动后会弹出图\ref{fig:insertdialog}所示的插入对话框,由于标题栏块是以文件形式存在的,所以点击“名称”左边的浏览按钮,调出图\ref{fig:blockfileselect}所示的选择图形文件对话框,选择标题栏块文件所在的位置并选择,点击打开即完成标题块的载入,结果如图\ref{fig:insertdialog2}所示。

标题栏块加载后,点击确定按钮,命令会提示指定插入点,此时按图\ref{fig:insertbiaotilan} 所示位置选择块的插入点,完成后结果如图\ref{fig:lunshitu3} 所示。
\begin{lstlisting}
命令: INSERT
指定插入点或 [基点(B)/比例(S)/旋转(R)]:
\end{lstlisting}

\begin{figure}[htbp]
\centering
\subfloat[]{\label{fig:insertbiaotilan}\includegraphics[scale=0.3]{insertbiaotilan}}\hspace{20pt}
\subfloat[]{\label{fig:lunshitu3}\includegraphics[scale=0.3]{lunshitu3}}
\caption{块插入过程}
\end{figure}
\end{procedure}
\subsection{制作视图}
\begin{procedure}
\item 删除当前视口

观察图\ref{fig:lunshitu3}可以看出,插入标题栏的标题与自动生成的视口存在交叉重叠,生成视图后会相互影响。由于标题栏的尺寸是国家校准规定的,是不能够更改的,所需要删除或修改自动生成的视口,以消除交叉重叠的情况。AutoCAD中调用删除命令的方法有:
\begin{itemize}
\item 键盘输入ERASE\index{erase,删除}或delete\index{delete,删除}或E
\item 【修改】$\rightarrow $【删除】
\item 【修改】\includegraphics[scale=0.45]{edittools} 工具栏中的\includegraphics[scale=0.45]{erase}图标
\end{itemize}

删除命令调用后用鼠标选择视口,选择完成后视口会以虚线的形式显示,如图\ref{fig:eraseselect}。删除后的结果如图\ref{fig:eraseresult}所示。

\begin{figure}[htbp]
\centering
\subfloat[]{\label{fig:eraseselect}\includegraphics[scale=0.3]{eraseselect}}\hspace{20pt}
\subfloat[]{\label{fig:eraseresult}\includegraphics[scale=0.3]{eraseresult}}
\caption{删除视口过程}
\end{figure}
\begin{lstlisting}
命令: ERASE
选择对象: 找到 1 个
选择对象:
\end{lstlisting}

\item 新建视口

由于视口被删除后,图纸空间中已经没有轮零件的模型显示,因此需要在左视图位置新建一个视口来显示轮零件的左视图。AutoCAD中新建视口命令的调用方法有:
\begin{itemize}
\item 键盘输入-vports\index{-vports,新建视口}
\item 【视图】$\rightarrow $【视口】$\rightarrow $【一个视口】
\item 【视口】\includegraphics[scale=0.45]{vportstools}工具栏中的【单个视口】\includegraphics[scale=0.45]{vportssingle}图标
\end{itemize}

新建视口命令调用后提示指定视口的角点,此时按图\ref{fig:vportsfirstnode} 所示位置指定角点。

\begin{figure}[htbp]
\centering
\subfloat[]{\label{fig:vportsfirstnode}\includegraphics[scale=0.2]{vportsfirstnode}}\hspace{20pt}
\subfloat[]{\label{fig:vportssecondnode}\includegraphics[scale=0.2]{vportssecondnode}}\hspace{20pt}
\subfloat[]{\label{fig:vportsresult}\includegraphics[scale=0.2]{vportsresult}}
\caption{新建视口过程}
\end{figure}
\begin{lstlisting}
命令: -VPORTS
指定视口的角点或 [开(ON)/关(OFF)/布满(F)/着色打印(S)/锁定(L)/对象(O)/多边形(P)/恢复(R)/图层(LA)/2/3/4] <布满>:
\end{lstlisting}

接下来,按图\ref{fig:vportssecondnode}所示位置指定对角点,完成后结果如图\ref{fig:vportsresult}所示。
\begin{lstlisting}
指定对角点:
\end{lstlisting}

\item 设置左视图

左视图视口建立完成后,通过双击或输入mspace命令进入视口模型空间。
\begin{lstlisting}
命令: MSPACE
\end{lstlisting}

接下来,将视图方向设置为左视图方向,并将视口比例设置为1:1。
\begin{lstlisting}
命令: -VIEW
输入选项 [?/删除(D)/正交(O)/恢复(R)/保存(S)/设置(E)/窗口(W)]: left
\end{lstlisting}

接下来,提取左视图轮廓,即完成左视图的生成。
\begin{lstlisting}
命令: solprof
选择对象: 找到 1 个
选择对象:
是否在单独的图层中显示隐藏的轮廓线?[是(Y)/否(N)] <是>:
是否将轮廓线投影到平面?[是(Y)/否(N)] <是>:
是否删除相切的边? [是(Y)/否(N)] <是>:
\end{lstlisting}

\item 生成全剖主视图

图\ref{fig:xiaolunlun.pdf}所示的轮零件图的主视图被称为全剖视图。剖视图是为克服由于物体内部形状较复杂时,视图虚线过多,读图和标尺寸困难的不足,而采取的有利于清晰表达物体内部形状的视图表达方法,它是假想用剖切面剖开物体,将位于观察者和剖切面之物的部分移去,而将其余部分向投影面投影所得的图形。所谓全剖视图是用剖切面将物体完全剖开后所得的剖视图。全剖视图适合表达内部开头比较复杂的物体。
\begin{enumerate}
\item 生成截面视图

在AutoCAD中,生成全剖主视图需要用到图纸视图命令,其调用方法有:
\begin{itemize}
\item 键盘输入solview\index{solview,图纸视图}
\item 【绘图】$\rightarrow $【建模】$\rightarrow $【设置】$\rightarrow $【视图】
\end{itemize}

图纸视图命令调用后,提示输入选项,由于要生成全部剖视图,需要指定其假想的剖切面,因此选择【截面(s)】选项。
\begin{lstlisting}
命令: solview
输入选项 [UCS(U)/正交(O)/辅助(A)/截面(S)]: s
\end{lstlisting}

接下来命令提示指定剪切平面的第一个点,此时按图\ref{fig:solviewfirstnode}所示的对象捕捉追踪方式设置第一个点。

\begin{figure}[htbp]
\centering
\subfloat[]{\label{fig:solviewfirstnode}\includegraphics[scale=0.4]{solviewfirstnode}}\hspace{20pt}
\subfloat[]{\label{fig:solviewsecondnode}\includegraphics[scale=0.4]{solviewsecondnode}}\hspace{20pt}
\subfloat[]{\label{fig:solview1}\includegraphics[scale=0.4]{solview1}}
\caption{截面视图生成过程(一)}
\end{figure}

\begin{lstlisting}
指定剪切平面的第一个点:
\end{lstlisting}

接下来,以图\ref{fig:solviewsecondnode}所示的方式指定剪切平面的第二个点,完成后的效果如图\ref{fig:solview1}所示。
\begin{lstlisting}
指定剪切平面的第二个点:
\end{lstlisting}

由于主视图位于左视视的左边,因此需要从左视图的右边向左边观察,因此用鼠标点击图\ref{fig:solview1}中虚线的右侧,以指定观察侧。
\begin{lstlisting}
指定要从哪侧查看:
\end{lstlisting}

接下来,指定视图比例,一般情况下都是直接确认默认值。
\begin{lstlisting}
输入视图比例 <1>:
\end{lstlisting}

接下来以图\ref{fig:solview2}所示的位置作为视图的中心。
\begin{lstlisting}
指定视图中心:
指定视图中心 <指定视口>:
\end{lstlisting}

\begin{figure}[htbp]
\centering
\subfloat[]{\label{fig:solview2}\includegraphics[scale=0.3]{solview2}}\hspace{20pt}
\subfloat[]{\label{fig:solview3}\includegraphics[scale=0.3]{solview3}}
\caption{截面视图生成过程(二)}
\end{figure}

接下来以用鼠标确主视图视口的两个角点。
\begin{lstlisting}
指定视口的第一个角点:
指定视口的对角点:
\end{lstlisting}

指定“F”作为视图名称即可完成截面视图的生成,其效果如图\ref{fig:solview3} 所示。
\begin{lstlisting}
输入视图名: f
输入选项 [UCS(U)/正交(O)/辅助(A)/截面(S)]:
\end{lstlisting}

\item 图形化截面视图

截面视图生成后,还需要将其图形化后,才能够真正地生成剖视图。AutoCAD中调用视图图形化命令的方式有:
\begin{itemize}
\item 键盘输入soldraw\index{soldraw,视图图形化}
\item 【绘图】$\rightarrow $【建模】$\rightarrow $【设置】$\rightarrow $【图形】
\end{itemize}

调用图形化命令后选择主视图所在的视口作为要绘图的视口,选择后的效果如图
\begin{lstlisting}
命令: soldraw
选择要绘图的视口...
选择对象: 找到 1 个
选择对象:
\end{lstlisting}

\begin{figure}[htbp]
\centering
\subfloat[]{\label{fig:soldraw1}\includegraphics[scale=0.3]{soldraw1}}\hspace{20pt}
\subfloat[]{\label{fig:soldraw2}\includegraphics[scale=0.3]{soldraw2}}
\caption{图形化操作}
\end{figure}

\item 修改填充图案

通常情况下,图化后的填充图案并不符合国家标准的要求,需要进行修改。AutoCAD中修改填充图案需要调用编辑填充图案命令,其方式有:
\begin{itemize}
\item 键盘输入hatchedit\index{hatchedit,编辑图案填充}
\item 【修改】$\rightarrow $【对象】$\rightarrow $【图案填空】
\item 【修改II】\includegraphics[scale=0.45]{modifyIItools}工具栏中的【编辑图案填充】\includegraphics[scale=0.45]{hatchedidtool}图标
\end{itemize}

编辑填充图案命令调用后,提示选择图案填充对象。选择主视图中的填充图案后会弹出图\ref{fig:hatcheditdialog}所示的图案填充编辑对话框,点击样例\includegraphics[scale=0.4]{hatchpicture}图案,弹出图\ref{fig:hatchfillpicture}所示的填充图案选项板,然后点选ANSI选项卡,如图\ref{fig:hatchfillselect}所示。选择ANSI37图案并确定,修改后的结果如图\ref{fig:lunshitu5}所示。
\begin{lstlisting}
命令: hatchedit
选择图案填充对象:
\end{lstlisting}

\begin{figure}[htbp]
\centering
\subfloat[]{\label{fig:hatcheditdialog}\includegraphics[scale=0.25]{hatcheditdialog}}\hspace{20pt}
\subfloat[]{\label{fig:hatchfillpicture}\includegraphics[scale=0.35]{hatchfillpicture}}\\
\subfloat[]{\label{fig:hatchfillselect}\includegraphics[scale=0.35]{hatchfillselect}}\hspace{20pt}
\subfloat[]{\label{fig:lunshitu5}\includegraphics[scale=0.3]{lunshitu5}}
\caption{修改图案填充}
\end{figure}

\end{enumerate}
\item 调整图层设置

首先退出模型空间,防止错误的鼠标缩放操作改变视口的显示比例,导致对应关系错误。

\begin{lstlisting}
命令: PSPACE
\end{lstlisting}

按图 所示结果设置图层,设置完成后的轮零件图效果如图 所示。
\begin{lstlisting}
命令:layer
\end{lstlisting}

\begin{figure}[htbp]
\centering
\begin{floatrow}[2]
\ffigbox{\caption{图层设置结果}\label{fig:lunshitulayer}}{\includegraphics[scale=0.3]{lunshitulayer}}
\ffigbox{\label{fig:lunshituresult1}}{\includegraphics[scale=0.3]{lunshituresult1}}
\end{floatrow}
\end{figure}

\end{procedure}
\subsection{标注尺寸}

\begin{figure}[htbp]
\centering
\subfloat[]{\label{fig:lunshitu6}\includegraphics[scale=0.3]{lunshitu6}}\hspace{20pt}
\subfloat[]{\label{fig:lunshitu7}\includegraphics[scale=0.3]{lunshitu7}}\\
\subfloat[]{\label{fig:lunshitu8}\includegraphics[scale=0.3]{lunshitu8}}\hspace{20pt}
\subfloat[]{\label{fig:lunshitu9}\includegraphics[scale=0.3]{lunshitu9}}
\caption{尺寸标注}

\end{figure}
\begin{procedure}
\item 设置标注样式

按\ref{sec:lianjieganshitu}节的步骤设置标注样式。

\item 标注线性尺寸

按\ref{sec:lianjieganshitu}节的步骤进行线性尺寸的标注,结果如图\ref{fig:lunshitu6} 所示。


\item 标注半径尺寸

圆弧的尺寸必须要使用半径方式进行标注。AutoCAD中标注半径需要调用半径标注命令,其调用方法有:
\begin{itemize}
\item 键盘输入dimradius\index{dimradius,半径标注}
\item 【标注】$\rightarrow $【半径】
\item 【标注】\includegraphics[scale=0.45]{dimtoolsbar}工具栏中的【半径】\includegraphics[scale=0.45]{dimradius}图标
\end{itemize}

半径标注命令调用后,选择主视图右下角的$R4$圆弧进行标注,结果如图\ref{fig:lunshitu7}所示。
\begin{lstlisting}
命令: dimradius
选择圆弧或圆:
标注文字 = 4
指定尺寸线位置或 [多行文字(M)/文字(T)/角度(A)]:
\end{lstlisting}

以相同的方式完成$R2$圆弧的标注,结果如图\ref{fig:lunshitu8}所示。
\item 标注直径尺寸

圆的标注必须使用直径的方式进行标注,AutoCAD中直径标注命令的调用方式有:
\begin{itemize}
\item 键盘输入dimdiameter\index{dimdiameter,直径标注}
\item 【标注】$\rightarrow $【直径】
\item 【标注】\includegraphics[scale=0.45]{dimtoolsbar}工具栏中的【半径】\includegraphics[scale=0.45]{dimdiameter}图标
\end{itemize}

直径标注命令调用后,选择左视图中最大的圆作为标注对象,结果如图\ref{fig:lunshitu9}所示。

\begin{lstlisting}
命令: dimdiameter
选择圆弧或圆:
标注文字 = 100
指定尺寸线位置或 [多行文字(M)/文字(T)/角度(A)]:
\end{lstlisting}
\end{procedure}
\endinput
%%%%%%%%%%%%%第五章%%%%%%%%%%%%%%%%%%%%%%%
\chapter{支架}

\endinput
\section{支架三维建模分析}

\subsection{切割法建模}
所谓切割法是从具备包容待建模模型基本形体之中逐步切除多余部分,从而实现三维模型构构建的方法。图\ref{fig:zhijiafenxi0}为构建支架模型的基本楔体,在此基础之上去除圆角之外多余部分材料构成图\ref{fig:zhijiafenxi1}的结果,图\ref{fig:zhijiafenxi2}是去除孔材料的结果,图\ref{fig:zhijiafenxi3}是去除中间部分多余材料的结果。其整个过程是不断的切除多余的材料来实现支架模型。

\begin{figure}[htbp]
\centering
\subfloat[]{\label{fig:zhijiafenxi0}\includegraphics[scale=0.5]{zhijiafenxi0}}\hspace{20pt}
\subfloat[]{\label{fig:zhijiafenxi1}\includegraphics[scale=0.5]{zhijiafenxi1}}\hspace{20pt}
\subfloat[]{\label{fig:zhijiafenxi2}\includegraphics[scale=0.6]{zhijiafenxi2}}\hspace{20pt}
\subfloat[]{\label{fig:zhijiafenxi3}\includegraphics[scale=0.6]{zhijiafenxi3}}
\caption{切割法建模支架}
\end{figure}

\subsection{叠加法建模}

所谓叠加法是将待建模的形体切割为多个组成部分,然后将各个组成部分叠加组合在一起来构建三维模型的建模方法。图\ref{fig:zhijiafenxi4}的顶板和图\ref{fig:zhijiafenxi5}的支耳是支架的基本组成部分,将两部分有效的组合在一起即可构建图\ref{fig:zhijiafenxi6}所示的支架三维模型。

\begin{figure}[htbp]
\centering
\subfloat[]{\label{fig:zhijiafenxi4}\includegraphics[scale=0.5]{zhijiafenxi4}}\hspace{20pt}
\subfloat[]{\label{fig:zhijiafenxi5}\includegraphics[scale=0.5]{zhijiafenxi5}}\hspace{20pt}
\subfloat[]{\label{fig:zhijiafenxi6}\includegraphics[scale=0.9]{zhijiafenxi3}}
\caption{叠加法建模支架}
\end{figure}

在实际的模型构建过程中经常需要将叠加建模法和切割建模法组合起来运用,如图\ref{fig:zhijiafenxi4}的平板和图\ref{fig:zhijiafenxi5}的支耳中的孔是运用切割法来完成。

\endinput
\section{支架三维建模}

切割建模法和叠加建模法不仅适用于实体建模,也适用于采用旋转和拉伸方式构建三维模型。切割建模法和叠加建模法是通用性的方法。本节将采用叠加建模法来构建支架的三维模型。

\subsection{构建顶板}
\begin{procedure}
\item 切换视图方向为西南等轴测

\begin{lstlisting}
命令: -VIEW
输入选项 [?/删除(D)/正交(O)/恢复(R)/保存(S)/设置(E)/窗口(W)]: swiso
\end{lstlisting}

\item 构建顶板基础

忽略顶板的倒角和孔后,其视图所表达的是一长方体。长方体是一种基本几何体,它属于平面立体中的直棱柱。所谓平面立体是指立体表面均由平面构成。在平面立体中,平面立体的表面是由若干个平面多边形构成的,多边形的边是平面立体的轮廓线,也是平面立体两个平面的交线,当轮廓的投影可见时,用粗实线表示;不可见时,用虚线表示;当实线与虚线重合时,应当用粗实线表示。棱柱体通常由顶面、底面及若干个侧棱面构成。棱柱体的各个侧棱相互平行,顶面和底面相互平行。如果棱柱的侧棱与顶面和底面垂直则称为直棱柱,否则称为斜棱柱。当直棱柱的顶面和底面为正多边形时则称为正棱柱。图\ref{fig:cube}和图\ref{fig:cubethreeview}分别为长方体的立体图和三视图。从图\ref{fig:cubethreeview}中可知,长方体的顶面与底面的水平投影重合并反映实形,为一长方形,其它棱面的水平投影积聚为长方形的四条边;前面与后面的正投影重合并反映实形,顶面、底面和两个侧面积聚为长方形的四条件边;左面和右面的侧面投影重合并反映实形,顶面、底面、前面和后面积聚为长方形的四条件边。

\begin{figure}[htbp]
\centering
\subfloat[]{\label{fig:cube}\includegraphics[scale=0.9]{cube.png}}\hspace{30pt}
\subfloat[]{\label{fig:cubethreeview}\includegraphics[scale=1]{cubethreeview.png}}
\caption{长方体的投影}
\end{figure}

因此构建长方体作为顶板建模的基础是比较方便快捷的。在AutoCAD中,调用绘制长方体命令的方法有:
\begin{itemize}
\item 键盘输入box\index{box,长方体}
\item 【绘图】$\rightarrow $【建模】$\rightarrow $【长方体】
\item 【实体】\includegraphics[scale=0.45]{solidtoolbar}工具栏中的【长方体】\includegraphics[scale=0.45]{boxtool}图标
\end{itemize}

调用长方体命令后,命令提示指定第一外角点,此时用鼠标在三维空间中任意指定一点。
\begin{lstlisting}
命令: BOX
指定第一个角点或 [中心(C)]:
\end{lstlisting}

接下来,参照支架零件图中的尺寸以相对坐指定另一个角点。
\begin{lstlisting}
指定其他角点或 [立方体(C)/长度(L)]: @44,36
\end{lstlisting}

然后,依据零件图中的尺寸指定支架顶板的高度,结果如图\ref{fig:zhijiajianmo1}所示。

\begin{lstlisting}
指定高度或 [两点(2P)]: 2
\end{lstlisting}

\begin{figure}[htbp]
\centering
\begin{floatrow}[2]
\ffigbox{\caption{顶板基础}\label{fig:zhijiajianmo1}}{\includegraphics[scale=0.7]{zhijiajianmo1}}
\ffigbox{\caption{三点创建UCS}}{
\subfloat[]{\label{fig:ucsselectnode1}\includegraphics[scale=0.3]{ucsselectnode1}}\hspace{20pt}
\subfloat[]{\label{fig:ucsselectnode2}\includegraphics[scale=0.3]{ucsselectnode2}}\\
\subfloat[]{\label{fig:ucsselectnode3}\includegraphics[scale=0.3]{ucsselectnode3}}\hspace{20pt}
\subfloat[]{\label{fig:zhijiajianmo2}\includegraphics[scale=0.3]{zhijiajianmo2}}
}
\end{floatrow}
\end{figure}

\item 制作顶板孔

\begin{enumerate}
\item 切换用户坐标系

为方便制作支架顶板上的孔,我们将用户坐标系定义到顶板的角点上,其操作方法是采用默认的三点方式,即按图\ref{fig:ucsselectnode1}的方式指定原点,按图\ref{fig:ucsselectnode2}的方式指定$x$轴上的点,按图\ref{fig:ucsselectnode3}的方式指定$y$轴上的点,最终的坐标系效果如图\ref{fig:zhijiajianmo2}所示。从图中可以看出,$z$轴的正方向是指向俯视投影面的。


\begin{lstlisting}
命令: UCS
当前 UCS 名称: *世界*
指定 UCS 的原点或 [面(F)/命名(NA)/对象(OB)/上一个(P)/视图(V)/世界(W)/X/Y/Z/Z 轴(ZA)] <世界>:
指定 X 轴上的点或 <接受>:
指定 XY 平面上的点或 <接受>:
\end{lstlisting}


\item 绘制孔圆柱体

由于用户坐标系已经移动到长方体的角点之上,因此可根据支架零件图中的定圆心定位尺寸用绝对坐标来指定圆柱体的底圆圆心。

\begin{lstlisting}
命令: CYLINDER
指定底面的中心点或 [三点(3P)/两点(2P)/切点、切点、半径(T)/椭圆(E)]: 20,18
指定底面半径或 [直径(D)]: 7
指定高度或 [两点(2P)/轴端点(A)] <2.0000>: 2
\end{lstlisting}

\yaodian{适时的切换用户坐标系能够利用更方便的定位。}

\item 做差集

完成差集操作后,顶板孔的效果如图\ref{fig:zhijiajianmo3}所示。
\begin{lstlisting}
命令: SUBTRACT 
选择要从中减去的实体、曲面和面域...
选择对象: 找到 1 个
选择对象:  选择要减去的实体、曲面和面域...
选择对象: 找到 1 个
选择对象:
\end{lstlisting}

\end{enumerate}

\begin{figure}[htbp]
\centering
\begin{floatrow}[2]
\ffigbox{\caption{制作顶板孔}\label{fig:zhijiajianmo3}}{\includegraphics[scale=0.7]{zhijiajianmo3}}
\ffigbox{\caption{制作倒角边}\label{fig:zhijiajianmo4}}{\includegraphics[scale=0.7]{zhijiajianmo4}}
\end{floatrow}
\end{figure}
\item 倒角边

根据支架零件图可知顶板的上的倒角边是整个面斜斜22度,由于倒角边命令不具备指定角度的功能。因此需要通过三角函数计算另一边的长度才能够准确地倒出22度的倒角边。直接用计算器计算出准确的值后输入的方法虽然能够完成,但这样做并不可取。主要是因为AutoCAD本身就具备计算功能,可以轻松地完成此类计算,故选择[表达式(E)]选项,并输入正确的计算表达式。顶板倒角边后的效果如图\ref{fig:zhijiajianmo4}所示。


\begin{lstlisting}
命令: CHAMFEREDGE
距离 1 = 1.0000,距离 2 = 1.0000
选择一条边或 [环(L)/距离(D)]: d
指定距离 1 或 [表达式(E)] <1.0000>: 2
指定距离 2 或 [表达式(E)] <1.0000>: e
输入表达式: 2*tan(22)
选择一条边或 [环(L)/距离(D)]:
选择同一个面上的其他边或 [环(L)/距离(D)]:
按 Enter 键接受倒角或 [距离(D)]:
\end{lstlisting}

\end{procedure}

\subsection{构建支耳}
\begin{procedure}
\item 制作支耳基础楔体

支架支耳忽略倒圆角后可以简化为一个基本的楔体。在AutoCAD中构建楔体命令的调用方法有:
\begin{itemize}
\item 键盘输入wedge\index{wedge,楔体}
\item 【绘图】$\rightarrow $【建模】$\rightarrow $【楔体】
\item 【实体】\includegraphics[scale=0.45]{solidtoolbar} 工具栏中的【楔体】\includegraphics[scale=0.45]{wedgetool}图标
\end{itemize}

楔体命令调用后以需要指定两个对角点来绘制底面,此操作与长方体定义底面的操作是一致的。此时在顶板的旁边任意点取一点来指定第一个角点,然后根据尺寸用相对坐标指定第二个角点。

\begin{lstlisting}
命令: wedge
指定第一个角点或 [中心(C)]:
指定其他角点或 [立方体(C)/长度(L)]: @44,3
\end{lstlisting}

接下来需要指定楔体的高度,由于无法从零件图中直接获取高度尺寸,因此也需要通过计算来获取高度值。但是楔体命令没有表达式选项,如何才能够使用AutoCAD的计算功能呢?其方法是调用计算器命令,即在命令行中输入\lstinline{'cal} \index{cal,计算器}来临时调用计算器,然后输入表达式进行计算。最终结果如图\ref{fig:zhijiajianmo5}所示。
\begin{lstlisting}
指定高度或 [两点(2P)] <2.0000>: 'cal
>>>> 表达式: 44/tang(22)
正在恢复执行 WEDGE 命令。
指定高度或 [两点(2P)] <2.0000>: 108.90382155032
\end{lstlisting}

\begin{figure}[htbp]
\centering
\begin{floatrow}[3]
\ffigbox{\caption{支耳基础楔体}\label{fig:zhijiajianmo5}}{\includegraphics[scale=0.4]{zhijiajianmo5}}
\ffigbox{\caption{楔体倒圆角}\label{fig:zhijiajianmo6}}{\includegraphics[scale=0.6]{zhijiajianmo6}}
\ffigbox{\caption{构建支耳孔}\label{fig:zhijiajianmo7}}{\includegraphics[scale=0.6]{zhijiajianmo7}}
\end{floatrow}
\end{figure}
\item 倒支耳圆角边

接下来,用圆角边命令制作$R10$的圆角边,效果如图\ref{fig:zhijiajianmo6}所示。

\begin{lstlisting}
命令: FILLETEDGE
半径 = 1.0000
选择边或 [链(C)/环(L)/半径(R)]: r
输入圆角半径或 [表达式(E)] <1.0000>: 10
选择边或 [链(C)/环(L)/半径(R)]:
选择边或 [链(C)/环(L)/半径(R)]:
已选定 1 个边用于圆角。
按 Enter 键接受圆角或 [半径(R)]:
\end{lstlisting}

\item 制作孔

\begin{enumerate}
\item 切换用户坐标系

由于AutoCAD中圆柱体底面是与$XY$平面平行的,而支耳中的圆柱孔的底面是与主视图平行的,因此需要切换用用户坐标系,使用户坐标系中的$XY$平面与主视图平行。

\begin{lstlisting}
命令: UCS
当前 UCS 名称: *没有名称*
指定 UCS 的原点或 [面(F)/命名(NA)/对象(OB)/上一个(P)/视图(V)/世界(W)/X/Y/Z/Z 轴(ZA)] <世界>:
指定 X 轴上的点或 <接受>:
指定 XY 平面上的点或 <接受>:
\end{lstlisting}

\yaodian{理解AutoCAD基本实体与$XY$平面之间的关系,能够有助于建立更有针对性的用户坐标系。}

\item 构建孔圆柱体

以支耳的圆角边的前圆心做为孔圆柱体的底圆圆心来构建孔圆柱体。
\begin{lstlisting}
命令: CYLINDER
指定底面的中心点或 [三点(3P)/两点(2P)/切点、切点、半径(T)/椭圆(E)]:
指定底面半径或 [直径(D)] <7.0000>: 4
指定高度或 [两点(2P)/轴端点(A)] <108.9038>: 3
\end{lstlisting}

\item 做差集

然后,从支耳中去除孔的圆柱体便可得到支耳的孔,效果如图\ref{fig:zhijiajianmo7} 所示。
\begin{lstlisting}
命令: SUBTRACT 
选择要从中减去的实体、曲面和面域...
选择对象: 找到 1 个
选择对象:  选择要减去的实体、曲面和面域...
选择对象: 找到 1 个
选择对象:
\end{lstlisting}

\end{enumerate}
\end{procedure}

\subsection{组合构建支架}
\begin{procedure}
\item 叠加支耳

由于顶板和支耳是分别制作的,故需要将两者组合起来构成一个整个。因此需要将支耳叠加到顶板之上;当然也可以将顶板叠加在支耳之上,两者效果是一样的。在AutoCAD中实现两对象叠加命令有移动和三维对齐。移命令只能用于已经建立的模型组件与实际需要方向一致的情况,三维对齐命令不仅能够适用于方向一致的情况,也适用于不一致的情况。当模型组件与实际方向一致时,使用移动命令则最为简便。AutoCAD中调用移动命令的方法有:
\begin{itemize}
\item 键盘输入move\index{move,移动}或M
\item 【修改】$\rightarrow $【移动】
\item 【修改】\includegraphics[scale=0.45]{edittools}工具栏中的【移动】\includegraphics[scale=0.45]{movetool}图标
\end{itemize}

\begin{figure}[htbp]
\centering
\subfloat[]{\label{fig:moveselectobj}\includegraphics[scale=0.5]{moveselectobj}}\hspace{20pt}
\subfloat[]{\label{fig:moveselectbase}\includegraphics[scale=0.5]{moveselectbase}}\hspace{20pt}
\subfloat[]{\label{fig:moveselectsec}\includegraphics[scale=0.5]{moveselectsec}}\hspace{20pt}
\subfloat[]{\label{fig:moveresult}\includegraphics[scale=0.5]{moveresult}}
\caption{移动操作过程}
\end{figure}

调用移动命令后,会提示选择要移动的对象,此时用鼠标选择支耳作为要移动的对象,选择后会以虚线形式显示,效果如图\ref{fig:moveselectobj}所示。

\begin{lstlisting}
命令: MOVE
选择对象: 找到 1 个
选择对象:
\end{lstlisting}

接下来选择图\ref{fig:moveselectbase}所示的点作为基点。

\begin{lstlisting}
指定基点或 [位移(D)] <位移>:
\end{lstlisting}

最后选择图\ref{fig:moveselectsec}所示的点作为第二点,操作完成后即可支耳移动到顶板之上,效果如图\ref{fig:moveresult}所示。

\begin{lstlisting}
指定第二个点或 <使用第一个点作为位移>:
\end{lstlisting}

\item 制作支耳镜像

支架具有两个支耳,因此需要使用镜像的方法直接生成另一个支耳,在本例中将使用三点方式来指定镜像面。即选择图\ref{fig:mirror3dselectnode1}所示中点作为镜像平面上的第一点,选择图\ref{fig:mirror3dselectnode2}所示中点作为镜像平面上的第二点,选择图\ref{fig:mirror3dselectnode3}所示中点作为镜像平面上的第三点。完成后效果如图\ref{fig:zhijiajianmo8}所示。

\begin{figure}[htbp]
\centering
\subfloat[]{\label{fig:mirror3dselectnode1}\includegraphics[scale=0.5]{mirror3dselectnode1}}\hspace{20pt}
\subfloat[]{\label{fig:mirror3dselectnode2}\includegraphics[scale=0.5]{mirror3dselectnode2}}\hspace{20pt}
\subfloat[]{\label{fig:mirror3dselectnode3}\includegraphics[scale=0.5]{mirror3dselectnode3}}\hspace{20pt}
\subfloat[]{\label{fig:zhijiajianmo8}\includegraphics[scale=0.5]{zhijiajianmo8}}
\end{figure}

\begin{lstlisting}
命令: MIRROR3D
选择对象: 找到 1 个
选择对象:
指定镜像平面 (三点) 的第一个点或
  [对象(O)/最近的(L)/Z 轴(Z)/视图(V)/XY 平面(XY)/YZ 平面(YZ)/ZX 平面(ZX)/三点(3)] <三点>: 
  在镜像平面上指定第二点: 
  在镜像平面上指定第三点:
是否删除源对象?[是(Y)/否(N)] <否>:
\end{lstlisting}

\item 合并实体

接下来将顶板和两个支耳组合在一起,形成一个整个体,为另外两圆边创造条件。

\begin{lstlisting}
命令: UNION
选择对象: 指定对角点: 找到 3 个
选择对象:
\end{lstlisting}

\item 倒圆角边

完成合并操作后,需要构建支架中两个$R2$圆角边的才能够最终完成支架的建模,完成后效果如图\ref{fig:zhijiajianmo9}所示。

\begin{lstlisting}
命令: FILLETEDGE
半径 = 10.0000
选择边或 [链(C)/环(L)/半径(R)]: r
输入圆角半径或 [表达式(E)] <10.0000>: 2
选择边或 [链(C)/环(L)/半径(R)]:
选择边或 [链(C)/环(L)/半径(R)]:
选择边或 [链(C)/环(L)/半径(R)]:
已选定 2 个边用于圆角。
按 Enter 键接受圆角或 [半径(R)]:
\end{lstlisting}

\begin{figure}[htbp]
\centering
\begin{floatrow}[2]
\ffigbox{\caption{支架倒圆角}\label{fig:zhijiajianmo9}}{\includegraphics[scale=0.5]{zhijiajianmo9}}
\ffigbox{\caption{支架三维模型}\label{fig:zhijiajianmo10}}{\includegraphics[scale=0.5]{zhijiajianmo10}}
\end{floatrow}
\end{figure}
\item 切换视觉样式为灰度


\begin{lstlisting}
命令: VSCURRENT
输入选项 [二维线框(2)/线框(W)/隐藏(H)/真实(R)/概念(C)/着色(S)/带边缘着色(E)/灰度(G)/勾画(SK)/X 射线(X)/其他(O)] <二维线框>: g
\end{lstlisting}

将视觉样式切换为灰度后,最终的支架三维模型效果如图\ref{fig:zhijiajianmo10}所示。
\item 保存支架模型

最后将建立的支架三维模型以“支架.dwg”的文件名予以保存。

\end{procedure}

\endinput
\section{支架零件图制作}
\begin{procedure}

\begin{figure}[htbp]
\centering
\subfloat[]{\label{fig:layoutwizard1}\includegraphics[scale=0.15]{layoutwizard1}}\hspace{20pt}
\subfloat[]{\label{fig:layoutwizard2}\includegraphics[scale=0.15]{layoutwizard2}}\hspace{20pt}
\subfloat[]{\label{fig:layoutwizard3}\includegraphics[scale=0.15]{layoutwizard3}}
\caption{新建布局过程(一)}
\end{figure}

\item 创建新布局

由于系统创建的窗口布局通常是不能够满足实际布局需求的,为适应支架三视图生成的需要,我们可以采用新建布局向导来进行设置。AutoCAD中调用新建布局向导的方法有:
\begin{itemize}
\item 键盘输入layoutwizard\index{layoutwizard,创建布局向导}
\item 【插入】$\rightarrow $【布局】$\rightarrow $【创建布局向导】
\end{itemize}

调用布局向导命令会弹出图\ref{fig:layoutwizard1}所示的创建布局对话框,在文本框中输入新布局的名称;点击下一步进入图\ref{fig:layoutwizard2}所示的打印机设置步骤,将打印机设置为“DWG TO PDF.pc3”;点击下一步进入图\ref{fig:layoutwizard3}所示的图纸尺寸设置步骤,将图纸尺寸设置为“ISO A4(297.00x210.00毫米)";点击下步进入图\ref{fig:layoutwizard4}所示的方向设置步骤,由于A4图纸比较小,对放置三视图而言设置为纵向比较合理,故选择纵向;点击下一下进入图\ref{fig:layoutwizard5}所示的标题栏设置步骤,由于无符合国家标准的标题栏,故选择无;点击下一步进入图\ref{fig:layoutwizard6}所示的定义视口步骤,为方便后续操作,将其设置为无视口;点击下一步进入图\ref{fig:layoutwizard7}完成布局创建。

\begin{figure}[htbp]
\centering
\subfloat[]{\label{fig:layoutwizard4}\includegraphics[scale=0.15]{layoutwizard4}}\hspace{20pt}
\subfloat[]{\label{fig:layoutwizard5}\includegraphics[scale=0.15]{layoutwizard5}}\hspace{20pt}
\subfloat[]{\label{fig:layoutwizard6}\includegraphics[scale=0.15]{layoutwizard5}}
\caption{新建布局过程(二)}
\end{figure}
\begin{lstlisting}
命令:LAYOUTWIZARD
\end{lstlisting}

\begin{figure}[htbp]
\centering
\subfloat[]{\label{fig:layoutwizard7}\includegraphics[scale=0.15]{layoutwizard7}}\hspace{20pt}
\subfloat[]{\label{fig:zhijiabuju1}\includegraphics[scale=0.2]{zhijiabuju1}}
\caption{新建布局过程(三)}
\end{figure}

通过向导创建的布局其可打印区域通常不能够与标准图幅的要求是不一致的,因此需要利用页面设置管理器设置图纸可打印区域,最终结果如图\ref{fig:zhijiabuju1}所示。

\item 新建图层

新建“尺寸标注”和“标题栏”图层,线型为“continuous",线宽为默认;新建“中心线”图层,线型为“center",线宽为默认; 新建“图框”图层,线型为“continuous",线宽为0.5mm。

\item 制作图框与标题栏

设置当前图层为“图框”,绘制图框矩形。
\begin{lstlisting}
命令: RECTANG
指定第一个角点或 [倒角(C)/标高(E)/圆角(F)/厚度(T)/宽度(W)]: 0,0
指定另一个角点或 [面积(A)/尺寸(D)/旋转(R)]: @180,287
\end{lstlisting}

设置当前图层为“标题栏”,插入“GB标题栏”块。
\begin{lstlisting}
命令: INSERT
指定插入点或 [基点(B)/比例(S)/旋转(R)]:
\end{lstlisting}

\begin{figure}[htbp]
\centering

\end{figure}
\item 加载虚线线型

为方便于生成视图时系统自动设置虚线,需要先加载虚线线型。在AutoCAD中加载线型可调用线型管理器,其调用方式为:
\begin{itemize}
\item 键盘输入linetype\index{linetype}
\item 【格式】$\rightarrow $【线型】
\end{itemize}

调用线型管理器命令会弹出图\ref{fig:linetypemanarge}所示的对话框,点击加载按钮,并加载“HIDDEN”线型,结果如图\ref{fig:loadhiddenline}所示。

\begin{figure}[htbp]
\centering
\subfloat[]{\label{fig:linetypemanarge}\includegraphics[scale=0.15]{linetypemanarge}}\hspace{20pt}
\subfloat[]{\label{fig:loadhiddenline}\includegraphics[scale=0.15]{loadhiddenline}}
\caption{线型管理}
\end{figure}

\begin{lstlisting}
命令: LINETYPE
\end{lstlisting}

\item 生成左视图

\begin{enumerate}
\item 新建左视图视口。

\begin{lstlisting}
命令: -VPORTS
指定视口的角点或 [开(ON)/关(OFF)/布满(F)/着色打印(S)/锁定(L)/对象(O)/多边形(P)/恢复(R)/图层(LA)/2/3/4] <布满>:
指定对角点:
\end{lstlisting}

\item 设置左视同图视口模型空间的视图方向为左视。
\begin{lstlisting}
命令: -VIEW
输入选项 [?/删除(D)/正交(O)/恢复(R)/保存(S)/设置(E)/窗口(W)]: left
\end{lstlisting}

\item 设置视口显示比例为1:1,结果如图\ref{fig:zhijiabuju3} 所示。

\end{enumerate}

\begin{figure}[htbp]
\centering
\begin{floatrow}[3]
\ffigbox{\caption{创建支架左视图显示}\label{fig:zhijiabuju3}}{\includegraphics[scale=0.2]{zhijiabuju3}}
\ffigbox{\caption{创建支架主视图显示}\label{fig:zhijiabuju4}}{\includegraphics[scale=0.2]{zhijiabuju4}}
\ffigbox{\caption{创建支架俯视图显示}\label{fig:zhijiabuju5}}{\includegraphics[scale=0.2]{zhijiabuju5}}
\end{floatrow}
\end{figure}

\item 生成全剖主视图

为更清楚地表达支架的内部结构,故采用全剖视图作为支架的主视图,能够实现既清楚表达支耳的结构又清楚表达顶板的孔。生成主视图的效果如图\ref{fig:zhijiabuju4}所示。

\begin{lstlisting}
命令: SOLVIEW
输入选项 [UCS(U)/正交(O)/辅助(A)/截面(S)]: s
指定剪切平面的第一个点:
指定剪切平面的第二个点:
指定要从哪侧查看:
输入视图比例 <1>:
指定视图中心:
指定视图中心 <指定视口>:
指定视口的第一个角点: 
指定视口的对角点:
输入视图名: front
输入选项 [UCS(U)/正交(O)/辅助(A)/截面(S)]:
\end{lstlisting}


\item 生成俯视图

双击进入主视图所在的视口模型空间,然后以主视图为基础用正交选项生成普通的俯视图,结果如图\ref{fig:zhijiabuju5}所示。

\begin{lstlisting}
命令: SOLVIEW
输入选项 [UCS(U)/正交(O)/辅助(A)/截面(S)]: o
指定视口要投影的那一侧:
指定视图中心:
指定视图中心 <指定视口>:
指定视口的第一个角点:
指定视口的对角点:
输入视图名: top
输入选项 [UCS(U)/正交(O)/辅助(A)/截面(S)]:
\end{lstlisting}

\item 生成左视图轮廓

到目前为止,还没有生成左视图。由于左视图是生成主视图的基础,因此只能用轮廓命令生成左视图。

\begin{lstlisting}
命令: SOLPROF
选择对象: 找到 1 个
选择对象:
是否在单独的图层中显示隐藏的轮廓线?[是(Y)/否(N)] <是>:
是否将轮廓线投影到平面?[是(Y)/否(N)] <是>:
是否删除相切的边? [是(Y)/否(N)] <是>:
\end{lstlisting}

\item 图形化主视图和俯视图

\begin{figure}[htbp]
\centering
\subfloat[]{\label{fig:zhijiabuju6}\includegraphics[scale=0.2]{zhijiabuju6}}\hspace{20pt}
\subfloat[]{\label{fig:zhijiabuju7}\includegraphics[scale=0.2]{zhijiabuju7}}
\caption{生成最终主俯视图}
\end{figure} 

选择主视图和俯视图作为图形化对象,选择后的效果如图\ref{fig:zhijiabuju6}所示。

\begin{lstlisting}
命令: SOLDRAW
选择要绘图的视口...
选择对象: 找到 1 个
选择对象: 找到 1 个,总计 2 个
选择对象:
\end{lstlisting}

\item 修改主视图图案填充

由于主视图的图案填充不符合国家标准的要求,故用图案填充编辑命令将填充图案修改为ANSI31。最终主视图和俯视图效果如图\ref{fig:zhijiabuju7}所示。

\begin{lstlisting}
命令: HATCHEDIT
选择图案填充对象:
\end{lstlisting}

\item 设置图层

按图\ref{fig:zhijiabuju8}进行支架的图层设置。
\begin{figure}[htbp]
\centering
\begin{floatrow}[2]
\ffigbox{\caption{支架图层设置}\label{fig:zhijiabuju8}}{\includegraphics[scale=0.2]{zhijiabuju8}}
\ffigbox{\caption{绘制中心线结果}\label{fig:zhijiabuju9}}{\includegraphics[scale=0.2]{zhijiabuju9}}
\end{floatrow}
\end{figure}
\item 绘制中心线

将当前图层设置为“中心线”图层,并绘制支架孔的中心线,结果如图\ref{fig:zhijiabuju9}所示。

\item 标注尺寸
\begin{enumerate}
\item 设置标注样式

按\ref{sec:lianjieganshitu}节的步骤设置标注样式。

\item 标注线性尺寸

完成支架三视图中所有线性尺寸的标注,结果如图\ref{fig:zhijiabuju10} 所示。

\begin{figure}[htbp]
\centering
\subfloat[]{\label{fig:zhijiabuju10}\includegraphics[scale=0.2]{zhijiabuju10}}\hspace{20pt}
\subfloat[]{\label{fig:zhijiabuju11}\includegraphics[scale=0.2]{zhijiabuju11}}\hspace{20pt}
\subfloat[]{\label{fig:zhijiabuju12}\includegraphics[scale=0.2]{zhijiabuju12}}\hspace{20pt}
\subfloat[]{\label{fig:zhijiabuju13}\includegraphics[scale=0.2]{zhijiabuju13}}
\caption{支架尺寸标注}
\end{figure}

\item 标注半径尺寸

完成支架三视图中所有圆弧的半径尺寸标注,结果如图\ref{fig:zhijiabuju11}所示。

\item 标注直径尺寸

完成支架三视图中所有圆的直径尺寸标注,结果如图\ref{fig:zhijiabuju12}所示。
\item 标注角度尺寸

对于支架中的角度尺寸需要调用AutoCAD中的角度标注命令,其方法有:
\begin{itemize}
\item 键盘输入dimangular\index{dimanglar,角度标注}
\item 【标注】$\rightarrow $【角度】
\item 【标注】\includegraphics[scale=0.45]{dimtoolsbar}工具栏中的【角度】\includegraphics[scale=0.45]{dimangular}图标
\end{itemize}

调用角度标注命令后,选择两条直线并指定标注线的位置即可完成角度标注,其结果如图\ref{fig:zhijiabuju13}所示。

\begin{lstlisting}
命令: dimangular
选择圆弧、圆、直线或 <指定顶点>:
选择第二条直线:
指定标注弧线位置或 [多行文字(M)/文字(T)/角度(A)/象限点(Q)]:
标注文字 = 22
\end{lstlisting}

\end{enumerate}

\end{procedure}
\endinput
\section{基本几何体}

根据表面形状的不同可以将基本几何体分为平面立体和曲面立体。如果立体表面均由平面构成,则称为平面立体,如长方体、正方体、棱柱、棱锥、棱台等。如果立体表面由平面和曲面共同构成或全部由曲面构成,则称为曲面立体,如圆柱、圆球、圆环等。
\subsection{平面立体}
在平面立体中,平面立体的表面是由若干个平面多边形构成的,多边形的边是平面立体的轮廓线,是平面立体两个平面的交线,当轮廓的投影可见时,用粗实线表示;不可见时,用虚线表示;当实线与虚线重合时,应当用粗实线表示。
\subsubsection{棱柱体}
棱柱体由顶面、底面及若干个侧棱面构成。棱柱体的各个侧棱相互平行,顶面和底面相互平行。如果棱柱的侧棱与顶面和底面垂直则称为直棱柱,否则称为斜棱柱。当直棱柱的顶面和底面为正多边形时则称为正棱柱。
\begin{figure}[htbp]
\centering
\subfloat[]{\label{fig:cube}\includegraphics[scale=0.9]{cube.png}}\hspace{30pt}
\subfloat[]{\label{fig:cubethreeview}\includegraphics[scale=1]{cubethreeview.png}}
\caption{长方体的投影}
\end{figure}

图\ref{fig:cubethreeview}所示的长方体,其顶面与底面的水平投影重合并反映衬形,为一长方形,其它棱面的水平投影积聚为长方形的四条边;前面与后面的正投影重合并反映实形,顶面、底面和两个侧面积聚为长方形的四条件边;左面和右面的侧面投影重合并反映实形,顶面、底面、前面和后面积聚为长方形的四条件边。长方体的三视图投影如图\ref{fig:cube}所示。

图\ref{fig:sannenzhu}所示的正三棱柱,其顶面与底面的水平面投影重合并反映实形,为一正三角形。三个棱面在水平投影面积聚为三角形的三条边。三棱柱的三视图投影如图\ref{fig:sannenzhuthreeview}所示。
\begin{figure}[htbp]
\centering
\subfloat[]{\label{fig:sannenzhu}\includegraphics[scale=0.6]{sannenzhu.png}}\hspace{30pt}
\subfloat[]{\label{fig:sannenzhuthreeview}\includegraphics[scale=1]{sannenzhuthreeview.png}}
\caption{正三棱柱的投影}
\end{figure}

图\ref{fig:sixnenzhu}所示的正六棱柱,其顶面与底面的水平面投影重合并反映实形,为一正六边形。六个棱面在水平投影面积聚为六边形的六条边。正六棱柱的三视较长投影如图\ref{fig:sixnenthreeview}所示。
\begin{figure}[htbp]
\centering
\subfloat[]{\label{fig:sixnenzhu}\includegraphics[scale=0.7]{sixnenzhu.png}}\hspace{30pt}
\subfloat[]{\label{fig:sixnenthreeview}\includegraphics[scale=1]{sixnenthreeview.png}}
\caption{正六棱柱的投影}
\end{figure}

由此可见棱柱体的投影特点是:一面投影反映底面实形,其余两面投影则为矩形或复合矩形。
\subsubsection{棱锥体}
棱锥体是由一个多边形底面和若干个共顶点的三角形棱面构成的。从棱锥体顶点到底面的垂直距离称为棱锥体的高。如果棱锥体的底面为正多边形,锥顶的投影位于多边形的中心,各棱面是等腰三角形,则该棱锥体称为正棱锥。正四棱锥的三面投影如图\ref{fig:fournenzhuithreeview}所示。

\begin{figure}[htbp]
\centering
\subfloat[]{\label{fig:fournengzhui}\includegraphics[scale=0.9]{fournengzhui.png}}\hspace{30pt}
\subfloat[]{\label{fig:fournenzhuithreeview}\includegraphics[scale=1]{fournenzhuithreeview.png}}
\caption{正四棱锥的投影}
\end{figure}

图\ref{fig:fournengzhui}所示的正四棱锥的底面与水平投影面平行,其投影反映实形,为正方形;底面在其它投影面积聚为一条直线;棱面的三面投影则为类似的三角形。

由此可见,棱锥体的投影特点是:一投影面为由三角形构成的复合多边形,其两投影为三角形或复合三角形。
\subsubsection{棱台体}
棱台体是由棱锥体被切掉顶部后所构成的一种形体。棱台体的投影特点是:一面投影为由梯形构成的内外相似复合多边形,其余两面投影则为梯形或复合梯形。图\ref{fig:fivenentai} 所示为五棱台的投影。
\begin{figure}[htbp]
\centering
\subfloat[]{\includegraphics[scale=0.9]{fivenentai.png}}\hspace{30pt}
\subfloat[]{\includegraphics[scale=1]{fivenentaithreeview.png}}
\caption{五棱台的投影}\label{fig:fivenentai}
\end{figure}

\subsection{曲面立体}
曲面体是由面或曲面和平面共同构成的立体,其中最常见的是回转曲面。回转体曲面是由一母线绕一空间轴线作旋转运动而形成的光滑曲面。母线在在回转曲面上任意位置称作素线,母线上任意一点的旋转轨迹都是一个圆,该圆称为纬圆。图\ref{fig:huizhuangti}所示为回转体的形成和术语。
\begin{figure}[htbp]
\centering
\subfloat[]{\includegraphics[scale=0.5]{huizhuangti.png}}\hspace{30pt}
\subfloat[]{\includegraphics[scale=0.5]{rotatetree.png}}
\caption{回转体的形成及术语}\label{fig:huizhuangti}
\end{figure}

画回转体的投影图时,首先画出回转轴线,然后画出反映实形的投影,最后画其余两个投影。回转曲面向某一投影面投影时,轮廓素线是回转曲面在该投影面上可见和不可见面的分界线的投影,在分界线之前的回转曲面为可见,反之则不可见。画图时,凡不属于该投影面的轮廓素线,一律不画出。
\subsubsection{圆柱体}
圆柱体是由圆柱面、顶面、底面所构成的。圆柱体可以看作一条与轴线平行的母线绕轴线旋转而成的。图\ref{fig:yuanzhutix}所示圆柱体的轴线垂直于水平投影面,其水平投影为圆;其正投影和侧投影均为矩形。
\begin{figure}[htbp]
\centering
\subfloat[]{\includegraphics[scale=0.9]{yuanzhuti.png}}\hspace{30pt}
\subfloat[]{\includegraphics[scale=0.7]{yuanzhutithreeview.png}}
\caption{圆柱体的投影}\label{fig:yuanzhutix}
\end{figure}

\subsubsection{球体}
球体是由圆形母线以其直径为回转轴旋转而成的。球体的三面投影均为圆。

\begin{figure}[tbp]
\centering
\subfloat[]{\includegraphics[scale=0.7]{qiouti.png}}\hspace{30pt}
\subfloat[]{\includegraphics[scale=0.7]{qioutithreeview.png}}
\caption{球体的投影}\label{fig:qiout}
\end{figure}

\endinput
%%%%%%%%%%%%%第六章%%%%%%%%%%%%%%%%%%%%%%%
\chapter{小轮组装配}

本章的目标是构建小轮组的三维装配模型,并介绍如何利用三维装配模型制作小轮组的装配图。本章将讲述以下内容:
\begin{itemize}
\item AutoCAD模板的定制
\item 小轮组三维装配模型的制作
\item 小轮组装配图的制作
\item 装配图知识
\end{itemize}

\section{定制标注样式样板}
在前面几章中,我们每次标注尺寸时都需要按照\ref{sec:lianjieganshitu}节的尺寸样式设置步骤进行标注样式设置,其过程比较繁琐,能否有一种方法可以实现一次定制重复使用呢?要实现这样的要求,可利用AutoCAD的模板功能。下面是制作带有国家标准标注样式的AutoCAD模板的基本步骤。
\begin{procedure}
\item 新建文档

在AutoCAD中新建文档的方式有很多,常用的方式有:
\begin{itemize}
\item 键盘输入new\index{new,新建}
\item \includegraphics[scale=0.3]{keyctrl}+\includegraphics[scale=0.3]{keyn}
\item 【文件】$\rightarrow $【新建】
\item 【标准】\includegraphics[scale=0.45]{biaozhuntools}工具栏中的\includegraphics[scale=0.45]{newtool}图标
\item 【快速访问】\includegraphics[scale=0.45]{quicktools}中的\includegraphics[scale=0.45]{newtool}图标
\end{itemize}

命令启动后弹出图\ref{fig:newfiletemplate}所示的选择样板对话框,选择acadiso.dwt文件并点击打开按钮。

\begin{figure}[htbp]
\centering
\includegraphics[scale=0.5]{newfiletemplate}
\caption{新建文件对话}\label{fig:newfiletemplate}
\end{figure}

\begin{lstlisting}
命令:new
\end{lstlisting}

\item 设置文字样式

按照\ref{sec:lianjieganshitu}节的方法设置文字样式。

\item 设置标注样式

按照\ref{sec:lianjieganshitu}节的方法设置标注样式。

\item 保存为样板文件

完成设置后,点击保存或另存为,弹出图\ref{fig:GBacadisotemplate}所示的图形另存为对话框,在文件下拉列表框中选择“AutoCAD 图形样板文件.dwt”,并以“GBacadiso.dwt”文件名进行保存,弹出图\ref{fig:GBtamplate}所示的样板选项对话框,在其中输入说明后点击确定即可生成样板文件。


\begin{figure}[htbp]
\centering
\begin{floatrow}[2]
\ffigbox{\caption{保存为样板}\label{fig:GBacadisotemplate}}{\includegraphics[scale=0.3]{GBacadisotemplate}}
\ffigbox{\caption{样板选项对话框}\label{fig:GBtamplate}}{\includegraphics[scale=0.4]{GBtamplate}}
\end{floatrow}
\end{figure}

\item 关闭样板文件

完成样板文件定制后需要关闭。在新建文件时选择此样板文件来创建新文件,新文件就已经具备标准的文字样式设置和标注样式设置。
\end{procedure}
\endinput
\section{构建小轮组三维模型}
\begin{procedure}
\item 初始化绘图环境

\begin{enumerate}
\item 新建带国家标准标注样式的文件。

以带标国家标准标注样式的样板文件“GBacadiso”为样板创建新的AutoCAD绘图文件。

\item 切换至西南等轴测图。

\begin{lstlisting}
命令: -VIEW
输入选项 [?/删除(D)/正交(O)/恢复(R)/保存(S)/设置(E)/窗口(W)]: swiso
\end{lstlisting}

\item 定义用户坐标系,使其$xy$平面与左视图投影面平行。

由于小轮组的组成零件中大多数的回转体特征面与左视图投影面平行,因此本例选择将用户坐标系定义为与左视图投影面平行。当然也可以不进行此操作,只是显示结果与本例会略有差异,但操作过程是基本一致的。

\begin{lstlisting}
命令: UCS
当前 UCS 名称: *世界*
指定 UCS 的原点或 [面(F)/命名(NA)/对象(OB)/上一个(P)/视图(V)/世界(W)/X/Y/Z/Z 轴(ZA)] <世界>: za
指定新原点或 [对象(O)] <0,0,0>:
在正 Z 轴范围上指定点 <0.0000,0.0000,1.0000>: -1,0,0
\end{lstlisting}
\end{enumerate}

\item 复制小轮组组件三维模型

用AutoCAD打开之前建立好的轮零件三维模型文件,调出轮零件的三维模型。为实现将轮零件的三维模型复制到新建的AutoCAD文件之中,需要先将轮零件的三维模型先复制来剪切板之中,然后再将其粘粘贴到新建到的AutoCAD文件之中。通常实现将AutoCAD对象复制到剪切板的方法有:
\begin{itemize}
\item 键盘输入copyclip\index{copyclip,复制到剪切板}
\item \includegraphics[scale=0.3]{keyctrl}+\includegraphics[scale=0.3]{keyc}
\item 【编辑】$\rightarrow $【复制】
\item 【标准】\includegraphics[scale=0.45]{biaozhuntools}工具栏中的\includegraphics[scale=0.45]{copyclip}图标
\end{itemize}

命令调用后选择轮零件的三维模型并结束选择即可将其复制到剪切板之中。
\begin{lstlisting}
命令: COPYCLIP
选择对象: 找到 1 个
选择对象:
\end{lstlisting}

接下来进入新建的AutoCAD文件之中,调用粘贴命令来实现将剪切板中的轮零件三维模型复制到文件之中。通常调用粘贴命令的方法有:
\begin{itemize}
\item 键盘输入pasteclip\index{pasteclip,粘贴自剪切板}
\item \includegraphics[scale=0.3]{keyctrl}+\includegraphics[scale=0.3]{keyv}
\item 【编辑】$\rightarrow $【粘贴】
\item 【标准】\includegraphics[scale=0.45]{biaozhuntools}工具栏中的\includegraphics[scale=0.45]{pasteclip}图标
\end{itemize}

命令调用后,在绘图区中任意选取一点作为插入点,即可完成轮零件三维的模型的复制,结果如图\ref{fig:xiaolunzhuangpei1} 所示。
\begin{lstlisting}
命令:  PASTECLIP
指定插入点:
\end{lstlisting}


\begin{figure}[htbp]
\centering
\subfloat[]{\label{fig:xiaolunzhuangpei1}\includegraphics[scale=0.4]{xiaolunzhuangpei1}}\hspace{20pt}
\subfloat[]{\label{fig:xiaolunzhuangpei2}\includegraphics[scale=0.5]{xiaolunzhuangpei2}}
\caption{复制小轮组组成零件}
\end{figure}

按照上述方法依次将小轮组的其它零件复制过来,结果如图\ref{fig:xiaolunzhuangpei2}所示。
\item 组装套筒与轮

在实际生产之中零件的装配是有先后顺序的。如果顺序错误是无法完的成部件的组装的。尽管在AutoCAD中不依照生产实际的顺序任然能够构建部件的三维装配模型,但是遵循生产实际的装配顺序能够有助于我们更深入地理解装配图的作用。在此,我们先将套筒和轮装配在一起。通过观察图\ref{fig:xiaolunzhuangpei2}可知,套筒的轴心线与轮孔的轴心线不是相互平行的,因此无法直接用移动命令来完成组装操作。一种可行的方法是:先将笔筒旋转至与轮轴心线平行的位置,然后再通过移动来完成操作。但是AutoCAD中还有更为方便的命令——三维对齐可以直接完成此项任务,其调用方法有:
\begin{itemize}
\item 键盘输入3dalign\index{3dalign,三维对齐}
\item 【修改】$\rightarrow $【三维操作】$\rightarrow $【三维对齐】
\item 【建模】\includegraphics[scale=0.45]{solidtoolbar}工具栏中的【三维对齐】\includegraphics[scale=0.45]{3dalign}
\end{itemize}

\begin{figure}[htbp]
\centering
\subfloat[]{\label{fig:3dalignselect}\includegraphics[scale=1]{3dalignselect}}\hspace{20pt}
\subfloat[]{\label{fig:3dalignselect1}\includegraphics[scale=0.3]{3dalignselect1}}\\
\subfloat[]{\label{fig:3dalignselect2}\includegraphics[scale=0.3]{3dalignselect2}}\hspace{20pt}
\subfloat[]{\label{fig:3dalignselect3}\includegraphics[scale=0.3]{3dalignselect3}}
\caption{三维对齐操作过程(一)}
\end{figure}

三维对齐命令调用后选择套筒作为要对齐的对象,选择后会以虚线的形式予以标识,效果如图\ref{fig:3dalignselect}所示。
\begin{lstlisting}
命令: 3DALIGN
选择对象: 找到 1 个
选择对象:
\end{lstlisting}

接下来是指定源平面和方向,为方便操作应向上轻推鼠标滚轮将套筒零件模型进行适当的放大,选择图\ref{fig:3dalignselect1}所示的圆心作为基点,选择图\ref{fig:3dalignselect2}所示的圆心作为第二个点,为方便选择与套筒轴心线垂直的点,用\includegraphics[scale=0.3]{keyf8}键开启正交模式,并以图\ref{fig:3dalignselect3}的方式选择第三点。
\begin{lstlisting}
 指定源平面和方向 ...
指定基点或 [复制(C)]:
指定第二个点或 [继续(C)] <C>:
指定第三个点或 [继续(C)] <C>:
\end{lstlisting}

\begin{figure}[htbp]
\centering
\subfloat[]{\label{fig:3dalignselect4}\includegraphics[scale=0.3]{3dalignselect4}}\hspace{20pt}
\subfloat[]{\label{fig:3dalignselect5}\includegraphics[scale=0.3]{3dalignselect5}}\hspace{20pt}
\subfloat[]{\label{fig:xiaolunzhuangpei3}\includegraphics[scale=0.3]{xiaolunzhuangpei3}}
\caption{三维对齐操作过程(二)}
\end{figure}

接下来指定目标平面和方向,按图\ref{fig:3dalignselect4}所示选择轮零件模型上的圆心作为第一个目标点,选择图\ref{fig:3dalignselect5}的圆心作为每个目标点,并以正交模式任意指定一个与轴线垂直的点作为第三个目标点,对齐后的效果如图\ref{fig:xiaolunzhuangpei3} 所示。
\begin{lstlisting}
 指定目标平面和方向 ...
指定第一个目标点:
指定第二个目标点或 [退出(X)] <X>:
指定第三个目标点或 [退出(X)] <X>:
指定第三个目标点或 [退出(X)] <X>:
\end{lstlisting}

\item 组装支架与连接杆

完成套筒与轮的组装后,需要将支架与连接杆组装在一起,任然使用三维对齐命令来完成。组装完成后效果如图\ref{fig:xiaolunzhuangpei4}所示。
\begin{lstlisting}
命令: 3DALIGN
选择对象: 找到 1 个
选择对象:
 指定源平面和方向 ...
指定基点或 [复制(C)]:
指定第二个点或 [继续(C)] <C>:
指定第三个点或 [继续(C)] <C>:
 指定目标平面和方向 ...
指定第一个目标点:
指定第二个目标点或 [退出(X)] <X>:
指定第三个目标点或 [退出(X)] <X>:
指定第三个目标点或 [退出(X)] <X>:
\end{lstlisting}

\begin{figure}[htbp]
\centering
\begin{floatrow}[3]
\ffigbox{\caption{组装支架与连接杆}\label{fig:xiaolunzhuangpei4}}{\includegraphics[scale=0.3]{xiaolunzhuangpei4}}
\ffigbox{\caption{支架辅助轴线}\label{fig:drawaidline1}}{\includegraphics[scale=0.3]{drawaidline1}}
\ffigbox{\caption{套筒辅助轴线}\label{fig:drawaidline2}}{\includegraphics[scale=0.3]{drawaidline2}}
\end{floatrow}
\end{figure}
\item 组装套筒轮组件与支架连杆组件

为方便将套筒和轮构成组件定位于支架安装孔的正中位置,需要先绘制辅助线。首先绘制支架的辅助线,结果如图\ref{fig:drawaidline1}所示。
\begin{lstlisting}
命令: line
指定第一个点:
指定下一点或 [放弃(U)]:
指定下一点或 [放弃(U)]:
\end{lstlisting}

接下来绘制套筒的辅助线,结果如图\ref{fig:drawaidline2}所示。
\begin{lstlisting}
命令: line
指定第一个点:
指定下一点或 [放弃(U)]:
指定下一点或 [放弃(U)]:
\end{lstlisting}

用移动命令来实现两个组件的组装。即同时选择套筒和轮作为移动对象,选择套筒辅助线的中点作为基点,选择支架辅助线的中点作为第二个点。完成后的效果如图\ref{fig:xiaolunzhuangpei5}所示。

\begin{lstlisting}
命令: move
选择对象: 指定对角点: 找到 3 个
选择对象:
指定基点或 [位移(D)] <位移>:
指定第二个点或 <使用第一个点作为位移>:  
\end{lstlisting}
\begin{figure}[htbp]
\begin{floatrow}[2]
\ffigbox{\caption{套筒轮组件与支架连杆组件组装结果}\label{fig:xiaolunzhuangpei5}}{\includegraphics[scale=0.3]{xiaolunzhuangpei5}}
\ffigbox{\caption{小轮组三维模型}\label{fig:xiaolunzhuangpei6}}{\includegraphics[scale=0.3]{xiaolunzhuangpei6}}
\end{floatrow}
\end{figure}
\item 组装轴

由于轴的轴线与图\ref{fig:xiaolunzhuangpei5}的组件轴心线不平行,因此用三维对齐命令来完成轴的组装。

\begin{lstlisting}
命令: 3DALIGN
选择对象: 找到 1 个
选择对象:
 指定源平面和方向 ...
指定基点或 [复制(C)]:
指定第二个点或 [继续(C)] <C>:
指定第三个点或 [继续(C)] <C>:
 指定目标平面和方向 ...
指定第一个目标点:
指定第二个目标点或 [退出(X)] <X>:
指定第三个目标点或 [退出(X)] <X>:
指定第三个目标点或 [退出(X)] <X>:
\end{lstlisting}

\item 切换视觉样式

为便于观察小轮组三维模型,将视觉样式切换为灰度,效果如图\ref{fig:xiaolunzhuangpei6}所示。

\begin{lstlisting}
命令: VSCURRENT
输入选项 [二维线框(2)/线框(W)/隐藏(H)/真实(R)/概念(C)/着色(S)/带边缘着色(E)/灰度(G)/勾画(SK)/X 射线(X)/其他(O)] <二维线框>: G
\end{lstlisting}
\item 保存模型结果

最后将建好的小轮组三维模型以“小轮组.dwg”予以保存。
\end{procedure}
\endinput
\section{制作小轮组装配图}

\endinput
\endinput
\section{定制标注样式样板}
在前面几章中,我们每次标注尺寸时都需要按照\ref{sec:lianjieganshitu}节的尺寸样式设置步骤进行标注样式设置,其过程比较繁琐,能否有一种方法可以实现一次定制重复使用呢?要实现这样的要求,可利用AutoCAD的模板功能。下面是制作带有国家标准标注样式的AutoCAD模板的基本步骤。
\begin{procedure}
\item 新建文档

在AutoCAD中新建文档的方式有很多,常用的方式有:
\begin{itemize}
\item 键盘输入new\index{new,新建}
\item \includegraphics[scale=0.3]{keyctrl}+\includegraphics[scale=0.3]{keyn}
\item 【文件】$\rightarrow $【新建】
\item 【标准】\includegraphics[scale=0.45]{biaozhuntools}工具栏中的\includegraphics[scale=0.45]{newtool}图标
\item 【快速访问】\includegraphics[scale=0.45]{quicktools}中的\includegraphics[scale=0.45]{newtool}图标
\end{itemize}

命令启动后弹出图\ref{fig:newfiletemplate}所示的选择样板对话框,选择acadiso.dwt文件并点击打开按钮。

\begin{figure}[htbp]
\centering
\includegraphics[scale=0.5]{newfiletemplate}
\caption{新建文件对话}\label{fig:newfiletemplate}
\end{figure}

\begin{lstlisting}
命令:new
\end{lstlisting}

\item 设置文字样式

按照\ref{sec:lianjieganshitu}节的方法设置文字样式。

\item 设置标注样式

按照\ref{sec:lianjieganshitu}节的方法设置标注样式。

\item 保存为样板文件

完成设置后,点击保存或另存为,弹出图\ref{fig:GBacadisotemplate}所示的图形另存为对话框,在文件下拉列表框中选择“AutoCAD 图形样板文件.dwt”,并以“GBacadiso.dwt”文件名进行保存,弹出图\ref{fig:GBtamplate}所示的样板选项对话框,在其中输入说明后点击确定即可生成样板文件。


\begin{figure}[htbp]
\centering
\begin{floatrow}[2]
\ffigbox{\caption{保存为样板}\label{fig:GBacadisotemplate}}{\includegraphics[scale=0.3]{GBacadisotemplate}}
\ffigbox{\caption{样板选项对话框}\label{fig:GBtamplate}}{\includegraphics[scale=0.4]{GBtamplate}}
\end{floatrow}
\end{figure}

\item 关闭样板文件

完成样板文件定制后需要关闭。在新建文件时选择此样板文件来创建新文件,新文件就已经具备标准的文字样式设置和标注样式设置。
\end{procedure}
\endinput
\section{构建小轮组三维模型}
\begin{procedure}
\item 初始化绘图环境

\begin{enumerate}
\item 新建带国家标准标注样式的文件。

以带标国家标准标注样式的样板文件“GBacadiso”为样板创建新的AutoCAD绘图文件。

\item 切换至西南等轴测图。

\begin{lstlisting}
命令: -VIEW
输入选项 [?/删除(D)/正交(O)/恢复(R)/保存(S)/设置(E)/窗口(W)]: swiso
\end{lstlisting}

\item 定义用户坐标系,使其$xy$平面与左视图投影面平行。

由于小轮组的组成零件中大多数的回转体特征面与左视图投影面平行,因此本例选择将用户坐标系定义为与左视图投影面平行。当然也可以不进行此操作,只是显示结果与本例会略有差异,但操作过程是基本一致的。

\begin{lstlisting}
命令: UCS
当前 UCS 名称: *世界*
指定 UCS 的原点或 [面(F)/命名(NA)/对象(OB)/上一个(P)/视图(V)/世界(W)/X/Y/Z/Z 轴(ZA)] <世界>: za
指定新原点或 [对象(O)] <0,0,0>:
在正 Z 轴范围上指定点 <0.0000,0.0000,1.0000>: -1,0,0
\end{lstlisting}
\end{enumerate}

\item 复制小轮组组件三维模型

用AutoCAD打开之前建立好的轮零件三维模型文件,调出轮零件的三维模型。为实现将轮零件的三维模型复制到新建的AutoCAD文件之中,需要先将轮零件的三维模型先复制来剪切板之中,然后再将其粘粘贴到新建到的AutoCAD文件之中。通常实现将AutoCAD对象复制到剪切板的方法有:
\begin{itemize}
\item 键盘输入copyclip\index{copyclip,复制到剪切板}
\item \includegraphics[scale=0.3]{keyctrl}+\includegraphics[scale=0.3]{keyc}
\item 【编辑】$\rightarrow $【复制】
\item 【标准】\includegraphics[scale=0.45]{biaozhuntools}工具栏中的\includegraphics[scale=0.45]{copyclip}图标
\end{itemize}

命令调用后选择轮零件的三维模型并结束选择即可将其复制到剪切板之中。
\begin{lstlisting}
命令: COPYCLIP
选择对象: 找到 1 个
选择对象:
\end{lstlisting}

接下来进入新建的AutoCAD文件之中,调用粘贴命令来实现将剪切板中的轮零件三维模型复制到文件之中。通常调用粘贴命令的方法有:
\begin{itemize}
\item 键盘输入pasteclip\index{pasteclip,粘贴自剪切板}
\item \includegraphics[scale=0.3]{keyctrl}+\includegraphics[scale=0.3]{keyv}
\item 【编辑】$\rightarrow $【粘贴】
\item 【标准】\includegraphics[scale=0.45]{biaozhuntools}工具栏中的\includegraphics[scale=0.45]{pasteclip}图标
\end{itemize}

命令调用后,在绘图区中任意选取一点作为插入点,即可完成轮零件三维的模型的复制,结果如图\ref{fig:xiaolunzhuangpei1} 所示。
\begin{lstlisting}
命令:  PASTECLIP
指定插入点:
\end{lstlisting}


\begin{figure}[htbp]
\centering
\subfloat[]{\label{fig:xiaolunzhuangpei1}\includegraphics[scale=0.4]{xiaolunzhuangpei1}}\hspace{20pt}
\subfloat[]{\label{fig:xiaolunzhuangpei2}\includegraphics[scale=0.5]{xiaolunzhuangpei2}}
\caption{复制小轮组组成零件}
\end{figure}

按照上述方法依次将小轮组的其它零件复制过来,结果如图\ref{fig:xiaolunzhuangpei2}所示。
\item 组装套筒与轮

在实际生产之中零件的装配是有先后顺序的。如果顺序错误是无法完的成部件的组装的。尽管在AutoCAD中不依照生产实际的顺序任然能够构建部件的三维装配模型,但是遵循生产实际的装配顺序能够有助于我们更深入地理解装配图的作用。在此,我们先将套筒和轮装配在一起。通过观察图\ref{fig:xiaolunzhuangpei2}可知,套筒的轴心线与轮孔的轴心线不是相互平行的,因此无法直接用移动命令来完成组装操作。一种可行的方法是:先将笔筒旋转至与轮轴心线平行的位置,然后再通过移动来完成操作。但是AutoCAD中还有更为方便的命令——三维对齐可以直接完成此项任务,其调用方法有:
\begin{itemize}
\item 键盘输入3dalign\index{3dalign,三维对齐}
\item 【修改】$\rightarrow $【三维操作】$\rightarrow $【三维对齐】
\item 【建模】\includegraphics[scale=0.45]{solidtoolbar}工具栏中的【三维对齐】\includegraphics[scale=0.45]{3dalign}
\end{itemize}

\begin{figure}[htbp]
\centering
\subfloat[]{\label{fig:3dalignselect}\includegraphics[scale=1]{3dalignselect}}\hspace{20pt}
\subfloat[]{\label{fig:3dalignselect1}\includegraphics[scale=0.3]{3dalignselect1}}\\
\subfloat[]{\label{fig:3dalignselect2}\includegraphics[scale=0.3]{3dalignselect2}}\hspace{20pt}
\subfloat[]{\label{fig:3dalignselect3}\includegraphics[scale=0.3]{3dalignselect3}}
\caption{三维对齐操作过程(一)}
\end{figure}

三维对齐命令调用后选择套筒作为要对齐的对象,选择后会以虚线的形式予以标识,效果如图\ref{fig:3dalignselect}所示。
\begin{lstlisting}
命令: 3DALIGN
选择对象: 找到 1 个
选择对象:
\end{lstlisting}

接下来是指定源平面和方向,为方便操作应向上轻推鼠标滚轮将套筒零件模型进行适当的放大,选择图\ref{fig:3dalignselect1}所示的圆心作为基点,选择图\ref{fig:3dalignselect2}所示的圆心作为第二个点,为方便选择与套筒轴心线垂直的点,用\includegraphics[scale=0.3]{keyf8}键开启正交模式,并以图\ref{fig:3dalignselect3}的方式选择第三点。
\begin{lstlisting}
 指定源平面和方向 ...
指定基点或 [复制(C)]:
指定第二个点或 [继续(C)] <C>:
指定第三个点或 [继续(C)] <C>:
\end{lstlisting}

\begin{figure}[htbp]
\centering
\subfloat[]{\label{fig:3dalignselect4}\includegraphics[scale=0.3]{3dalignselect4}}\hspace{20pt}
\subfloat[]{\label{fig:3dalignselect5}\includegraphics[scale=0.3]{3dalignselect5}}\hspace{20pt}
\subfloat[]{\label{fig:xiaolunzhuangpei3}\includegraphics[scale=0.3]{xiaolunzhuangpei3}}
\caption{三维对齐操作过程(二)}
\end{figure}

接下来指定目标平面和方向,按图\ref{fig:3dalignselect4}所示选择轮零件模型上的圆心作为第一个目标点,选择图\ref{fig:3dalignselect5}的圆心作为每个目标点,并以正交模式任意指定一个与轴线垂直的点作为第三个目标点,对齐后的效果如图\ref{fig:xiaolunzhuangpei3} 所示。
\begin{lstlisting}
 指定目标平面和方向 ...
指定第一个目标点:
指定第二个目标点或 [退出(X)] <X>:
指定第三个目标点或 [退出(X)] <X>:
指定第三个目标点或 [退出(X)] <X>:
\end{lstlisting}

\item 组装支架与连接杆

完成套筒与轮的组装后,需要将支架与连接杆组装在一起,任然使用三维对齐命令来完成。组装完成后效果如图\ref{fig:xiaolunzhuangpei4}所示。
\begin{lstlisting}
命令: 3DALIGN
选择对象: 找到 1 个
选择对象:
 指定源平面和方向 ...
指定基点或 [复制(C)]:
指定第二个点或 [继续(C)] <C>:
指定第三个点或 [继续(C)] <C>:
 指定目标平面和方向 ...
指定第一个目标点:
指定第二个目标点或 [退出(X)] <X>:
指定第三个目标点或 [退出(X)] <X>:
指定第三个目标点或 [退出(X)] <X>:
\end{lstlisting}

\begin{figure}[htbp]
\centering
\begin{floatrow}[3]
\ffigbox{\caption{组装支架与连接杆}\label{fig:xiaolunzhuangpei4}}{\includegraphics[scale=0.3]{xiaolunzhuangpei4}}
\ffigbox{\caption{支架辅助轴线}\label{fig:drawaidline1}}{\includegraphics[scale=0.3]{drawaidline1}}
\ffigbox{\caption{套筒辅助轴线}\label{fig:drawaidline2}}{\includegraphics[scale=0.3]{drawaidline2}}
\end{floatrow}
\end{figure}
\item 组装套筒轮组件与支架连杆组件

为方便将套筒和轮构成组件定位于支架安装孔的正中位置,需要先绘制辅助线。首先绘制支架的辅助线,结果如图\ref{fig:drawaidline1}所示。
\begin{lstlisting}
命令: line
指定第一个点:
指定下一点或 [放弃(U)]:
指定下一点或 [放弃(U)]:
\end{lstlisting}

接下来绘制套筒的辅助线,结果如图\ref{fig:drawaidline2}所示。
\begin{lstlisting}
命令: line
指定第一个点:
指定下一点或 [放弃(U)]:
指定下一点或 [放弃(U)]:
\end{lstlisting}

用移动命令来实现两个组件的组装。即同时选择套筒和轮作为移动对象,选择套筒辅助线的中点作为基点,选择支架辅助线的中点作为第二个点。完成后的效果如图\ref{fig:xiaolunzhuangpei5}所示。

\begin{lstlisting}
命令: move
选择对象: 指定对角点: 找到 3 个
选择对象:
指定基点或 [位移(D)] <位移>:
指定第二个点或 <使用第一个点作为位移>:  
\end{lstlisting}
\begin{figure}[htbp]
\begin{floatrow}[2]
\ffigbox{\caption{套筒轮组件与支架连杆组件组装结果}\label{fig:xiaolunzhuangpei5}}{\includegraphics[scale=0.3]{xiaolunzhuangpei5}}
\ffigbox{\caption{小轮组三维模型}\label{fig:xiaolunzhuangpei6}}{\includegraphics[scale=0.3]{xiaolunzhuangpei6}}
\end{floatrow}
\end{figure}
\item 组装轴

由于轴的轴线与图\ref{fig:xiaolunzhuangpei5}的组件轴心线不平行,因此用三维对齐命令来完成轴的组装。

\begin{lstlisting}
命令: 3DALIGN
选择对象: 找到 1 个
选择对象:
 指定源平面和方向 ...
指定基点或 [复制(C)]:
指定第二个点或 [继续(C)] <C>:
指定第三个点或 [继续(C)] <C>:
 指定目标平面和方向 ...
指定第一个目标点:
指定第二个目标点或 [退出(X)] <X>:
指定第三个目标点或 [退出(X)] <X>:
指定第三个目标点或 [退出(X)] <X>:
\end{lstlisting}

\item 切换视觉样式

为便于观察小轮组三维模型,将视觉样式切换为灰度,效果如图\ref{fig:xiaolunzhuangpei6}所示。

\begin{lstlisting}
命令: VSCURRENT
输入选项 [二维线框(2)/线框(W)/隐藏(H)/真实(R)/概念(C)/着色(S)/带边缘着色(E)/灰度(G)/勾画(SK)/X 射线(X)/其他(O)] <二维线框>: G
\end{lstlisting}
\item 保存模型结果

最后将建好的小轮组三维模型以“小轮组.dwg”予以保存。
\end{procedure}
\endinput
\section{制作小轮组装配图}

\endinput
\part{零件三维建模实例}
\chapter{台虎钳零件三维建模}

\chapter{钳口三维建模}

\endinput
\chapter{螺母}

\endinput
\chapter{圆螺丝钉}

\endinput
\section{丝杠}

\endinput
\chapter{滑块}

\endinput
\chapter{动掌}

\endinput
\chapter{底座}

\endinput
\chapter{台虎钳装配}

\endinput

\endinput
\chapter{钳口三维建模}

\endinput
%\part{航模发动机三维建模}
\chapter{缸盖}

\endinput
\chapter{发动机主轴}

\endinput
\chapter{驱动盒}
\begin{figure}[htbp]
\centering
\includegraphics[scale=0.45]{qudonghe.pdf}
\caption{驱动盒零件图}\label{fig:qudonghe}
\end{figure}
%AutoCAD作为国际上广泛使用的流行的绘图工具,它具有较强的二维绘图和三维绘图功能,欢迎来到AutoCAD的三维世界,
本章我们的目标是用AutoCAD制作图\ref{fig:qudonghe}所示的航模发动机的驱动盒零件的三维模型。本章将学习以下内容:
\begin{itemize}
	\item 拉伸操作
	\item 拉伸面操作
	\item 三维对象的环形阵列操作

\end{itemize}
\endinput
\chapter{燃料入管}

\endinput
\chapter{汽化器}

\endinput
\chapter{活塞}

\endinput
\chapter{活塞缸套筒}

\endinput
\chapter{上端盖}

\endinput
\chapter{发动机机匣}
\backmatter
\appendix
\noindent
%\begin{landscape}
%%\chapter{图纸集}
%\begin{figure}[htbp]
\centering
\includegraphics[scale=0.7]{xiaoluntaotong.pdf}
\end{figure}
\begin{figure}[htbp]
\centering
\includegraphics[scale=0.7]{xiaolunzhou.pdf}
\end{figure}
%\end{landscape}
\printindex
\begin{thebibliography}{99}
\bibitem{zhao} 赵灼辉,杨文瑜,房延,电力工程制图与CAD,北京:中国电图出版社,2007
\bibitem{li}李会文,程时甘,AutoCAD 2011应用教程,北京:机械工业出版社,2012.1
\bibitem{chen}程光远,AutoCAD绘图要义,北京:电子工业出版社,2012.3
\bibitem{liwei}李伟,余洪等,AutoCAD 2010从入门到精通,北京:清华大学出版社,2010.11
\bibitem{yao}姚辉学,鲁金忠,潘金彪等,AutoCAD 2008中文版基础教程,北京:化学工业出版社,2008.1
\bibitem{xuyuanpu}徐元甫,郭玲,陈杰,机械制图,北京:中国水力水电出版社,2005
\bibitem{wang}王菁,乔建军,AutoCAD2012电气设计绘图基础入门与落伍精通,北京:科学出版社,2011.6
\end{thebibliography}
\endinput
\end{document}
