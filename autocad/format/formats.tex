%%%%%%%%%% 定理类环境的定义 %%%%%%%%%% 
%% 必须在导入中文环境之后 
\newtheorem{example}{例}             % 整体编号 
\newtheorem{algorithm}{算法} 
\newtheorem{theorem}{定理}[section]  % 按 section 编号 
\newtheorem{definition}{定义} 
\newtheorem{axiom}{公理} 
\newtheorem{property}{性质} 
\newtheorem{proposition}{命题} 
\newtheorem{lemma}{引理} 
\newtheorem{corollary}{推论} 
\newtheorem{remark}{注解} 
\newtheorem{condition}{条件} 
\newtheorem{conclusion}{结论} 
\newtheorem{assumption}{假设} 
 
%%%%%%%%%% 一些重定义 %%%%%%%%%% 

%% 必须在导入中文环境之后 
\renewcommand{\contentsname}{目\quad 录}% 将 Contents 改为目录 
%\renewcommand{\abstractname}{摘\ \ 要} % 将 Abstract 改为摘要 
\renewcommand{\bibname}{参考文献}      % 将 References 改为参考文献 
\renewcommand{\indexname}{索\quad 引} %将Idex 改为索引
\renewcommand{\figurename}{图} 
\renewcommand{\tablename}{表} 
\renewcommand{\appendixname}{附\quad 录} 
\renewcommand{\proofname}{\hei 证明} 
\renewcommand{\algorithm}{\hei 算法} 
 
%%%%%%%%%% 重定义字号命令 %%%%%%%%%% 

\newcommand{\yihao}{\fontsize{26pt}{36pt}\selectfont}%一号, 1.4 倍行距 
\newcommand{\erhao}{\fontsize{22pt}{28pt}\selectfont}%二号, 1.25 倍行距 
\newcommand{\xiaoer}{\fontsize{18pt}{18pt}\selectfont}%小二, 单倍行距 
\newcommand{\sanhao}{\fontsize{16pt}{24pt}\selectfont}%三号, 1.5 倍行距
\newcommand{\xiaosan}{\fontsize{15pt}{22pt}\selectfont}%小三, 1.5 倍行距 
\newcommand{\sihao}{\fontsize{14pt}{21pt}\selectfont}%四号, 1.5 倍行距 
\newcommand{\bansi}{\fontsize{13pt}{19.5pt}\selectfont}%半四, 1.5 倍行距 
\newcommand{\xiaosi}{\fontsize{12pt}{18pt}\selectfont}% 小四, 1.5 倍行距 
\newcommand{\dawu}{\fontsize{11pt}{11pt}\selectfont} % 大五, 单倍行距 
\newcommand{\wuhao}{\fontsize{10.5pt}{10.5pt}\selectfont}%五号, 单倍行距

%%%%%%%%%%%%%重定义章节的格式%%%%%%%%%%%%%%%%%%%

\titlecontents{chapter}[1em]{\vspace{.5\baselineskip}\bfseries}
{第\thecontentslabel 章\quad}{}
{\hspace{.5em}\titlerule*[10pt]{$\cdot$}\contentspage}
\renewcommand{\chaptername}{第\,\thechapter \,章}
\titleformat{\chapter}[hang]{\centering\huge\bfseries}{\chaptername}{1em}{}
\titlespacing{\chapter}{0pt}{*0}{*4}
\pagestyle{fancy}
\fancyhf{}
\renewcommand{\chaptermark}[1]{\markboth{第\,\thechapter\,章\ #1}{}}
\renewcommand{\sectionmark}[1]{\markright{\thesection\ #1}{}}
%%%%%%%%%%%%定义页眉而脚%%%%%%%%%%%%%%%%%%%%%%%%%%%%%%%%%

\fancyhead[ER]{\leftmark}
\fancyhead[OL]{\rightmark}
\fancyhead[EL,OR]{$\cdot$\ \thepage\ $\cdot$}
\renewcommand{\headrulewidth}{0.4pt}

%%%%%%%%%%%%%设置列表格式%%%%%%%%%%%%%%%%

\setenumerate[1]{itemsep=0pt,partopsep=0pt,parsep=\parskip,topsep=5pt}

\setitemize[1]{itemsep=0pt,partopsep=0pt,parsep=\parskip,topsep=5pt}

\setdescription{itemsep=0pt,partopsep=0pt,parsep=\parskip,topsep=5pt}

%%%%%%%%%%%定义过程列表%%%%%%%%%%%%%%%%%

\newcommand{\gorectangle}[1]{
\tikz\node[text=white,font=\sffamily\bfseries,inner sep=0.5mm,draw,rounded corners,fill=black]{\small #1};}
\newcounter{procedurecounter}
\newenvironment{procedure}{
\begin{list}{\gorectangle{Step\,\arabic{procedurecounter}}}{
\setlength{\parsep}{\parskip}
\setlength{\itemsep}{0ex plus 0.1ex}
\setlength{\labelwidth}{4em}
\setlength{\labelsep}{0.2em}
\setlength{\leftmargin}{6.2em}
\usecounter{procedurecounter}
\setcounter{procedurecounter}{0}}}
{\end{list}}

\newcounter{yaodiancounter}
\tikzstyle{mybox} = [draw=black, fill=gray!20, very thick,
    rectangle, rounded corners, inner sep=10pt, inner ysep=20pt]
\tikzstyle{fancytitle} =[fill=gray, text=white]
\newcommand{\yaodian}[1]{
\addtocounter{yaodiancounter}{1}
\begin{tikzpicture}
\node [mybox] (box){%
    \begin{minipage}{0.75\textwidth}
       \lishu{#1}
    \end{minipage}
};
\node[fancytitle, right=10pt] at (box.north west) {\hei{要点\arabic{yaodiancounter}}};
\end{tikzpicture}
}
%源代码格式设置
\lstdefinelanguage{autocad}{
morekeywords={line, circle,xline,view,subtract,cylinder,vscurrent,union,chamferedge,save,saveas},
sensitive=false,
}
\lstloadlanguages{autocad}                  % 所要粘贴代码的编程语言 
\lstset{language=autocad,tabsize=4, keepspaces=true, 
    xleftmargin=2em,xrightmargin=2em, aboveskip=1em, 
    backgroundcolor=\color{lightgray},    % 定义背景颜色 
    frame=none,                      % 表示不要边框 
    keywordstyle=\color{blue}\bfseries,     breakindent=22pt,stepnumber=1,numberstyle=\tiny, 
    basicstyle=\footnotesize, 
    showspaces=false, 
    flexiblecolumns=true, 
    breaklines=true, breakautoindent=true,breakindent=4em, 
    escapeinside={/*@}{@*/} 
} 
\endinput