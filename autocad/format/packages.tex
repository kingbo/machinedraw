\usepackage[CJKnumber,CJKchecksingle]{xeCJK}
\usepackage[paperwidth=185mm,paperheight=260mm,text={148mm,210mm},
left=21mm,vmarginratio=1:1]{geometry}%页面尺寸控制
\usepackage[a4,cam,center,noinfo,graphics]{crop}
\usepackage[perpage,symbol]{footmisc}%脚注控制
\usepackage[sf]{titlesec}%控制标题
\usepackage{titletoc}%控制目录
\usepackage{fancyhdr}%控制页眉页脚
\usepackage{type1cm}%控制字体大小
%\usepackage{indentfirst}%首行缩进
\usepackage{makeidx}%建立索引
\usepackage{textcomp}%千分号等特殊符号
\usepackage{layouts}%打印当前页面格式
%\usepackage{bbding}%一些特殊符号
\usepackage{cite}%支持引用
\usepackage{color,xcolor}%支持彩色文本、底色、文本框等
\usepackage{listings}%粘贴源代码
\lstdefinelanguage{autocad}{
morekeywords={line, circle,xline},
sensitive=true,
}
\lstloadlanguages{autocad}                  % 所要粘贴代码的编程语言 
\lstset{language=autocad,tabsize=4, keepspaces=true, 
    xleftmargin=2em,xrightmargin=2em, aboveskip=1em, 
    backgroundcolor=\color{lightgray},    % 定义背景颜色 
    frame=none,                      % 表示不要边框 
    keywordstyle=\color{blue}\bfseries,     breakindent=22pt, 
    numbers=left,stepnumber=1,numberstyle=\tiny, 
    basicstyle=\footnotesize, 
    showspaces=false, 
    flexiblecolumns=true, 
    breaklines=true, breakautoindent=true,breakindent=4em, 
    escapeinside={/*@}{@*/} 
} 
%\usepackage{CJK,CJKnumb,CJKulem}%中文支持
\usepackage{times}%包括Times Roman +Helvetica +Courier
\usepackage{latexsym}
\usepackage{amsmath}%AMS LaTeX 宏包
\usepackage{amssymb}% 用来排版漂亮的数学公式 
\usepackage{amsbsy} 
\usepackage{amsthm} 
\usepackage{amsfonts} 
\usepackage{mathrsfs}% 英文花体字体 
\usepackage{bm}% 数学公式中的黑斜体 
\usepackage{relsize}% 调整公式字体大小:
%\mathsmaller, \mathlarger 
\usepackage{enumitem}
\usepackage{caption}% 浮动图形和表格标题样式 
\usepackage{multirow}
\usepackage{floatrow}
\usepackage{subfig}
\usepackage[toc,page,title,titletoc,header]{appendix}
\usepackage{fontspec,xunicode,xltxtra}%XeLaTeX 相关字体库
\newif\ifpdf
\ifx\pdfoutput\undefined
   \pdffalse
\else
   \pdfoutput=1
   \pdftrue
\fi
\ifpdf
   \usepackage[pdftex]{graphicx}
   %\usepackage[pdftex,unicode=true,colorlinks,linkcolor=red,anchorcolor=blue,citecolor=green]{hyperref}
\else
   \usepackage{graphicx}
   %\usepackage[unicode={true},colorlinks,linkcolor=red,anchorcolor=blue,citecolor=green]{hyperref}
\fi
%%%%%%%%%% 中文字体设置  %%%%%%%%%% 
\XeTeXlinebreaklocale "zh"
\XeTeXlinebreakskip=0pt plus 1pt minus 0.1pt
\newfontfamily\song{SimSun} %设置宋体字
\newfontfamily\hei{SimHei} %设置黑体字
\newfontfamily\kai{KaiTi} %设置楷体字
\newfontfamily\lishu{LiSu} %设置隶书字体
\newfontfamily\fangsong{FangSong} %设置仿宋字体
\newfontfamily\tnroman{Times New Roman}
\setmainfont{Times New Roman}%设置默认英文字体
\setCJKmainfont[BoldFont={SimHei},ItalicFont={KaiTi}]{SimSun}%设置默认中文字体
\setCJKmonofont{SimSun}%设置等宽字体
%%%%%%%%%% 一些距离设置 %%%%%%%%%%% 
\setlength{\floatsep}{10pt plus 3pt minus 2pt} % 图形之间或图形与正文之间的距离 
\setlength{\abovecaptionskip}{2pt plus 1pt minus 1pt}% 图形中的图与标题之间的距离 
\setlength{\belowcaptionskip}{3pt plus 1pt minus 2pt}% 表格中的表与标题之间的距 

\endinput