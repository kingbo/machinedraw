\usepackage{xeCJK}
\usepackage[paperwidth=185mm,paperheight=260mm,text={148mm,210mm},
left=21mm,vmarginratio=1:1]{geometry}%页面尺寸控制
%\usepackage[a4,cam,center,noinfo,graphics]{crop}
\usepackage[perpage,symbol]{footmisc}%脚注控制
\usepackage[sf,newparttoc,explicit]{titlesec}%控制标题
\usepackage{stdclsdv}
%\usepackage[Bjornstrup]{fncychap}
\usepackage{titletoc}%控制目录
\usepackage{fancyhdr}%控制页眉页脚
\usepackage{type1cm}%控制字体大小
\usepackage{indentfirst}%首行缩进
\usepackage{makeidx}%建立索引
\usepackage{textcomp}%千分号等特殊符号
\usepackage{layouts}%打印当前页面格式
%\usepackage{bbding}%一些特殊符号
\usepackage{cite}%支持引用
\usepackage{tikz}%绘图宏包
\usetikzlibrary{shapes,decorations}
\usepackage{color,xcolor}%支持彩色文本、底色、文本框等
\usepackage{listings}%粘贴源代码
%\usepackage{CJK,CJKnumb,CJKulem}%中文支持
\usepackage{times}%包括Times Roman +Helvetica +Courier
\usepackage{latexsym}
\usepackage{amsmath}%AMS LaTeX 宏包
\usepackage{amssymb}% 用来排版漂亮的数学公式 
\usepackage{amsbsy} 
\usepackage{amsthm} 
\usepackage{amsfonts} 
\usepackage{pdflscape}
\usepackage{mathrsfs}% 英文花体字体 
\usepackage{bm}% 数学公式中的黑斜体 
\usepackage{relsize}% 调整公式字体大小:
%\mathsmaller, \mathlarger 
\usepackage{enumitem}
\usepackage{multirow}
\usepackage{floatrow}
\usepackage{subfig}
%\usepackage{float}
\usepackage{caption}%[2004/07/16]% 浮动图形和表格标题样式 
\usepackage[toc,page,title,titletoc,header]{appendix}
\usepackage{fontspec,xunicode,xltxtra}%XeLaTeX 相关字体库
\defaultfontfeatures{Mapping=tex-text} %如果没有它,会有一些 tex 特殊字符无法正常使用,比如连字符。
\usepackage[unicode={true},colorlinks,linkcolor=black]{hyperref}
\usepackage{graphicx}
%\usepackage{booktabs}
\usepackage{tabu}
\usepackage{exsheets}%习题宏包
\usepackage{calc}
%\usepackage[os=win, mackeys=symbols]{menukeys}
%\usepackage{shorttoc}
\usepackage{ifthen}
\NewCounterPattern{testa}{ta}
%%%%%%%%%% 中文字体设置  %%%%%%%%%% 
\XeTeXlinebreaklocale "zh"
\XeTeXlinebreakskip=0pt plus 1pt minus 0.1pt
\newfontfamily\song{SimSun} %设置宋体字
\newfontfamily\hei{SimHei} %设置黑体字
\newfontfamily\kai{KaiTi} %设置楷体字
\newfontfamily\lishu{LiSu} %设置隶书字体
\newfontfamily\fangsong{FangSong} %设置仿宋字体
\newfontfamily\tnroman{Times New Roman}
\setmainfont{Times New Roman}%设置默认英文字体
\setCJKmainfont[BoldFont={SimHei},ItalicFont={KaiTi}]{SimSun}%设置默认中文字体
\setCJKmonofont[BoldFont={SimHei},ItalicFont={KaiTi}]{SimSun}%设置等宽字体
\setCJKfamilyfont{song}{SimSun}
%%%%%%%%%% 一些距离设置 %%%%%%%%%%% 
\setlength{\floatsep}{10pt plus 3pt minus 2pt} % 图形之间或图形与正文之间的距离 
\setlength{\abovecaptionskip}{2pt plus 1pt minus 1pt}% 图形中的图与标题之间的距离 
\setlength{\belowcaptionskip}{3pt plus 1pt minus 2pt}% 表格中的表与标题之间的距 
%%%%%%%去掉图表标题中的冒号
\DeclareCaptionLabelSeparator{twospace}{\ ~}
\captionsetup{labelsep=twospace} 
\floatsetup[table]{style=Plaintop}%设置表格标题位于表格上方

%%%%%%%%%%%%选择题格式设置%%%%%%%%%%%%
\SetupExSheets{
  question/type=exam ,
  headings=plain
}

% The following declares an empty question heading. If you don't
% want that leave the code alone and load one of the other heading
% styles provided by exsheets in the options above.
\DeclareInstance{exsheets-heading}{plain}{default}
  {
    runin = true ,
    number-post-code = \space ,
    attach = { main[l,vc]points[l,vc](\linewidth+\marginparsep,0pt) }
}

% The following declares an environment `choices' similar to list but
% instead of \item you use \choice
\NewTasks[
  counter-format = (tsk[A]) ,
  label-width = 2em
]{choices}[\choice]

\endinput