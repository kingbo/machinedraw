\section{小结}
本章通过小轮组轴零件的三维建模,进一步熟悉了圆柱体的应用,首先学习了如何利用圆柱体这类简单的回转体构建轴类零件的三维模型。整个三维模型构建过程是:
\begin{enumerate}
\item 分析轴零件的组成部分
\item 分段构建轴零件实体
\item 组合各个组成部分实体,构成轴零件整体
\end{enumerate} 

其次是学习了如何利用三维模型生成平面基本的平面视图,更深入地理解了三维模型与平面图形之间的对应关系。应用三维模型构建单一基本视图的基本过程是:
\begin{enumerate}
\item 从模型空间切换至图纸空间
\item 修改页面设置
\item 进入图纸视口模型空间切换视图方向
\item 设置视口图形显示比例
\item 提出模型轮廓
\item 修改图层设置
\end{enumerate}
最后介绍图纸幅面和比例的相关国家标准。
\endinput