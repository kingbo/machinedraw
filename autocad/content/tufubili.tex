\section{图幅与比例}
在\ref{sec:zhoushitu}节中,我们将图纸尺寸设置为ISO A4(210.00x297.00毫米),视口的显示比例设置成了5:1。之所以这样设置,其目的是确保使用的图纸和比例都符合国家制定的制图标准。因此本节将介绍图纸幅面和比例相关的国家标准。
\subsection{图纸幅面与格式}
\subsubsection{图纸幅面}
图纸幅面是指整张图纸的尺寸大小。工程中为了便于图纸的装订、保管及合理利用图纸,要求图纸幅面大小要符合表\ref{tab:tuzhifumian}的规定。
\begin{table}[htbp]
\caption{图纸幅面}\label{tab:tuzhifumian}
\begin{tabu}to \linewidth {X[cm]*5{|X[cm]}}
\tabucline -
幅面代号&A0&A1&A2&A3&A4\\
\tabucline -
$B\times L$&$841\times 1189$&$594\times 841$& $420\times 594$&$297\times 420$&$210\times 297$\\
\tabucline -
$e$&\multicolumn{2}{c|}{20}&\multicolumn{3}{c}{10}\\
\tabucline -
$c$&\multicolumn{3}{c|}{10}&\multicolumn{2}{c}{5}\\
\tabucline -
$a$&\multicolumn{5}{c}{25}\\
\tabucline -
\tabuphantomline
\end{tabu}
\end{table}
\subsubsection{图框}
\subsection{比例}
比例是图形与其实物相应要素的线性尺寸之比。线性尺寸是能够用直线表达的尺寸。例如圆弧的半径,直线的长度等。

通常图样的比例分为原值比例、放大比例和缩小比例三种。画图时应优先采用1:1的比例进行绘制,以便能够看出物体的真实大小。若无法采用1:1的比例时,则应优先选用表\ref{tab:biaozhunbili}中规定的比例系列,必要时也可采用表\ref{tab:biaozhunbili} 规定的比例系列。
\begin{table}[htbp]
\caption{标准比例系列}\label{tab:biaozhunbili}

\begin{tabu} to \linewidth {X[cm]|X[c m]|X[c m]|X[c m]}
\tabucline -
种\qquad 类&\multicolumn{3}{c}{ 比\qquad 例 } \\
\tabucline -
原始比例&\multicolumn{3}{c}{1:1}\\
\tabucline -
放大比例&
$\begin{tabu}{c}
2:1\\
2\times 10^n:1
\end{tabu}$
&
$\begin{tabu}{c}
5:1\\
5\times 10^n:1
\end{tabu}$
&$1\times 10^n$:1\\
\tabucline -
缩小比例&
$\begin{tabu}{c}
1:2\\
1:2\times 10^n
\end{tabu}$
&
$\begin{tabu}{c}
1:5\\
1:5\times 10^n
\end{tabu}$
&
$\begin{tabu}{c}
1:10\\
1:10\times 10^n
\end{tabu}$\\
\tabucline -
\tabuphantomline
\end{tabu}
\end{table}

\begin{table}[htbp]

\begin{tabu}to \linewidth {X[cm]|X[2cm]|X[cm]|X[cm]|X[2cm]}
\tabucline -
种\qquad 类&\multicolumn{4}{c}{比\qquad 例}\\
\tabucline -
放大比例&\multicolumn{2}{c|}{4:1\quad $4\times 10^n$:1}&\multicolumn{2}{c}{2.5:1\quad $2.5\times 10^n$:1}\\
\tabucline -
缩小比例&$\begin{tabu}{c}
1:3\\
1:3\times 10^n
\end{tabu}$
&
\multicolumn{2}{c|}{$\begin{tabu}{c}
1:4\\
1:4\times 10^n
\end{tabu}$}
&
$\begin{tabu}{c}
1:6\\
1:6\times 10^n
\end{tabu}$\\
\tabucline -
\tabuphantomline
\end{tabu}
\caption{比例系列}\label{tab:biaoxilei}
\end{table}
\endinput