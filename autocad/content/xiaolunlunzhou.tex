\chapter{轮轴}
\noindent
\begin{figure}[htbp]
\centering
\includegraphics[scale=0.45]{xiaolunzhou.pdf}
\caption{轮轴零件图}\label{fig:xiaolunzhou}
\end{figure}

本简的目标是用AutoCAD制作图\ref{fig:xiaolunzhou}所示轴零件的三维模型,并用布局空间生成该三维模型的主视图,以验证模型的正确性,使学习者进一步理解三维模型与工程零件图之间的关系。因此,本章将讲述以下内容:
\begin{itemize}
	\item 轴零件三维模型的构建
	\item 并集操作
	\item 视图的生成
	\item 图幅和比例的标准
\end{itemize}
\section{轴建模过程分析}\label{sec:zhoufengxi}
图\ref{fig:xiaolunzhou}所示的轴零件由一个主视图构成,根据\ref{sec:lijieshitu}节的知识可知一个视图通常是不能够唯一表达物体的。但仔细观察轴零件图的垂直方向的尺寸标注,可以发现垂直方向尺寸标注的一个共同点是均有表示直径的符号$\phi$,由此可知轴零件的各个组成部分都是圆柱体。 基于\ref{sec:taotongjianmo}节套筒零件的三维建模经验,可以采用以下两种方式进行三维建模。

\yaodian{合理清晰的尺寸标注有助于视图表达。}
\subsection{运用叠加方式构建}
从图\ref{fig:xiaolunzhou}中可以直观的看出整个轴零件由三段叠加而成,分别是直径$\phi 6$长3的圆柱体、直径$\phi 8$长35的圆柱体和直径$\phi 16$长2的圆柱体。结果如图\ref{fig:zhoufengxi1}所示。
\begin{figure}[htbp]
\centering
\includegraphics[scale=0.6]{zhoufengxi1.png}
\caption{分段建模}\label{fig:zhoufengxi1}
\end{figure}
\subsection{运用包含关系构建}
轴零件整个都是实体,因此可以将两个轴零件看作这样一种包含关系,即$\phi 16$的圆柱体包含了$\phi 8$和$\phi 6$两个圆柱体的部分实体,$\phi 8$的圆柱体包含了$\phi 6$的部分实体。因此,$\phi 8$圆柱体的长度要加上$\phi 16$圆柱体的长度,故长为37。与此类似,可得$\phi 6$圆柱体的长度则为40。结果如图\ref{fig:zhoufengxi2}所示。
\begin{figure}[htbp]
\centering
\includegraphics[scale=0.6]{zhoufengxi2.png}
\caption{按包含关系建模}\label{fig:zhoufengxi2}
\end{figure}

对于图\ref{fig:xiaolunzhou}所示的轴零件而言,用分段建模更为直观和直接,但实际中两建模方式并没有优劣之分,应当根据所需建模零件的实际情况灵活地综合运用。
\endinput
\section{轴三维模型构建}

\begin{lstlisting}
命令: -VIEW
输入选项 [?/删除(D)/正交(O)/恢复(R)/保存(S)/设置(E)/窗口(W)]: left
\end{lstlisting}

\begin{lstlisting}
命令: -VIEW
输入选项 [?/删除(D)/正交(O)/恢复(R)/保存(S)/设置(E)/窗口(W)]: swiso
\end{lstlisting}

\begin{lstlisting}
命令: CYLINDER
指定底面的中心点或 [三点(3P)/两点(2P)/切点、切点、半径(T)/椭圆(E)]:
指定底面半径或 [直径(D)]: 8
指定高度或 [两点(2P)/轴端点(A)]: 2
\end{lstlisting}

\begin{lstlisting}
命令: CYLINDER
指定底面的中心点或 [三点(3P)/两点(2P)/切点、切点、半径(T)/椭圆(E)]:
指定底面半径或 [直径(D)] <8.0000>: 4
指定高度或 [两点(2P)/轴端点(A)] <2.0000>: 35
\end{lstlisting}

\begin{lstlisting}
命令: CYLINDER
指定底面的中心点或 [三点(3P)/两点(2P)/切点、切点、半径(T)/椭圆(E)]:
指定底面半径或 [直径(D)] <4.0000>: 3
指定高度或 [两点(2P)/轴端点(A)] <35.0000>: 3
\end{lstlisting}

\begin{lstlisting}
命令:  UNION
选择对象: 指定对角点: 找到 3 个
选择对象:
\end{lstlisting}

\begin{lstlisting}
命令: CHAMFEREDGE
距离 1 = 1.0000,距离 2 = 1.0000
选择一条边或 [环(L)/距离(D)]: d
指定距离 1 或 [表达式(E)] <1.0000>: 0.5
指定距离 2 或 [表达式(E)] <1.0000>: 0.5
选择一条边或 [环(L)/距离(D)]:
选择同一个面上的其他边或 [环(L)/距离(D)]:
按 Enter 键接受倒角或 [距离(D)]:
\end{lstlisting}

\begin{lstlisting}
命令: CHAMFEREDGE
距离 1 = 0.5000,距离 2 = 0.5000
选择一条边或 [环(L)/距离(D)]:
选择同一个面上的其他边或 [环(L)/距离(D)]:
按 Enter 键接受倒角或 [距离(D)]:
\end{lstlisting}

\begin{lstlisting}
命令: VSCURRENT
输入选项 [二维线框(2)/线框(W)/隐藏(H)/真实(R)/概念(C)/着色(S)/带边缘着色(E)/灰度(G)/勾画(SK)/X 射线(X)/其他(O)] <二维线框>: g
\end{lstlisting}

\endinput
\section{轴主视图生成}

\endinput
\section{小结}
本章通过小轮组轴零件的三维建模,进一步熟悉了圆柱体的应用,首先学习了如何利用圆柱体这类简单的回转体构建轴类零件的三维模型。整个三维模型构建过程是:
\begin{enumerate}
\item 分析轴零件的组成部分
\item 分段构建轴零件实体
\item 组合各个组成部分实体,构成轴零件整体
\end{enumerate} 

其次是学习了如何利用三维模型生成平面基本的平面视图,更深入地理解了三维模型与平面图形之间的对应关系。应用三维模型构建单一基本视图的基本过程是:
\begin{enumerate}
\item 从模型空间切换至图纸空间
\item 修改页面设置
\item 进入图纸视口模型空间切换视图方向
\item 设置视口图形显示比例
\item 提出模型轮廓
\item 修改图层设置
\end{enumerate}
最后介绍图纸幅面和比例的相关国家标准。
\endinput
\endinput