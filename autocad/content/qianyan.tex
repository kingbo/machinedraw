\chapter*{前言}
\section*{本书结构}
\section*{目标读者}
\section*{命令提示说明}
书中用到的相关命令及提示都会以代码的形式贴出。对于第一次出现的命令则会根据命令的提示进行逐步讲解,其形式如下:
\begin{lstlisting}
命令:CYLINDER
\end{lstlisting}

命令提示指定圆柱体的底面中心点或者绘图选项。
\begin{lstlisting}
指定底面的中心点或 [三点(3P)/两点(2P)/切点、切点、半径(T)/椭圆(E)]:
\end{lstlisting}

接下来,根据命令提示输入底面的半径
\begin{lstlisting}
指定底面半径或 [直径(D)]: 7
\end{lstlisting}

最后,根据命令提示指定圆柱体的高度,并按回车或空格键结束命令。
\begin{lstlisting}
指定高度或 [两点(2P)/轴端点(A)]: 28
\end{lstlisting}

对于已经使用过的命令,则直接贴整个命令提示,其形式如下:

\begin{lstlisting}
命令: CYLINDER
指定底面的中心点或 [三点(3P)/两点(2P)/切点、切点、半径(T)/椭圆(E)]:
指定底面半径或 [直径(D)]: 8
指定高度或 [两点(2P)/轴端点(A)]: 2
\end{lstlisting}
\section*{关于勘误}
由于编者的时间和水平比较有限,书中难免会出现一些纰漏和错误。如果读者在阅读过程中发现任何错误,请及时与本人联系,提出修改意见和建议。本人会在本书后续的版本中加以改正。本人专门为本书设立的电子邮箱是:kingbo2001@gmail.com。本人欢迎并希望和大家一起学习和讨论AutoCAD的三维建模功能,促进大家的共同进步。
\section*{致谢}
\endinput