\documentclass[14pt,CJKutf8,mathserif]{beamer}
\usepackage{CJKutf8,CJKnumb}
\usepackage{amsmath,amssymb,amsthm,amscd}
%\usepackage{asymptote}
\usepackage{wasysym}
\usepackage{gensymb}
\usepackage{enumitem}
\usepackage{tikz}
\usepackage{circuitikz}
\usepackage{epstopdf}
\usepackage[utf8]{inputenc}
\usepackage{manfnt}
\usetikzlibrary{positioning,calc}
\usetikzlibrary{shapes,snakes}
\tikzset{
>=stealth,
nonterminal/.style={rectangle,minimum size=6mm,draw=black,font=\ttfamily},
circleterminal/.style={circle,minimum size=1mm,draw=black},
circlenode/.style={circle,inner sep =0pt,minimum size=3pt,draw=black,fill=black},
noshapenode/.style={circle,inner sep =0pt,minimum size=0pt}
}
\mode<presentation>
{
  \usetheme{PaloAlto}
}
%\includeonlylecture{yinlun}
%\includeonly{content/chapter2-1}
%\includeonly{content/chapter2-2}
%\includeonly{content/chapter2-3}
%\includeonlylecture{wulimoxing}
\begin{document}

\begin{CJK}{UTF8}{song}
\title{计算机控制原理与应用}
\author{金波 \inst{1}}
\institute[西藏职业技术学院]{
\inst{1}
教务处\\ 西藏职业技术学院 }
\date{\today}
\begin{frame}
\titlepage
\end{frame}
\begin{frame}[allowframebreaks]{主要内容}
  \tableofcontents
\end{frame}
\chapter{绘制杯立体图}\label{chap:bei}

盛夏,太阳无情的炙烤着大地,树上的知了有气无力的鸣叫着,热浪袭卷着整个城市,人们大多都呆家中或有空调的地方以躲避酷热。秦奋前几日收到了重点高中的取通知书,兴奋过后生活又归于平淡。没有学习压力的他每天除了看电视、上网之外便无事可做。没过几日,好学地他便寻思着找到点来什么事来做或者学点什么。正琢磨着,秦奋的爸爸下班回家了,他灵光一闪:爸爸是高级工程师,他的AutoCAD用得很好,我为什么不跟他学学用AutoCAD绘图,一来可以消磨无聊的生活,二来也可以学项本事,何乐而不为呢!
{\bfseries 知识目标}
\begin{itemize}
\item 掌握line命令的使用
\item 掌握offset命令和trim命令的使用
\item 掌握circle命令
\item 掌握端点和圆心捕捉方法
\item 掌握AutoCAD视图切换知识
\item 掌握AutoCAD旋转和拉伸建模命令的使用方法
\item 掌握AutoCAD差集命令的使用方法
\end{itemize}

{\bfseries 技能目标}
\begin{itemize}
\item 能够根据给定的回转体零件图,运用投影知识选择正确的投影图
\item 能够根据块零件特征,运用AutoCAD命令,绘制特征图
\item 能够运用AutoCAD旋转和拉伸命令,完成回转体零件和平面体块零件的三绘建模
\end{itemize}

{\bfseries 本章提要}

本章将对调压阀中的杯零件进行三维建模,使读者能够掌握line命令、offset命令和trim命令绘制简单回转体零件特征视图的方法,并通过实体建模中revolve命令和extude命令来生成三维模型。通过完成该项目读者能够基本掌握使用AutoCAD进行三维建模的基本步骤。图\ref{fig:tiaoyafabei}所示即为本项目的块零件图。
\noindent
\begin{figure}[htbp]
\centering
\includegraphics[scale=0.6]{tiaoyafabei.pdf}
\caption{杯零件图}\label{fig:tiaoyafabei}
\end{figure}
\endinput
\section{电气元件}\label{sec:dianqiyuanjian}

{\bfseries 知识目标}
\begin{itemize}
\item 掌握图块的概念
\item 掌握图块的定义、插入、输出方法
\item 掌握图块属性定义与设置知识
\end{itemize}

{\bfseries 技能目标}
\begin{itemize}
\item 能够完成电子元件元件的定制
\end{itemize}

本任务以绘制\ref{fig:zhumodianlu}所示电路图中的电子元件为目标,主要是帮助读者掌握AutoCAD的图块的概念,以便于在绘图过程中将大量重复的图形定义为图块,以提高图形的绘制速度。

\subsection{电容元件}

\subsection{接地元件}

\subsection{电阻元件}

\subsection{电灯元件}

\subsection{二极管元件}

\subsection{三极管元件}

\section{触摸延时开关电路图}

\endinput
\section{触摸延时开关电路图}
上一节,我们了解了绘制电子电器元件的方法及电子元器件图块的制作方法。这一节中,我们将应用我们所做的电子元器件图块和CAD绘图技巧完成本章开始所展示的触摸延时开关电路图。
\subsection{绘制参考线}
\subsection{绘制电器元件}
\subsection{绘制线路}

\endinput
%%%%%%%%%%%%%%%教案头%%%%%%%%%%%%%%%%%%%%%%%%%%%%%%%
\mode<article>{

\begin{longtable}{|m{20mm}|m{20mm}|m{20mm}|m{20mm}|m{20mm}|m{28mm}|}
\caption*{\huge 教案头}\\
\hline
\endfirsthead
\multicolumn{6}{l}{(续表)}\\
\hline
\endhead
\hline
\multicolumn{6}{l}{\itshape 接下一页表格.......}\\ [2ex]
\endfoot
\hline
\endlastfoot
\centering{授课单元}&\multicolumn{3}{m{60mm}|}{\centering 2.4.4方框图简化}&\centering{授课日期}&2014年03月31日 \\
\hline
\centering 授课地点 & \multicolumn{3}{m{60mm}|}{B6-204}&\centering 授课学时 & 2 \\
\hline
& \multicolumn{2}{m{40mm}|}{能力目标} & \multicolumn{2}{m{40mm}|}{知识目标}&素质目标 \\
\cline{2-6}
\centering 教学目标&\multicolumn{2}{m{40mm}|}{\begin{enumerate}
\item 能够进行方框图的简化
\item 能够用MatLab进行系统方框图简化
\end{enumerate} }&\multicolumn{2}{m{40mm}|}{\begin{enumerate}
\item 掌握方框图简化的方法
\item 掌握用MatLab进行方框图简化的方法
\end{enumerate}} & {\qquad}\\
\hline
\centering 能力训练任务或案例 &\multicolumn{5}{m{108mm}|}{ }\\
\hline
\centering 教学重点 & \multicolumn{5}{m{108mm}|}{\begin{enumerate}
\item 方框图的简化
\end{enumerate}}\\
\hline
\centering 教学难点与解决办法 &\multicolumn{5}{m{108mm}|}{\begin{enumerate}
\item 难点:方框图的简化
\item 解决方法:用实例进行分析讲解
\end{enumerate}}\\
\hline
\centering 德育内容 &\multicolumn{5}{m{108mm}|}{无}\\
\hline
 &教材 & \multicolumn{4}{m{88mm}|}{计算机控制原理与应用}\\
\cline{2-6}& 教学资源 &\multicolumn{4}{m{88mm}|}{PPT}\\
\cline{2-6}\centering 使用的教学材料& 主要教学仪器设备和工具等 &\multicolumn{4}{m{88mm}|}{投影机、MATLAB}\\
\cline{2-6}& 主要耗材 &\multicolumn{4}{m{88mm}|}{无}\\
\hline
\centering 教学模式 &\multicolumn{2}{m{40mm}|}{知识讲授}&\centering 教学手段 &\multicolumn{2}{m{48mm}|}{多媒体教学}\\
\hline
\centering 学生成果与过程考核方式 &\multicolumn{5}{m{108mm}|}{无}
\end{longtable}
\clearpage

%%%%%%%%%%%%%%%教学实施过程%%%%%%%%%%%%%%%%%%%%%%%%%%%%
\begin{landscape}

\begin{longtable}{|m{10mm}|m{50mm}|m{50mm}|m{50mm}|m{15mm}|}
\caption*{\huge 教学组织与实施}\\
\hline
\endfirsthead
\multicolumn{5}{l}{\small 接上页}\\
\hline
\multicolumn{1}{|c|}{步骤}&\multicolumn{1}{c|}{教学内容}&\multicolumn{1}{c|}{教师活动}&\multicolumn{1}{c|}{学生活动}&\multicolumn{1}{c|}{时间}\\
\hline
\endhead

\multicolumn{5}{r}{\small 接下页}\\
\endfoot
\hline
\endlastfoot
\multicolumn{1}{|c|}{步骤}&\multicolumn{1}{c|}{教学内容}&\multicolumn{1}{c|}{教师活动}&\multicolumn{1}{c|}{学生活动}&\multicolumn{1}{c|}{时间}\\\hline
引入&\begin{enumerate}
\item 系统反馈较多时不便于进行系统分析
\end{enumerate} &\begin{enumerate}
\item 讲解方框简化的原则
\end{enumerate} &\begin{enumerate}
\item 学生记录
\end{enumerate} &10 \\\hline
讲解&\begin{enumerate}
\item 方框图简化方法
\end{enumerate}
 &\begin{enumerate}
\item 通过实例讲解方框图的简化
\end{enumerate} &\begin{enumerate}
\item 学生倾听并记录
\end{enumerate} &15 \\\hline
练习&\begin{enumerate}
\item 方框图的化简
\end{enumerate}
&\begin{enumerate}
\item 指需要简化的方框图
\item 指导学生进行方框图简化
\item 讲解正确结果
\end{enumerate} &\begin{enumerate}
\item 学生尝试方框图简化
\item 学生展示简化结果
\item 学生进行记录
\end{enumerate} &20 \\\hline
讲解&\begin{enumerate}
\item 用MatLab进行系统方框图简化
\end{enumerate}
 &\begin{enumerate}
\item 讲解相关的MatLab指令
\item 讲解用MatLab进行系统方框图化的方法
\end{enumerate} &\begin{enumerate}
\item 学生记录笔记
\end{enumerate} &20 \\\hline
练习&
\begin{enumerate}
\item 用MatLab进行方框图简化
\end{enumerate}
 &\begin{enumerate}
\item 指导学生用MatLab进行方框图简化
\item 讲解要点
\end{enumerate} &\begin{enumerate}
\item 学生尝试用MatLab进行方框图简化
\item 学生记录笔记
\end{enumerate} &20 \\\hline
\centering 本次课总结(评价)&总结本课程内容 &进行知识总结 &学生倾听 &5 \\\hline
\centering 学生学习笔记或工单等检查情况&\multicolumn{4}{m{165mm}|}{\quad}\\\hline
\centering 课后作业&\multicolumn{4}{m{165mm}|}{2-19,2-20,2-21}\\\hline
\centering 教学体会&\multicolumn{4}{m{165mm}|}{\quad}\\
\end{longtable}

\end{landscape}
\clearpage
%%%%%%%%%%%%%%%%%%%%板书设计%%%%%%%%%%%%%%%%%%%%%%%%%%
\lecture{传递函数与方框图}{chuandihanshu}
\begin{center}
{\huge 板书设计}
\end{center}
}
\mode<presentation>{ \section{方框图简化}
 \subsection{方框图简化}}
 \begin{frame}
 \begin{block}{简化标准}
 \begin{itemize}
 \item 简化成标准形式
 \end{itemize}
 \end{block}
 \begin{block}{方框图简化原则}
 \begin{itemize}
 \item 简化前与简化后的数学关系不变
 \end{itemize}
 \end{block}
 \begin{block}{简化规则}
 \begin{itemize}
 \item 简化规则见P59页至P60页
 \end{itemize}
 \end{block}
 \end{frame}
 \begin{frame}
 \begin{block}{简化的特点}
 \begin{itemize}
 \item 传递函数具有唯一性
 \item 方框图不唯一
 \end{itemize}
 \end{block}
 \end{frame}
 \begin{frame}{用MatLab进行方框图化简}
 \begin{block}{串联}
 sys1=tf(num1,den1)
 
 sys2=tf(num2,den2)
 
 sys=series(sys1,sys2)
 \end{block}
 \begin{block}{并联}
 sys1=tf(num1,den1)
 
 sys2=tf(num2,den2)
 
 sys=parallel(sys1,sys2)
 \end{block}
 \end{frame}
 \begin{frame}
 \begin{block}{反馈连接}
sysg=[numg,deng]

sysh=[numh,denh]

sys=feedback(sysg,sysh)
 \end{block}
 \end{frame}

%%%%%%%%%%%%%%教案头%%%%%%%%%%%%%%%%%%%%%%%%%%%%%%%
\mode<article>{

\begin{longtable}{|m{20mm}|m{20mm}|m{20mm}|m{20mm}|m{20mm}|m{28mm}|}
\caption*{\huge 教案头}\\
\hline
\endfirsthead
\multicolumn{6}{l}{(续表)}\\
\hline
\endhead
\hline
\multicolumn{6}{l}{\itshape 接下一页表格.......}\\ [2ex]
\endfoot
\hline
\endlastfoot
\centering{授课单元}&\multicolumn{3}{m{60mm}|}{\centering 2.5传递函数和信号流图}&\centering{授课日期}&2014年03月13日 \\
\hline
\centering 授课地点 & \multicolumn{3}{m{60mm}|}{B6-204}&\centering 授课学时 & 2 \\
\hline
& \multicolumn{2}{m{40mm}|}{能力目标} & \multicolumn{2}{m{40mm}|}{知识目标}&素质目标 \\
\cline{2-6}
\centering 教学目标&\multicolumn{2}{m{40mm}|}{\begin{enumerate}
\item 能够将方框图转化为信号流图
\item 能够应用梅逊增益公式求解系统的传递函数
\end{enumerate} }&\multicolumn{2}{m{40mm}|}{\begin{enumerate}
\item 掌握信号流图的画法
\item 掌握梅逊增益公式
\end{enumerate}} & {\qquad}\\
\hline
\centering 能力训练任务或案例 &\multicolumn{5}{m{108mm}|}{ }\\
\hline
\centering 教学重点 & \multicolumn{5}{m{108mm}|}{\begin{enumerate}
\item 梅逊增益公式
\end{enumerate}}\\
\hline
\centering 教学难点与解决办法 &\multicolumn{5}{m{108mm}|}{\begin{enumerate}
\item 难点:梅逊增益公式
\item 解决方法:用实例进行分析讲解
\end{enumerate}}\\
\hline
\centering 德育内容 &\multicolumn{5}{m{108mm}|}{无}\\
\hline
 &教材 & \multicolumn{4}{m{88mm}|}{计算机控制原理与应用}\\
\cline{2-6}& 教学资源 &\multicolumn{4}{m{88mm}|}{PPT}\\
\cline{2-6}\centering 使用的教学材料& 主要教学仪器设备和工具等 &\multicolumn{4}{m{88mm}|}{投影机、MATLAB}\\
\cline{2-6}& 主要耗材 &\multicolumn{4}{m{88mm}|}{无}\\
\hline
\centering 教学模式 &\multicolumn{2}{m{40mm}|}{知识讲授}&\centering 教学手段 &\multicolumn{2}{m{48mm}|}{多媒体教学}\\
\hline
\centering 学生成果与过程考核方式 &\multicolumn{5}{m{108mm}|}{无}
\end{longtable}
\clearpage

%%%%%%%%%%%%%%%教学实施过程%%%%%%%%%%%%%%%%%%%%%%%%%%%%
\begin{landscape}

\begin{longtable}{|m{10mm}|m{50mm}|m{50mm}|m{50mm}|m{15mm}|}
\caption*{\huge 教学组织与实施}\\
\hline
\endfirsthead
\multicolumn{5}{l}{\small 接上页}\\
\hline
\multicolumn{1}{|c|}{步骤}&\multicolumn{1}{c|}{教学内容}&\multicolumn{1}{c|}{教师活动}&\multicolumn{1}{c|}{学生活动}&\multicolumn{1}{c|}{时间}\\
\hline
\endhead

\multicolumn{5}{r}{\small 接下页}\\
\endfoot
\hline
\endlastfoot
\multicolumn{1}{|c|}{步骤}&\multicolumn{1}{c|}{教学内容}&\multicolumn{1}{c|}{教师活动}&\multicolumn{1}{c|}{学生活动}&\multicolumn{1}{c|}{时间}\\\hline
引入&\begin{enumerate}
\item 复习方框图的简化
\end{enumerate} &\begin{enumerate}
\item 通过提问了解学生对方框图的简化掌握情况
\end{enumerate} &\begin{enumerate}
\item 学生回简问题
\end{enumerate} &10 \\\hline
讲解&\begin{enumerate}
\item 信息号流图的定义和绘制方法
\end{enumerate}
 &\begin{enumerate}
\item 通过实例讲解信号流图的定义和绘制
\end{enumerate} &\begin{enumerate}
\item 学生倾听并记录
\end{enumerate} &15 \\\hline
练习&\begin{enumerate}
\item 将方框图转换为信号流图
\end{enumerate}
&\begin{enumerate}
\item 指定需要转换的方框图
\item 指导学生进行信号流图的绘制
\item 讲解正确结果
\end{enumerate} &\begin{enumerate}
\item 学生尝试将方框图转为信号流图
\item 学生展示绘制结果
\item 学生进行记录
\end{enumerate} &20 \\\hline
讲解&\begin{enumerate}
\item 用梅逊增益求解系统的传递函数
\end{enumerate}
 &\begin{enumerate}
\item 讲解梅逊增益公式
\item 讲解用梅逊增益公式求解系统传递函数的实例
\end{enumerate} &\begin{enumerate}
\item 学生记录笔记
\end{enumerate} &20 \\\hline
练习&
\begin{enumerate}
\item 用梅逊增益公式求解系统传递函数
\end{enumerate}
 &\begin{enumerate}
\item 指导学生进行系统传递函数求解
\item 讲解要点
\end{enumerate} &\begin{enumerate}
\item 学生尝试用梅逊公式求解系统传递函数
\item 学生记录笔记
\end{enumerate} &20 \\\hline
\centering 本次课总结(评价)&总结本课程内容 &进行知识总结 &学生倾听 &5 \\\hline
\centering 学生学习笔记或工单等检查情况&\multicolumn{4}{m{165mm}|}{\quad}\\\hline
\centering 课后作业&\multicolumn{4}{m{165mm}|}{2-19,2-20,2-21}\\\hline
\centering 教学体会&\multicolumn{4}{m{165mm}|}{\quad}\\
\end{longtable}

\end{landscape}
\clearpage
%%%%%%%%%%%%%%%%%%%%板书设计%%%%%%%%%%%%%%%%%%%%%%%%%%
\lecture{传递函数与信号流图}{chuandihanshu}
\begin{center}
{\huge 板书设计}
\end{center}
}
\mode<presentation>{ \section{传递函数与信号流图}
 \subsection{信号流图的定义}}
 \begin{frame}{信号流图}
 \uncover<+->{\begin{block}{信息流图的组成}
 \begin{itemize}
 \item<+-> 节点:表示变量或信号
 \item<+-> 支路:连接两个接点的有向线段
 \end{itemize}
 \end{block}}
 \uncover<+->{\begin{block}{支路的特点}
 \begin{itemize}
 \item<+-> 联接有因果关系的节点,相当于乘法器
 \item<+-> 有方向性,信号只能沿箭头单向传递
 \item<+-> 有加权性
 \end{itemize}
 \end{block}  }
 \end{frame}
 \begin{frame}{信号流图} 
 \uncover<+->{\begin{block}{信号流图的特点}
 \begin{itemize}
 \item<+-> 是系统代数方程的图形表示
 \item<+-> 不是唯一的
 \item<+-> 便于用梅逊增益公式求解系统传递函数
 \end{itemize}
 \end{block}}
 \end{frame}
 \begin{frame}{信号流图}
 \uncover<+->{\begin{block}{信号流图术语}
 \begin{itemize}
 \item<+-> 输入节点:表示独立变量,仅有输入支路
 \item<+-> 输出节点:表示因变量,仅有输出支路
 \item<+-> 混合节点:既有输入支路也有输出支路
 \item<+-> 通路:表示一个沿支路箭头方向,穿过各个相连支路的线路。
 \item<+-> 前向通路:输入节点和输出节点之间的通路,遇到或穿过任何节点不多于一次。
 \end{itemize}
 \end{block}}
 \end{frame}
 \begin{frame}{信号流图}
 \begin{block}{信号流图术语}
 \begin{itemize}
 \item<+-> 通路增益:是中通路中各支路传输的乘积。
 \item<+-> 回路:起点和终点在同一节点。
 \end{itemize}
 \end{block}
 \begin{block}{并联}
 sys1=tf(num1,den1)
 
 sys2=tf(num2,den2)
 
 sys=parallel(sys1,sys2)
 \end{block}
 \end{frame}
 \begin{frame}
 \begin{block}{反馈连接}
sysg=[numg,deng]

sysh=[numh,denh]

sys=feedback(sysg,sysh)
 \end{block}
 \end{frame}

%%%%%%%%%%%%%%教案头%%%%%%%%%%%%%%%%%%%%%%%%%%%%%%%
\mode<article>{

\begin{longtable}{|m{20mm}|m{20mm}|m{20mm}|m{20mm}|m{20mm}|m{28mm}|}
\caption*{\huge 教案头}\\
\hline
\endfirsthead
\multicolumn{6}{l}{(续表)}\\
\hline
\endhead
\hline
\multicolumn{6}{l}{\itshape 接下一页表格.......}\\ [2ex]
\endfoot
\hline
\endlastfoot
\centering{授课单元}&\multicolumn{3}{m{60mm}|}{\centering 2.6控制系统的时域分析2.6.1脉新华路响应和阶路响应2.6.2时域性能指标2.6.3一阶系统的动态响应}&\centering{授课日期}&2014年03月13日 \\
\hline
\centering 授课地点 & \multicolumn{3}{m{60mm}|}{B6-204}&\centering 授课学时 & 2 \\
\hline
& \multicolumn{2}{m{40mm}|}{能力目标} & \multicolumn{2}{m{40mm}|}{知识目标}&素质目标 \\
\cline{2-6}
\centering 教学目标&\multicolumn{2}{m{40mm}|}{\begin{enumerate}
\item 能够分析一阶系统的动态响应
\end{enumerate} }&\multicolumn{2}{m{40mm}|}{\begin{enumerate}
\item 了解脉冲响应函数和阶跃响应函数
\item 了解时域性能指标
\end{enumerate}} & {\qquad}\\
\hline
\centering 能力训练任务或案例 &\multicolumn{5}{m{108mm}|}{ }\\
\hline
\centering 教学重点 & \multicolumn{5}{m{108mm}|}{\begin{enumerate}
\item 一阶系统的动态响应
\end{enumerate}}\\
\hline
\centering 教学难点与解决办法 &\multicolumn{5}{m{108mm}|}{\begin{enumerate}
\item 难点:一阶系统的动态响应分析
\item 解决方法:用实例进行分析讲解
\end{enumerate}}\\
\hline
\centering 德育内容 &\multicolumn{5}{m{108mm}|}{无}\\
\hline
 &教材 & \multicolumn{4}{m{88mm}|}{计算机控制原理与应用}\\
\cline{2-6}& 教学资源 &\multicolumn{4}{m{88mm}|}{PPT}\\
\cline{2-6}\centering 使用的教学材料& 主要教学仪器设备和工具等 &\multicolumn{4}{m{88mm}|}{投影机、MATLAB}\\
\cline{2-6}& 主要耗材 &\multicolumn{4}{m{88mm}|}{无}\\
\hline
\centering 教学模式 &\multicolumn{2}{m{40mm}|}{知识讲授}&\centering 教学手段 &\multicolumn{2}{m{48mm}|}{多媒体教学}\\
\hline
\centering 学生成果与过程考核方式 &\multicolumn{5}{m{108mm}|}{无}
\end{longtable}
\clearpage

%%%%%%%%%%%%%%%教学实施过程%%%%%%%%%%%%%%%%%%%%%%%%%%%%
\begin{landscape}

\begin{longtable}{|m{10mm}|m{50mm}|m{50mm}|m{50mm}|m{15mm}|}
\caption*{\huge 教学组织与实施}\\
\hline
\endfirsthead
\multicolumn{5}{l}{\small 接上页}\\
\hline
\multicolumn{1}{|c|}{步骤}&\multicolumn{1}{c|}{教学内容}&\multicolumn{1}{c|}{教师活动}&\multicolumn{1}{c|}{学生活动}&\multicolumn{1}{c|}{时间}\\
\hline
\endhead

\multicolumn{5}{r}{\small 接下页}\\
\endfoot
\hline
\endlastfoot
\multicolumn{1}{|c|}{步骤}&\multicolumn{1}{c|}{教学内容}&\multicolumn{1}{c|}{教师活动}&\multicolumn{1}{c|}{学生活动}&\multicolumn{1}{c|}{时间}\\\hline
讲解&\begin{enumerate}
\item 脉冲响应和阶跃响应
\end{enumerate} &\begin{enumerate}
\item 讲解分析单位脉冲响应和阶跃响应函数
\end{enumerate} &\begin{enumerate}
\item 学生倾听并记录
\end{enumerate} &20 \\\hline
讲解&\begin{enumerate}
\item 时域性能指标
\end{enumerate}
 &\begin{enumerate}
\item 通过图示讲解时域性能指标
\end{enumerate} &\begin{enumerate}
\item 学生倾听并记录
\end{enumerate} &25 \\\hline
讲解&\begin{enumerate}
\item 一阶系统的数学模型
\end{enumerate}
&\begin{enumerate}
\item 讲解一阶系统的数学模型
\end{enumerate} &\begin{enumerate}
\item 学生倾听并记录
\end{enumerate} &10 \\\hline
讲解&\begin{enumerate}
\item 一阶系统的单位阶跃响应
\end{enumerate}
 &\begin{enumerate}
\item 讲解阶系统的单位阶跃响应分析
\end{enumerate} &\begin{enumerate}
\item 学生记录笔记
\end{enumerate} &20 \\\hline
讲解&
\begin{enumerate}
\item 一阶系统的单位脉冲响应
\end{enumerate}
 &\begin{enumerate}
\item 讲解一阶系统的单位脉冲响应分析
\end{enumerate} &\begin{enumerate}
\item 学生记录笔记
\end{enumerate} &10 \\\hline
\centering 本次课总结(评价)&总结本课程内容 &进行知识总结 &学生倾听 &5 \\\hline
\centering 学生学习笔记或工单等检查情况&\multicolumn{4}{m{165mm}|}{\quad}\\\hline
\centering 课后作业&\multicolumn{4}{m{165mm}|}{2-19,2-20,2-21}\\\hline
\centering 教学体会&\multicolumn{4}{m{165mm}|}{\quad}\\
\end{longtable}

\end{landscape}
\clearpage
%%%%%%%%%%%%%%%%%%%%板书设计%%%%%%%%%%%%%%%%%%%%%%%%%%
\lecture{传递函数与信号流图}{chuandihanshu}
\begin{center}
{\huge 板书设计}
\end{center}
}
\mode<presentation>{ \section{控制系统的时域分析}
 \subsection{一阶系统的动态响应}}
 \begin{frame}{脉冲响应}
 \begin{block}{}
 \begin{itemize}
 \item<+-> 线性定常系统的传递函数为:$G(s)$
 \[r(t)=L^{-1}[R(s)]\]
 \item<+-> 则系统输出为:
 \[C(s)=G(s)R(s)\]
 \end{itemize}
 \end{block}
 \end{frame}
 
 \begin{frame}
 \begin{block}{}
 \begin{itemize}
 \item<+-> 输入为单位脉冲函数$\delta(t)$,输出记为$g(t)$,则:
 \[L[\delta]=1\]
 \[C(s)=G(s)\cdot 1\]
 \item<+-> 得:
 \[c(t)=g(t)\]
\end{itemize}  
\end{block}
\end{frame}

\begin{frame}
\begin{block}{}
\begin{itemize}
\item<+-> 系统传递函数等于单位脉冲函数$g(t)$的拉氏变换,即:
\[G(s)=L[g(t)]\]
\item<+-> 单位脉冲函数$g(t)$是传递函数$G(s)$拉氏逆变换,即:
\[g(t)=L^{-1}[G(s)]\]
\end{itemize}
\end{block}
\end{frame}
\begin{frame}
\begin{block}{}
推广:
\begin{eqnarray*}
c(t)&=&L^{-1}[C(s)]\\
&=&L^{-1}[G(s)R(s)]\\
&=&\int^{\infty}_{0}g(t-\tau)r(\tau)d\tau \\
&=&\int^{\infty}_{0}r(t-\tau)g(\tau)d\tau
\end{eqnarray*}
\end{block}
\end{frame}
\begin{frame}{阶跃响应}
\begin{block}{}
\begin{eqnarray*}
L[u(t)]=\frac{1}{s}\\
C(s)=G(s)R(s)=G(s)\frac{1}{s}=H(s)\\
h(t)=L^{-1}[C(s)]=L^{-1}[H(s)]=L^{-1}[\frac{G(s)}{s}]\\
h(t)=L^{-1}[\frac{G(s)}{s}]=\int^t_0g(t)dt
\end{eqnarray*}
\end{block}
\end{frame}
\begin{frame}
\begin{block}{}
\begin{eqnarray*}
G(s)=sH(s)\\
g(t)=L^{-1}[G(s)]=L^{-1}[sH(s)]=\frac{dh(t)}{dt}
\end{eqnarray*}
\end{block}
\end{frame}
\begin{frame}{时域性能指标}
\begin{block}{具有误差振荡的单位阶跃响应}
\includegraphics[scale=0.8]{danweijiyao}
\end{block}
\end{frame}
\begin{frame}
\begin{block}{}
\begin{itemize}
\item<+-> 延迟时间$t_d$:达到终值一半所需要的时间
\item<+-> 上升时间$t_r$:从终值的10\%到90\%所需要的时间
\item<+-> 峰值时间$t_p$:终值达到超调量的第一个峰值的时间
\item<+-> 最大超调量$M_p$:
\[M_p\%=\frac{c(t_p)-c(\infty)}{c(\infty)}\%\]
\end{itemize}
\end{block}
\end{frame}
\begin{frame}
\begin{block}{具有单调变化的单位阶跃响应}
\includegraphics[scale=1.2]{onejieyaoimpulse}
\end{block}
\end{frame}
%%%%%%%%%%%%%%教案头%%%%%%%%%%%%%%%%%%%%%%%%%%%%%%%
\mode<article>{

\begin{longtable}{|m{20mm}|m{20mm}|m{20mm}|m{20mm}|m{20mm}|m{28mm}|}
\caption*{\huge 教案头}\\
\hline
\endfirsthead
\multicolumn{6}{l}{(续表)}\\
\hline
\endhead
\hline
\multicolumn{6}{l}{\itshape 接下一页表格.......}\\ [2ex]
\endfoot
\hline
\endlastfoot
\centering{授课单元}&\multicolumn{3}{m{60mm}|}{\centering 2.6.4二阶系统的动态响应}&\centering{授课日期}&2014年03月13日 \\
\hline
\centering 授课地点 & \multicolumn{3}{m{60mm}|}{B6-204}&\centering 授课学时 & 2 \\
\hline
& \multicolumn{2}{m{40mm}|}{能力目标} & \multicolumn{2}{m{40mm}|}{知识目标}&素质目标 \\
\cline{2-6}
\centering 教学目标&\multicolumn{2}{m{40mm}|}{\begin{enumerate}
\item 能够分析二阶系统的动态响应
\item 能够计算二阶系统的性能指标
\end{enumerate} }&\multicolumn{2}{m{40mm}|}{\begin{enumerate}
\item 了解二阶系统的数学模型
\item 掌握二阶系统的响应分析
\end{enumerate}} & {\qquad}\\
\hline
\centering 能力训练任务或案例 &\multicolumn{5}{m{108mm}|}{ }\\
\hline
\centering 教学重点 & \multicolumn{5}{m{108mm}|}{\begin{enumerate}
\item 二阶系统的动态响应
\end{enumerate}}\\
\hline
\centering 教学难点与解决办法 &\multicolumn{5}{m{108mm}|}{\begin{enumerate}
\item 难点:二阶系统的动态响应分析
\item 解决方法:用实例进行分析讲解
\end{enumerate}}\\
\hline
\centering 德育内容 &\multicolumn{5}{m{108mm}|}{无}\\
\hline
 &教材 & \multicolumn{4}{m{88mm}|}{计算机控制原理与应用}\\
\cline{2-6}& 教学资源 &\multicolumn{4}{m{88mm}|}{PPT}\\
\cline{2-6}\centering 使用的教学材料& 主要教学仪器设备和工具等 &\multicolumn{4}{m{88mm}|}{投影机、MATLAB}\\
\cline{2-6}& 主要耗材 &\multicolumn{4}{m{88mm}|}{无}\\
\hline
\centering 教学模式 &\multicolumn{2}{m{40mm}|}{知识讲授}&\centering 教学手段 &\multicolumn{2}{m{48mm}|}{多媒体教学}\\
\hline
\centering 学生成果与过程考核方式 &\multicolumn{5}{m{108mm}|}{无}
\end{longtable}
\clearpage

%%%%%%%%%%%%%%%教学实施过程%%%%%%%%%%%%%%%%%%%%%%%%%%%%
\begin{landscape}

\begin{longtable}{|m{10mm}|m{50mm}|m{50mm}|m{50mm}|m{15mm}|}
\caption*{\huge 教学组织与实施}\\
\hline
\endfirsthead
\multicolumn{5}{l}{\small 接上页}\\
\hline
\multicolumn{1}{|c|}{步骤}&\multicolumn{1}{c|}{教学内容}&\multicolumn{1}{c|}{教师活动}&\multicolumn{1}{c|}{学生活动}&\multicolumn{1}{c|}{时间}\\
\hline
\endhead

\multicolumn{5}{r}{\small 接下页}\\
\endfoot
\hline
\endlastfoot
\multicolumn{1}{|c|}{步骤}&\multicolumn{1}{c|}{教学内容}&\multicolumn{1}{c|}{教师活动}&\multicolumn{1}{c|}{学生活动}&\multicolumn{1}{c|}{时间}\\\hline
讲解&\begin{enumerate}
\item 二阶系统的数学模型
\end{enumerate} &\begin{enumerate}
\item 讲解分析二阶系统的数学模型
\end{enumerate} &\begin{enumerate}
\item 学生倾听并记录
\end{enumerate} &15 \\\hline
讲解&\begin{enumerate}
\item 二阶系统的单位阶跃响应
\end{enumerate}
 &\begin{enumerate}
\item 通过数学分析和图示讲解二阶系统的单位阶跃响应
\end{enumerate} &\begin{enumerate}
\item 学生倾听并记录
\end{enumerate} &30 \\\hline
讲解&\begin{enumerate}
\item 典型二阶系统的动态性能指标
\end{enumerate}
&\begin{enumerate}
\item 讲解二阶系统的动态性能指标
\end{enumerate} &\begin{enumerate}
\item 学生倾听并记录
\end{enumerate} &20 \\\hline
讲解&\begin{enumerate}
\item 二阶系统的单位脉冲响应
\end{enumerate}
 &\begin{enumerate}
\item 讲解阶系统的单位阶跃响应分析
\end{enumerate} &\begin{enumerate}
\item 学生记录笔记
\end{enumerate} &20 \\\hline
总结&
\begin{enumerate}
\item 二阶系统的动态响应
\end{enumerate}
 &\begin{enumerate}
\item 总结二阶系统的动态响应
\end{enumerate} &\begin{enumerate}
\item 学生记录笔记
\end{enumerate} &5 \\\hline
\centering 本次课总结(评价)&总结本课程内容 &进行知识总结 &学生倾听 &5 \\\hline
\centering 学生学习笔记或工单等检查情况&\multicolumn{4}{m{165mm}|}{\quad}\\\hline
\centering 课后作业&\multicolumn{4}{m{165mm}|}{2-19,2-20,2-21}\\\hline
\centering 教学体会&\multicolumn{4}{m{165mm}|}{\quad}\\
\end{longtable}

\end{landscape}
\clearpage
%%%%%%%%%%%%%%%%%%%%板书设计%%%%%%%%%%%%%%%%%%%%%%%%%%
\begin{center}
{\huge 板书设计}
\end{center}
}
\mode<presentation>{ \section{控制系统的时域分析}
 \subsection{二阶系统的动态响应}}
 \begin{frame}{二阶系统的数学模型}
 \begin{block}{}
 \begin{itemize}
 \item<+-> RLC网络的二阶微分方程:
 \[\frac{d^2c(t)}{dt^2}+\frac{R}{L}\frac{dc(t)}{dt}+\frac{1}{LC}c(t)=\frac{1}{LC}r(t)\]
 \item<+-> 典型形式为:
 \[\frac{d^2c(t)}{dt^2}+2\zeta\omega_n\frac{dc(t)}{dt}+\omega_n^2=\omega_n^2r(t)\]
 \end{itemize}
 \end{block}
 \end{frame}
 
 \begin{frame}
 \begin{block}{}
 \begin{itemize}
 \item<+-> 其中:
 \[\omega_n=\frac{1}{\sqrt{LC}}\]
 \[2\zeta\omega_n=\frac{R}{L}\]
 \[\zeta=\frac{R}{2}\sqrt{\frac{C}{L}}\]
\end{itemize}  
\end{block}
\end{frame}

\begin{frame}
\begin{block}{}
\begin{itemize}
\item<+-> 进行拉氏变换得:
\[s^2C(s)+2\zeta\omega_nsC(s)+\omega_n^2C(s)=\omega_n^2R(s)\]
\item<+-> 系统开环传递函数为:
\[G_o(s)=\frac{C(s)}{E(s)}=\frac{\omega^2_n}{s(s+2\zeta\omega_n)}\]
\end{itemize}
\end{block}
\end{frame}
\begin{frame}
\begin{block}{}
\begin{itemize}
\item<+-> 系统闭环传递函数为:
\[\Phi(s)=\frac{C(s)}{R(s)}=\frac{\omega_n^2}{s^2+2\zeta\omega_ns+\omega_n^2}\]
\item<+-> 系统特征方程为:
\[s^2+2\zeta\omega_ns+\omega_n^2=0\]
\end{itemize}
\end{block}
\end{frame}
\begin{frame}
\uncover<+->{\begin{block}{}
特征根为:
\[s_{1,2}=-\zeta\omega_n\pm\omega_n\sqrt{\zeta^2-1}\]
\end{block}}
\uncover<+->{\begin{block}{}
系统的特征取决于$\zeta$和$\omega_n$两个参数
\end{block}}
\end{frame}

\begin{frame}{二阶系统的单位阶跃响应}
\begin{block}{}
\begin{itemize}
\item<+-> 单位阶跃函数的拉氏变换为:
\[R(s)=\frac{1}{s}\]
\item<+-> 系统输入的拉氏变换为:
\[C(s)=\Phi(s)R(s)=\frac{\omega_n^2}{s^2+2\zeta\omega_ns+\omega_n^2}\cdot\frac{1}{s}\]
\end{itemize}
\end{block}
\end{frame}
\begin{frame}
\begin{block}{当$\zeta=0$时}
\begin{itemize}
\item<+-> 则:
\[C(s)=\frac{\omega_n^2}{s^2+\omega_n^2}\cdot\frac{1}{s}\]
\item<+-> 展开得:
\[C(s)=\frac{\omega_n^2}{s(s^2+\omega_n^2)}=\frac{1}{s}-\frac{s}{s^2+\omega_n^2}\]
\end{itemize}
\end{block}
\end{frame}
\begin{frame}
\begin{block}{}
\begin{itemize}
\item<+-> 拉氏逆变换得:
\[c(t)=1-\cos\omega_nt\]
\item<+-> 系统响应曲线:

\includegraphics[scale=0.25]{wuzuni.png}
\end{itemize}
\end{block}
\end{frame}
\begin{frame}
\begin{block}{当$0<\zeta<1$时}
\begin{itemize}
\item<+-> 系统特征方程根为:
\[s_1=-\zeta\omega_n+j\omega_n\sqrt{1-\zeta^2}=-\zeta\omega_n+j\omega_d\]
\[s_2=-\zeta\omega_n-j\omega_n\sqrt{1-\zeta^2}=-\zeta\omega_n-j\omega_d\]
\end{itemize}
其中:$\omega_d=\omega_n\sqrt{1-\zeta^2}$
\end{block}
\end{frame}
\begin{frame}
\begin{block}{}
\begin{itemize}
\item<+-> 特征方程可写为:
\begin{eqnarray*}
s^2+2\zeta\omega_ns+\omega_n^2\\
=(s+\zeta\omega_n-j\omega_d)(s+\zeta\omega_n+j\omega_d)
\end{eqnarray*}
\end{itemize}
\end{block}
\end{frame}
\begin{frame}
\begin{block}{}
系统输出$C(s)$为:
\begin{eqnarray*}
C(s)=\Phi(s)R(s)=\frac{\omega_n^2}{s^2+2\zeta\omega_ns+\omega_n^2}\frac{1}{s}\\
=\frac{1}{s}-\frac{s+\zeta\omega_n}{(s+\zeta\omega_n)^2+\omega_d^2}-\frac{\zeta\omega_n}{(s+\zeta\omega_n)^2+\omega_d^2}
\end{eqnarray*}
\end{block}
\end{frame}
\begin{frame}
\begin{block}{}
\begin{itemize}
\item<+-> 拉氏逆变换为:
\begin{eqnarray*}
c(t)=L^{-1}[C(s)]\\
=1-e^{-\zeta\omega_nt}\frac{1}{\sqrt{1-\zeta^2}}\sin(\omega_dt+\varphi),t\geq 0
\end{eqnarray*}
\end{itemize}
\end{block}
\end{frame}
\begin{frame}
\begin{block}{当$\zeta=1$时}
\begin{itemize}
\item<+-> 系统的特征方程为:
\[s^2+2\zeta\omega_ns+\omega_n^2=s^2+2\omega_n+\omega_n^2=(s+\omega_n)^2\]
\item<+-> 系统输出为:
\[C(s)=\Phi(s)R(s)=\frac{\omega_n^2}{(s+\omega_n)^2}\cdot\frac{1}{s}\]
\end{itemize}
\end{block}
\end{frame}
\begin{frame}
\begin{block}{}
\begin{itemize}
\item<+-> 展开得:
\begin{eqnarray*}
C(s)=\frac{\omega_n^2}{s(s+\omega_n)^2}\\
=\frac{1}{s}-\frac{1}{s+\omega_n}-\frac{\omega_n}{(s+\omega_n)^2}
\end{eqnarray*} 
\item<+-> 拉氏逆变换得:
\[c(t)=1-e^{-\omega_nt}-e^{\omega_nt}\omega_nt,t\geq 0\]
\end{itemize}
\end{block}
\end{frame}
\begin{frame}{典型二阶系统的动态性能指标}
\begin{block}{}
\begin{itemize}
\item<+-> 延迟时间$t_d$:
\[t_d=\frac{0.7\zeta+1}{\omega_n}\]
\item<+-> 上升时间$t_r$:
\[t_r=\frac{1}{\omega_n}\arctan(1-\frac{\sqrt{1-\zeta^2}}{\zeta}=\frac{\pi-\varphi}{\omega_n}\]
\end{itemize}
\end{block}
\end{frame}
\begin{frame}
\begin{block}{}
\begin{itemize}
\item<+-> 峰值时间$t_p$:
\[t_p=\frac{\pi}{\omega_d}=\frac{\pi}{\omega_n\sqrt{1-\zeta^2}}\]
\item<+-> 最大超调量$M_p$:
\[M_p\%=e^{-\frac{\zeta\pi}{\sqrt{1-\zeta^2}}}\%\]
\end{itemize}
\end{block}
\end{frame}
\begin{frame}
\begin{block}{}
\begin{itemize}
\item<+-> 5\%调整时间:
\[t_s\approx\frac{1}{\zeta\omega_n}\ln(0.05\sqrt{1-\zeta^2})\]
\item<+-> 2\%调整时间:
\[t_s\approx\frac{1}{\zeta\omega_n}\ln(0.02\sqrt{1-\zeta^2})\]
\end{itemize}
\end{block}
\end{frame}
\endinput
%%%%%%%%%%%%%%教案头%%%%%%%%%%%%%%%%%%%%%%%%%%%%%%%
\mode<article>{

\begin{longtable}{|m{20mm}|m{20mm}|m{20mm}|m{20mm}|m{20mm}|m{28mm}|}
\caption*{\huge 教案头}\\
\hline
\endfirsthead
\multicolumn{6}{l}{(续表)}\\
\hline
\endhead
\hline
\multicolumn{6}{l}{\itshape 接下一页表格.......}\\ [2ex]
\endfoot
\hline
\endlastfoot
\centering{授课单元}&\multicolumn{3}{m{60mm}|}{\centering 2.6.5高阶系统分析2.6.6稳态误差分析}&\centering{授课日期}&2014年04月18日 \\
\hline
\centering 授课地点 & \multicolumn{3}{m{60mm}|}{B6-204}&\centering 授课学时 & 2 \\
\hline
& \multicolumn{2}{m{40mm}|}{能力目标} & \multicolumn{2}{m{40mm}|}{知识目标}&素质目标 \\
\cline{2-6}
\centering 教学目标&\multicolumn{2}{m{40mm}|}{\begin{enumerate}
\item 能够分析系统的稳态误差
\end{enumerate} }&\multicolumn{2}{m{40mm}|}{\begin{enumerate}
\item 了解高附系统的分析方法
\item 掌握系统的稳态误差分析
\end{enumerate}} & {\qquad}\\
\hline
\centering 能力训练任务或案例 &\multicolumn{5}{m{108mm}|}{ }\\
\hline
\centering 教学重点 & \multicolumn{5}{m{108mm}|}{\begin{enumerate}
\item 系统的稳态误差分析
\end{enumerate}}\\
\hline
\centering 教学难点与解决办法 &\multicolumn{5}{m{108mm}|}{\begin{enumerate}
\item 难点:系统的稳态误差分析
\item 解决方法:数学推理加实例讲解
\end{enumerate}}\\
\hline
\centering 德育内容 &\multicolumn{5}{m{108mm}|}{无}\\
\hline
 &教材 & \multicolumn{4}{m{88mm}|}{计算机控制原理与应用}\\
\cline{2-6}& 教学资源 &\multicolumn{4}{m{88mm}|}{PPT}\\
\cline{2-6}\centering 使用的教学材料& 主要教学仪器设备和工具等 &\multicolumn{4}{m{88mm}|}{投影机、MATLAB}\\
\cline{2-6}& 主要耗材 &\multicolumn{4}{m{88mm}|}{无}\\
\hline
\centering 教学模式 &\multicolumn{2}{m{40mm}|}{知识讲授}&\centering 教学手段 &\multicolumn{2}{m{48mm}|}{多媒体教学}\\
\hline
\centering 学生成果与过程考核方式 &\multicolumn{5}{m{108mm}|}{无}
\end{longtable}
\clearpage

%%%%%%%%%%%%%%%教学实施过程%%%%%%%%%%%%%%%%%%%%%%%%%%%%
\begin{landscape}

\begin{longtable}{|m{10mm}|m{50mm}|m{50mm}|m{50mm}|m{15mm}|}
\caption*{\huge 教学组织与实施}\\
\hline
\endfirsthead
\multicolumn{5}{l}{\small 接上页}\\
\hline
\multicolumn{1}{|c|}{步骤}&\multicolumn{1}{c|}{教学内容}&\multicolumn{1}{c|}{教师活动}&\multicolumn{1}{c|}{学生活动}&\multicolumn{1}{c|}{时间}\\
\hline
\endhead

\multicolumn{5}{r}{\small 接下页}\\
\endfoot
\hline
\endlastfoot
\multicolumn{1}{|c|}{步骤}&\multicolumn{1}{c|}{教学内容}&\multicolumn{1}{c|}{教师活动}&\multicolumn{1}{c|}{学生活动}&\multicolumn{1}{c|}{时间}\\\hline
讲解&\begin{enumerate}
\item 高阶系统的单位阶跃响应
\end{enumerate} &\begin{enumerate}
\item 讲解高阶系统的单位阶跃响应
\end{enumerate} &\begin{enumerate}
\item 学生倾听并记录
\end{enumerate} &30 \\\hline
讲解&\begin{enumerate}
\item 高阶系统分析
\end{enumerate}
 &\begin{enumerate}
\item 讲解高阶系统分析方法
\end{enumerate} &\begin{enumerate}
\item 学生倾听并记录
\end{enumerate} &15 \\\hline
讲解&\begin{enumerate}
\item 稳态误差分析
\end{enumerate}
&\begin{enumerate}
\item 讲解系统稳态分析
\end{enumerate} &\begin{enumerate}
\item 学生倾听并记录
\end{enumerate} &20 \\\hline
讲解&\begin{enumerate}
\item 稳态误差分析实例
\end{enumerate}
 &\begin{enumerate}
\item 讲解稳态误差分析实例
\end{enumerate} &\begin{enumerate}
\item 学生记录笔记
\end{enumerate} &20 \\\hline

\centering 本次课总结(评价)&总结本课程内容 &进行知识总结 &学生倾听 &5 \\\hline
\centering 学生学习笔记或工单等检查情况&\multicolumn{4}{m{165mm}|}{\quad}\\\hline
\centering 课后作业&\multicolumn{4}{m{165mm}|}{2-19,2-20,2-21}\\\hline
\centering 教学体会&\multicolumn{4}{m{165mm}|}{\quad}\\
\end{longtable}

\end{landscape}
\clearpage
%%%%%%%%%%%%%%%%%%%%板书设计%%%%%%%%%%%%%%%%%%%%%%%%%%
\begin{center}
{\huge 板书设计}
\end{center}
}
\mode<presentation>{ \section{高阶系统分析}
 \subsection{高阶系统分析}}
 \begin{frame}{高阶系统的单位阶跃响应}
 \begin{block}{高阶闭环传递函数}
 \begin{eqnarray*}
 \Phi(s)&=&\frac{C(s)}{R(s)}=\frac{B(s)}{A(s)}\\
 &=&\frac{b_0s^m+b_1s^{m-1}+\cdots +b_{m-1}s+b_m}{s^n+a_1s^{n-1}+\cdots +a_{n-1}s+a_n}
 \end{eqnarray*}
 其中:$B(s)$为分子多项式,$A(s)$为分母多项式
 \end{block}
 \end{frame}
 
 \begin{frame}
 \begin{block}{极点、零点形式}
 \begin{eqnarray*}
 \Phi(s)&=&\frac{C(s)}{R(s)}=\frac{B(s)}{A(s)}\\
&=&\frac{K\prod_{i=1}^m(s-z_i)}{\prod_{j=1}^{n_1}(s-p_j)\prod_{i=1}^{n_2}(s^2+2\zeta_l\omega_ls+\omega_l^2)}
\end{eqnarray*}  
\end{block}
\end{frame}

\begin{frame}
\begin{block}{单位阶跃输出部分展开式}
\begin{eqnarray*}
C(s)=\frac{K\prod_{i=1}^m(s-z_i)}{s[\prod_{j=1}^{n_1}(s-p_j)\prod_{i=1}^{n_2}(s^2+2\zeta_l\omega_ls+\omega_l^2)]}\\
=\frac{a_0}{s}+\sum_{j=1}^{n_1}\frac{a_j}{s-p_j}+\sum_{l=1}^{n_2}\frac{\beta_ls+r_l}{(s-p_{1l})(s-p_{2l})}
\end{eqnarray*}
其中:$a_0,a_j,\beta_l,r_l$为待定系数
\end{block}
\end{frame}
\begin{frame}
待定系统求法:
\[a_0=[sC(s)]\mid_{s=0}\]
\[a_j=[(s-p_j)C(s)]\mid_{s=p_j},j=1,2,\cdots ,n\]
\[\beta_ls+r_l \]
\[=[(s-p_{1l})(s-P_{2l}C(s)]\mid_{s=p_{1l}(or s=p_{2l})},\]
\[l=1,2,\cdots ,n\]
\end{frame}
\begin{frame}
\begin{block}{拉氏变换}
\begin{eqnarray*}
c(t)&=&L^{-1}[C(s)]\\
&=&a_0+\sum_{j=1}^{n_1}a_je^{p_jt}+\\
&&\sum_{l=1}^{n_1}\beta_le^{-\zeta_l\omega_lt}\cos\omega_l\sqrt{1-\zeta^2_l}t+\\
&&\sum_{l=1}^{n_2}\frac{r_l-\beta_l\zeta_l\omega_l}{\omega_l\sqrt{1-\zeta_l^2}}e^{-\zeta_l\omega_lt}\sin\omega_l\sqrt{1-\zeta_l^2}t
\end{eqnarray*}
\end{block}
\end{frame}

\begin{frame}{高阶系统分析}
\begin{block}{}
\begin{itemize}
\item<+-> 位于左半s平面,系统稳定
\item<+-> 若系统稳定:离虚轴越远,则指数项衰减走越快、
\item<+-> 实际中,常用根轨迹法、频率法等
\end{itemize}
\end{block}
\end{frame}
\begin{frame}{稳态误差分析}
\begin{block}{系统开环传递函数}
\begin{eqnarray*}
G_o(s)=\frac{K(1+\tau_1s)(1+\tau_2s)\cdots(1+\tau_ms}{s^q(1+T_1s)(1+T_2s)\cdots(1+T_{n-q}s)}
\end{eqnarray*}
其中:$K$为开环增益;\\
$\tau_1,\tau_2,\cdots,\tau_m$和$T_1,T_2,\cdots,T_n$为时间常数;\\
$q$为开环系统在$s$平面坐标原点上重极点数
\end{block}
\end{frame}
\begin{frame}
\begin{block}{稳态误差}
\begin{itemize}
\item<+-> 闭环传递函数为:
\[\Phi(s)=\frac{C(s)}{R(s)}=\frac{G(s)}{1+H(s)G(s)}\]
\item<+-> 则开环为:
\[G_o(s)=G(s)H(s)\]
\end{itemize}
\end{block}
\end{frame}
\begin{frame}
\begin{block}{}
\begin{itemize}
\item<+-> 误差信息$e(t)$为:
\[e(t)=r(t)-b(t)\]
\item<+-> 拉氏变换为:
\[E(s)=R(s)-B(s)=R(s)-H(s)C(s)\]
\end{itemize}
\end{block}
\end{frame}
\begin{frame}
\begin{block}{误差传递函数为}
\begin{eqnarray*}
\Phi_E(s)&=&\frac{E(s)}{R(s)}=\frac{1}{1+H(s)G(s)}\\
E(s)&=&\Phi_E(s)R(s)\\
&=&\frac{1}{1+H(s)G(s)}R(s)\\
&=&\frac{1}{1+G_o(s)}R(s)
\end{eqnarray*}
\end{block}
\end{frame}
\begin{frame}
\begin{block}{系统稳态误差}
\begin{eqnarray*}
e_{ss}&=&\lim_{s\rightarrow \infty}e(t)=\lim_{s\rightarrow 0}sE(s)\\
&=&\lim_{s\rightarrow 0}\frac{sR(s)}{1+H(s)R(s)}\\
&=&\lim_{s\rightarrow 0}\frac{sR(s)}{1+G_o(s)}
\end{eqnarray*}
\end{block}
\end{frame}
\begin{frame}
\begin{block}{阶跃输入的稳态误差}
\begin{eqnarray*}
e_{ss}(step)&=&\lim\limits_{s\rightarrow 0}\frac{s}{1+G_o(s)}\frac{R_0}{s}\\
&=&\lim\limits_{s\rightarrow 0}\frac{R_0}{1+G_o(s)}\\
&=&\lim\limits_{s\rightarrow 0}\frac{R_0}{1+K_p}\\
K_p&=&\lim\limits_{s\rightarrow 0}G_o(s)=\lim\limits_{s\rightarrow 0}H(s)G(s)
\end{eqnarray*}
\end{block}
\end{frame}
\begin{frame}
\begin{block}{斜坡输入时的稳态误差}
\begin{eqnarray*}
e_{ss}(ramp)&=&\lim\limits_{s\rightarrow 0}\frac{s}{1+G_o(s)}\frac{R_0}{s^2}\\
&=&\lim\limits_{s\rightarrow 0}\frac{R_0}{sG_o(s)}\\
&=&\frac{R_0}{K_V}\\
K_V&=&\lim\limits_{s\rightarrow 0}sG_o(s)=\lim\limits_{s\rightarrow 0}sH(s)G(s)
\end{eqnarray*}
\end{block}
\end{frame}
\begin{frame}
\begin{block}{抛物线输入时的稳态误差}
\begin{eqnarray*}
&&e_{ss}(parabolic)=\lim\limits_{s\rightarrow 0}\frac{s}{1+G_o(s)}\frac{R_0}{s^3}\\
&=&\lim\limits_{s\rightarrow 0}\frac{R_0}{s^2G_o(s)}\\
&=&\frac{R_0}{K_a}\\
K_a&=&\lim\limits_{s\rightarrow 0}s^2G_o(s)=\lim\limits_{s\rightarrow 0}s^2H(s)G(s)
\end{eqnarray*}
\end{block}
\end{frame}
\endinput
%%%%%%%%%%%%%%教案头%%%%%%%%%%%%%%%%%%%%%%%%%%%%%%%
\mode<article>{

\begin{longtable}{|m{20mm}|m{20mm}|m{20mm}|m{20mm}|m{20mm}|m{28mm}|}
\caption*{\huge 教案头}\\
\hline
\endfirsthead
\multicolumn{6}{l}{(续表)}\\
\hline
\endhead
\hline
\multicolumn{6}{l}{\itshape 接下一页表格.......}\\ [2ex]
\endfoot
\hline
\endlastfoot
\centering{授课单元}&\multicolumn{3}{m{60mm}|}{\centering 2.6.8稳定性分析}&\centering{授课日期}&2014年04月25日 \\
\hline
\centering 授课地点 & \multicolumn{3}{m{60mm}|}{B6-204}&\centering 授课学时 & 2 \\
\hline
& \multicolumn{2}{m{40mm}|}{能力目标} & \multicolumn{2}{m{40mm}|}{知识目标}&素质目标 \\
\cline{2-6}
\centering 教学目标&\multicolumn{2}{m{40mm}|}{\begin{enumerate}
\item 能够分析系统的稳定性
\end{enumerate} }&\multicolumn{2}{m{40mm}|}{\begin{enumerate}
\item 了解系统的极点、零点和稳定性的概念
\item 掌握劳斯稳定性判据
\end{enumerate}} & {\qquad}\\
\hline
\centering 能力训练任务或案例 &\multicolumn{5}{m{108mm}|}{ }\\
\hline
\centering 教学重点 & \multicolumn{5}{m{108mm}|}{\begin{enumerate}
\item 劳斯稳定性判据
\end{enumerate}}\\
\hline
\centering 教学难点与解决办法 &\multicolumn{5}{m{108mm}|}{\begin{enumerate}
\item 难点:劳斯稳定性判据
\item 解决方法:实例讲解
\end{enumerate}}\\
\hline
\centering 德育内容 &\multicolumn{5}{m{108mm}|}{无}\\
\hline
 &教材 & \multicolumn{4}{m{88mm}|}{计算机控制原理与应用}\\
\cline{2-6}& 教学资源 &\multicolumn{4}{m{88mm}|}{PPT}\\
\cline{2-6}\centering 使用的教学材料& 主要教学仪器设备和工具等 &\multicolumn{4}{m{88mm}|}{投影机、MATLAB}\\
\cline{2-6}& 主要耗材 &\multicolumn{4}{m{88mm}|}{无}\\
\hline
\centering 教学模式 &\multicolumn{2}{m{40mm}|}{知识讲授}&\centering 教学手段 &\multicolumn{2}{m{48mm}|}{多媒体教学}\\
\hline
\centering 学生成果与过程考核方式 &\multicolumn{5}{m{108mm}|}{无}
\end{longtable}
\clearpage

%%%%%%%%%%%%%%%教学实施过程%%%%%%%%%%%%%%%%%%%%%%%%%%%%
\begin{landscape}

\begin{longtable}{|m{10mm}|m{50mm}|m{50mm}|m{50mm}|m{15mm}|}
\caption*{\huge 教学组织与实施}\\
\hline
\endfirsthead
\multicolumn{5}{l}{\small 接上页}\\
\hline
\multicolumn{1}{|c|}{步骤}&\multicolumn{1}{c|}{教学内容}&\multicolumn{1}{c|}{教师活动}&\multicolumn{1}{c|}{学生活动}&\multicolumn{1}{c|}{时间}\\
\hline
\endhead

\multicolumn{5}{r}{\small 接下页}\\
\endfoot
\hline
\endlastfoot
\multicolumn{1}{|c|}{步骤}&\multicolumn{1}{c|}{教学内容}&\multicolumn{1}{c|}{教师活动}&\multicolumn{1}{c|}{学生活动}&\multicolumn{1}{c|}{时间}\\\hline
讲解&\begin{enumerate}
\item 极点、零点和稳定性的概念
\end{enumerate} &\begin{enumerate}
\item 讲解极点、零点和稳定性的概念
\end{enumerate} &\begin{enumerate}
\item 学生倾听并记录
\end{enumerate} &15 \\\hline
讲解&\begin{enumerate}
\item 稳定性判定的数学理论
\end{enumerate}
 &\begin{enumerate}
\item 讲解稳定性判定的数学理论
\end{enumerate} &\begin{enumerate}
\item 学生倾听并记录
\end{enumerate} &30 \\\hline
讲解&\begin{enumerate}
\item 劳斯稳定性判据
\end{enumerate}
&\begin{enumerate}
\item 讲解劳斯稳定性判据
\end{enumerate} &\begin{enumerate}
\item 学生倾听并记录
\end{enumerate} &20 \\\hline
讲解&\begin{enumerate}
\item 劳斯稳定性判据的应用实例
\end{enumerate}
 &\begin{enumerate}
\item 讲解劳斯稳定性判据的应用实例
\end{enumerate} &\begin{enumerate}
\item 学生记录笔记
\end{enumerate} &20 \\\hline

\centering 本次课总结(评价)&总结本课程内容 &进行知识总结 &学生倾听 &5 \\\hline
\centering 学生学习笔记或工单等检查情况&\multicolumn{4}{m{165mm}|}{\quad}\\\hline
\centering 课后作业&\multicolumn{4}{m{165mm}|}{2-28,2-29}\\\hline
\centering 教学体会&\multicolumn{4}{m{165mm}|}{\quad}\\
\end{longtable}

\end{landscape}
\clearpage
%%%%%%%%%%%%%%%%%%%%板书设计%%%%%%%%%%%%%%%%%%%%%%%%%%
\begin{center}
{\huge 板书设计}
\end{center}
}
\mode<presentation>{ \section{稳定性分析}
 \subsection{稳定性分析}}
 \begin{frame}{极点、零点和稳定性}
 \begin{block}{$n$阶传递函数的因子形式}
 \begin{eqnarray*}
 \Phi(s)&=&\frac{C(s)}{R(s)}\\
 &=&\frac{K\prod\limits_{i=1}^m(s-z_i)}{\prod\limits_{j=1}^{n_1}(s-p_j)\prod\limits_{l=1}^{n_2}(s^2+2\zeta_l\omega_ls+\omega_l^2)}
 \end{eqnarray*}
 \end{block}
 \end{frame}
 
 \begin{frame}
 \begin{block}{极点、零点的定义}
 设:
 \begin{eqnarray*}
 &&s^2+2\zeta_l\omega_l+\omega_l^2=(s-p_{1l})(s-p_{2l})\\
 &&s_{1l,2l}=-\zeta_l\omega_l\pm j\omega_l\sqrt{1-\zeta_l^2}\\
 &&l=1,2,\cdots ,n_2
\end{eqnarray*}
\begin{itemize}
\item $z_1,z_2,\cdots,z_m$称为零点
\item $p_{11},p_{12},\cdots,p_{1n_2},p_{21},p_{22},\cdots,p_{2n_2}$和\\
$p_1,p_2,\cdots,p_{n_1}$称为极点
\end{itemize}  
\end{block}
\end{frame}

\begin{frame}
\begin{block}{稳定性}
\begin{itemize}
\item 稳定系统
$\lim\limits_{t\rightarrow \infty}g(t)=0$
\item 不稳定系统
$\lim\limits_{t\rightarrow \infty}g(t)=\infty$
\item 临界稳定系统
$\lim\limits_{t\rightarrow \infty}g(t)=P$
\end{itemize}
\end{block}
\end{frame}
\begin{frame}
\begin{block}{单位脉冲输出部分展开式}
\begin{eqnarray*}
\Phi(s)=\frac{K\prod\limits_{i=1}^m(s-z_i)}{\prod\limits_{j=1}^{n_1}(s-p_j)\prod\limits_{i=1}^{n_2}(s^2+2\zeta_l\omega_ls+\omega_l^2)}\\
=\sum_{j=1}^{n_1}\frac{a_j}{s-p_j}+\sum_{l=1}^{n_2}\frac{\beta_ls+r_l}{(s-p_{1l})(s-p_{2l})}
\end{eqnarray*}
其中:$a_0,a_j,\beta_l,r_l$为待定系数
\end{block}
\end{frame}

\begin{frame}
\begin{block}{拉氏变换}
\begin{eqnarray*}
g(t)&=&L^{-1}[C(s)]=L^{-1}[\Phi(s)]\\
&=&\sum_{j=1}^{n_1}a_je^{p_jt}+\\
&&\sum_{l=1}^{n_1}\beta_le^{-\zeta_l\omega_lt}\cos\omega_l\sqrt{1-\zeta^2_l}t+\\
&&\sum_{l=1}^{n_2}\frac{r_l-\beta_l\zeta_l\omega_l}{\omega_l\sqrt{1-\zeta_l^2}}e^{-\zeta_l\omega_lt}\sin\omega_l\sqrt{1-\zeta_l^2}t
\end{eqnarray*}
\end{block}
\end{frame}

\begin{frame}{劳斯稳定性判据}
\begin{block}{$n$阶线性定常系统的特征方程}
\[a_0s^n+a_1+s^{n-1}+\cdots+a_{n-1}s+a_n=0\]
\end{block}
\end{frame}
\begin{frame}
\begin{block}{劳斯表}
\begin{equation*}
\begin{array}{cccccc}
s^n&a_0&a_2&a_4&a_6&\cdots\\
s^{n-1}&a_1&a_3&a_5&a_7&\cdots\\
S^{n-2}&T_{3,1}&T_{3,2}&T_{3,3}&T_{3,4}&\cdots\\
S^{n-3}&T_{4,1}&T_{4,2}&T_{4,3}&T_{4,4}&\cdots\\
\cdots&\cdots&\cdots&\cdots&\cdots&\cdots\\
s^2&T_{n-1,1}&T_{n-1,2}\\
s^1&T_{n,1}\\
s^0&T_{n+1,1}
\end{array}
\end{equation*}
\end{block}
\end{frame}

\begin{frame}
\begin{block}{}
\begin{eqnarray*}
T_{3,1}=-\frac{\left|\begin{array}{cc}
a_0&a_2\\
a_1&a_3
\end{array}\right|}{a_1}=\frac{a_1a_2-a_0a_3}{a_1}\\
T_{3,2}=-\frac{\left|\begin{array}{cc}
a_0&a_4\\
a_1&a_5
\end{array}\right|}{a_1}=\frac{a_1a_4-a_0a_5}{a_1}\\
T_{3,3}=-\frac{\left|\begin{array}{cc}
a_0&a_6\\
a_1&a_7
\end{array}\right|}{a_1}=\frac{a_1a_6-a_0a_7}{a_1}
\end{eqnarray*}
\end{block}
\end{frame}
\begin{frame}
\begin{block}{}
\begin{eqnarray*}
T_{4,1}=-\frac{\left|\begin{array}{cc}
a_1&a_3\\
T_{3,1}&T_{3,2}
\end{array}\right|}{T_{3,1}}=\frac{T_{3,1}a_3-T_{3,2}a_1}{T_{3,1}}\\
T_{4,2}=-\frac{\left|\begin{array}{cc}
a_1&a_5\\
T_{3,1}&T_{3,3}
\end{array}\right|}{T_{3,1}}=\frac{T_{3,1}a_5-a_1T_{3,3}}{T_{3,1}}\\
\vdots
\end{eqnarray*}
\end{block}
\end{frame}

\begin{frame}
\begin{block}{}
\begin{eqnarray*}
T_{5,1}=-\frac{\left|\begin{array}{cc}
T_{3,1}&T_{3,2}\\
T_{4,1}&T_{4,2}
\end{array}\right|}{T_{4,1}}=\frac{T_{4,1}T_{3,2}-T_{3,1}T_{4,2}}{T_{4,1}}\\
T_{5,2}=-\frac{\left|\begin{array}{cc}
T_{3,1}&T_{3,3}\\
T_{4,1}&T_{4,3}
\end{array}\right|}{T_{4,1}}=\frac{T_{4,1}T_{3,3}-T_{3,1}T_{4,3}}{T_{4,1}}\\
T_{5,3}=-\frac{\left|\begin{array}{cc}
T_{3,1}&T_{3,4}\\
T_{4,1}&T_{4,4}
\end{array}\right|}{T_{4,1}}=\frac{T_{4,1}T_{3,4}-T_{3,1}T_{4,4}}{T_{4,1}}
\end{eqnarray*}
\end{block}
\end{frame}

\endinput
\end{CJK}
\end{document}