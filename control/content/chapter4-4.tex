%%%%%%%%%%%%%%教案头%%%%%%%%%%%%%%%%%%%%%%%%%%%%%%%
\mode<article>{

\begin{longtable}{|m{20mm}|m{20mm}|m{20mm}|m{20mm}|m{20mm}|m{28mm}|}
\caption*{\huge 教案头}\\
\hline
\endfirsthead
\multicolumn{6}{l}{(续表)}\\
\hline
\endhead
\hline
\multicolumn{6}{l}{\itshape 接下一页表格.......}\\ [2ex]
\endfoot
\hline
\endlastfoot
\centering{授课单元}&\multicolumn{3}{m{60mm}|}{\centering 4.4z变换}&\centering{授课日期}&2014年06月4日 \\
\hline
\centering 授课地点 & \multicolumn{3}{m{60mm}|}{B6-204}&\centering 授课学时 & 2 \\
\hline
& \multicolumn{2}{m{40mm}|}{能力目标} & \multicolumn{2}{m{40mm}|}{知识目标}&素质目标 \\
\cline{2-6}
\centering 教学目标&\multicolumn{2}{m{40mm}|}{\begin{enumerate}
\item  能够进行离散信号的Z变换
\end{enumerate} }&\multicolumn{2}{m{40mm}|}{\begin{enumerate}
\item 了解Z变换的定义
\item 了解Z变换的求解方法
\item 了解Z变换的逆变换
\end{enumerate}} & {\qquad}\\
\hline
\centering 能力训练任务或案例 &\multicolumn{5}{m{108mm}|}{ }\\
\hline
\centering 教学重点 & \multicolumn{5}{m{108mm}|}{\begin{enumerate}
\item Z变换
\item Z变换的逆变换
\end{enumerate}}\\
\hline
\centering 教学难点与解决办法 &\multicolumn{5}{m{108mm}|}{\begin{enumerate}
\item 难点:Z变换的逆变换
\item 解决方法:实例讲解
\end{enumerate}}\\
\hline
\centering 德育内容 &\multicolumn{5}{m{108mm}|}{无}\\
\hline
 &教材 & \multicolumn{4}{m{88mm}|}{计算机控制原理与应用}\\
\cline{2-6}& 教学资源 &\multicolumn{4}{m{88mm}|}{PPT}\\
\cline{2-6}\centering 使用的教学材料& 主要教学仪器设备和工具等 &\multicolumn{4}{m{88mm}|}{投影机、MATLAB}\\
\cline{2-6}& 主要耗材 &\multicolumn{4}{m{88mm}|}{无}\\
\hline
\centering 教学模式 &\multicolumn{2}{m{40mm}|}{知识讲授}&\centering 教学手段 &\multicolumn{2}{m{48mm}|}{多媒体教学}\\
\hline
\centering 学生成果与过程考核方式 &\multicolumn{5}{m{108mm}|}{无}
\end{longtable}
\clearpage

%%%%%%%%%%%%%%%教学实施过程%%%%%%%%%%%%%%%%%%%%%%%%%%%%
\begin{landscape}

\begin{longtable}{|m{10mm}|m{50mm}|m{50mm}|m{50mm}|m{15mm}|}
\caption*{\huge 教学组织与实施}\\
\hline
\endfirsthead
\multicolumn{5}{l}{\small 接上页}\\
\hline
\multicolumn{1}{|c|}{步骤}&\multicolumn{1}{c|}{教学内容}&\multicolumn{1}{c|}{教师活动}&\multicolumn{1}{c|}{学生活动}&\multicolumn{1}{c|}{时间}\\
\hline
\endhead

\multicolumn{5}{r}{\small 接下页}\\
\endfoot
\hline
\endlastfoot
\multicolumn{1}{|c|}{步骤}&\multicolumn{1}{c|}{教学内容}&\multicolumn{1}{c|}{教师活动}&\multicolumn{1}{c|}{学生活动}&\multicolumn{1}{c|}{时间}\\\hline
讲解&\begin{enumerate}
\item Z变换的定义
\end{enumerate} &\begin{enumerate}
\item 讲解Z变换的定义
\end{enumerate} &\begin{enumerate}
\item 学生倾听并记录
\end{enumerate} &25\\\hline
讲解&\begin{enumerate}
\item Z变换的实例
\end{enumerate}
 &\begin{enumerate}
\item 讲解Z变换的例子
\end{enumerate} &\begin{enumerate}
\item 学生倾听并记录
\end{enumerate} &20 \\\hline
讲解&\begin{enumerate}
\item Z变换定理
\end{enumerate}
&\begin{enumerate}
\item 讲解Z变换定理
\end{enumerate} &\begin{enumerate}
\item 学生倾听并记录
\end{enumerate} &40 \\\hline

\centering 本次课总结(评价)&总结本课程内容 &进行知识总结 &学生倾听 &5 \\\hline
\centering 学生学习笔记或工单等检查情况&\multicolumn{4}{m{165mm}|}{\quad}\\\hline
\centering 课后作业&\multicolumn{4}{m{165mm}|}{2-28,2-29}\\\hline
\centering 教学体会&\multicolumn{4}{m{165mm}|}{\quad}\\
\end{longtable}

\end{landscape}
\clearpage
%%%%%%%%%%%%%%%%%%%%板书设计%%%%%%%%%%%%%%%%%%%%%%%%%%
\begin{center}
{\huge 板书设计}
\end{center}
}

 \begin{frame}{4.4Z变换} 
 \begin{block}{}
\[f^*(t)=\sum\limits_{k=0}^\infty f(kT)\delta(t-kT)\]
 \end{block}
 \begin{block}{拉氏变换得}
 \[F^*(s)=\sum\limits_{k=0}^\infty f(kT)e^{-kTs}\]
 \end{block}
 \end{frame}
 
 \begin{frame}
 \begin{block}{引入复变量Z}
\begin{eqnarray*}
z=e^{sT}\\
s=\frac{1}{T}\ln z
\end{eqnarray*}
\end{block}
\begin{block}{Z变换定义为}
\[F(z)=\sum\limits_{k=0}^\infty f(kT)z^{-k}\]
\end{block}
\end{frame}

\begin{frame}{Z变换定理}
\begin{block}{线性定理}
\[\pounds[af_1(t)+bf_2(t)]=aF_1(z)+bF_2(z)\]
\end{block}
\begin{block}{超前定理}
\[\pounds[f(t+nT)]=z^n\left[F(z)-\sum\limits_{q=0}^{n-1}f(qT)z^{-q}\right]\]
\end{block}

\end{frame}

\begin{frame}
\begin{block}{滞后定理}
\begin{equation*}
\pounds[f(t-nT)u(t-nT)]=z^{-n}F(z)
\end{equation*}
\end{block}
\begin{block}{有限和定理}
\begin{equation*}
\pounds\left[\sum\limits_{k=0}^nf(kT)\right]=\frac{F(z)}{1-z^{-1}}
\end{equation*}
\end{block}
\end{frame}

\begin{frame}
\begin{block}{阻尼定理}
\begin{eqnarray*}
\pounds[e^{at}f(t)]=F(e^{-aT}z)
\end{eqnarray*}
\end{block}
\begin{block}{复微分定理}
\begin{eqnarray*}
\pounds[tf(t)]=-Tz\frac{d}{dz}F(z)
\end{eqnarray*}
\end{block}
\end{frame}

\begin{frame}
\begin{block}{初值定理}
\begin{eqnarray*}
f(0)=\lim_{k\to 0}f(kT)=\lim_{z\to\infty}F(z)
\end{eqnarray*}
\end{block}
\begin{block}{终值定理}
\begin{eqnarray*}
f(\infty)=\lim_{k\to\infty}f(kT)=\lim_{z\to 1}(z-1)F(z)
\end{eqnarray*}
\end{block}
\end{frame}

\begin{frame}{Z逆变换}
\begin{block}{长除法}
先将$F(z)$展开为无穷幂级数,再逐项求$Z$逆变换,实际中只计算几项就够了。
\end{block}
\begin{block}{部分展开法}
若$F(z)$为有理函数,先将其展开,再逐项求$Z$逆变换。
\end{block}
\end{frame}
\begin{frame}{}
\begin{block}{留数法}
是一种公式法
\begin{eqnarray*}
f(kT)=\pounds^{-1}[F(z)]=\frac{1}{2\pi j}\oint_{\Gamma}F(z)z^{k-1}dz
\end{eqnarray*}
\end{block}
\begin{block}{由留数定理得}
\begin{eqnarray*}
f(kT)=f(k)=f_1+f_2+\cdots +f_n\\
\sum\limits_{k=1}^nRes[F(z)z^{k-1}]|_{z=p_i}
\end{eqnarray*}
\end{block}
\end{frame}
\begin{frame}
\begin{block}{当$p_i$为非重极点时}
\[Res[F(z)z^{k-1}]|_{z=p_i}=\lim_{z\to p_i}(z-p_i)F(z)z^{k-1}\]
则
\[f(kT)=\sum\limits_{i=1}^n\lim_{z\to p_i}(z-p_i)F(z)z^{k-1}\]
\end{block}
\end{frame}

\begin{frame}
\begin{block}{有m重极点时}
\begin{eqnarray*}
&&Res[F(z)z^{k-1}]=\\
&&\frac{1}{(m-1)!}\lim_{z\to p_j}\left\lbrace\frac{d^{m-1}}{dz^{m-1}}[(z-p_j)^mF(z)z^{k-1}]\right\rbrace\\
&&f(kT)=\sum\limits_{i=1}^{n-m}\lim_{z\to p_i}[(z-p_i)F(z)z^{k-1}+\\
&&\lim_{z\to p_j}\frac{1}{(m-1)!}\left\lbrace\frac{d^{m-1}}{dz^{m-1}}[(z-p_j)^mF(z)z^{k-1}]\right\rbrace
\end{eqnarray*}
\end{block}
\end{frame}
\begin{frame}
\begin{block}{计算法}
\begin{itemize}
\item Matlab法
\item 差分方程法
\end{itemize}
\end{block}
\end{frame}
