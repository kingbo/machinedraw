%%%%%%%%%%%%%%%教案头%%%%%%%%%%%%%%%%%%%%%%%%%%%%%%%
\mode<article>{

\begin{longtable}{|m{20mm}|m{20mm}|m{20mm}|m{20mm}|m{20mm}|m{28mm}|}
\caption*{\huge 教案头}\\
\hline
\endfirsthead
\multicolumn{6}{l}{(续表)}\\
\hline
\endhead
\hline
\multicolumn{6}{l}{\itshape 接下一页表格.......}\\ [2ex]
\endfoot
\hline
\endlastfoot
\centering{授课单元}&\multicolumn{3}{m{60mm}|}{\centering 2.4.4方框图简化}&\centering{授课日期}&2014年03月13日 \\
\hline
\centering 授课地点 & \multicolumn{3}{m{60mm}|}{B6-204}&\centering 授课学时 & 2 \\
\hline
& \multicolumn{2}{m{40mm}|}{能力目标} & \multicolumn{2}{m{40mm}|}{知识目标}&素质目标 \\
\cline{2-6}
\centering 教学目标&\multicolumn{2}{m{40mm}|}{\begin{enumerate}
\item 能够进行方框图的简化
\item 能够用MatLab进行系统方框图简化
\end{enumerate} }&\multicolumn{2}{m{40mm}|}{\begin{enumerate}
\item 掌握方框图简化的方法
\item 掌握用MatLab进行方框图简化的方法
\end{enumerate}} & {\qquad}\\
\hline
\centering 能力训练任务或案例 &\multicolumn{5}{m{108mm}|}{ }\\
\hline
\centering 教学重点 & \multicolumn{5}{m{108mm}|}{\begin{enumerate}
\item 方框图的简化
\end{enumerate}}\\
\hline
\centering 教学难点与解决办法 &\multicolumn{5}{m{108mm}|}{\begin{enumerate}
\item 难点:方框图的简化
\item 解决方法:用实例进行分析讲解
\end{enumerate}}\\
\hline
\centering 德育内容 &\multicolumn{5}{m{108mm}|}{无}\\
\hline
 &教材 & \multicolumn{4}{m{88mm}|}{计算机控制原理与应用}\\
\cline{2-6}& 教学资源 &\multicolumn{4}{m{88mm}|}{PPT}\\
\cline{2-6}\centering 使用的教学材料& 主要教学仪器设备和工具等 &\multicolumn{4}{m{88mm}|}{投影机、MATLAB}\\
\cline{2-6}& 主要耗材 &\multicolumn{4}{m{88mm}|}{无}\\
\hline
\centering 教学模式 &\multicolumn{2}{m{40mm}|}{知识讲授}&\centering 教学手段 &\multicolumn{2}{m{48mm}|}{多媒体教学}\\
\hline
\centering 学生成果与过程考核方式 &\multicolumn{5}{m{108mm}|}{无}
\end{longtable}
\clearpage

%%%%%%%%%%%%%%%教学实施过程%%%%%%%%%%%%%%%%%%%%%%%%%%%%
\begin{landscape}

\begin{longtable}{|m{10mm}|m{50mm}|m{50mm}|m{50mm}|m{15mm}|}
\caption*{\huge 教学组织与实施}\\
\hline
\endfirsthead
\multicolumn{5}{l}{\small 接上页}\\
\hline
\multicolumn{1}{|c|}{步骤}&\multicolumn{1}{c|}{教学内容}&\multicolumn{1}{c|}{教师活动}&\multicolumn{1}{c|}{学生活动}&\multicolumn{1}{c|}{时间}\\
\hline
\endhead

\multicolumn{5}{r}{\small 接下页}\\
\endfoot
\hline
\endlastfoot
\multicolumn{1}{|c|}{步骤}&\multicolumn{1}{c|}{教学内容}&\multicolumn{1}{c|}{教师活动}&\multicolumn{1}{c|}{学生活动}&\multicolumn{1}{c|}{时间}\\\hline
引入&\begin{enumerate}
\item 系统反馈较多时不便于进行系统分析
\end{enumerate} &\begin{enumerate}
\item 讲解方框简化的原则
\end{enumerate} &\begin{enumerate}
\item 学生记录
\end{enumerate} &10 \\\hline
讲解&\begin{enumerate}
\item 方框图简化方法
\end{enumerate}
 &\begin{enumerate}
\item 通过实例讲解方框图的简化
\end{enumerate} &\begin{enumerate}
\item 学生倾听并记录
\end{enumerate} &15 \\\hline
练习&\begin{enumerate}
\item 方框图的化简
\end{enumerate}
&\begin{enumerate}
\item 指需要简化的方框图
\item 指导学生进行方框图简化
\item 讲解正确结果
\end{enumerate} &\begin{enumerate}
\item 学生尝试方框图简化
\item 学生展示简化结果
\item 学生进行记录
\end{enumerate} &20 \\\hline
讲解&\begin{enumerate}
\item 用MatLab进行系统方框图简化
\end{enumerate}
 &\begin{enumerate}
\item 讲解相关的MatLab指令
\item 讲解用MatLab进行系统方框图化的方法
\end{enumerate} &\begin{enumerate}
\item 学生记录笔记
\end{enumerate} &20 \\\hline
练习&
\begin{enumerate}
\item 用MatLab进行方框图简化
\end{enumerate}
 &\begin{enumerate}
\item 指导学生用MatLab进行方框图简化
\item 讲解要点
\end{enumerate} &\begin{enumerate}
\item 学生尝试用MatLab进行方框图简化
\item 学生记录笔记
\end{enumerate} &20 \\\hline
\centering 本次课总结(评价)&总结本课程内容 &进行知识总结 &学生倾听 &5 \\\hline
\centering 学生学习笔记或工单等检查情况&\multicolumn{4}{m{165mm}|}{\quad}\\\hline
\centering 课后作业&\multicolumn{4}{m{165mm}|}{2-19,2-20,2-21}\\\hline
\centering 教学体会&\multicolumn{4}{m{165mm}|}{\quad}\\
\end{longtable}

\end{landscape}
\clearpage
%%%%%%%%%%%%%%%%%%%%板书设计%%%%%%%%%%%%%%%%%%%%%%%%%%
\lecture{传递函数与方框图}{chuandihanshu}
\begin{center}
{\huge 板书设计}
\end{center}
}
\mode<presentation>{ \section{方框图简化}
 \subsection{方框图简化}}
 \begin{frame}
 \begin{block}{简化标准}
 \begin{itemize}
 \item 简化成标准形式
 \end{itemize}
 \end{block}
 \begin{block}{方框图简化原则}
 \begin{itemize}
 \item 简化前与简化后的数学关系不变
 \end{itemize}
 \end{block}
 \begin{block}{简化规则}
 \begin{itemize}
 \item 简化规则见P59页至P60页
 \end{itemize}
 \end{block}
 \end{frame}
 \begin{frame}
 \begin{block}{简化的特点}
 \begin{itemize}
 \item 传递函数具有唯一性
 \item 方框图不唯一
 \end{itemize}
 \end{block}
 \end{frame}
 \begin{frame}{用MatLab进行方框图化简}
 \begin{block}{串联}
 sys1=tf(num1,den1)
 
 sys2=tf(num2,den2)
 
 sys=series(sys1,sys2)
 \end{block}
 \begin{block}{并联}
 sys1=tf(num1,den1)
 
 sys2=tf(num2,den2)
 
 sys=parallel(sys1,sys2)
 \end{block}
 \end{frame}
 \begin{frame}
 \begin{block}{反馈连接}
sysg=[numg,deng]

sysh=[numh,denh]

sys=feedback(sysg,sysh)
 \end{block}
 \end{frame}
