%%%%%%%%%%%%%%教案头%%%%%%%%%%%%%%%%%%%%%%%%%%%%%%%
\mode<article>{

\begin{longtable}{|m{20mm}|m{20mm}|m{20mm}|m{20mm}|m{20mm}|m{28mm}|}
\caption*{\huge 教案头}\\
\hline
\endfirsthead
\multicolumn{6}{l}{(续表)}\\
\hline
\endhead
\hline
\multicolumn{6}{l}{\itshape 接下一页表格.......}\\ [2ex]
\endfoot
\hline
\endlastfoot
\centering{授课单元}&\multicolumn{3}{m{60mm}|}{\centering 3.5描述函数法}&\centering{授课日期}&2014年05月23日 \\
\hline
\centering 授课地点 & \multicolumn{3}{m{60mm}|}{B6-204}&\centering 授课学时 & 2 \\
\hline
& \multicolumn{2}{m{40mm}|}{能力目标} & \multicolumn{2}{m{40mm}|}{知识目标}&素质目标 \\
\cline{2-6}
\centering 教学目标&\multicolumn{2}{m{40mm}|}{\begin{enumerate}
\item  能够进行相轨迹图的绘制
\item  能够用MatLab分析非线性系统
\end{enumerate} }&\multicolumn{2}{m{40mm}|}{\begin{enumerate}
\item 了解相平面法的基本概念
\item 了解相轨图的绘制
\item 了解用MatLab分析非线性系统
\end{enumerate}} & {\qquad}\\
\hline
\centering 能力训练任务或案例 &\multicolumn{5}{m{108mm}|}{ }\\
\hline
\centering 教学重点 & \multicolumn{5}{m{108mm}|}{\begin{enumerate}
\item 用MatLab分析非线性系统
\end{enumerate}}\\
\hline
\centering 教学难点与解决办法 &\multicolumn{5}{m{108mm}|}{\begin{enumerate}
\item 难点:相平面法的基本概念
\item 解决方法:实例讲解
\end{enumerate}}\\
\hline
\centering 德育内容 &\multicolumn{5}{m{108mm}|}{无}\\
\hline
 &教材 & \multicolumn{4}{m{88mm}|}{计算机控制原理与应用}\\
\cline{2-6}& 教学资源 &\multicolumn{4}{m{88mm}|}{PPT}\\
\cline{2-6}\centering 使用的教学材料& 主要教学仪器设备和工具等 &\multicolumn{4}{m{88mm}|}{投影机、MATLAB}\\
\cline{2-6}& 主要耗材 &\multicolumn{4}{m{88mm}|}{无}\\
\hline
\centering 教学模式 &\multicolumn{2}{m{40mm}|}{知识讲授}&\centering 教学手段 &\multicolumn{2}{m{48mm}|}{多媒体教学}\\
\hline
\centering 学生成果与过程考核方式 &\multicolumn{5}{m{108mm}|}{无}
\end{longtable}
\clearpage

%%%%%%%%%%%%%%%教学实施过程%%%%%%%%%%%%%%%%%%%%%%%%%%%%
\begin{landscape}

\begin{longtable}{|m{10mm}|m{50mm}|m{50mm}|m{50mm}|m{15mm}|}
\caption*{\huge 教学组织与实施}\\
\hline
\endfirsthead
\multicolumn{5}{l}{\small 接上页}\\
\hline
\multicolumn{1}{|c|}{步骤}&\multicolumn{1}{c|}{教学内容}&\multicolumn{1}{c|}{教师活动}&\multicolumn{1}{c|}{学生活动}&\multicolumn{1}{c|}{时间}\\
\hline
\endhead

\multicolumn{5}{r}{\small 接下页}\\
\endfoot
\hline
\endlastfoot
\multicolumn{1}{|c|}{步骤}&\multicolumn{1}{c|}{教学内容}&\multicolumn{1}{c|}{教师活动}&\multicolumn{1}{c|}{学生活动}&\multicolumn{1}{c|}{时间}\\\hline
讲解&\begin{enumerate}
\item 相平面法的基本概念
\end{enumerate} &\begin{enumerate}
\item 讲解检点平面法的基本概念
\end{enumerate} &\begin{enumerate}
\item 学生倾听并记录
\end{enumerate} &20\\\hline
讲解&\begin{enumerate}
\item 相轨迹图的绘制
\end{enumerate}
 &\begin{enumerate}
\item 讲解相轨迹图的绘制
\end{enumerate} &\begin{enumerate}
\item 学生倾听并记录
\end{enumerate} &25 \\\hline
讲解&\begin{enumerate}
\item 用MatLab分析非线性系统
\end{enumerate}
&\begin{enumerate}
\item 讲解用MatLab分析非线性系统
\end{enumerate} &\begin{enumerate}
\item 学生倾听并记录
\end{enumerate} &40 \\\hline

\centering 本次课总结(评价)&总结本课程内容 &进行知识总结 &学生倾听 &5 \\\hline
\centering 学生学习笔记或工单等检查情况&\multicolumn{4}{m{165mm}|}{\quad}\\\hline
\centering 课后作业&\multicolumn{4}{m{165mm}|}{2-28,2-29}\\\hline
\centering 教学体会&\multicolumn{4}{m{165mm}|}{\quad}\\
\end{longtable}

\end{landscape}
\clearpage
%%%%%%%%%%%%%%%%%%%%板书设计%%%%%%%%%%%%%%%%%%%%%%%%%%
\begin{center}
{\huge 板书设计}
\end{center}
}

 \begin{frame}{相平面法的基本概念} 
 \begin{block}{二阶时不变系统的常微分方程}
 \[\ddot{x}+f(x,\dot{x})=0\]
 \end{block}
 \begin{block}{相平面}
 由$x$和$\dot{x}$组成的平面称为相平面。
 \end{block}
 \begin{block}{相轨迹}
 当$t$变化时,在相平面给的出关系曲线称为相轨迹。
 \end{block}
 \end{frame}
 
 \begin{frame}
 \begin{block}{相平面图}
 不同的初如条件,对应的一簇相轨迹所组成的图像
 \end{block}
 \begin{block}{相平面法}
 利用相平面分析系统的方法
 \end{block}
 \end{frame}

\begin{frame}{相轨迹的性质}
\begin{block}{相轨迹的斜率}
由二阶微分式:
\[\frac{d\dot{x}}{dt}=-f(x,\dot{x})\]

两边同除$\frac{dx}{dt}$
\[\frac{d\dot{x}}{dx}=-\frac{f(x,\dot{x}}{\dot{x}}\]

\end{block}
\end{frame}

\begin{frame}
\begin{block}{}
相平面的横坐标为$x$,纵坐标为$\dot{x}$,则$\frac{d\dot{x}}{dx}$表示在$(x,\dot{x})$点上的根轨迹斜率。
\end{block}
\begin{block}{根轨迹对称条件}
\begin{itemize}
\item $x$轴对称:$f(x,\dot{x})=f(x,-\dot{x})$
\item $\dot{x}$轴对称:$f(x,\dot{x})=-f(-x,\dot{x})$
\item 原点对称:$f(x,\dot{x}=-f(-x,-\dot{x})$
\end{itemize}
\end{block}
\begin{block}{相平面上的奇点}
如果同时满足$\dot{x}=0$和$f(x,\dot{x})=0$的点
\end{block}
\end{frame}

\begin{frame}{相轨迹图的绘制}
\begin{block}{绘制方法}
\begin{itemize}
\item 解析法
\item 等倾线法
\item 用计算机绘制相轨迹图
\end{itemize}
\end{block}
\end{frame}

