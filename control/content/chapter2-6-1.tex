%%%%%%%%%%%%%%教案头%%%%%%%%%%%%%%%%%%%%%%%%%%%%%%%
\mode<article>{

\begin{longtable}{|m{20mm}|m{20mm}|m{20mm}|m{20mm}|m{20mm}|m{28mm}|}
\caption*{\huge 教案头}\\
\hline
\endfirsthead
\multicolumn{6}{l}{(续表)}\\
\hline
\endhead
\hline
\multicolumn{6}{l}{\itshape 接下一页表格.......}\\ [2ex]
\endfoot
\hline
\endlastfoot
\centering{授课单元}&\multicolumn{3}{m{60mm}|}{\centering 2.6控制系统的时域分析2.6.1脉新华路响应和阶路响应2.6.2时域性能指标2.6.3一阶系统的动态响应}&\centering{授课日期}&2014年03月13日 \\
\hline
\centering 授课地点 & \multicolumn{3}{m{60mm}|}{B6-204}&\centering 授课学时 & 2 \\
\hline
& \multicolumn{2}{m{40mm}|}{能力目标} & \multicolumn{2}{m{40mm}|}{知识目标}&素质目标 \\
\cline{2-6}
\centering 教学目标&\multicolumn{2}{m{40mm}|}{\begin{enumerate}
\item 能够分析一阶系统的动态响应
\end{enumerate} }&\multicolumn{2}{m{40mm}|}{\begin{enumerate}
\item 了解脉冲响应函数和阶跃响应函数
\item 了解时域性能指标
\end{enumerate}} & {\qquad}\\
\hline
\centering 能力训练任务或案例 &\multicolumn{5}{m{108mm}|}{ }\\
\hline
\centering 教学重点 & \multicolumn{5}{m{108mm}|}{\begin{enumerate}
\item 一阶系统的动态响应
\end{enumerate}}\\
\hline
\centering 教学难点与解决办法 &\multicolumn{5}{m{108mm}|}{\begin{enumerate}
\item 难点:一阶系统的动态响应分析
\item 解决方法:用实例进行分析讲解
\end{enumerate}}\\
\hline
\centering 德育内容 &\multicolumn{5}{m{108mm}|}{无}\\
\hline
 &教材 & \multicolumn{4}{m{88mm}|}{计算机控制原理与应用}\\
\cline{2-6}& 教学资源 &\multicolumn{4}{m{88mm}|}{PPT}\\
\cline{2-6}\centering 使用的教学材料& 主要教学仪器设备和工具等 &\multicolumn{4}{m{88mm}|}{投影机、MATLAB}\\
\cline{2-6}& 主要耗材 &\multicolumn{4}{m{88mm}|}{无}\\
\hline
\centering 教学模式 &\multicolumn{2}{m{40mm}|}{知识讲授}&\centering 教学手段 &\multicolumn{2}{m{48mm}|}{多媒体教学}\\
\hline
\centering 学生成果与过程考核方式 &\multicolumn{5}{m{108mm}|}{无}
\end{longtable}
\clearpage

%%%%%%%%%%%%%%%教学实施过程%%%%%%%%%%%%%%%%%%%%%%%%%%%%
\begin{landscape}

\begin{longtable}{|m{10mm}|m{50mm}|m{50mm}|m{50mm}|m{15mm}|}
\caption*{\huge 教学组织与实施}\\
\hline
\endfirsthead
\multicolumn{5}{l}{\small 接上页}\\
\hline
\multicolumn{1}{|c|}{步骤}&\multicolumn{1}{c|}{教学内容}&\multicolumn{1}{c|}{教师活动}&\multicolumn{1}{c|}{学生活动}&\multicolumn{1}{c|}{时间}\\
\hline
\endhead

\multicolumn{5}{r}{\small 接下页}\\
\endfoot
\hline
\endlastfoot
\multicolumn{1}{|c|}{步骤}&\multicolumn{1}{c|}{教学内容}&\multicolumn{1}{c|}{教师活动}&\multicolumn{1}{c|}{学生活动}&\multicolumn{1}{c|}{时间}\\\hline
讲解&\begin{enumerate}
\item 脉冲响应和阶跃响应
\end{enumerate} &\begin{enumerate}
\item 讲解分析单位脉冲响应和阶跃响应函数
\end{enumerate} &\begin{enumerate}
\item 学生倾听并记录
\end{enumerate} &20 \\\hline
讲解&\begin{enumerate}
\item 时域性能指标
\end{enumerate}
 &\begin{enumerate}
\item 通过图示讲解时域性能指标
\end{enumerate} &\begin{enumerate}
\item 学生倾听并记录
\end{enumerate} &25 \\\hline
讲解&\begin{enumerate}
\item 一阶系统的数学模型
\end{enumerate}
&\begin{enumerate}
\item 讲解一阶系统的数学模型
\end{enumerate} &\begin{enumerate}
\item 学生倾听并记录
\end{enumerate} &10 \\\hline
讲解&\begin{enumerate}
\item 一阶系统的单位阶跃响应
\end{enumerate}
 &\begin{enumerate}
\item 讲解阶系统的单位阶跃响应分析
\end{enumerate} &\begin{enumerate}
\item 学生记录笔记
\end{enumerate} &20 \\\hline
讲解&
\begin{enumerate}
\item 一阶系统的单位脉冲响应
\end{enumerate}
 &\begin{enumerate}
\item 讲解一阶系统的单位脉冲响应分析
\end{enumerate} &\begin{enumerate}
\item 学生记录笔记
\end{enumerate} &10 \\\hline
\centering 本次课总结(评价)&总结本课程内容 &进行知识总结 &学生倾听 &5 \\\hline
\centering 学生学习笔记或工单等检查情况&\multicolumn{4}{m{165mm}|}{\quad}\\\hline
\centering 课后作业&\multicolumn{4}{m{165mm}|}{2-19,2-20,2-21}\\\hline
\centering 教学体会&\multicolumn{4}{m{165mm}|}{\quad}\\
\end{longtable}

\end{landscape}
\clearpage
%%%%%%%%%%%%%%%%%%%%板书设计%%%%%%%%%%%%%%%%%%%%%%%%%%
\lecture{传递函数与信号流图}{chuandihanshu}
\begin{center}
{\huge 板书设计}
\end{center}
}
\mode<presentation>{ \section{传递函数与信号流图}
 \subsection{信号流图的定义}}
 \begin{frame}{信号流图}
 \uncover<+->{\begin{block}{方框图模型}
 \begin{tikzpicture}
 [node distance=10mm,>=stealth]
 \node(start)[circleterminal,label=above left:\tiny +,label=below left:-]{};
 \draw(start.north east)--(start.south west)(start.south east)--(start.north west);
 \draw[<-](start.west)--++(-10mm,0)node[above,midway]{\tiny R(s)};
 \node(GC)[nonterminal,right=of start]{\tiny $G_c(s)$};
 \draw[->](start.east)--(GC.west)node[above,midway]{\tiny $E(s)$};
 \node(GP)[nonterminal,right=of GC]{\tiny $G_p(s)$};
 \draw[->](GC.east)--(GP.west)node[above,midway]{\tiny $M(s)$};
 \draw[->](GP.east)--++(5mm,0)coordinate(a)--++(5mm,0)node[above,midway]{\tiny $C(s)$};
 \filldraw(a)circle(2pt);
 \node(H)[nonterminal,below=of GC]{\tiny $H(s)$};
 \draw[->](a)|-(H.east);
 \draw[->](H.west)-|(start.south);
 \end{tikzpicture}
 \end{block}}
 \uncover<+->{
 \begin{block}{信号流图模型}
 \begin{tikzpicture}
 [node distance=10mm,>=stealth]
 \node(RS)[circlenode,label=above:\tiny $R(s)$]{};
 \node(a)[noshapenode,right=of RS,label=above:\tiny 1]{};
 \draw[->](RS)--(a);
 \node(ES)[circlenode,right=of a,label=above:\tiny $E(s)$]{};
 \draw(a.west)--(ES);
\node(GC)[noshapenode,right=of ES,label=above:\tiny $G_c(s)$]{};
\draw[->](ES)--(GC);
 \node(MS)[circlenode,right=of GC,label=above:\tiny $M(s)$]{};
 \draw(GC)--(MS);
\node(GP)[noshapenode,right=of MS,label=above:\tiny $G_p(s)$]{};
\draw[->](MS)--(GP);
 \node(CS1)[circlenode,right=of GP,label=above:\tiny $C(s)$]{};
 \draw(GP)--(CS1);
 \node(b)[noshapenode,right=of CS1,label=above:\tiny 1]{};
 \draw[->](CS1)--(b);
 \node(CS2)[circlenode,right=of b,label=above:\tiny $C(s)$]{};
 \draw(b)--(CS2);
 \node(H)[noshapenode,below=of MS,label=above:\tiny $H(s)$]{};
 \draw[->](CS1)to[bend left=25](H);
 \draw(ES)to[bend right=25](H);
 \end{tikzpicture}
 \end{block}
 }
 \end{frame}
 \begin{frame}{信号流图}
 \uncover<+->{\begin{block}{信息流图的组成}
 \begin{itemize}
 \item<+-> 节点:表示变量或信号
 \item<+-> 支路:连接两个接点的有向线段
 \end{itemize}
 \end{block}}
 \uncover<+->{\begin{block}{支路的特点}
 \begin{itemize}
 \item<+-> 联接有因果关系的节点,相当于乘法器
 \item<+-> 有方向性,信号只能沿箭头单向传递
 \item<+-> 有加权性
 \end{itemize}
 \end{block}  }
 \end{frame}
 \begin{frame}{信号流图} 
 \uncover<+->{\begin{block}{信号流图的特点}
 \begin{itemize}
 \item<+-> 是系统代数方程的图形表示
 \item<+-> 不是唯一的
 \item<+-> 便于用梅逊增益公式求解系统传递函数
 \end{itemize}
 \end{block}}
 \end{frame}
 \begin{frame}{信号流图}
 \uncover<+->{\begin{block}{信号流图术语}
 \begin{itemize}
 \item<+-> 输入节点:表示独立变量,仅有输入支路
 \item<+-> 输出节点:表示因变量,仅有输出支路
 \item<+-> 混合节点:既有输入支路也有输出支路
 \item<+-> 通路:表示一个沿支路箭头方向,穿过各个相连支路的线路。
 \item<+-> 前向通路:输入节点和输出节点之间的通路,遇到或穿过任何节点不多于一次。
 \end{itemize}
 \end{block}}
 \end{frame}
 \begin{frame}{信号流图}
 \begin{block}{信号流图术语}
 \begin{itemize}
 \item<+-> 通路增益:是中通路中各支路传输的乘积。
 \item<+-> 回路:起点和终点在同一节点。
 \item<+-> 回路增益:是回路中的各支路传输的乘积。
 \item<+-> 不接触回路:两个回路没有共同节点。
 \end{itemize}
 \end{block}
 \end{frame}
 \begin{frame}{信号流图化简}
 \uncover<+->{\begin{block}{串联支路合并}
 串联相乘
 \begin{tikzpicture}
 [node distance=10mm,>=stealth]
 \node(x1)[circlenode,label=below:\tiny $x_1$]{};
 \node(a)[noshapenode,right=of x1,label=above:\tiny $a$]{};
 \draw[->](x1)--(a);
 \node(x2)[circlenode,right=of a,label=below:\tiny $x_2$]{};
 \draw(a)--(x2);
 \node(b)[noshapenode,right=of x2,label=above:\tiny $b$]{};
 \draw[->](x2)--(b);
 \node(x3)[circlenode,right=of b,label=below:\tiny $x_3$]{};
 \draw(b)--(x3);
 \node(c)[noshapenode,right=of x3]{\tiny $=$};
 \node(x4)[circlenode,right=of c,label=below:\tiny $x_1$]{};
 \node(d)[noshapenode,right=of x4,label=above:\tiny $ab$]{};
 \draw[->](x4)--(d);
 \node(x5)[circlenode,right=of d,label=below:\tiny $x_3$]{};
 \draw(d)--(x5);
 \end{tikzpicture}
 \end{block}}
 \uncover<+->{
 \begin{block}{并联支路合并}
 并联相加
 
 \begin{tikzpicture}
 [node distance=10mm,>=stealth]
 \node(x1)[circlenode,label=left:\tiny $x_1$]{};
 \node(x3)[noshapenode,right=of x1]{};
 \node(x2)[circlenode,right=of x3,label=right:\tiny $x_2$]{};
 \node(a)[noshapenode,above=7mm of x3,label=above:\tiny $a$]{};
 \node(b)[noshapenode,below=7mm of x3,label=below:\tiny $b$]{};
 \draw[->](x1)to[bend left=30](a);
 \draw(a)to[bend left=30](x2);
 \draw[->](x1)to[bend right=30](b);
 \draw(b)to[bend right=30](x2);
 \node(c)[noshapenode,right= of x2]{\tiny $=$};
 \node(x4)[circlenode,right=7mm of c,label=below:\tiny $x_1$]{};
 \node(d)[noshapenode,right=of x4,label=above:\tiny $a+b$]{};
 \draw[->](x4)--(d);
 \node(x5)[circlenode,right=of d,label=below:\tiny $x_2$]{};
 \draw(d)--(x5);
 \end{tikzpicture}
 \end{block}} 
 \end{frame}
 \begin{frame}{信号流图简化}
 \begin{block}{环路的消除}
 \begin{tikzpicture}
 [node distance=10mm,>=stealth]
  \node(x1)[circlenode,label=below:\tiny $x_1$]{};
 \node(a)[noshapenode,right=of x1,label=above:\tiny $a$]{};
 \draw[->](x1)--(a);
 \node(x2)[circlenode,right=of a,label=below:\tiny $x_2$]{};
 \draw(a)--(x2);
 \node(b)[noshapenode,right=of x2,label=above:\tiny $b$]{};
 \draw[->](x2)--(b);
 \node(x3)[circlenode,right=of b,label=below:\tiny $x_3$]{};
 \draw(b)--(x3);
 \node(e)[noshapenode,below=7mm of b,label=above:\tiny $c$]{};
 \draw[->](x3)to[bend left=25](e);
 \draw(x2)to[bend right=25](e);
 \node(c)[noshapenode,right=of x3]{\tiny $=$};
 \node(x4)[circlenode,right=of c,label=below:\tiny $x_1$]{};
 \node(d)[noshapenode,right=of x4,label=above:\tiny $ab$]{};
 \draw[->](x4)--(d);
 \node(x5)[circlenode,right=of d,label=below:\tiny $x_3$]{};
 \draw(d)--(x5);
 \node(f)[noshapenode,below=of x5,label=above:\tiny $bc$]{};
 \draw[->](x5)to[bend left=90](f);
 \draw(f)to[bend left=90](x5);
 \end{tikzpicture}
 \begin{tikzpicture}
 [node distance=10mm,>=stealth]
 \node(x1)[circlenode,label=below:\tiny $x_1$]{};
 \node(a)[noshapenode,right=of x1,label=above:\tiny $\frac{ab}{1-bc}$]{};
 \draw[->](x1)--(a);
 \node(x3)[circlenode,right=of a,label=below:\tiny $x_3$]{};
 \draw(a)--(x3);
 \node(b)[noshapenode,left=of x1]{\tiny $=$};
 \end{tikzpicture}
 \end{block}
 \end{frame}
 \begin{frame}{梅逊增益公式}
 \begin{block}{}
 \[P(s)=\frac{Y(s)}{X(s)}=\frac{\sum_{i=1}^n P_i(s)\Delta_i(s)}{\Delta(s)}\]
 $P_i(s)$为第$i$条前向通路的增益;
 \[\Delta=1-\sum_jL_j+\sum_{m,n}L_mL_n-\sum_{p,q,r}L_pL_qL_r+\cdots\]
 $L$为回路;\\
 $\Delta_i(s)$为第$i$条前向通路行列式的余因子,其值为消去与第$i$条通路相接触的回路后的行列式。
 \end{block}
 \end{frame}