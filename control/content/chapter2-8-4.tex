%%%%%%%%%%%%%%教案头%%%%%%%%%%%%%%%%%%%%%%%%%%%%%%%
\mode<article>{

\begin{longtable}{|m{20mm}|m{20mm}|m{20mm}|m{20mm}|m{20mm}|m{28mm}|}
\caption*{\huge 教案头}\\
\hline
\endfirsthead
\multicolumn{6}{l}{(续表)}\\
\hline
\endhead
\hline
\multicolumn{6}{l}{\itshape 接下一页表格.......}\\ [2ex]
\endfoot
\hline
\endlastfoot
\centering{授课单元}&\multicolumn{3}{m{60mm}|}{\centering 2.8.4频域性能指标2.8.5开环传递函数的频率特性2.8.7用MatLab进行频域分析}&\centering{授课日期}&2014年05月09日 \\
\hline
\centering 授课地点 & \multicolumn{3}{m{60mm}|}{B6-204}&\centering 授课学时 & 2 \\
\hline
& \multicolumn{2}{m{40mm}|}{能力目标} & \multicolumn{2}{m{40mm}|}{知识目标}&素质目标 \\
\cline{2-6}
\centering 教学目标&\multicolumn{2}{m{40mm}|}{\begin{enumerate}
\item 能够分析开环传递函数的频率特性
\item 能够用MatLab分析系统的频率特性
\end{enumerate} }&\multicolumn{2}{m{40mm}|}{\begin{enumerate}
\item 掌握频频域性能指标的概念
\item 掌握奈奎斯特判据
\end{enumerate}} & {\qquad}\\
\hline
\centering 能力训练任务或案例 &\multicolumn{5}{m{108mm}|}{ }\\
\hline
\centering 教学重点 & \multicolumn{5}{m{108mm}|}{\begin{enumerate}
\item 开环传递函数的频率特性
\item 用MatLab进行频率特性分析
\end{enumerate}}\\
\hline
\centering 教学难点与解决办法 &\multicolumn{5}{m{108mm}|}{\begin{enumerate}
\item 难点:奈奎斯判据
\item 解决方法:适当的例题演示
\end{enumerate}}\\
\hline
\centering 德育内容 &\multicolumn{5}{m{108mm}|}{无}\\
\hline
 &教材 & \multicolumn{4}{m{88mm}|}{计算机控制原理与应用}\\
\cline{2-6}& 教学资源 &\multicolumn{4}{m{88mm}|}{PPT}\\
\cline{2-6}\centering 使用的教学材料& 主要教学仪器设备和工具等 &\multicolumn{4}{m{88mm}|}{投影机、MATLAB}\\
\cline{2-6}& 主要耗材 &\multicolumn{4}{m{88mm}|}{无}\\
\hline
\centering 教学模式 &\multicolumn{2}{m{40mm}|}{知识讲授}&\centering 教学手段 &\multicolumn{2}{m{48mm}|}{多媒体教学}\\
\hline
\centering 学生成果与过程考核方式 &\multicolumn{5}{m{108mm}|}{无}
\end{longtable}
\clearpage

%%%%%%%%%%%%%%%教学实施过程%%%%%%%%%%%%%%%%%%%%%%%%%%%%
\begin{landscape}

\begin{longtable}{|m{10mm}|m{50mm}|m{50mm}|m{50mm}|m{15mm}|}
\caption*{\huge 教学组织与实施}\\
\hline
\endfirsthead
\multicolumn{5}{l}{\small 接上页}\\
\hline
\multicolumn{1}{|c|}{步骤}&\multicolumn{1}{c|}{教学内容}&\multicolumn{1}{c|}{教师活动}&\multicolumn{1}{c|}{学生活动}&\multicolumn{1}{c|}{时间}\\
\hline
\endhead

\multicolumn{5}{r}{\small 接下页}\\
\endfoot
\hline
\endlastfoot
\multicolumn{1}{|c|}{步骤}&\multicolumn{1}{c|}{教学内容}&\multicolumn{1}{c|}{教师活动}&\multicolumn{1}{c|}{学生活动}&\multicolumn{1}{c|}{时间}\\\hline
讲解&\begin{enumerate}
\item 频域性能指标
\end{enumerate} &\begin{enumerate}
\item 讲解频域性能指标
\end{enumerate} &\begin{enumerate}
\item 学生倾听并记录
\end{enumerate} &15 \\\hline
讲解&\begin{enumerate}
\item 开环传递函数的频率特性
\end{enumerate}
 &\begin{enumerate}
\item 讲解开环传递函数的频率特性
\end{enumerate} &\begin{enumerate}
\item 学生倾听并记录
\end{enumerate} &30 \\\hline
讲解&\begin{enumerate}
\item 用MatLab进行频率特性分析
\end{enumerate}
&\begin{enumerate}
\item 指导学生用MatLab进行频率特性分析
\end{enumerate} &\begin{enumerate}
\item 学生用MatLab实践频率特性分析
\end{enumerate} &40 \\\hline

\centering 本次课总结(评价)&总结本课程内容 &进行知识总结 &学生倾听 &5 \\\hline
\centering 学生学习笔记或工单等检查情况&\multicolumn{4}{m{165mm}|}{\quad}\\\hline
\centering 课后作业&\multicolumn{4}{m{165mm}|}{2-28,2-29}\\\hline
\centering 教学体会&\multicolumn{4}{m{165mm}|}{\quad}\\
\end{longtable}

\end{landscape}
\clearpage
%%%%%%%%%%%%%%%%%%%%板书设计%%%%%%%%%%%%%%%%%%%%%%%%%%
\begin{center}
{\huge 板书设计}
\end{center}
}
\mode<presentation>{ \section{频率响应的概念}
 \subsection{频率响应的概念}}
 \begin{frame}{频域性能指标} 
 \begin{block}{带宽($\omega_b$)}
 定义为频率$\omega=0$到系统的幅值响应是它的零频率值的0.707时的频率区间。
 \end{block}
 \begin{block}{截止速率}
当幅值减小,在超越带宽时的速率。
 \end{block}
 \begin{block}{谐振峰($M_r$)} 
 定义为系统阻尼比在$0<\zeta <0.707$时,幅值响应的最大值。
 \end{block}
 \end{frame}
 \begin{frame}
 \begin{block}{谐振频率$\omega_r$}
定义为阻尼比在$0<\zeta <0.707$时,谐振峰对应的频率。
 \end{block}
 \end{frame}
 
 \begin{frame}{开环传递函数的频率特性}
 \begin{block}{开环传递函数}
\begin{eqnarray*}
G_o(s)&=&G(s)H(s)\\
&=&\frac{K(1+\tau_1s)(1+\tau_2s)\cdots(1+\tau_ms)}{(1+T_1s)(1+T_2s)\cdots(1+T_ns)}
\end{eqnarray*}
因子式:
\[G_o(s)=G_1(s)G_2(s)\cdots G_n(s)\]
\end{block}
\end{frame}

\begin{frame}
\begin{block}{开环幅频特性}
 \begin{eqnarray*}
G(j\omega)=|G_1(j\omega)||G_2(j\omega)|\cdots |G_n(j\omega)|
\end{eqnarray*}
\end{block}
\begin{block}{开环相频特性}
 \begin{eqnarray*}
\varphi(\omega)=\varphi_1(\omega)+\varphi_2(\omega)+\cdots +\varphi_n(\omega)
\end{eqnarray*}
\end{block}
\end{frame}

\begin{frame}{系统的相对稳定性}
\begin{block}{增益穿越频率($\omega_g$)}
开环传递函数幅值为1时的频率。
 \begin{eqnarray*}
|G_o(j\omega_g)|=1
\end{eqnarray*}
\end{block}
\begin{block}{相位穿越频率($\omega_f$)}
开环传递函数相角为$-180^0$时的频率。
 \begin{eqnarray*}
\varphi(\omega_f)=-180^0
\end{eqnarray*}
\end{block}
\end{frame}

\begin{frame}
\begin{block}{增益裕度$G_m$}
定义为在相位穿越频率点,计算的开环传递函数幅值的倒数
 \begin{eqnarray*}
G_m=\frac{1}{|G_o(j\omega_f)|}
\end{eqnarray*}
\end{block}
\begin{block}{增益裕度$P_m$}
$180^0$加开环传递数在增益穿越频率点的相角
 \begin{eqnarray*}
P_m=180^o+P(\omega_g)
\end{eqnarray*}
\end{block}
\end{frame}

\begin{frame}
\begin{block}{系统稳定的条件}
相位裕度和增益裕度必须是正值。
\end{block}
\end{frame}

\begin{frame}
\begin{block}{奈奎斯特判据1}
单位负反馈系统,如果开环传递函数$G_o(s)$在右半$s$平面无极点,而开环传递函数的极坐标频率特性曲线$G_o(j\omega)$不包围$-1+j0$点,则系统是稳定的 。
\end{block}
\begin{block}{奈奎斯特判据2}
单位负反馈系统,如果开环传递函数$G_o(s)$在右半$s$平面e有$k$个极点,而开环传递函数的极坐标频率特性曲线$G_o(j\omega)$反时针包围$-1+j0$点$k$圈,则系统是稳定的。
\end{block}
\end{frame}