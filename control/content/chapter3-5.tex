%%%%%%%%%%%%%%教案头%%%%%%%%%%%%%%%%%%%%%%%%%%%%%%%
\mode<article>{

\begin{longtable}{|m{20mm}|m{20mm}|m{20mm}|m{20mm}|m{20mm}|m{28mm}|}
\caption*{\huge 教案头}\\
\hline
\endfirsthead
\multicolumn{6}{l}{(续表)}\\
\hline
\endhead
\hline
\multicolumn{6}{l}{\itshape 接下一页表格.......}\\ [2ex]
\endfoot
\hline
\endlastfoot
\centering{授课单元}&\multicolumn{3}{m{60mm}|}{\centering 3.5描述函数法}&\centering{授课日期}&2014年05月19日 \\
\hline
\centering 授课地点 & \multicolumn{3}{m{60mm}|}{B6-204}&\centering 授课学时 & 2 \\
\hline
& \multicolumn{2}{m{40mm}|}{能力目标} & \multicolumn{2}{m{40mm}|}{知识目标}&素质目标 \\
\cline{2-6}
\centering 教学目标&\multicolumn{2}{m{40mm}|}{\begin{enumerate}
\item  能够进行建立非线性控制系统的描述函数
\item  能够用描述函数法分析非线性系统的稳定性
\end{enumerate} }&\multicolumn{2}{m{40mm}|}{\begin{enumerate}
\item 了解描述函数法的基本概念
\item 了解典型非线性环节的描述函数
\item 了解非线性系统的稳定性分析
\end{enumerate}} & {\qquad}\\
\hline
\centering 能力训练任务或案例 &\multicolumn{5}{m{108mm}|}{ }\\
\hline
\centering 教学重点 & \multicolumn{5}{m{108mm}|}{\begin{enumerate}
\item 典型非线性环节的描述函数
\end{enumerate}}\\
\hline
\centering 教学难点与解决办法 &\multicolumn{5}{m{108mm}|}{\begin{enumerate}
\item 难点:用描述函数法分析非线性系统的稳定性
\item 解决方法:实例讲解
\end{enumerate}}\\
\hline
\centering 德育内容 &\multicolumn{5}{m{108mm}|}{无}\\
\hline
 &教材 & \multicolumn{4}{m{88mm}|}{计算机控制原理与应用}\\
\cline{2-6}& 教学资源 &\multicolumn{4}{m{88mm}|}{PPT}\\
\cline{2-6}\centering 使用的教学材料& 主要教学仪器设备和工具等 &\multicolumn{4}{m{88mm}|}{投影机、MATLAB}\\
\cline{2-6}& 主要耗材 &\multicolumn{4}{m{88mm}|}{无}\\
\hline
\centering 教学模式 &\multicolumn{2}{m{40mm}|}{知识讲授}&\centering 教学手段 &\multicolumn{2}{m{48mm}|}{多媒体教学}\\
\hline
\centering 学生成果与过程考核方式 &\multicolumn{5}{m{108mm}|}{无}
\end{longtable}
\clearpage

%%%%%%%%%%%%%%%教学实施过程%%%%%%%%%%%%%%%%%%%%%%%%%%%%
\begin{landscape}

\begin{longtable}{|m{10mm}|m{50mm}|m{50mm}|m{50mm}|m{15mm}|}
\caption*{\huge 教学组织与实施}\\
\hline
\endfirsthead
\multicolumn{5}{l}{\small 接上页}\\
\hline
\multicolumn{1}{|c|}{步骤}&\multicolumn{1}{c|}{教学内容}&\multicolumn{1}{c|}{教师活动}&\multicolumn{1}{c|}{学生活动}&\multicolumn{1}{c|}{时间}\\
\hline
\endhead

\multicolumn{5}{r}{\small 接下页}\\
\endfoot
\hline
\endlastfoot
\multicolumn{1}{|c|}{步骤}&\multicolumn{1}{c|}{教学内容}&\multicolumn{1}{c|}{教师活动}&\multicolumn{1}{c|}{学生活动}&\multicolumn{1}{c|}{时间}\\\hline
讲解&\begin{enumerate}
\item 描述函数法的基本概念
\end{enumerate} &\begin{enumerate}
\item 讲解描述函数法的基本概念
\end{enumerate} &\begin{enumerate}
\item 学生倾听并记录
\end{enumerate} &20\\\hline
讲解&\begin{enumerate}
\item 典型非线性环节的描述函数
\end{enumerate}
 &\begin{enumerate}
\item 讲解典型非线性环节的描述函数
\end{enumerate} &\begin{enumerate}
\item 学生倾听并记录
\end{enumerate} &25 \\\hline
讲解&\begin{enumerate}
\item 用描述函数法分析非线性系统的稳定性
\end{enumerate}
&\begin{enumerate}
\item 讲解用描述函数法分析非线性系统的稳定性
\end{enumerate} &\begin{enumerate}
\item 学生倾听并记录
\end{enumerate} &40 \\\hline

\centering 本次课总结(评价)&总结本课程内容 &进行知识总结 &学生倾听 &5 \\\hline
\centering 学生学习笔记或工单等检查情况&\multicolumn{4}{m{165mm}|}{\quad}\\\hline
\centering 课后作业&\multicolumn{4}{m{165mm}|}{2-28,2-29}\\\hline
\centering 教学体会&\multicolumn{4}{m{165mm}|}{\quad}\\
\end{longtable}

\end{landscape}
\clearpage
%%%%%%%%%%%%%%%%%%%%板书设计%%%%%%%%%%%%%%%%%%%%%%%%%%
\begin{center}
{\huge 板书设计}
\end{center}
}

 \begin{frame}{描述函数法的基本概念} 
 \begin{block}{应用函数描述法的基本条件}
非线性系统经过变换其结构可简化为只有一个非线性环节N和一个线性环节G(s)。
 \end{block}
 \begin{block}{输入信号}
 \[x(t)=X\sin \omega t\]
 \end{block}
 \end{frame}
 
 \begin{frame}
 \begin{block}{输出信号的傅里叶展开}
\begin{eqnarray*}
y(t)=A_0+\sum\limits_{n=1}^\infty(A_n\cos n\omega t+B_n\sin n\omega t)\\
A_n=\frac{1}{\pi}\int\limits_0^{2\pi}y(t)\cos n\omega t d(\omega t)\\
B_n=\frac{1}{\pi}\int\limits_0^{2\pi}y(t)\sin n\omega t d(\omega t)
\end{eqnarray*}
\end{block}
\end{frame}

\begin{frame}
\begin{block}{}
若非线性特性是中心对称的,则$A_0$=0;

若线性部分具有低通滤波特性,可用基波进行近似。
\[y(t)\approx y_1(t)=A_1\cos\omega t+B_1\sin\omega t\]
\[=Y_1\sin (\omega t+\varphi_1)\]
非线性环节的近似等效频率特性称为描述函数,即
\[N(X)=\frac{Y_1}{X}e^{j\varphi}=\frac{B_1}{X}+j\frac{A_1}{X}\]
\end{block}
\end{frame}

\begin{frame}{典型非线性环节的描述函数}
\begin{block}{饱和限幅特性的描述函数}
\begin{equation*}
y=\begin{cases}
kX\sin \omega t & 0\leq\omega t\leq \alpha\\
ka & \varphi\leq\omega t\leq \frac{\pi}{2}
\end{cases}
\end{equation*}
\end{block}
\begin{block}{输出具有奇对称性,只求$B_1$}
\begin{eqnarray*}
B_1=\frac{1}{\pi}\int_{0}^{2\pi}y(t)\sin \omega td(\omega t)\\
=\frac{2kX}{\pi}\left[\arcsin\frac{a}{X}+\frac{a}{X}\sqrt{1-(\frac{a}{X})^2}\right]
\end{eqnarray*}
\end{block}
\end{frame}

\begin{frame}
\begin{block}{饱和百线性的描述函数}
\begin{equation*}
N(x)=\frac{B_1}{X}=\frac{2k}{\pi}\left[\arcsin\frac{a}{X}+\frac{a}{X}\sqrt{1-(\frac{a}{X})^2}\right]
\end{equation*}
\end{block}
\end{frame}

\begin{frame}{继电器非线性特性}
\begin{block}{理想二位继电器}
\begin{equation*}
y=\begin{cases}
+M & 0\leq\omega t\leq \pi\\
-M & \pi\leq\omega t\leq 2\pi
\end{cases}
\end{equation*}
\end{block}
\begin{block}{输出具有奇对称性,只求$B_1$}
\begin{eqnarray*}
B_1=\frac{1}{\pi}\int_{0}^{2\pi}y(t)\sin \omega td(\omega t)\\
=\frac{4M}{\pi}
\end{eqnarray*}
\end{block}
\end{frame}

\begin{frame}{用描述函数法分析非线性系统的稳定性}
\begin{block}{闭环系统的频率特性}
\begin{equation*}
\phi(j\omega)=\frac{C(j\omega)}{R(j\omega)}=\frac{N(X)G(j\omega)}{1+N(X)G(j\omega)}
\end{equation*}
\end{block}
\begin{block}{特征方程}
\begin{equation*}
1+N(X)G(j\omega)=0
\end{equation*} 
\end{block}
\end{frame}

\begin{frame}
\begin{block}{稳定性判定}
\begin{itemize}
\item $G(j\omega)$曲线不包围$\frac{-1}{N(X)}$曲线时,闭环系统稳定
\item $G(j\omega)$曲线包围$\frac{-1}{N(X)}$曲线时,闭环系统不稳定
\item $G(j\omega)$曲线与$\frac{-1}{N(X)}$曲线相交,闭环系统临界稳定
\end{itemize}
\end{block}
\end{frame}
