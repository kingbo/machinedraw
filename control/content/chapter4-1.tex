%%%%%%%%%%%%%%教案头%%%%%%%%%%%%%%%%%%%%%%%%%%%%%%%
\mode<article>{

\begin{longtable}{|m{20mm}|m{20mm}|m{20mm}|m{20mm}|m{20mm}|m{28mm}|}
\caption*{\huge 教案头}\\
\hline
\endfirsthead
\multicolumn{6}{l}{(续表)}\\
\hline
\endhead
\hline
\multicolumn{6}{l}{\itshape 接下一页表格.......}\\ [2ex]
\endfoot
\hline
\endlastfoot
\centering{授课单元}&\multicolumn{3}{m{60mm}|}{\centering 4.1引言4.2差分方程}&\centering{授课日期}&2014年05月26日 \\
\hline
\centering 授课地点 & \multicolumn{3}{m{60mm}|}{B6-204}&\centering 授课学时 & 2 \\
\hline
& \multicolumn{2}{m{40mm}|}{能力目标} & \multicolumn{2}{m{40mm}|}{知识目标}&素质目标 \\
\cline{2-6}
\centering 教学目标&\multicolumn{2}{m{40mm}|}{\begin{enumerate}
\item  能够建立差分方程
\end{enumerate} }&\multicolumn{2}{m{40mm}|}{\begin{enumerate}
\item 了解离散时间系统的基本概念
\item 了解差分方程的基本概念
\end{enumerate}} & {\qquad}\\
\hline
\centering 能力训练任务或案例 &\multicolumn{5}{m{108mm}|}{ }\\
\hline
\centering 教学重点 & \multicolumn{5}{m{108mm}|}{\begin{enumerate}
\item 差分方程
\end{enumerate}}\\
\hline
\centering 教学难点与解决办法 &\multicolumn{5}{m{108mm}|}{\begin{enumerate}
\item 难点:差分方程 
\item 解决方法:实例讲解
\end{enumerate}}\\
\hline
\centering 德育内容 &\multicolumn{5}{m{108mm}|}{无}\\
\hline
 &教材 & \multicolumn{4}{m{88mm}|}{计算机控制原理与应用}\\
\cline{2-6}& 教学资源 &\multicolumn{4}{m{88mm}|}{PPT}\\
\cline{2-6}\centering 使用的教学材料& 主要教学仪器设备和工具等 &\multicolumn{4}{m{88mm}|}{投影机、MATLAB}\\
\cline{2-6}& 主要耗材 &\multicolumn{4}{m{88mm}|}{无}\\
\hline
\centering 教学模式 &\multicolumn{2}{m{40mm}|}{知识讲授}&\centering 教学手段 &\multicolumn{2}{m{48mm}|}{多媒体教学}\\
\hline
\centering 学生成果与过程考核方式 &\multicolumn{5}{m{108mm}|}{无}
\end{longtable}
\clearpage

%%%%%%%%%%%%%%%教学实施过程%%%%%%%%%%%%%%%%%%%%%%%%%%%%
\begin{landscape}

\begin{longtable}{|m{10mm}|m{50mm}|m{50mm}|m{50mm}|m{15mm}|}
\caption*{\huge 教学组织与实施}\\
\hline
\endfirsthead
\multicolumn{5}{l}{\small 接上页}\\
\hline
\multicolumn{1}{|c|}{步骤}&\multicolumn{1}{c|}{教学内容}&\multicolumn{1}{c|}{教师活动}&\multicolumn{1}{c|}{学生活动}&\multicolumn{1}{c|}{时间}\\
\hline
\endhead

\multicolumn{5}{r}{\small 接下页}\\
\endfoot
\hline
\endlastfoot
\multicolumn{1}{|c|}{步骤}&\multicolumn{1}{c|}{教学内容}&\multicolumn{1}{c|}{教师活动}&\multicolumn{1}{c|}{学生活动}&\multicolumn{1}{c|}{时间}\\\hline
讲解&\begin{enumerate}
\item 离散时间系统的基本概念
\end{enumerate} &\begin{enumerate}
\item 讲解离散时间系统的基本概念
\end{enumerate} &\begin{enumerate}
\item 学生倾听并记录
\end{enumerate} &45\\\hline
讲解&\begin{enumerate}
\item 差分方程
\end{enumerate}
 &\begin{enumerate}
\item 讲解差分方程的建立
\end{enumerate} &\begin{enumerate}
\item 学生倾听并记录
\end{enumerate} &40 \\\hline

\centering 本次课总结(评价)&总结本课程内容 &进行知识总结 &学生倾听 &5 \\\hline
\centering 学生学习笔记或工单等检查情况&\multicolumn{4}{m{165mm}|}{\quad}\\\hline
\centering 课后作业&\multicolumn{4}{m{165mm}|}{2-28,2-29}\\\hline
\centering 教学体会&\multicolumn{4}{m{165mm}|}{\quad}\\
\end{longtable}

\end{landscape}
\clearpage
%%%%%%%%%%%%%%%%%%%%板书设计%%%%%%%%%%%%%%%%%%%%%%%%%%
\begin{center}
{\huge 板书设计}
\end{center}
}

 \begin{frame}{第四章线性离散时间系统} 
 \begin{block}{离散时间信号}
 在时间上是离散的信号
 \end{block}
 \begin{block}{采样瞬时}
 离散信号只在某些不连续的瞬时给出函数值,其他时间没有定义
 \end{block}
 \begin{block}{数字信号}
 若离散信号在每个采样瞬时,仅是任一个有限的数字
 \end{block}
 \end{frame}
 
 \begin{frame}
 \begin{block}{采样数据信号}
 若离散信号在每个采样瞬间,取连续敬意中的任意一个值
 \end{block}
 \begin{block}{离散时间系统}
输入信号和输出信号都是离散的系统
 \end{block}
 \begin{block}{离散时间系统的函数表示} 
 \[y(k)=P[x(k)]\]
 P为变换算子。
 \end{block}
 \end{frame}

\begin{frame}{}
\begin{block}{线性离散系统}
变换算子是线性的
\end{block}
\begin{block}{非线性离散系统}
变换算子是非线性的
\end{block}
\end{frame}

\begin{frame}
\begin{block}{线性离散系统满足叠加原理}
若:
\begin{eqnarray*}
y_1(k)=P[x_1(k)]\\
y_2(k)=P[x_2(k)]
\end{eqnarray*}
当输入为:
\[x(k)=ax_1(k)+bx_2(k)\]
\end{block}
\end{frame}

\begin{frame}{}
\begin{block}{}
输出为:
\begin{eqnarray*}
y(k)&=&P[x(k)]\\
&=&P[ax_1(k)+bx_2(k)]\\
&=&aP[x_1(k)]+bP[x_2(k)]
\end{eqnarray*}
\end{block}
\end{frame}
\begin{frame}
\begin{block}{线性定常离散系统}
系统的输出与输入关系不随时间而改变,对时间移位输入信号的响应是时间移位输出信息号的线性系统。满足:
\[y(k-m)=P[x(k-m)]\] 
\end{block}
\end{frame}
\begin{frame}
\begin{block}{离散单位阶跃函数}
\begin{equation*}
u(kT)=u(k)=\begin{cases}
1,k\geq 0\\
0,k<0
\end{cases}
\end{equation*}
\end{block}
\begin{block}{离散延迟单位阶跃函数}
\begin{equation*}
u(k-j)=\begin{cases}
1,k\geq j\\
0,k<j
\end{cases}
\end{equation*}
\end{block}
\end{frame}

\begin{frame}
\begin{block}{离散单位脉冲函数}
\begin{equation*}
\delta(kT)=\delta(k)=\begin{cases}
1,k= 0\\
0,k\neq 0
\end{cases}
\end{equation*}
\end{block}
\begin{block}{离散延迟单位脉冲函数}
\begin{equation*}
\delta(k-j)=\begin{cases}
1,k= j\\
0,k\neq j
\end{cases}
\end{equation*}
\end{block}
\end{frame}

\begin{frame}
\begin{block}{离散单位斜坡函数}
\begin{equation*}
f(kT)=f(k)=\begin{cases}
k,k\geq 0\\
0,k<0
\end{cases}
\end{equation*}
\[f(k)=ku(k)\]
\end{block}
\begin{block}{离散延迟单位斜坡函数}
\begin{equation*}
f(k)=(k-j)u(k-j)
\end{equation*}
\end{block}
\end{frame}

\begin{frame}{4.2差分方程}
\begin{block}{}
\begin{itemize}
\item 微分方程表示连续输入与连续输出函数之间的关系。
\item 差分方程表示离散输入 与离散输出函数之间的关系。
\end{itemize}
\end{block}
\begin{block}{前向差分模型}
\begin{equation*}
y(k+n)=\sum\limits _{i=0}^nb_ix(k+n-i)-\sum\limits_{i=1}^na_i(k+n-i)
\end{equation*}
\end{block}
\end{frame}

\begin{frame}
\begin{block}{后向差分模型}
\begin{equation*}
y(k)=\sum\limits _{i=0}^nb_ix(k-i)-\sum\limits_{i=1}^na_i(k-i)
\end{equation*}
\end{block}
\end{frame}