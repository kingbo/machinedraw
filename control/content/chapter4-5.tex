%%%%%%%%%%%%%%教案头%%%%%%%%%%%%%%%%%%%%%%%%%%%%%%%
\mode<article>{

\begin{longtable}{|m{20mm}|m{20mm}|m{20mm}|m{20mm}|m{20mm}|m{28mm}|}
\caption*{\huge 教案头}\\
\hline
\endfirsthead
\multicolumn{6}{l}{(续表)}\\
\hline
\endhead
\hline
\multicolumn{6}{l}{\itshape 接下一页表格.......}\\ [2ex]
\endfoot
\hline
\endlastfoot
\centering{授课单元}&\multicolumn{3}{m{60mm}|}{\centering 4.5离散系统的传递函数和方框图}&\centering{授课日期}&2014年06月6日 \\
\hline
\centering 授课地点 & \multicolumn{3}{m{60mm}|}{B6-204}&\centering 授课学时 & 2 \\
\hline
& \multicolumn{2}{m{40mm}|}{能力目标} & \multicolumn{2}{m{40mm}|}{知识目标}&素质目标 \\
\cline{2-6}
\centering 教学目标&\multicolumn{2}{m{40mm}|}{\begin{enumerate}
\item  能够建立离散系统的传递函数
\end{enumerate} }&\multicolumn{2}{m{40mm}|}{\begin{enumerate}
\item 了解离散系统的传递函数
\item 了解离散系统的方框图
\end{enumerate}} & {\qquad}\\
\hline
\centering 能力训练任务或案例 &\multicolumn{5}{m{108mm}|}{ }\\
\hline
\centering 教学重点 & \multicolumn{5}{m{108mm}|}{\begin{enumerate}
\item 离散系统的传递函数
\item 离散系统的方框图
\end{enumerate}}\\
\hline
\centering 教学难点与解决办法 &\multicolumn{5}{m{108mm}|}{\begin{enumerate}
\item 难点:离散系统的传递函数
\item 解决方法:实例讲解
\end{enumerate}}\\
\hline
\centering 德育内容 &\multicolumn{5}{m{108mm}|}{无}\\
\hline
 &教材 & \multicolumn{4}{m{88mm}|}{计算机控制原理与应用}\\
\cline{2-6}& 教学资源 &\multicolumn{4}{m{88mm}|}{PPT}\\
\cline{2-6}\centering 使用的教学材料& 主要教学仪器设备和工具等 &\multicolumn{4}{m{88mm}|}{投影机、MATLAB}\\
\cline{2-6}& 主要耗材 &\multicolumn{4}{m{88mm}|}{无}\\
\hline
\centering 教学模式 &\multicolumn{2}{m{40mm}|}{知识讲授}&\centering 教学手段 &\multicolumn{2}{m{48mm}|}{多媒体教学}\\
\hline
\centering 学生成果与过程考核方式 &\multicolumn{5}{m{108mm}|}{无}
\end{longtable}
\clearpage

%%%%%%%%%%%%%%%教学实施过程%%%%%%%%%%%%%%%%%%%%%%%%%%%%
\begin{landscape}

\begin{longtable}{|m{10mm}|m{50mm}|m{50mm}|m{50mm}|m{15mm}|}
\caption*{\huge 教学组织与实施}\\
\hline
\endfirsthead
\multicolumn{5}{l}{\small 接上页}\\
\hline
\multicolumn{1}{|c|}{步骤}&\multicolumn{1}{c|}{教学内容}&\multicolumn{1}{c|}{教师活动}&\multicolumn{1}{c|}{学生活动}&\multicolumn{1}{c|}{时间}\\
\hline
\endhead

\multicolumn{5}{r}{\small 接下页}\\
\endfoot
\hline
\endlastfoot
\multicolumn{1}{|c|}{步骤}&\multicolumn{1}{c|}{教学内容}&\multicolumn{1}{c|}{教师活动}&\multicolumn{1}{c|}{学生活动}&\multicolumn{1}{c|}{时间}\\\hline
讲解&\begin{enumerate}
\item Z传递函数
\end{enumerate} &\begin{enumerate}
\item 讲解Z传递函数
\end{enumerate} &\begin{enumerate}
\item 学生倾听并记录
\end{enumerate} &25\\\hline
讲解&\begin{enumerate}
\item Z传递函数与差分方程
\end{enumerate}
 &\begin{enumerate}
\item 讲解Z传递函数与差分方程
\end{enumerate} &\begin{enumerate}
\item 学生倾听并记录
\end{enumerate} &20 \\\hline
讲解&\begin{enumerate}
\item Z采样系统的传递函数
\end{enumerate}
&\begin{enumerate}
\item 讲解Z变换定理
\end{enumerate} &\begin{enumerate}
\item 学生倾听并记录
\end{enumerate} &40 \\\hline

\centering 本次课总结(评价)&总结本课程内容 &进行知识总结 &学生倾听 &5 \\\hline
\centering 学生学习笔记或工单等检查情况&\multicolumn{4}{m{165mm}|}{\quad}\\\hline
\centering 课后作业&\multicolumn{4}{m{165mm}|}{2-28,2-29}\\\hline
\centering 教学体会&\multicolumn{4}{m{165mm}|}{\quad}\\
\end{longtable}

\end{landscape}
\clearpage
%%%%%%%%%%%%%%%%%%%%板书设计%%%%%%%%%%%%%%%%%%%%%%%%%%
\begin{center}
{\huge 板书设计}
\end{center}
}

 \begin{frame}{4.5离散系统的传递函数和方框图} 
 \begin{block}{Z传递函数}
 \begin{eqnarray*}
&& G(z)=\frac{Y(z)}{X(z)}\\
 &&=\frac{b_0+b_1z^{-1}+\cdots +b_mz^{-m}}{1+a_1z^{-1}+\cdots +a_nz^{-n}}
 \end{eqnarray*}
 \end{block}
 \end{frame}
 
 \begin{frame}
 \begin{block}{差分方程}
 令$m=n$
\begin{eqnarray*}
y(k)+a_1y(k-1)+\cdots +a_ny(k-n)\\
=b_0x(k)+b_1x(k-1)+\cdots +b_nx(k-n)
\end{eqnarray*}
\end{block}
\begin{block}{采样数据系统的Z传递函数}
\begin{eqnarray*}
G(s)=G_h(s)G_p(s)\\
G_h(s)=\frac{1-e^{-Ts}}{s}
\end{eqnarray*}
\end{block}
\end{frame}

\begin{frame}
\begin{block}{系统的输出序列}
\[y(kT)=g(kT)*x(kT)=\sum_{j=0}^kg(jT)x(kT-jT)\]
由卷积定理得:
\[Y(z)=G(Z)X(Z)\]
\end{block}
\end{frame}

\begin{frame}
\begin{block}{可将$X^*$从离散符号中提出来}
\begin{equation*}
[G(s)X^*(s)]^*=G^*(s)X^*(s)
\end{equation*}
\end{block}
\begin{block}{若$G(s)=G_1(s)G_2(s)$}
\begin{equation*}
[G_1(s)G_2(s)X^*(s)]^*=[G_1(s)G_2(s)]^*X^*(s)
\end{equation*}
\end{block}
\end{frame}

\begin{frame}
\begin{block}{求Z传递函数的一般步骤}
\begin{itemize}
\item 求出连续部分的传递函数$G(s)$
\item 求出它的单位脉冲响应函数$g(t)=L^{-1}[G(s)]$
\item 求出单位脉冲函数$g(t)$的采样函数$g^*(t)$或$g(kT)$的z变换。
\end{itemize}
\end{block}
\end{frame}

\begin{frame}{4.5.2开环z传递函数}
\begin{block}{串联环节的z传递函数}
\begin{eqnarray*}
G(z)=\pounds[G_1(s)G_2(s)]=G_1(z)G_2(s)
\end{eqnarray*}
\end{block}
\begin{block}{并联环节的z传递函数}
\begin{eqnarray*}
G(z)=\pounds[G_1(s)]+\pounds[G_2(s)]=G_1(z)+G_2(z)
\end{eqnarray*}
\end{block}
\end{frame}

