%%%%%%%%%%%%%%教案头%%%%%%%%%%%%%%%%%%%%%%%%%%%%%%%
\mode<article>{

\begin{longtable}{|m{20mm}|m{20mm}|m{20mm}|m{20mm}|m{20mm}|m{28mm}|}
\caption*{\huge 教案头}\\
\hline
\endfirsthead
\multicolumn{6}{l}{(续表)}\\
\hline
\endhead
\hline
\multicolumn{6}{l}{\itshape 接下一页表格.......}\\ [2ex]
\endfoot
\hline
\endlastfoot
\centering{授课单元}&\multicolumn{3}{m{60mm}|}{\centering 2.6.8稳定性分析}&\centering{授课日期}&2014年04月25日 \\
\hline
\centering 授课地点 & \multicolumn{3}{m{60mm}|}{B6-204}&\centering 授课学时 & 2 \\
\hline
& \multicolumn{2}{m{40mm}|}{能力目标} & \multicolumn{2}{m{40mm}|}{知识目标}&素质目标 \\
\cline{2-6}
\centering 教学目标&\multicolumn{2}{m{40mm}|}{\begin{enumerate}
\item 能够分析系统的稳定性
\end{enumerate} }&\multicolumn{2}{m{40mm}|}{\begin{enumerate}
\item 了解系统的极点、零点和稳定性的概念
\item 掌握劳斯稳定性判据
\end{enumerate}} & {\qquad}\\
\hline
\centering 能力训练任务或案例 &\multicolumn{5}{m{108mm}|}{ }\\
\hline
\centering 教学重点 & \multicolumn{5}{m{108mm}|}{\begin{enumerate}
\item 劳斯稳定性判据
\end{enumerate}}\\
\hline
\centering 教学难点与解决办法 &\multicolumn{5}{m{108mm}|}{\begin{enumerate}
\item 难点:劳斯稳定性判据
\item 解决方法:实例讲解
\end{enumerate}}\\
\hline
\centering 德育内容 &\multicolumn{5}{m{108mm}|}{无}\\
\hline
 &教材 & \multicolumn{4}{m{88mm}|}{计算机控制原理与应用}\\
\cline{2-6}& 教学资源 &\multicolumn{4}{m{88mm}|}{PPT}\\
\cline{2-6}\centering 使用的教学材料& 主要教学仪器设备和工具等 &\multicolumn{4}{m{88mm}|}{投影机、MATLAB}\\
\cline{2-6}& 主要耗材 &\multicolumn{4}{m{88mm}|}{无}\\
\hline
\centering 教学模式 &\multicolumn{2}{m{40mm}|}{知识讲授}&\centering 教学手段 &\multicolumn{2}{m{48mm}|}{多媒体教学}\\
\hline
\centering 学生成果与过程考核方式 &\multicolumn{5}{m{108mm}|}{无}
\end{longtable}
\clearpage

%%%%%%%%%%%%%%%教学实施过程%%%%%%%%%%%%%%%%%%%%%%%%%%%%
\begin{landscape}

\begin{longtable}{|m{10mm}|m{50mm}|m{50mm}|m{50mm}|m{15mm}|}
\caption*{\huge 教学组织与实施}\\
\hline
\endfirsthead
\multicolumn{5}{l}{\small 接上页}\\
\hline
\multicolumn{1}{|c|}{步骤}&\multicolumn{1}{c|}{教学内容}&\multicolumn{1}{c|}{教师活动}&\multicolumn{1}{c|}{学生活动}&\multicolumn{1}{c|}{时间}\\
\hline
\endhead

\multicolumn{5}{r}{\small 接下页}\\
\endfoot
\hline
\endlastfoot
\multicolumn{1}{|c|}{步骤}&\multicolumn{1}{c|}{教学内容}&\multicolumn{1}{c|}{教师活动}&\multicolumn{1}{c|}{学生活动}&\multicolumn{1}{c|}{时间}\\\hline
讲解&\begin{enumerate}
\item 极点、零点和稳定性的概念
\end{enumerate} &\begin{enumerate}
\item 讲解极点、零点和稳定性的概念
\end{enumerate} &\begin{enumerate}
\item 学生倾听并记录
\end{enumerate} &15 \\\hline
讲解&\begin{enumerate}
\item 稳定性判定的数学理论
\end{enumerate}
 &\begin{enumerate}
\item 讲解稳定性判定的数学理论
\end{enumerate} &\begin{enumerate}
\item 学生倾听并记录
\end{enumerate} &30 \\\hline
讲解&\begin{enumerate}
\item 劳斯稳定性判据
\end{enumerate}
&\begin{enumerate}
\item 讲解劳斯稳定性判据
\end{enumerate} &\begin{enumerate}
\item 学生倾听并记录
\end{enumerate} &20 \\\hline
讲解&\begin{enumerate}
\item 劳斯稳定性判据的应用实例
\end{enumerate}
 &\begin{enumerate}
\item 讲解劳斯稳定性判据的应用实例
\end{enumerate} &\begin{enumerate}
\item 学生记录笔记
\end{enumerate} &20 \\\hline

\centering 本次课总结(评价)&总结本课程内容 &进行知识总结 &学生倾听 &5 \\\hline
\centering 学生学习笔记或工单等检查情况&\multicolumn{4}{m{165mm}|}{\quad}\\\hline
\centering 课后作业&\multicolumn{4}{m{165mm}|}{2-28,2-29}\\\hline
\centering 教学体会&\multicolumn{4}{m{165mm}|}{\quad}\\
\end{longtable}

\end{landscape}
\clearpage
%%%%%%%%%%%%%%%%%%%%板书设计%%%%%%%%%%%%%%%%%%%%%%%%%%
\begin{center}
{\huge 板书设计}
\end{center}
}
\mode<presentation>{ \section{稳定性分析}
 \subsection{稳定性分析}}
 \begin{frame}{极点、零点和稳定性}
 \begin{block}{$n$阶传递函数的因子形式}
 \begin{eqnarray*}
 \Phi(s)&=&\frac{C(s)}{R(s)}\\
 &=&\frac{K\prod\limits_{i=1}^m(s-z_i)}{\prod\limits_{j=1}^{n_1}(s-p_j)\prod\limits_{l=1}^{n_2}(s^2+2\zeta_l\omega_ls+\omega_l^2)}
 \end{eqnarray*}
 \end{block}
 \end{frame}
 
 \begin{frame}
 \begin{block}{极点、零点的定义}
 设:
 \begin{eqnarray*}
 &&s^2+2\zeta_l\omega_l+\omega_l^2=(s-p_{1l})(s-p_{2l})\\
 &&s_{1l,2l}=-\zeta_l\omega_l\pm j\omega_l\sqrt{1-\zeta_l^2}\\
 &&l=1,2,\cdots ,n_2
\end{eqnarray*}
\begin{itemize}
\item $z_1,z_2,\cdots,z_m$称为零点
\item $p_{11},p_{12},\cdots,p_{1n_2},p_{21},p_{22},\cdots,p_{2n_2}$和\\
$p_1,p_2,\cdots,p_{n_1}$称为极点
\end{itemize}  
\end{block}
\end{frame}

\begin{frame}
\begin{block}{稳定性}
\begin{itemize}
\item 稳定系统
$\lim\limits_{t\rightarrow \infty}g(t)=0$
\item 不稳定系统
$\lim\limits_{t\rightarrow \infty}g(t)=\infty$
\item 临界稳定系统
$\lim\limits_{t\rightarrow \infty}g(t)=P$
\end{itemize}
\end{block}
\end{frame}
\begin{frame}
\begin{block}{单位脉冲输出部分展开式}
\begin{eqnarray*}
\Phi(s)=\frac{K\prod\limits_{i=1}^m(s-z_i)}{\prod\limits_{j=1}^{n_1}(s-p_j)\prod\limits_{i=1}^{n_2}(s^2+2\zeta_l\omega_ls+\omega_l^2)}\\
=\sum_{j=1}^{n_1}\frac{a_j}{s-p_j}+\sum_{l=1}^{n_2}\frac{\beta_ls+r_l}{(s-p_{1l})(s-p_{2l})}
\end{eqnarray*}
其中:$a_0,a_j,\beta_l,r_l$为待定系数
\end{block}
\end{frame}

\begin{frame}
\begin{block}{拉氏变换}
\begin{eqnarray*}
g(t)&=&L^{-1}[C(s)]=L^{-1}[\Phi(s)]\\
&=&\sum_{j=1}^{n_1}a_je^{p_jt}+\\
&&\sum_{l=1}^{n_1}\beta_le^{-\zeta_l\omega_lt}\cos\omega_l\sqrt{1-\zeta^2_l}t+\\
&&\sum_{l=1}^{n_2}\frac{r_l-\beta_l\zeta_l\omega_l}{\omega_l\sqrt{1-\zeta_l^2}}e^{-\zeta_l\omega_lt}\sin\omega_l\sqrt{1-\zeta_l^2}t
\end{eqnarray*}
\end{block}
\end{frame}

\begin{frame}{劳斯稳定性判据}
\begin{block}{$n$阶线性定常系统的特征方程}
\[a_0s^n+a_1+s^{n-1}+\cdots+a_{n-1}s+a_n=0\]
\end{block}
\end{frame}
\begin{frame}
\begin{block}{劳斯表}
\begin{equation*}
\begin{array}{cccccc}
s^n&a_0&a_2&a_4&a_6&\cdots\\
s^{n-1}&a_1&a_3&a_5&a_7&\cdots\\
S^{n-2}&T_{3,1}&T_{3,2}&T_{3,3}&T_{3,4}&\cdots\\
S^{n-3}&T_{4,1}&T_{4,2}&T_{4,3}&T_{4,4}&\cdots\\
\cdots&\cdots&\cdots&\cdots&\cdots&\cdots\\
s^2&T_{n-1,1}&T_{n-1,2}\\
s^1&T_{n,1}\\
s^0&T_{n+1,1}
\end{array}
\end{equation*}
\end{block}
\end{frame}

\begin{frame}
\begin{block}{}
\begin{eqnarray*}
T_{3,1}=-\frac{\left|\begin{array}{cc}
a_0&a_2\\
a_1&a_3
\end{array}\right|}{a_1}=\frac{a_1a_2-a_0a_3}{a_1}\\
T_{3,2}=-\frac{\left|\begin{array}{cc}
a_0&a_4\\
a_1&a_5
\end{array}\right|}{a_1}=\frac{a_1a_4-a_0a_5}{a_1}\\
T_{3,3}=-\frac{\left|\begin{array}{cc}
a_0&a_6\\
a_1&a_7
\end{array}\right|}{a_1}=\frac{a_1a_6-a_0a_7}{a_1}
\end{eqnarray*}
\end{block}
\end{frame}
\begin{frame}
\begin{block}{}
\begin{eqnarray*}
T_{4,1}=-\frac{\left|\begin{array}{cc}
a_1&a_3\\
T_{3,1}&T_{3,2}
\end{array}\right|}{T_{3,1}}=\frac{T_{3,1}a_3-T_{3,2}a_1}{T_{3,1}}\\
T_{4,2}=-\frac{\left|\begin{array}{cc}
a_1&a_5\\
T_{3,1}&T_{3,3}
\end{array}\right|}{T_{3,1}}=\frac{T_{3,1}a_5-a_1T_{3,3}}{T_{3,1}}\\
\vdots
\end{eqnarray*}
\end{block}
\end{frame}

\begin{frame}
\begin{block}{}
\begin{eqnarray*}
T_{5,1}=-\frac{\left|\begin{array}{cc}
T_{3,1}&T_{3,2}\\
T_{4,1}&T_{4,2}
\end{array}\right|}{T_{4,1}}=\frac{T_{4,1}T_{3,2}-T_{3,1}T_{4,2}}{T_{4,1}}\\
T_{5,2}=-\frac{\left|\begin{array}{cc}
T_{3,1}&T_{3,3}\\
T_{4,1}&T_{4,3}
\end{array}\right|}{T_{4,1}}=\frac{T_{4,1}T_{3,3}-T_{3,1}T_{4,3}}{T_{4,1}}\\
T_{5,3}=-\frac{\left|\begin{array}{cc}
T_{3,1}&T_{3,4}\\
T_{4,1}&T_{4,4}
\end{array}\right|}{T_{4,1}}=\frac{T_{4,1}T_{3,4}-T_{3,1}T_{4,4}}{T_{4,1}}
\end{eqnarray*}
\end{block}
\end{frame}

\endinput