%%%%%%%%%%%%%%教案头%%%%%%%%%%%%%%%%%%%%%%%%%%%%%%%
\mode<article>{

\begin{longtable}{|m{20mm}|m{20mm}|m{20mm}|m{20mm}|m{20mm}|m{28mm}|}
\caption*{\huge 教案头}\\
\hline
\endfirsthead
\multicolumn{6}{l}{(续表)}\\
\hline
\endhead
\hline
\multicolumn{6}{l}{\itshape 接下一页表格.......}\\ [2ex]
\endfoot
\hline
\endlastfoot
\centering{授课单元}&\multicolumn{3}{m{60mm}|}{\centering 第3章非线性控制系统,3.1非线性数学模型的线性化3.2控制系统中的典型非线性化3.3非线性系统的特性,非线性系统的分析方法}&\centering{授课日期}&2014年05月16日 \\
\hline
\centering 授课地点 & \multicolumn{3}{m{60mm}|}{B6-204}&\centering 授课学时 & 2 \\
\hline
& \multicolumn{2}{m{40mm}|}{能力目标} & \multicolumn{2}{m{40mm}|}{知识目标}&素质目标 \\
\cline{2-6}
\centering 教学目标&\multicolumn{2}{m{40mm}|}{\begin{enumerate}
\item  
\end{enumerate} }&\multicolumn{2}{m{40mm}|}{\begin{enumerate}
\item 了解非线性数学模型的线性化
\item 了解控制系统跌典型非线性牲
\item 了解非线性系统的特性
\item 了解非线性系统的分析方法
\end{enumerate}} & {\qquad}\\
\hline
\centering 能力训练任务或案例 &\multicolumn{5}{m{108mm}|}{ }\\
\hline
\centering 教学重点 & \multicolumn{5}{m{108mm}|}{\begin{enumerate}
\item 非线性数学模型的线性化
\end{enumerate}}\\
\hline
\centering 教学难点与解决办法 &\multicolumn{5}{m{108mm}|}{\begin{enumerate}
\item 难点:非线性数学模型的线性化
\item 解决方法:讲解
\end{enumerate}}\\
\hline
\centering 德育内容 &\multicolumn{5}{m{108mm}|}{无}\\
\hline
 &教材 & \multicolumn{4}{m{88mm}|}{计算机控制原理与应用}\\
\cline{2-6}& 教学资源 &\multicolumn{4}{m{88mm}|}{PPT}\\
\cline{2-6}\centering 使用的教学材料& 主要教学仪器设备和工具等 &\multicolumn{4}{m{88mm}|}{投影机、MATLAB}\\
\cline{2-6}& 主要耗材 &\multicolumn{4}{m{88mm}|}{无}\\
\hline
\centering 教学模式 &\multicolumn{2}{m{40mm}|}{知识讲授}&\centering 教学手段 &\multicolumn{2}{m{48mm}|}{多媒体教学}\\
\hline
\centering 学生成果与过程考核方式 &\multicolumn{5}{m{108mm}|}{无}
\end{longtable}
\clearpage

%%%%%%%%%%%%%%%教学实施过程%%%%%%%%%%%%%%%%%%%%%%%%%%%%
\begin{landscape}

\begin{longtable}{|m{10mm}|m{50mm}|m{50mm}|m{50mm}|m{15mm}|}
\caption*{\huge 教学组织与实施}\\
\hline
\endfirsthead
\multicolumn{5}{l}{\small 接上页}\\
\hline
\multicolumn{1}{|c|}{步骤}&\multicolumn{1}{c|}{教学内容}&\multicolumn{1}{c|}{教师活动}&\multicolumn{1}{c|}{学生活动}&\multicolumn{1}{c|}{时间}\\
\hline
\endhead

\multicolumn{5}{r}{\small 接下页}\\
\endfoot
\hline
\endlastfoot
\multicolumn{1}{|c|}{步骤}&\multicolumn{1}{c|}{教学内容}&\multicolumn{1}{c|}{教师活动}&\multicolumn{1}{c|}{学生活动}&\multicolumn{1}{c|}{时间}\\\hline
讲解&\begin{enumerate}
\item 非线性数学模型的线性化
\end{enumerate} &\begin{enumerate}
\item 讲解非线性数学模型的线性化
\end{enumerate} &\begin{enumerate}
\item 学生倾听并记录
\end{enumerate} &20\\\hline
讲解&\begin{enumerate}
\item 控制系统中的典型非线性特性
\end{enumerate}
 &\begin{enumerate}
\item 讲解控制系统中的典型非线性特性
\end{enumerate} &\begin{enumerate}
\item 学生倾听并记录
\end{enumerate} &25 \\\hline
讲解&\begin{enumerate}
\item 非线性系统的特性
\end{enumerate}
&\begin{enumerate}
\item 讲解非线性系统的特性
\end{enumerate} &\begin{enumerate}
\item 学生倾听并记录
\end{enumerate} &20 \\\hline
讲解&\begin{enumerate}
\item 非线性系统的分析方法
\end{enumerate}
&\begin{enumerate}
\item 讲解非线性系统的分析方法
\end{enumerate} &\begin{enumerate}
\item 学生倾听并记录
\end{enumerate} &15 \\\hline

\centering 本次课总结(评价)&总结本课程内容 &进行知识总结 &学生倾听 &5 \\\hline
\centering 学生学习笔记或工单等检查情况&\multicolumn{4}{m{165mm}|}{\quad}\\\hline
\centering 课后作业&\multicolumn{4}{m{165mm}|}{2-28,2-29}\\\hline
\centering 教学体会&\multicolumn{4}{m{165mm}|}{\quad}\\
\end{longtable}

\end{landscape}
\clearpage
%%%%%%%%%%%%%%%%%%%%板书设计%%%%%%%%%%%%%%%%%%%%%%%%%%
\begin{center}
{\huge 板书设计}
\end{center}
}

 \begin{frame}{非线性控制系统} 
 \begin{block}{非线性数学模型的线性化}
 存在连续变化的非线性函数:
 \[ y=f(x)\]
 作微元
 \begin{eqnarray*}
 x=x_0+\Delta x\\
 y=y_0+\Delta y
 \end{eqnarray*}
 \end{block}
 \end{frame}
 
 \begin{frame}
 \begin{block}{}
 设函数$y=f(x)$在$(x_0,y_0)$处连续可微,则作泰勒展开
\begin{eqnarray*}
y=f(x)=f(x_0)+f'(x_0)(x-x_0)+\\
\frac{1}{2!}f"(x_0)(x-x_0)^2+\frac{1}{3!}f^{(3)}(x_0)(x-x_0)^3+\cdots
\end{eqnarray*}
\end{block}
\end{frame}

\begin{frame}
\begin{block}{}
由$y_0=f(x_0)$,并略去高阶得:
\[y-y_0=f'(x_0)(x-x_0)\]
该方程称为非线性化方程的线性化方程。
\end{block}
\end{frame}

\begin{frame}{控制系统中的典型非线性特性}
\begin{block}{饱和非线性特性}
\begin{equation*}
y=\begin{cases}
M & x>a\\
kx & |x|\leq a\\
-M & x<-a
\end{cases}
\end{equation*}
\end{block}
\begin{block}{死区非线特性}
\begin{equation*}
y=\begin{cases}
k(x-a) & x>a\\
0 & |x|\leq a\\
k(x+a) & x<-a
\end{cases}
\end{equation*}
\end{block}
\end{frame}

\begin{frame}
\begin{block}{间隙非线性特性}
\begin{equation*}
y=\begin{cases}
k(x-a) \\
0 \\
k(x+a) \\
(x-\frac{y}{k})=a \\
-a<(x-\frac{y}{k} )<a\\
(x-\frac{y}{k})=-a
\end{cases}
\end{equation*}
\end{block}
\end{frame}

\begin{frame}{继电器非线性特性}
\begin{block}{理想二位继电器}
\begin{equation*}
y=\begin{cases}
+M & x>0\\
-M & x<0
\end{cases}
\end{equation*}
\end{block}
\begin{block}{理想三位继电器}
\begin{equation*}
y=\begin{cases}
+M & x>a\\
0 & -a<x< a\\
-M & x<-a
\end{cases}
\end{equation*}
\end{block}
\end{frame}

\begin{frame}{非线性系统的特性}
\begin{block}{}
\begin{itemize}
\item 叠加原理不成立
\item 稳定性问题
\item 极限环
\item 畸变现象
\end{itemize}
\end{block}
\end{frame}

\begin{frame}{非线性系统的分析方法}
\begin{itemize}
\item 描述函数法
\item 相平面法
\item 李雅普诺夫方法
\item 反馈线性化方法
\item 计算机仿真
\end{itemize}
\end{frame}
