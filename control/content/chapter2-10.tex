%%%%%%%%%%%%%%教案头%%%%%%%%%%%%%%%%%%%%%%%%%%%%%%%
\mode<article>{

\begin{longtable}{|m{20mm}|m{20mm}|m{20mm}|m{20mm}|m{20mm}|m{28mm}|}
\caption*{\huge 教案头}\\
\hline
\endfirsthead
\multicolumn{6}{l}{(续表)}\\
\hline
\endhead
\hline
\multicolumn{6}{l}{\itshape 接下一页表格.......}\\ [2ex]
\endfoot
\hline
\endlastfoot
\centering{授课单元}&\multicolumn{3}{m{60mm}|}{\centering 2.10控制系统设计}&\centering{授课日期}&2014年05月12日 \\
\hline
\centering 授课地点 & \multicolumn{3}{m{60mm}|}{B6-204}&\centering 授课学时 & 2 \\
\hline
& \multicolumn{2}{m{40mm}|}{能力目标} & \multicolumn{2}{m{40mm}|}{知识目标}&素质目标 \\
\cline{2-6}
\centering 教学目标&\multicolumn{2}{m{40mm}|}{\begin{enumerate}
\item 能够进行系统根轨迹校正
\item 能够进行系统频率校正
\end{enumerate} }&\multicolumn{2}{m{40mm}|}{\begin{enumerate}
\item 掌握校正装置的结构
\item 掌握校正装置的特性
\item 掌握根轨迹校正的方法
\item 掌握频率校正的方法
\end{enumerate}} & {\qquad}\\
\hline
\centering 能力训练任务或案例 &\multicolumn{5}{m{108mm}|}{ }\\
\hline
\centering 教学重点 & \multicolumn{5}{m{108mm}|}{\begin{enumerate}
\item 根轨迹的校正
\item 频率校正
\end{enumerate}}\\
\hline
\centering 教学难点与解决办法 &\multicolumn{5}{m{108mm}|}{\begin{enumerate}
\item 难点:校正装置的特性
\item 解决方法:实例讲解
\end{enumerate}}\\
\hline
\centering 德育内容 &\multicolumn{5}{m{108mm}|}{无}\\
\hline
 &教材 & \multicolumn{4}{m{88mm}|}{计算机控制原理与应用}\\
\cline{2-6}& 教学资源 &\multicolumn{4}{m{88mm}|}{PPT}\\
\cline{2-6}\centering 使用的教学材料& 主要教学仪器设备和工具等 &\multicolumn{4}{m{88mm}|}{投影机、MATLAB}\\
\cline{2-6}& 主要耗材 &\multicolumn{4}{m{88mm}|}{无}\\
\hline
\centering 教学模式 &\multicolumn{2}{m{40mm}|}{知识讲授}&\centering 教学手段 &\multicolumn{2}{m{48mm}|}{多媒体教学}\\
\hline
\centering 学生成果与过程考核方式 &\multicolumn{5}{m{108mm}|}{无}
\end{longtable}
\clearpage

%%%%%%%%%%%%%%%教学实施过程%%%%%%%%%%%%%%%%%%%%%%%%%%%%
\begin{landscape}

\begin{longtable}{|m{10mm}|m{50mm}|m{50mm}|m{50mm}|m{15mm}|}
\caption*{\huge 教学组织与实施}\\
\hline
\endfirsthead
\multicolumn{5}{l}{\small 接上页}\\
\hline
\multicolumn{1}{|c|}{步骤}&\multicolumn{1}{c|}{教学内容}&\multicolumn{1}{c|}{教师活动}&\multicolumn{1}{c|}{学生活动}&\multicolumn{1}{c|}{时间}\\
\hline
\endhead

\multicolumn{5}{r}{\small 接下页}\\
\endfoot
\hline
\endlastfoot
\multicolumn{1}{|c|}{步骤}&\multicolumn{1}{c|}{教学内容}&\multicolumn{1}{c|}{教师活动}&\multicolumn{1}{c|}{学生活动}&\multicolumn{1}{c|}{时间}\\\hline
讲解&\begin{enumerate}
\item 校正装置的结构
\end{enumerate} &\begin{enumerate}
\item 讲解校正装置的结构
\end{enumerate} &\begin{enumerate}
\item 学生倾听并记录
\end{enumerate} &10 \\\hline
讲解&\begin{enumerate}
\item 超前校正装置
\end{enumerate}
 &\begin{enumerate}
\item 讲解超前校正装置
\end{enumerate} &\begin{enumerate}
\item 学生倾听并记录
\end{enumerate} &10 \\\hline
讲解&\begin{enumerate}
\item 滞后校正装置
\end{enumerate}
&\begin{enumerate}
\item 讲解滞后装置
\end{enumerate} &\begin{enumerate}
\item 学生倾听并记录
\end{enumerate} &10 \\\hline
讲解&\begin{enumerate}
\item 滞后-超前校正装置
\end{enumerate}
&\begin{enumerate}
\item 讲解滞后-超前校正装置
\end{enumerate} &\begin{enumerate}
\item 学生倾听并记录
\end{enumerate} &15 \\\hline
讲解&\begin{enumerate}
\item 根轨迹校正
\end{enumerate}
&\begin{enumerate}
\item 讲解根轨迹校正
\end{enumerate} &\begin{enumerate}
\item 学生倾听并记录
\end{enumerate} &20 \\\hline
讲解&\begin{enumerate}
\item 频率校正
\end{enumerate}
&\begin{enumerate}
\item 讲解频率校正
\end{enumerate} &\begin{enumerate}
\item 学生倾听并记录
\end{enumerate} &20 \\\hline
\centering 本次课总结(评价)&总结本课程内容 &进行知识总结 &学生倾听 &5 \\\hline
\centering 学生学习笔记或工单等检查情况&\multicolumn{4}{m{165mm}|}{\quad}\\\hline
\centering 课后作业&\multicolumn{4}{m{165mm}|}{2-28,2-29}\\\hline
\centering 教学体会&\multicolumn{4}{m{165mm}|}{\quad}\\
\end{longtable}

\end{landscape}
\clearpage
%%%%%%%%%%%%%%%%%%%%板书设计%%%%%%%%%%%%%%%%%%%%%%%%%%
\begin{center}
{\huge 板书设计}
\end{center}
}

 \begin{frame}{控制系统设计} 
 \begin{block}{校正装置的结构}
 \begin{itemize}
 \item 串联校正装置:校正装置与被控对象串联
 \item 并联校正装置:校正装置接在系统的局部反馈回
 \end{itemize}
 \end{block}
 \begin{block}{常用校正装置}
\begin{itemize}
\item 超前校正装置
\item 滞后校正装置
\item 滞后-超前装置
\end{itemize}
 \end{block}
 \end{frame}
 
 \begin{frame}
 \begin{block}{超前校正装置}
\begin{circuitikz}[american]
\draw(0,0)to[short,o-](1,0)to[short](1,1)to[C,l=$C$](2.5,1)to[short](2.5,-1)to[generic,l=$R_1$](1,-1)to[short](1,0)(2.5,0)to[short](3.5,0)to[generic,l=$R_2$](3.5,-3)to[short,-o](0,-3)to[open,l=$V_1$](0,0)(3.5,-3)to[short,-o](4.5,-3)to[open,-o,l_=$V_2$](4.5,0)to[short](3.5,0);
\end{circuitikz}
\end{block}
\end{frame}

\begin{frame}
\begin{block}{滞后校正装置}
\begin{circuitikz}[american]
\draw(0,0)to[generic,l=$R_1$,o-](2,0)to[generic,l=$R_2$](2,-1.5)to[C,l=$C$](2,-3)to[short,-o](0,-3)to[open,l=$V_1$](0,0)(2,0)to[short,-o](3,0)to[open,-o,l^=$V_2$](3,-3)to[short](2,-3);
\end{circuitikz}
\end{block}
\end{frame}

\begin{frame}
\begin{block}{滞后-超前校正装置}
\begin{circuitikz}[american]
\draw(0,0)to[short,o-](1,0)to[short](1,1)to[C,l=$C_1$](2.5,1)to[short](2.5,-1)to[generic,l=$R_1$](1,-1)to[short](1,0)(2.5,0)to[short](3.5,0)to[generic,l=$R_2$](3.5,-2)to[C,l=$C_2$](3.5,-3.5)to[short,-o](0,-3.5)to[open,l=$V_1$](0,0)(3.5,-3.5)to[short,-o](4.5,-3.5)to[open,-o,l_=$V_2$](4.5,0)to[short](3.5,0);
\end{circuitikz}
\end{block}
\end{frame}

\begin{frame}{根轨迹校正}
\begin{block}{增加极点对根轨迹的影响}
随着极点的增加,根轨迹右移。
\end{block}
\begin{block}{增加零点对根轨迹的影响}
增加零点,根轨迹向左移,改善系统的动态性能和稳定性。
\end{block}
\end{frame}

\begin{frame}{根轨迹法设计校正装置的基本步骤}
\begin{itemize}
\item 绘制未加校正装置的根轨迹。
\item 根据性能指标,确定期望的主导极点。
\item 加入校正装置。
\item 检验。检验稳态指标是否满足要求。检验动态指标是否满足要求。
\item 确定校正装置的参数。
\end{itemize}
\end{frame}
\begin{frame}
\begin{example}
设系统开环传递函数为$G_o(s)=\frac{k}{s(s+1)}$,现要求:
(1)单位斜坡输入时,位置输入出误差为$e_{ss}\leq 0.1$;
(2)开环系统截止频率$\omega_c\geq 4.4rad/s$;
(3)相位裕量$\gamma \geq 45^o$,幅值裕量$\geq 4。4dB$。试设计一个超前校正装置。
\end{example}
\end{frame}
