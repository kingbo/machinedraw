%%%%%%%%%%%%%%教案头%%%%%%%%%%%%%%%%%%%%%%%%%%%%%%%
\mode<article>{

\begin{longtable}{|m{20mm}|m{20mm}|m{20mm}|m{20mm}|m{20mm}|m{28mm}|}
\caption*{\huge 教案头}\\
\hline
\endfirsthead
\multicolumn{6}{l}{(续表)}\\
\hline
\endhead
\hline
\multicolumn{6}{l}{\itshape 接下一页表格.......}\\ [2ex]
\endfoot
\hline
\endlastfoot
\centering{授课单元}&\multicolumn{3}{m{60mm}|}{\centering 4.3采样定理}&\centering{授课日期}&2014年05月30日 \\
\hline
\centering 授课地点 & \multicolumn{3}{m{60mm}|}{B6-204}&\centering 授课学时 & 2 \\
\hline
& \multicolumn{2}{m{40mm}|}{能力目标} & \multicolumn{2}{m{40mm}|}{知识目标}&素质目标 \\
\cline{2-6}
\centering 教学目标&\multicolumn{2}{m{40mm}|}{\begin{enumerate}
\item  能理解采样信号的频谱
\item  能理解零阶保持器
\end{enumerate} }&\multicolumn{2}{m{40mm}|}{\begin{enumerate}
\item 了解采样过程及采样信号的频谱
\item 了解香农采样定理
\item 了解零阶保持器
\end{enumerate}} & {\qquad}\\
\hline
\centering 能力训练任务或案例 &\multicolumn{5}{m{108mm}|}{ }\\
\hline
\centering 教学重点 & \multicolumn{5}{m{108mm}|}{\begin{enumerate}
\item 采样过程及采样信号的频谱
\item 零阶保持器
\end{enumerate}}\\
\hline
\centering 教学难点与解决办法 &\multicolumn{5}{m{108mm}|}{\begin{enumerate}
\item 难点:采样过程及采样信号的频谱
\item 解决方法:实例讲解
\end{enumerate}}\\
\hline
\centering 德育内容 &\multicolumn{5}{m{108mm}|}{无}\\
\hline
 &教材 & \multicolumn{4}{m{88mm}|}{计算机控制原理与应用}\\
\cline{2-6}& 教学资源 &\multicolumn{4}{m{88mm}|}{PPT}\\
\cline{2-6}\centering 使用的教学材料& 主要教学仪器设备和工具等 &\multicolumn{4}{m{88mm}|}{投影机、MATLAB}\\
\cline{2-6}& 主要耗材 &\multicolumn{4}{m{88mm}|}{无}\\
\hline
\centering 教学模式 &\multicolumn{2}{m{40mm}|}{知识讲授}&\centering 教学手段 &\multicolumn{2}{m{48mm}|}{多媒体教学}\\
\hline
\centering 学生成果与过程考核方式 &\multicolumn{5}{m{108mm}|}{无}
\end{longtable}
\clearpage

%%%%%%%%%%%%%%%教学实施过程%%%%%%%%%%%%%%%%%%%%%%%%%%%%
\begin{landscape}

\begin{longtable}{|m{10mm}|m{50mm}|m{50mm}|m{50mm}|m{15mm}|}
\caption*{\huge 教学组织与实施}\\
\hline
\endfirsthead
\multicolumn{5}{l}{\small 接上页}\\
\hline
\multicolumn{1}{|c|}{步骤}&\multicolumn{1}{c|}{教学内容}&\multicolumn{1}{c|}{教师活动}&\multicolumn{1}{c|}{学生活动}&\multicolumn{1}{c|}{时间}\\
\hline
\endhead

\multicolumn{5}{r}{\small 接下页}\\
\endfoot
\hline
\endlastfoot
\multicolumn{1}{|c|}{步骤}&\multicolumn{1}{c|}{教学内容}&\multicolumn{1}{c|}{教师活动}&\multicolumn{1}{c|}{学生活动}&\multicolumn{1}{c|}{时间}\\\hline
讲解&\begin{enumerate}
\item 采样过程及采样信号的频谱
\end{enumerate} &\begin{enumerate}
\item 讲解采样过程及采样信号的频谱
\end{enumerate} &\begin{enumerate}
\item 学生倾听并记录
\end{enumerate} &35\\\hline
讲解&\begin{enumerate}
\item 香农采样定理
\end{enumerate}
 &\begin{enumerate}
\item 讲解得农采样定理
\end{enumerate} &\begin{enumerate}
\item 学生倾听并记录
\end{enumerate} &10 \\\hline
讲解&\begin{enumerate}
\item 零阶保持器
\end{enumerate}
&\begin{enumerate}
\item 讲解零阶保持器
\end{enumerate} &\begin{enumerate}
\item 学生倾听并记录
\end{enumerate} &40 \\\hline

\centering 本次课总结(评价)&总结本课程内容 &进行知识总结 &学生倾听 &5 \\\hline
\centering 学生学习笔记或工单等检查情况&\multicolumn{4}{m{165mm}|}{\quad}\\\hline
\centering 课后作业&\multicolumn{4}{m{165mm}|}{2-28,2-29}\\\hline
\centering 教学体会&\multicolumn{4}{m{165mm}|}{\quad}\\
\end{longtable}

\end{landscape}
\clearpage
%%%%%%%%%%%%%%%%%%%%板书设计%%%%%%%%%%%%%%%%%%%%%%%%%%
\begin{center}
{\huge 板书设计}
\end{center}
}

 \begin{frame}{采样过程及采样信号的频谱} 
 \begin{block}{采样数学表达式}
\[x^*(t)=x(t)p(t)\]
 \end{block}
 \begin{block}{单位脉冲序列}
 \[p(t)=\delta_T(t)=\sum\limits_{k=0}^\infty\delta(t-kT)\]
 \end{block}
 \end{frame}
 
 \begin{frame}
 \begin{block}{}
\begin{eqnarray*}
x*(t)=x(t)\delta_T(t)=x(t)\sum\limits_{k=0}^\infty\delta(t-kT)\\
=\sum\limits_{k=0}^\infty x(kT)\delta(t-kT)
\end{eqnarray*}
\end{block}
\end{frame}

\begin{frame}
\begin{block}{作拉氏变换}
\[X^*(s)=L[x^*(t)]=\sum\limits_{k=0}^\infty x(kT)\int\limits_0^\infty\delta(t-kT)e^{-st}\]

\[X^*(s)=\sum\limits_{k=0}^\infty x(kT)e^{-kTs}\]
\end{block}
\end{frame}

\begin{frame}
\begin{block}{$\delta_T(t)$傅里叶级数展}
若采样周其为$T$,采样频率为$\omega_s$,则
\[\omega_s=\frac{2\pi}{T}\]
傅里叶级数展开:
\begin{equation*}
\delta_T(t)=\sum\limits_{k=0}^\infty\delta(t-kT)=\sum\limits_{k=-\infty}^\infty C_k e^{-jkw_st}
\end{equation*}
\end{block}
\end{frame}

\begin{frame}
\begin{block}{}
\begin{eqnarray*}
C_k=\frac{1}{T}\int_{-\frac{T}{2}}^{\frac{T}{2}}\delta_T(t)e^{-jk\omega_st}dt\\
=\frac{1}{T}\int_{-\frac{T}{2}}^{\frac{T}{2}}\sum\limits_{k=0}^\infty\delta(t-kT)e^{-jk\omega_st}dt\\
=\frac{1}{T}\int_{0^-}^{0^+}\delta(t)e^{-jk\omega_st}dt=\frac{1}{T}
\end{eqnarray*}
\end{block}
\end{frame}

\begin{frame}
\begin{block}{}
\begin{eqnarray*}
\delta_T=\frac{1}{T}\sum\limits_{k=-\infty}^\infty e^{jk\omega_st}\\
x^*(t)=x(t)\frac{1}{T}\sum\limits_{k=-\infty}^\infty e^{jk\omega_st}=\frac{1}{T}\sum\limits_{k=-\infty}^\infty x(t)e^{jk\omega_st}
\end{eqnarray*}
\end{block}
\end{frame}

\begin{frame}
\begin{block}{}
\begin{eqnarray*}
X^*(s)=L[\frac{1}{T}\sum\limits_{k=-\infty}^\infty x(\zeta)e^{jk\omega_st}]\\
=\frac{1}{T}\sum\limits_{k=-\infty}^\infty X(s-jk\omega_s)\\
x^*(j\omega)=\frac{1}{T}\sum\limits_{k=-\infty}^\infty X[j(\omega+k\omega_s)]
\end{eqnarray*}
\end{block}
\end{frame}

\begin{frame}{香农采样定理}
\begin{block}{}
要示采样速率$\omega_s$至少是信号$x(t)$最高频率分量$\omega_h$的两倍。
\[\omega_s\geq 2\omega_h\]
\end{block}
\end{frame}
\begin{frame}{零阶保持器}
\begin{block}{解码}
根据DAC(数模变换器)采用的解码规则,将数字变成相对应的电压或电流。
\end{block}
\begin{block}{保持}
解决相邻采样时刻之间的插值。
\end{block}
\begin{block}{保持器}
实现保持功能的装置
\end{block}
\end{frame}
\begin{frame}
\begin{block}{保持器外推公式}
\[x(kT+\Delta t)=a_0+a_1\Delta t+a_2\Delta t^2\cdots+a_n\Delta t^n\]
\end{block}
\begin{block}{当$n=0$时}
\[x(kT+\Delta t)=a_0\]
\end{block}

\begin{block}{当$\Delta t=0$时}
\[x(kT)=a_0\]
\end{block}
\end{frame}

\begin{frame}
\begin{block}{零阶保持器的数学描述}
\begin{equation*}
m(t)=\sum\limits_{k=0}^\infty x(kT)\{u(t-kT)-u[t-(k+1)T]\}
\end{equation*}
\end{block}
\end{frame}
\begin{frame}
\begin{block}{求拉氏变换}
\begin{eqnarray*}
&&M(s)=L[m(t)]\\
&&=\int_0^\infty\sum\limits_{k=0}^\infty x(kT)\{u(t-kT)\\
&&-u[t-(k+1)T]\}e^{-st}dt\\
&&=\sum\limits_{k=0}^\infty x(kT)\left[\frac{e^{-ksT}-e^{-(k+1)sT}}{s}\right]
\end{eqnarray*}
\end{block}
\end{frame}
\begin{frame}
\begin{block}{}
\begin{eqnarray*}
&=&\left[\frac{1-e^{-sT}}{s}\right]\sum\limits_{k=0}^\infty x(kT)e^{-skT}\\
&=&\left[\frac{1-e^{-sT}}{s}\right]X^*(s)
\end{eqnarray*}
\end{block}
\begin{block}{零阶保持器的传递函数}
\[G_h(s)=\frac{M(s)}{X^*(s)}=\frac{1-e^{-sT}}{s}\]
\end{block}
\end{frame}