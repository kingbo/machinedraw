%%%%%%%%%%%%%%教案头%%%%%%%%%%%%%%%%%%%%%%%%%%%%%%%
\mode<article>{

\begin{longtable}{|m{20mm}|m{20mm}|m{20mm}|m{20mm}|m{20mm}|m{28mm}|}
\caption*{\huge 教案头}\\
\hline
\endfirsthead
\multicolumn{6}{l}{(续表)}\\
\hline
\endhead
\hline
\multicolumn{6}{l}{\itshape 接下一页表格.......}\\ [2ex]
\endfoot
\hline
\endlastfoot
\centering{授课单元}&\multicolumn{3}{m{60mm}|}{\centering 2.8.1频率响应法2.8.2频率特性的图形表示2.8.3典型环节的对数频率特性}&\centering{授课日期}&2014年05月05日 \\
\hline
\centering 授课地点 & \multicolumn{3}{m{60mm}|}{B6-204}&\centering 授课学时 & 2 \\
\hline
& \multicolumn{2}{m{40mm}|}{能力目标} & \multicolumn{2}{m{40mm}|}{知识目标}&素质目标 \\
\cline{2-6}
\centering 教学目标&\multicolumn{2}{m{40mm}|}{\begin{enumerate}
\item 能够分析典型环节的对数频率特性
\end{enumerate} }&\multicolumn{2}{m{40mm}|}{\begin{enumerate}
\item 掌握频率响应法的概念
\item 掌握根典型环节的对数频率特性
\end{enumerate}} & {\qquad}\\
\hline
\centering 能力训练任务或案例 &\multicolumn{5}{m{108mm}|}{ }\\
\hline
\centering 教学重点 & \multicolumn{5}{m{108mm}|}{\begin{enumerate}
\item 典型环节的对数频率特性
\end{enumerate}}\\
\hline
\centering 教学难点与解决办法 &\multicolumn{5}{m{108mm}|}{\begin{enumerate}
\item 难点:频率响应法
\item 解决方法:适应的数学知识补充
\end{enumerate}}\\
\hline
\centering 德育内容 &\multicolumn{5}{m{108mm}|}{无}\\
\hline
 &教材 & \multicolumn{4}{m{88mm}|}{计算机控制原理与应用}\\
\cline{2-6}& 教学资源 &\multicolumn{4}{m{88mm}|}{PPT}\\
\cline{2-6}\centering 使用的教学材料& 主要教学仪器设备和工具等 &\multicolumn{4}{m{88mm}|}{投影机、MATLAB}\\
\cline{2-6}& 主要耗材 &\multicolumn{4}{m{88mm}|}{无}\\
\hline
\centering 教学模式 &\multicolumn{2}{m{40mm}|}{知识讲授}&\centering 教学手段 &\multicolumn{2}{m{48mm}|}{多媒体教学}\\
\hline
\centering 学生成果与过程考核方式 &\multicolumn{5}{m{108mm}|}{无}
\end{longtable}
\clearpage

%%%%%%%%%%%%%%%教学实施过程%%%%%%%%%%%%%%%%%%%%%%%%%%%%
\begin{landscape}

\begin{longtable}{|m{10mm}|m{50mm}|m{50mm}|m{50mm}|m{15mm}|}
\caption*{\huge 教学组织与实施}\\
\hline
\endfirsthead
\multicolumn{5}{l}{\small 接上页}\\
\hline
\multicolumn{1}{|c|}{步骤}&\multicolumn{1}{c|}{教学内容}&\multicolumn{1}{c|}{教师活动}&\multicolumn{1}{c|}{学生活动}&\multicolumn{1}{c|}{时间}\\
\hline
\endhead

\multicolumn{5}{r}{\small 接下页}\\
\endfoot
\hline
\endlastfoot
\multicolumn{1}{|c|}{步骤}&\multicolumn{1}{c|}{教学内容}&\multicolumn{1}{c|}{教师活动}&\multicolumn{1}{c|}{学生活动}&\multicolumn{1}{c|}{时间}\\\hline
讲解&\begin{enumerate}
\item 频率响应法
\end{enumerate} &\begin{enumerate}
\item 讲解频率响应法的理论
\end{enumerate} &\begin{enumerate}
\item 学生倾听并记录
\end{enumerate} &25 \\\hline
讲解&\begin{enumerate}
\item 频率特性的图形表示
\end{enumerate}
 &\begin{enumerate}
\item 讲解频率特性的图形表示
\end{enumerate} &\begin{enumerate}
\item 学生倾听并记录
\end{enumerate} &20 \\\hline
讲解&\begin{enumerate}
\item 典型环节的对数频率特性
\end{enumerate}
&\begin{enumerate}
\item 讲解典型环节的对数频率特性
\end{enumerate} &\begin{enumerate}
\item 学生倾听并记录
\end{enumerate} &40 \\\hline

\centering 本次课总结(评价)&总结本课程内容 &进行知识总结 &学生倾听 &5 \\\hline
\centering 学生学习笔记或工单等检查情况&\multicolumn{4}{m{165mm}|}{\quad}\\\hline
\centering 课后作业&\multicolumn{4}{m{165mm}|}{2-28,2-29}\\\hline
\centering 教学体会&\multicolumn{4}{m{165mm}|}{\quad}\\
\end{longtable}

\end{landscape}
\clearpage
%%%%%%%%%%%%%%%%%%%%板书设计%%%%%%%%%%%%%%%%%%%%%%%%%%
\begin{center}
{\huge 板书设计}
\end{center}
}
\mode<presentation>{ \section{频率响应的概念}
 \subsection{频率响应的概念}}
 \begin{frame}{频率响应法} 
 \begin{block}{传递函数的频率特性表示}
 \begin{eqnarray*}
 G(j\omega)=G(s)|_{s=j\omega}=\frac{C(j\omega)}{R(j\omega)}
 \end{eqnarray*}
 \end{block}
  \begin{block}{频率特性的直角坐标表示}
 \begin{eqnarray*}
 G(j\omega)=R_e(\omega)+jI_m(\omega)
 \end{eqnarray*}
 \end{block}
 \end{frame}
 \begin{frame}
 \begin{block}{频率特性的极坐标表示}
 \begin{eqnarray*}
 G(j\omega)=\left|G(j\omega)\right|\underline{\diagup\varphi(\omega)}=\left|G(j\omega)\right|e^{j\varphi(\omega)}
 \end{eqnarray*}
 其中:
 \begin{eqnarray*}
 \left|G(j\omega)\right|=\sqrt{[R_e(\omega)]^2+[I_m(\omega)]^2}\\
 \varphi(\omega)=\arctan\left[\frac{I_m(\omega)}{R_e(\omega)}\right]
 \end{eqnarray*}
 \end{block}
 \end{frame}
 
 \begin{frame}
 \begin{block}{频率特性的图形表示}
\begin{itemize}
\item 极坐标图
\item 对数坐标图
\item 对数幅相图
\end{itemize}
\end{block}
\end{frame}

\begin{frame}
\begin{block}{惯性环节}
 \begin{eqnarray*}
G(s)=\frac{a}{s+a}=\frac{1}{1+Ts}\\
G(j\omega)=\frac{1}{1+j\omega T}=\frac{1}{\sqrt{1+(\omega T)^2}}e^{-j\varphi(\omega)}\\
L(\omega)=20\lg \frac{1}{\sqrt{1+(T\omega)^2}}=\approx -20\lg (T\omega)
\end{eqnarray*}
\end{block}
\end{frame}

\begin{frame}
\begin{block}{比例环节}
 \begin{eqnarray*}
G(s)=K\\
G(j\omega)=K\\
L(\omega)=20\lg K
\end{eqnarray*}
\end{block}
\end{frame}

\begin{frame}
\begin{block}{积分环节}
 \begin{eqnarray*}
G(s)=\frac{1}{s}\\
G(j\omega)=\frac{1}{j\omega}=-j\frac{1}{\omega}\\
L(\omega)=-20\lg \omega
\end{eqnarray*}
\end{block}
\end{frame}

\begin{frame}
\begin{block}{积分环节}
 \begin{eqnarray*}
G(s)=\frac{\omega_n^2}{s^2+2\zeta\omega_ns+\omega_n^2}=\frac{1}{1+2\zeta Ts+T^2s^2}\\
G(j\omega)=\frac{1}{(1-T^2\omega_n^2)+j\zeta T\omega}\\
L(\omega)=-20\lg\sqrt{(1-T^2\omega_n^2)^2+(2\zeta T\omega)^2}
\end{eqnarray*}
\end{block}
\end{frame}