%%%%%%%%%%%%%%教案头%%%%%%%%%%%%%%%%%%%%%%%%%%%%%%%
\mode<article>{

\begin{longtable}{|m{20mm}|m{20mm}|m{20mm}|m{20mm}|m{20mm}|m{28mm}|}
\caption*{\huge 教案头}\\
\hline
\endfirsthead
\multicolumn{6}{l}{(续表)}\\
\hline
\endhead
\hline
\multicolumn{6}{l}{\itshape 接下一页表格.......}\\ [2ex]
\endfoot
\hline
\endlastfoot
\centering{授课单元}&\multicolumn{3}{m{60mm}|}{\centering 2.6.5高阶系统分析2.6.6稳态误差分析}&\centering{授课日期}&2014年04月18日 \\
\hline
\centering 授课地点 & \multicolumn{3}{m{60mm}|}{B6-204}&\centering 授课学时 & 2 \\
\hline
& \multicolumn{2}{m{40mm}|}{能力目标} & \multicolumn{2}{m{40mm}|}{知识目标}&素质目标 \\
\cline{2-6}
\centering 教学目标&\multicolumn{2}{m{40mm}|}{\begin{enumerate}
\item 能够分析系统的稳态误差
\end{enumerate} }&\multicolumn{2}{m{40mm}|}{\begin{enumerate}
\item 了解高附系统的分析方法
\item 掌握系统的稳态误差分析
\end{enumerate}} & {\qquad}\\
\hline
\centering 能力训练任务或案例 &\multicolumn{5}{m{108mm}|}{ }\\
\hline
\centering 教学重点 & \multicolumn{5}{m{108mm}|}{\begin{enumerate}
\item 系统的稳态误差分析
\end{enumerate}}\\
\hline
\centering 教学难点与解决办法 &\multicolumn{5}{m{108mm}|}{\begin{enumerate}
\item 难点:系统的稳态误差分析
\item 解决方法:数学推理加实例讲解
\end{enumerate}}\\
\hline
\centering 德育内容 &\multicolumn{5}{m{108mm}|}{无}\\
\hline
 &教材 & \multicolumn{4}{m{88mm}|}{计算机控制原理与应用}\\
\cline{2-6}& 教学资源 &\multicolumn{4}{m{88mm}|}{PPT}\\
\cline{2-6}\centering 使用的教学材料& 主要教学仪器设备和工具等 &\multicolumn{4}{m{88mm}|}{投影机、MATLAB}\\
\cline{2-6}& 主要耗材 &\multicolumn{4}{m{88mm}|}{无}\\
\hline
\centering 教学模式 &\multicolumn{2}{m{40mm}|}{知识讲授}&\centering 教学手段 &\multicolumn{2}{m{48mm}|}{多媒体教学}\\
\hline
\centering 学生成果与过程考核方式 &\multicolumn{5}{m{108mm}|}{无}
\end{longtable}
\clearpage

%%%%%%%%%%%%%%%教学实施过程%%%%%%%%%%%%%%%%%%%%%%%%%%%%
\begin{landscape}

\begin{longtable}{|m{10mm}|m{50mm}|m{50mm}|m{50mm}|m{15mm}|}
\caption*{\huge 教学组织与实施}\\
\hline
\endfirsthead
\multicolumn{5}{l}{\small 接上页}\\
\hline
\multicolumn{1}{|c|}{步骤}&\multicolumn{1}{c|}{教学内容}&\multicolumn{1}{c|}{教师活动}&\multicolumn{1}{c|}{学生活动}&\multicolumn{1}{c|}{时间}\\
\hline
\endhead

\multicolumn{5}{r}{\small 接下页}\\
\endfoot
\hline
\endlastfoot
\multicolumn{1}{|c|}{步骤}&\multicolumn{1}{c|}{教学内容}&\multicolumn{1}{c|}{教师活动}&\multicolumn{1}{c|}{学生活动}&\multicolumn{1}{c|}{时间}\\\hline
讲解&\begin{enumerate}
\item 高阶系统的单位阶跃响应
\end{enumerate} &\begin{enumerate}
\item 讲解高阶系统的单位阶跃响应
\end{enumerate} &\begin{enumerate}
\item 学生倾听并记录
\end{enumerate} &30 \\\hline
讲解&\begin{enumerate}
\item 高阶系统分析
\end{enumerate}
 &\begin{enumerate}
\item 讲解高阶系统分析方法
\end{enumerate} &\begin{enumerate}
\item 学生倾听并记录
\end{enumerate} &15 \\\hline
讲解&\begin{enumerate}
\item 稳态误差分析
\end{enumerate}
&\begin{enumerate}
\item 讲解系统稳态分析
\end{enumerate} &\begin{enumerate}
\item 学生倾听并记录
\end{enumerate} &20 \\\hline
讲解&\begin{enumerate}
\item 稳态误差分析实例
\end{enumerate}
 &\begin{enumerate}
\item 讲解稳态误差分析实例
\end{enumerate} &\begin{enumerate}
\item 学生记录笔记
\end{enumerate} &20 \\\hline

\centering 本次课总结(评价)&总结本课程内容 &进行知识总结 &学生倾听 &5 \\\hline
\centering 学生学习笔记或工单等检查情况&\multicolumn{4}{m{165mm}|}{\quad}\\\hline
\centering 课后作业&\multicolumn{4}{m{165mm}|}{2-19,2-20,2-21}\\\hline
\centering 教学体会&\multicolumn{4}{m{165mm}|}{\quad}\\
\end{longtable}

\end{landscape}
\clearpage
%%%%%%%%%%%%%%%%%%%%板书设计%%%%%%%%%%%%%%%%%%%%%%%%%%
\begin{center}
{\huge 板书设计}
\end{center}
}
\mode<presentation>{ \section{高阶系统分析}
 \subsection{高阶系统分析}}
 \begin{frame}{高阶系统的单位阶跃响应}
 \begin{block}{高阶闭环传递函数}
 \begin{eqnarray*}
 \Phi(s)&=&\frac{C(s)}{R(s)}=\frac{B(s)}{A(s)}\\
 &=&\frac{b_0s^m+b_1s^{m-1}+\cdots +b_{m-1}s+b_m}{s^n+a_1s^{n-1}+\cdots +a_{n-1}s+a_n}
 \end{eqnarray*}
 其中:$B(s)$为分子多项式,$A(s)$为分母多项式
 \end{block}
 \end{frame}
 
 \begin{frame}
 \begin{block}{极点、零点形式}
 \begin{eqnarray*}
 \Phi(s)&=&\frac{C(s)}{R(s)}=\frac{B(s)}{A(s)}\\
&=&\frac{K\prod\limits_{i=1}^m(s-z_i)}{\prod\limits_{j=1}^{n_1}(s-p_j)\prod\limits_{i=1}^{n_2}(s^2+2\zeta_l\omega_ls+\omega_l^2)}
\end{eqnarray*}  
\end{block}
\end{frame}

\begin{frame}
\begin{block}{单位阶跃输出部分展开式}
\begin{eqnarray*}
C(s)=\frac{K\prod\limits_{i=1}^m(s-z_i)}{s[\prod\limits_{j=1}^{n_1}(s-p_j)\prod\limits_{i=1}^{n_2}(s^2+2\zeta_l\omega_ls+\omega_l^2)]}\\
=\frac{a_0}{s}+\sum_{j=1}^{n_1}\frac{a_j}{s-p_j}+\sum_{l=1}^{n_2}\frac{\beta_ls+r_l}{(s-p_{1l})(s-p_{2l})}
\end{eqnarray*}
其中:$a_0,a_j,\beta_l,r_l$为待定系数
\end{block}
\end{frame}
\begin{frame}
待定系统求法:
\[a_0=[sC(s)]\mid_{s=0}\]
\[a_j=[(s-p_j)C(s)]\mid_{s=p_j},j=1,2,\cdots ,n\]
\[\beta_ls+r_l \]
\[=[(s-p_{1l})(s-P_{2l}C(s)]\mid_{s=p_{1l}(or s=p_{2l})},\]
\[l=1,2,\cdots ,n\]
\end{frame}
\begin{frame}
\begin{block}{拉氏变换}
\begin{eqnarray*}
c(t)&=&L^{-1}[C(s)]\\
&=&a_0+\sum_{j=1}^{n_1}a_je^{p_jt}+\\
&&\sum_{l=1}^{n_1}\beta_le^{-\zeta_l\omega_lt}\cos\omega_l\sqrt{1-\zeta^2_l}t+\\
&&\sum_{l=1}^{n_2}\frac{r_l-\beta_l\zeta_l\omega_l}{\omega_l\sqrt{1-\zeta_l^2}}e^{-\zeta_l\omega_lt}\sin\omega_l\sqrt{1-\zeta_l^2}t
\end{eqnarray*}
\end{block}
\end{frame}

\begin{frame}{高阶系统分析}
\begin{block}{}
\begin{itemize}
\item<+-> 位于左半s平面,系统稳定
\item<+-> 若系统稳定:离虚轴越远,则指数项衰减走越快、
\item<+-> 实际中,常用根轨迹法、频率法等
\end{itemize}
\end{block}
\end{frame}
\begin{frame}{稳态误差分析}
\begin{block}{系统开环传递函数}
\begin{eqnarray*}
G_o(s)=\frac{K(1+\tau_1s)(1+\tau_2s)\cdots(1+\tau_ms}{s^q(1+T_1s)(1+T_2s)\cdots(1+T_{n-q}s)}
\end{eqnarray*}
其中:$K$为开环增益;\\
$\tau_1,\tau_2,\cdots,\tau_m$和$T_1,T_2,\cdots,T_n$为时间常数;\\
$q$为开环系统在$s$平面坐标原点上重极点数
\end{block}
\end{frame}
\begin{frame}
\begin{block}{稳态误差}
\begin{itemize}
\item<+-> 闭环传递函数为:
\[\Phi(s)=\frac{C(s)}{R(s)}=\frac{G(s)}{1+H(s)G(s)}\]
\item<+-> 则开环为:
\[G_o(s)=G(s)H(s)\]
\end{itemize}
\end{block}
\end{frame}
\begin{frame}
\begin{block}{}
\begin{itemize}
\item<+-> 误差信息$e(t)$为:
\[e(t)=r(t)-b(t)\]
\item<+-> 拉氏变换为:
\[E(s)=R(s)-B(s)=R(s)-H(s)C(s)\]
\end{itemize}
\end{block}
\end{frame}
\begin{frame}
\begin{block}{误差传递函数为}
\begin{eqnarray*}
\Phi_E(s)&=&\frac{E(s)}{R(s)}=\frac{1}{1+H(s)G(s)}\\
E(s)&=&\Phi_E(s)R(s)\\
&=&\frac{1}{1+H(s)G(s)}R(s)\\
&=&\frac{1}{1+G_o(s)}R(s)
\end{eqnarray*}
\end{block}
\end{frame}
\begin{frame}
\begin{block}{系统稳态误差}
\begin{eqnarray*}
e_{ss}&=&\lim_{s\rightarrow \infty}e(t)=\lim_{s\rightarrow 0}sE(s)\\
&=&\lim_{s\rightarrow 0}\frac{sR(s)}{1+H(s)R(s)}\\
&=&\lim_{s\rightarrow 0}\frac{sR(s)}{1+G_o(s)}
\end{eqnarray*}
\end{block}
\end{frame}
\begin{frame}
\begin{block}{阶跃输入的稳态误差}
\begin{eqnarray*}
e_{ss}(step)&=&\lim\limits_{s\rightarrow 0}\frac{s}{1+G_o(s)}\frac{R_0}{s}\\
&=&\lim\limits_{s\rightarrow 0}\frac{R_0}{1+G_o(s)}\\
&=&\lim\limits_{s\rightarrow 0}\frac{R_0}{1+K_p}\\
K_p&=&\lim\limits_{s\rightarrow 0}G_o(s)=\lim\limits_{s\rightarrow 0}H(s)G(s)
\end{eqnarray*}
\end{block}
\end{frame}
\begin{frame}
\begin{block}{斜坡输入时的稳态误差}
\begin{eqnarray*}
e_{ss}(ramp)&=&\lim\limits_{s\rightarrow 0}\frac{s}{1+G_o(s)}\frac{R_0}{s^2}\\
&=&\lim\limits_{s\rightarrow 0}\frac{R_0}{sG_o(s)}\\
&=&\frac{R_0}{K_V}\\
K_V&=&\lim\limits_{s\rightarrow 0}sG_o(s)=\lim\limits_{s\rightarrow 0}sH(s)G(s)
\end{eqnarray*}
\end{block}
\end{frame}
\begin{frame}
\begin{block}{抛物线输入时的稳态误差}
\begin{eqnarray*}
&&e_{ss}(parabolic)=\lim\limits_{s\rightarrow 0}\frac{s}{1+G_o(s)}\frac{R_0}{s^3}\\
&=&\lim\limits_{s\rightarrow 0}\frac{R_0}{s^2G_o(s)}\\
&=&\frac{R_0}{K_a}\\
K_a&=&\lim\limits_{s\rightarrow 0}s^2G_o(s)=\lim\limits_{s\rightarrow 0}s^2H(s)G(s)
\end{eqnarray*}
\end{block}
\end{frame}
\endinput