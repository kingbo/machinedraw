%%%%%%%%%%%%%%%教案头%%%%%%%%%%%%%%%%%%%%%%%%%%%%%%%
\mode<article>{

\begin{longtable}{|m{20mm}|m{20mm}|m{20mm}|m{20mm}|m{20mm}|m{28mm}|}
\caption*{\huge 教案头}\\
\hline
\endfirsthead
\multicolumn{6}{l}{(续表)}\\
\hline
\endhead
\hline
\multicolumn{6}{l}{\itshape 接下一页表格.......}\\ [2ex]
\endfoot
\hline
\endlastfoot
\centering{授课单元}&\multicolumn{3}{m{60mm}|}{\centering 2.4.1传递函数和微分方程2.4.2电子网络的传递函数2.4.3简单方框图的传递函数}&\centering{授课日期}&2014年03月28日 \\
\hline
\centering 授课地点 & \multicolumn{3}{m{60mm}|}{B6-204}&\centering 授课学时 & 2 \\
\hline
& \multicolumn{2}{m{40mm}|}{能力目标} & \multicolumn{2}{m{40mm}|}{知识目标}&素质目标 \\
\cline{2-6}
\centering 教学目标&\multicolumn{2}{m{40mm}|}{\begin{enumerate}
\item 能够求取电子网络的传递函数
\item 能够求取方框图的传递函数
\item 能够绘制传递函数方框图
\end{enumerate} }&\multicolumn{2}{m{40mm}|}{\begin{enumerate}
\item 掌握传递函数与微分方程之间的关系方法
\item 掌握方框图的绘制方法
\item 掌握传递函数的概念
\end{enumerate}} & {\qquad}\\
\hline
\centering 能力训练任务或案例 &\multicolumn{5}{m{108mm}|}{\begin{enumerate}
\item RC网络电路图
\item RLC网络电路图
\end{enumerate}}\\
\hline
\centering 教学重点 & \multicolumn{5}{m{108mm}|}{\begin{enumerate}
\item 方框图的传递函数
\item 方框图的绘制
\end{enumerate}}\\
\hline
\centering 教学难点与解决办法 &\multicolumn{5}{m{108mm}|}{\begin{enumerate}
\item 难点:方框图的绘制
\item 解决方法:用实例进行分析讲解
\end{enumerate}}\\
\hline
\centering 德育内容 &\multicolumn{5}{m{108mm}|}{无}\\
\hline
 &教材 & \multicolumn{4}{m{88mm}|}{计算机控制原理与应用}\\
\cline{2-6}& 教学资源 &\multicolumn{4}{m{88mm}|}{PPT}\\
\cline{2-6}\centering 使用的教学材料& 主要教学仪器设备和工具等 &\multicolumn{4}{m{88mm}|}{投影机、MATLAB}\\
\cline{2-6}& 主要耗材 &\multicolumn{4}{m{88mm}|}{无}\\
\hline
\centering 教学模式 &\multicolumn{2}{m{40mm}|}{知识讲授}&\centering 教学手段 &\multicolumn{2}{m{48mm}|}{多媒体教学}\\
\hline
\centering 学生成果与过程考核方式 &\multicolumn{5}{m{108mm}|}{无}
\end{longtable}
\clearpage

%%%%%%%%%%%%%%%教学实施过程%%%%%%%%%%%%%%%%%%%%%%%%%%%%
\begin{landscape}

\begin{longtable}{|m{10mm}|m{50mm}|m{50mm}|m{50mm}|m{15mm}|}
\caption*{\huge 教学组织与实施}\\
\hline
\endfirsthead
\multicolumn{5}{l}{\small 接上页}\\
\hline
\multicolumn{1}{|c|}{步骤}&\multicolumn{1}{c|}{教学内容}&\multicolumn{1}{c|}{教师活动}&\multicolumn{1}{c|}{学生活动}&\multicolumn{1}{c|}{时间}\\
\hline
\endhead

\multicolumn{5}{r}{\small 接下页}\\
\endfoot
\hline
\endlastfoot
\multicolumn{1}{|c|}{步骤}&\multicolumn{1}{c|}{教学内容}&\multicolumn{1}{c|}{教师活动}&\multicolumn{1}{c|}{学生活动}&\multicolumn{1}{c|}{时间}\\\hline
引入&\begin{enumerate}
\item 分析系统微分方程存在的不足
\end{enumerate} &\begin{enumerate}
\item 系统微分方程不便于表示复杂的系统
\item 建立比较复杂
\item 不便于进行系统特性分析
\end{enumerate} &\begin{enumerate}
\item 学生记录
\end{enumerate} &10 \\\hline
讲解&\begin{enumerate}
\item 传递函数和微分方程
\end{enumerate}
 &\begin{enumerate}
\item 通过数学推导讲明传递函数与微分方程之间的关系
\end{enumerate} &\begin{enumerate}
\item 学生倾听并记录
\end{enumerate} &15 \\\hline
讲解&\begin{enumerate}
\item 电子网络的传递函数
\end{enumerate}
&\begin{enumerate}
\item 展示RC和RCL电子网络电路图
\item 指导学生求取传递函数
\item 讲解求取要点
\end{enumerate} &\begin{enumerate}
\item 学生尝试求取传递函数
\item 学生展示传递函数的结果
\item 学生进行记录
\end{enumerate} &20 \\\hline
讲解&\begin{enumerate}
\item 简单方框图的传递函数
\end{enumerate}
 &\begin{enumerate}
\item 讲解方框图的概念
\item 讲解方框图传递函数的求取方法
\end{enumerate} &\begin{enumerate}
\item 学生记录笔记
\end{enumerate} &20 \\\hline
讲解&
\begin{enumerate}
\item 方框图的绘制
\end{enumerate}
 &\begin{enumerate}
\item 以RC电路为例讲解方框图的绘方法
\item 指导学生绘制RLC电路的方框图
\item 讲解要点
\end{enumerate} &\begin{enumerate}
\item 学生倾听并记录
\item 学生尝试绘制RLC电路的方框图
\item 学生记录笔记
\end{enumerate} &20 \\\hline
\centering 本次课总结(评价)&总结本课程内容 &进行知识总结 &学生倾听 &5 \\\hline
\centering 学生学习笔记或工单等检查情况&\multicolumn{4}{m{165mm}|}{\quad}\\\hline
\centering 课后作业&\multicolumn{4}{m{165mm}|}{2-19,2-20,2-21}\\\hline
\centering 教学体会&\multicolumn{4}{m{165mm}|}{\quad}\\
\end{longtable}

\end{landscape}
\clearpage
%%%%%%%%%%%%%%%%%%%%板书设计%%%%%%%%%%%%%%%%%%%%%%%%%%
\lecture{传递函数与方框图}{chuandihanshu}
\begin{center}
{\huge 板书设计}
\end{center}
}
\mode<presentation>{ \section{传递函数与微分方程}
 \subsection{传递函数和微分方程}}
 \begin{frame}[containsverbatim]{传递函数和微分方程}
\begin{eqnarray*}
 \frac{d^n y(t)}{dt^n}+a_1\frac{d^{n-1}y(t)}{dt^{n-1}}+\cdots \\ 
 +a_{n-1}\frac{dy(t)}{dt}+a_ny(t)=
 \\  b_0\frac{d^mx(t)}{dt^m}+b_1\frac{d^{m-1}x(t)}{dt^{m-1}}+\cdots \\
 +b_{m-1}\frac{dx(t)}{dt}+b_mx(t)
 \end{eqnarray*}
 \end{frame}
 \begin{frame}
 \begin{eqnarray*}
 G(s)=\frac{Y(s)}{X(s)}\\ 
 =\frac{b_0s^m+b_1s^{m-1}+\cdots +b_{m-1}s+b_m}{s^n+a_1s^{n-1}+\cdots +a_{n-1}s+a_n}
 \end{eqnarray*}
 \end{frame}
 \begin{frame}{电子网络的传递函数}
 \begin{block}{RC无源网络}
 \[\frac{dV_2(t)}{dt}+\frac{1}{RC}V_2(t)=\frac{1}{RC}v_1(t)\]
 \[sV_2(s)+\frac{1}{RC}V_2(s)=\frac{1}{RC}V_1(s)\]
 \[\frac{V_2(s)}{V_1(s)}=\frac{1}{RCs+1}\]
 \end{block}
 \end{frame}
 \begin{frame}
 \begin{block}{RLC无源网络}
 \[\frac{d^2V_2(t)}{dt^2}+\frac{RdV_2(t)}{Ldt}+\frac{V_2(t)}{LC}=\frac{V_1(t)}{LC}
 \]
 \[\frac{V_2(s)}{V_1(s)}=\frac{1}{LCs^2+RCs+1}
 \]
 \end{block}
 \end{frame}
 \begin{frame}{简单方框图的传递函数}
 \begin{definition}
 \begin{itemize}
 \item 是一种图形
 \item 表示元件或子系统的功能
 \item 表示信号流向
 \end{itemize} 
 \end{definition}
 \begin{block}{方框图的特点}
 \begin{itemize}
 \item 保持系统的数学描述
 \item 信号流向用箭头表示
 \item 包含系统动态特性信息,不含物理结构
 \end{itemize} 
 \end{block}
 \end{frame}
 \begin{frame}{方框图的传递函数}
 \begin{block}{开环系统}
 \begin{tikzpicture}
 [node distance=10mm,>=stealth]
 \node(GS1)[nonterminal]{\tiny $G_c(s)$};
 \draw[<-](GS1.west)--++(-10mm,0)node[midway,above]{\tiny R(s)};
 \node(GS2)[nonterminal,right=of GS1]{\tiny $G_p(s)$};
 \draw[->](GS1.east)--(GS2.west);node[midway,above]{\tiny M(s)};
 \draw[->](GS2.east)--++(10mm,0)node[midway,above]{\tiny C(s)};
 \end{tikzpicture}
 \end{block}
 \begin{block}{传递函数}
 \[G(s)=G_c(s)G_p(s)\]
 \end{block}
 \end{frame}
 \begin{frame}[allowframebreaks]
 \begin{block}{闭环系统}
 \begin{tikzpicture}
 [node distance=10mm,>=stealth]
 \node(star)[circleterminal,label=above left:\tiny +,label=below left:-]{};
 \draw(star.north east)--(star.south west)(star.south east)--(star.north west);
 \draw[<-](star.west)--++(-10mm,0)node[above,midway]{\tiny R(s)};
 \node(GS1)[nonterminal,right=of star]{\tiny $G_c(s)$};
 \draw[->](star.east)--(GS1.west)node[midway,above]{\tiny E(s)};
 \node(GS2)[nonterminal,right=of GS1]{\tiny $G_p(s)$};
 \draw[->](GS1.east)--(GS2.west)node[midway,above]{\tiny M(s)};
 \draw[->](GS2.east)--++(10mm,0)node[near end,above]{\tiny C(s)};
 \node(feadback)[nonterminal,below=of GS1]{\tiny H(s)};
 \draw[->]($(GS2.east)+(5mm,0)$)coordinate(a)|-(feadback.east);
 \filldraw(a)circle(2pt);
 \draw[->](feadback.west)-|(star.south)node[near end,right]{\tiny B(s)};
 \end{tikzpicture}
 \end{block}
 \begin{block}{传递函数}
 \[C(s)=E(s)G_c(s)G_p(s)\]
 \[B(s)=C(s)H(s)\]
 \[E(s)=R(s)-B(s)\]
 \[G(s)=\frac{G_c(s)G_p(s)}{1+G_c(s)G_p(s)H(s)}\]
 \end{block}
 \end{frame}
 \begin{frame}
 \begin{block}{方框图的画法}
 \begin{itemize}
 \item 找出被控对象
 \item 明确输入输出关系
 \item 找出控制环节
 \item 根据信号流连接各个环节
 \end{itemize}
 \end{block}
 \end{frame}
 \begin{frame}
 \begin{example}
 \begin{circuitikz}[american]
\draw(0,0)to[generic,l^=$R$,o-*](3,0)to[C,l=$C$](3,-2)to[short,*-o](0,-2)to[open,o-,l=$v_{1}(t)$](0,0)(3,0)to[short,-o](5,0)to[open,-o,l=$v_{o}(t)$](5,-2)to[short](3,-2);
\end{circuitikz}
\end{example}
 \end{frame}
\begin{frame}
\begin{block}{绘制电阻R控制图}
\begin{tikzpicture}
[node distance=10mm,>=stealth]
\node(start)[circleterminal,label=above left:\tiny +,label=below left:-]{};
\draw(start.north west)--(start.south east);
\draw(start.north east)--(start.south west);
\draw[<-](start.west)--++(-10mm,0)node[midway,above]{\tiny $V_1(s)$};
\node(R)[nonterminal,right=of start]{\tiny $\frac{1}{R}$};
\draw[->](start.east)--(R.west);
\draw[->](R.east)--++(10mm,0)coordinate(a)node[above]{\tiny $I(s)$};
\draw($(a)+(0,-10mm)$)-|(start.south)node[very near start,above]{\tiny $V_o(s)$}; 
\end{tikzpicture}
\end{block}
\begin{block}{加入电容C的控制图}
\begin{tikzpicture}
[node distance=10mm,>=stealth]
\node(start)[circleterminal,label=above left:\tiny +,label=below left:-]{};
\draw(start.north west)--(start.south east);
\draw(start.north east)--(start.south west);
\draw[<-](start.west)--++(-10mm,0)node[midway,above]{\tiny $V_1(s)$};
\node(R)[nonterminal,right=of start]{\tiny $\frac{1}{R}$};
\draw[->](start.east)--(R.west);
\node(C)[nonterminal,right=of R]{\tiny $\frac{1}{Cs}$};
\draw[->](R.east)--(C.west);
\draw[->](C.east)--++(10mm,0)coordinate(a)node[very near end,above]{\tiny $V_o(s)$};
\draw[->]($(C.east)!.5!(a)$)coordinate(b)--++(0,-10mm)-|(start.south);
\filldraw(b)circle(2pt);
\end{tikzpicture}
\end{block}
\end{frame}
\begin{frame}
\begin{example}
\begin{circuitikz}[american]
 \draw(0,0)to[generic,l^=$R$,o-](2,0)to[L,l^=$L$](4,0)to[C,l=$C$,*-*](4,-2)to[short,-o](0,-2)to[open,l=$v_{1}(t)$](0,0)(4,0)to[short,-o](6,0)to[open,l=$v_{o}(t)$,-o](6,-2)to[short](4,-2);
 \end{circuitikz}
\end{example}
\end{frame}
\begin{frame}
\begin{block}{绘制电阻R控制图}
\begin{tikzpicture}
[node distance=10mm,>=stealth]
\node(start)[circleterminal,label=above left:\tiny +,label=below left:-]{};
\draw(start.north west)--(start.south east);
\draw(start.north east)--(start.south west);
\draw[<-](start.west)--++(-10mm,0)node[midway,above]{\tiny $V_1(s)$};
\node(R)[nonterminal,right=of start]{\tiny $\frac{1}{Ls+R}$};
\draw[->](start.east)--(R.west);
\draw[->](R.east)--++(10mm,0)coordinate(a)node[above]{\tiny $I(s)$};
\draw($(a)+(0,-10mm)$)-|(start.south)node[very near start,above]{\tiny $V_o(s)$}; 
\end{tikzpicture}
\end{block}
\begin{block}{加入电容C的控制图}
\begin{tikzpicture}
[node distance=10mm,>=stealth]
\node(start)[circleterminal,label=above left:\tiny +,label=below left:-]{};
\draw(start.north west)--(start.south east);
\draw(start.north east)--(start.south west);
\draw[<-](start.west)--++(-10mm,0)node[midway,above]{\tiny $V_1(s)$};
\node(R)[nonterminal,right=of start]{\tiny $\frac{1}{Ls+R}$};
\draw[->](start.east)--(R.west);
\node(C)[nonterminal,right=of R]{\tiny $\frac{1}{Cs}$};
\draw[->](R.east)--(C.west);
\draw[->](C.east)--++(10mm,0)coordinate(a)node[very near end,above]{\tiny $V_o(s)$};
\draw[->]($(C.east)!.5!(a)$)coordinate(b)--++(0,-10mm)-|(start.south);
\filldraw(b)circle(2pt);
\end{tikzpicture}
\end{block}
\end{frame}
\begin{frame}
\begin{block}{最终结果}
\begin{tikzpicture}
[node distance=8mm,>=stealth]
\node(start1)[circleterminal,label=above left:\tiny +,label=below left:-]{};
\draw(start1.north west)--(start1.south east);
\draw(start1.north east)--(start1.south west);
\draw[<-](start1.west)--++(-10mm,0)node[midway,above]{\tiny $V_1(s)$};
\node(L)[nonterminal,right=of start1]{\tiny $\frac{1}{L}$};
\draw[->](start1.east)--(L.west);
\node(start2)[circleterminal,label=above left:\tiny +,label=below left:-,right=of L]{};
\draw(start2.north west)--(start2.south east);
\draw(start2.north east)--(start2.south west);
\draw[->](L.east)--(start2.west);
\node(S)[nonterminal,right=of start2]{\tiny $\frac{1}{s}$};
\draw[->](start2.east)--(S.west);
\node(C)[nonterminal,right=of S]{\tiny $\frac{1}{C}$};
\draw[->](S.east)--(C.west);
\node(fback)[nonterminal,below=of S]{\tiny $\frac{R}{L}$};
\draw($(S.east)!.5!(C.west)$)coordinate(a)node[above]{\tiny $I(s)$}|-(fback.east);
\draw[->](fback.west)-|(start2.south);
\filldraw(a)circle(2pt);
\node(S1)[nonterminal,right=of C]{\tiny $\frac{1}{s}$};
\draw[->](C.east)--(S1.west);
\draw[->](S1.east)--++(12mm,0)coordinate(b);
\draw[->]($(S1.east)!.7!(b)$)coordinate(c)node[above]{\tiny $V_2(s)$}--++(0,-25mm)-|(start1.south);
\filldraw(c)circle(2pt);
\end{tikzpicture}
\end{block}
\end{frame}