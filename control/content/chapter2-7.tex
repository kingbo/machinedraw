%%%%%%%%%%%%%%教案头%%%%%%%%%%%%%%%%%%%%%%%%%%%%%%%
\mode<article>{

\begin{longtable}{|m{20mm}|m{20mm}|m{20mm}|m{20mm}|m{20mm}|m{28mm}|}
\caption*{\huge 教案头}\\
\hline
\endfirsthead
\multicolumn{6}{l}{(续表)}\\
\hline
\endhead
\hline
\multicolumn{6}{l}{\itshape 接下一页表格.......}\\ [2ex]
\endfoot
\hline
\endlastfoot
\centering{授课单元}&\multicolumn{3}{m{60mm}|}{\centering 2.7根轨迹}&\centering{授课日期}&2014年04月28日 \\
\hline
\centering 授课地点 & \multicolumn{3}{m{60mm}|}{B6-204}&\centering 授课学时 & 2 \\
\hline
& \multicolumn{2}{m{40mm}|}{能力目标} & \multicolumn{2}{m{40mm}|}{知识目标}&素质目标 \\
\cline{2-6}
\centering 教学目标&\multicolumn{2}{m{40mm}|}{\begin{enumerate}
\item 能够分析系统的根轨迹
\end{enumerate} }&\multicolumn{2}{m{40mm}|}{\begin{enumerate}
\item 掌握根轨迹法
\item 掌握根轨迹的绘制要点
\end{enumerate}} & {\qquad}\\
\hline
\centering 能力训练任务或案例 &\multicolumn{5}{m{108mm}|}{ }\\
\hline
\centering 教学重点 & \multicolumn{5}{m{108mm}|}{\begin{enumerate}
\item 根轨迹的绘制
\end{enumerate}}\\
\hline
\centering 教学难点与解决办法 &\multicolumn{5}{m{108mm}|}{\begin{enumerate}
\item 难点:根轨迹的绘制
\item 解决方法:实例讲解
\end{enumerate}}\\
\hline
\centering 德育内容 &\multicolumn{5}{m{108mm}|}{无}\\
\hline
 &教材 & \multicolumn{4}{m{88mm}|}{计算机控制原理与应用}\\
\cline{2-6}& 教学资源 &\multicolumn{4}{m{88mm}|}{PPT}\\
\cline{2-6}\centering 使用的教学材料& 主要教学仪器设备和工具等 &\multicolumn{4}{m{88mm}|}{投影机、MATLAB}\\
\cline{2-6}& 主要耗材 &\multicolumn{4}{m{88mm}|}{无}\\
\hline
\centering 教学模式 &\multicolumn{2}{m{40mm}|}{知识讲授}&\centering 教学手段 &\multicolumn{2}{m{48mm}|}{多媒体教学}\\
\hline
\centering 学生成果与过程考核方式 &\multicolumn{5}{m{108mm}|}{无}
\end{longtable}
\clearpage

%%%%%%%%%%%%%%%教学实施过程%%%%%%%%%%%%%%%%%%%%%%%%%%%%
\begin{landscape}

\begin{longtable}{|m{10mm}|m{50mm}|m{50mm}|m{50mm}|m{15mm}|}
\caption*{\huge 教学组织与实施}\\
\hline
\endfirsthead
\multicolumn{5}{l}{\small 接上页}\\
\hline
\multicolumn{1}{|c|}{步骤}&\multicolumn{1}{c|}{教学内容}&\multicolumn{1}{c|}{教师活动}&\multicolumn{1}{c|}{学生活动}&\multicolumn{1}{c|}{时间}\\
\hline
\endhead

\multicolumn{5}{r}{\small 接下页}\\
\endfoot
\hline
\endlastfoot
\multicolumn{1}{|c|}{步骤}&\multicolumn{1}{c|}{教学内容}&\multicolumn{1}{c|}{教师活动}&\multicolumn{1}{c|}{学生活动}&\multicolumn{1}{c|}{时间}\\\hline
讲解&\begin{enumerate}
\item 根轨迹法
\end{enumerate} &\begin{enumerate}
\item 讲解讲解根轨迹法的理论
\end{enumerate} &\begin{enumerate}
\item 学生倾听并记录
\end{enumerate} &20 \\\hline
讲解&\begin{enumerate}
\item 根轨迹图的绘制
\end{enumerate}
 &\begin{enumerate}
\item 讲解根轨迹图的绘制方法
\end{enumerate} &\begin{enumerate}
\item 学生倾听并记录
\end{enumerate} &25 \\\hline
讲解&\begin{enumerate}
\item 用MatLab绘制根轨迹图
\end{enumerate}
&\begin{enumerate}
\item 讲解用MatLab绘制根轨迹图
\end{enumerate} &\begin{enumerate}
\item 学生倾听并记录
\end{enumerate} &10 \\\hline
讲解&\begin{enumerate}
\item MatLab根轨迹图绘制实践
\end{enumerate}
 &\begin{enumerate}
\item 指导学生用MatLab绘制根轨迹图
\end{enumerate} &\begin{enumerate}
\item 学生用MatLab绘制根轨迹图
\end{enumerate} &30 \\\hline

\centering 本次课总结(评价)&总结本课程内容 &进行知识总结 &学生倾听 &5 \\\hline
\centering 学生学习笔记或工单等检查情况&\multicolumn{4}{m{165mm}|}{\quad}\\\hline
\centering 课后作业&\multicolumn{4}{m{165mm}|}{2-28,2-29}\\\hline
\centering 教学体会&\multicolumn{4}{m{165mm}|}{\quad}\\
\end{longtable}

\end{landscape}
\clearpage
%%%%%%%%%%%%%%%%%%%%板书设计%%%%%%%%%%%%%%%%%%%%%%%%%%
\begin{center}
{\huge 板书设计}
\end{center}
}
\mode<presentation>{ \section{根轨迹}
 \subsection{根轨迹}}
 \begin{frame}{根轨迹法} 
 \begin{block}{开环传递函数展开式}
 \begin{eqnarray*}
 \Phi(s)&=&\frac{G(s)}{1+G(s)H(s)}=\frac{G(s)}{1+G_o(s)}\\
 G_o(s)&=&G(s)H(s)=\frac{K\prod\limits_{i=1}^m(\tau s-1)}{\prod\limits_{j=1}^{n}(T_js-1)}
 \end{eqnarray*}
 \end{block}
 \end{frame}
 \begin{frame}
 \begin{block}{}
 \[=K_g\frac{\prod\limits_{i=1}^m(s-z_i)}{\prod\limits_{j=1}^{n}(s-p_j)}=K_g\frac{N_o(s)}{D_o(s)}\]
 令$s=0$得:
 \[K=K_g\frac{\prod\limits_{i=1}^mz_i}{\prod\limits_{j=1}^np_j}\]
 \end{block}
 \end{frame}
 
 \begin{frame}
 \begin{block}{根轨迹方程}
 \begin{eqnarray*}
K_g\frac{\prod\limits_{i=1}^m(s-z_i)}{\prod\limits_{j=1}^n(s-p_j)}=-1
\end{eqnarray*}
由:
\[-1^{\pm j(2k+1)\pi},k=0,1,2,\cdots\]
\end{block}
\end{frame}

\begin{frame}
\begin{block}{根轨迹方程幅角式}
 \begin{eqnarray*}
\left|G_o(s)\right|=\left|K_g\frac{\prod\limits_{i=1}^m(s-z_i)}{\prod\limits_{j=1}^n(s-p_j)}\right|=K_g\frac{\prod\limits_{i=1}^m\left|s-z_i\right|}{\prod\limits_{j=1}^n\left|s-p_j\right|}=1\\
\underline{\diagup G_o(s)}=\sum\limits_{i=1}^m\underline{\diagup s-z_i}-\sum\limits_{j=1}^n\underline{\diagup s-p_j}\\
=\pm 180^o(2k+1),k=0,1,2,\cdots
\end{eqnarray*}
\end{block}
\end{frame}
