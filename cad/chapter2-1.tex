\section{电气元件}\label{sec:dianqiyuanjian}

{\bfseries 知识目标}
\begin{itemize}
\item 掌握图块的概念
\item 掌握图块的定义、插入、输出方法
\item 掌握图块属性定义与设置知识
\end{itemize}

{\bfseries 技能目标}
\begin{itemize}
\item 能够完成电子元件的定制
\end{itemize}

本任务以绘制\ref{fig:zhumodianlu}所示电路图中的电子元件为目标,主要是帮助读者掌握AutoCAD的图块的概念,以便于在绘图过程中将大量重复的图形定义为图块,以提高图形的绘制速度。

\subsection{电容元件}
首先,让我们来绘制图\ref{fig:zhumodianlu}所示电路图中的电容元件。

第一步,先绘制电容元件的对称中心线。

\noindent
\begin{figure}[htbp]
\centering
\subfloat[]{\label{fig:dianrong1}\includegraphics[scale=0.3]{dianrong1.png}}\hspace{20pt}
\subfloat[]{\label{fig:dianrong2}\includegraphics[scale=0.5]{dianrong2.png}}\hspace{20pt}
\subfloat[]{\label{fig:dianrong3}\includegraphics[scale=0.5]{dianrong3.png}}
\subfloat[]{\label{fig:dianrong4}\includegraphics[scale=0.5]{dianrong4.png}}
\caption{电容元件绘制}
\end{figure}
\noindent

命令:line\\
指定第一点:\\
指定下一点或 [放弃(U)]: @6$<90$\\
指定下一点或 [放弃(U)]:\\
\indent

第二步,以对称中心做为偏移对象,向左右各偏移2mm,如图\ref{fig:dianrong2} 所示。

\noindent
命令: offset
当前设置: 删除源=否  图层=源  OFFSETGAPTYPE=0\\
指定偏移距离或 [通过(T)/删除(E)/图层(L)] $<$通过$>$:  2\\
选择要偏移的对象,或 [退出(E)/放弃(U)] $<$退出$>$:\\
指定要偏移的那一侧上的点,或 [退出(E)/多个(M)/放弃(U)] $<$退出$>$:\\
选择要偏移的对象,或 [退出(E)/放弃(U)] $<$退出$>$:\\
指定要偏移的那一侧上的点,或 [退出(E)/多个(M)/放弃(U)] $<$退出$>$:\\
选择要偏移的对象,或 [退出(E)/放弃(U)] $<$退出$>$:\\
\indent

第三步,绘电容接线时,以其最左侧竖线的中心点为起点向左绘制10mm的线条,然后以中心线做另一电容接线,最终结果如图\ref{fig:dianrong3}所示。

\noindent
命令: line\\
指定第一点: mid\\
于\\
指定下一点或 [放弃(U)]: @10$<$-180\\
指定下一点或 [放弃(U)]:\\
命令: mirror\\
选择对象: 找到 1 个\\
选择对象:  指定镜像线的第一点: 指定镜像线的第二点:\\
要删除源对象吗?[是(Y)/否(N)] $<N>:$

\indent
最后,删除中心线,完成电容元件的绘制,其结果如图\ref{fig:dianrong4}所示。

由于电器元件通常都需要重复使用,为提高绘图速度,我们需要将绘制的电气元件定义为块并写入文件中,才能够实现其重复利用的目的。下面我们将通过电容块的定义来说明块的定义过程。当我们输入block命令后,会出现图\ref{fig:kuaidingyi}所示的块定义窗体。我们在窗体中命名块的名称为电容,点击拾取点并选择电容元件的最左点来完成基点的定义,点击选择对象并选整个电容图形来完成对象的定义,最后点击确定。

\noindent
命令: block 指定插入基点:\\
选择对象: 指定对角点: 找到 4 个\\
选择对象:\\

\indent
完成块定义后,还需要通过块写命令wblock将定义的块写入dwg文件中才能够实现随时调用和重复利用。输入wblock命令后,会出现图\ref{fig:kuaixie}所示对话框,选择块并选择好块文件保存的路径即可完成块写操作。当块写为文件后就可以在新的CAD文件中调用,真正实现一处定义多处使用,提高绘图效率。

\noindent
\begin{figure}[htbp]
\centering
\begin{floatrow}
\ffigbox{\caption{块定义}\label{fig:kuaidingyi}}{
\includegraphics[scale=0.4]{kuaidingyi.png}
}
\ffigbox{\caption{块写}\label{fig:kuaixie}}{
\includegraphics[scale=0.35]{kuaixie.png}
}
\end{floatrow}
\end{figure}
\indent

\subsection{接地元件}
接地元件的绘制比较简单,先绘制一根做为基础,接下来通过偏移得到另外两根,通过缩放将偏移得两的线调整为相应的长度即可,并最长的一根上绘制连接线,然后将其定义为块并写入文件,如图\ref{fig:jiedi}所示。

\noindent
\begin{figure}[htbp]
\centering
\subfloat[]{\label{fig:jiedicicu}\includegraphics[scale=0.6]{jiedicicun.png}}\hspace{30pt}
\subfloat[]{\label{fig:jiedi1}\includegraphics[scale=1]{suofang.png}}\hspace{30pt}
\subfloat[]{\label{fig:jiedi}\includegraphics[scale=1]{jiedixian.png}}
\caption{接地元件绘制}
\end{figure}

\noindent
命令: line \\
指定第一点:\\
指定下一点或 [放弃(U)]: $@6<0$\\
指定下一点或 [放弃(U)]:\\
命令: offset\\
当前设置: 删除源=否  图层=源  OFFSETGAPTYPE=0\\
指定偏移距离或 [通过(T)/删除(E)/图层(L)] $<$通过$>$:  3\\
选择要偏移的对象,或 [退出(E)/放弃(U)]$<$退出$>$:\\
指定要偏移的那一侧上的点,或 [退出(E)/多个(M)/放弃(U)]$<$退出$>$:\\
选择要偏移的对象,或 [退出(E)/放弃(U)] $<$退出$>$:\\
指定要偏移的那一侧上的点,或 [退出(E)/多个(M)/放弃(U)]$<$退出$>$:\\
选择要偏移的对象,或 [退出(E)/放弃(U)] $<$退出$>$:\\
令: scale\\
选择对象: 找到 1 个\\
选择对象:\\
指定基点: mid\\
于\\
指定比例因子或 [复制(C)/参照(R)]: r\\
指定参照长度 $<$1.0000$>$: 10\\
指定新的长度或 [点(P)] $<$1.0000$>$:  5\\
令: scale\\
选择对象: 找到 1 个\\
选择对象:\\
指定基点: mid\\
于\\
指定比例因子或 [复制(C)/参照(R)]: r\\
指定参照长度 $<$10.0000$>$: 10\\
指定新的长度或 [点(P)]$ <5.0000>$:  2\\
命令: line \\
指定第一点: mid\\
于\\
指定下一点或 [放弃(U)]: $@10<90$\\
指定下一点或 [放弃(U)]:\\
\indent
\subsection{电阻元件}
电阻元件的绘制方法是:先绘制一个长10宽4的矩形,然后再绘制两根长10的直线,具体尺寸和结果如图\ref{fig:dianzucicu}和\ref{fig:dianzu}所示。

\noindent
\begin{figure}[htbp]
\centering
\subfloat[]{\label{fig:dianzucicu}\includegraphics[scale=0.4]{dianzucichun.png}}\hspace{30pt}
\subfloat[]{\label{fig:dianzu}\includegraphics[scale=0.5]{dianzu.png}}
\caption{电阻元件绘制}
\end{figure}

\noindent
命令: rectang\\
指定第一个角点或 [倒角(C)/标高(E)/圆角(F)/厚度(T)/宽度(W)]:0,0\\
指定另一个角点或 [面积(A)/尺寸(D)/旋转(R)]: @10,4\\
命令:line 指定第一点: mid\\
于\\
指定下一点或 [放弃(U)]:$ @10<180$\\
指定下一点或 [放弃(U)]:\\
命令: mirror\\
选择对象: 找到 1 个\\
选择对象:  指定镜像线的第一点: mid\\
于 指定镜像线的第二点: mid\\
于\\
要删除源对象吗?[是(Y)/否(N)] $<N>:$\\

\subsection{电灯元件}
电灯元件的绘制过程:先绘制圆,接下来将其等分为8等分,然后绘制两条相交直线,最后绘制电源连接线即可。电灯元件的具体尺寸和结果如图\ref{fig:diandengcicu}和\ref{fig:diandeng3}所示。

\noindent
\begin{figure}[htbp]
\centering
\subfloat[]{\label{fig:diandengcicu}\includegraphics[scale=0.3]{diandeng.png}}\hspace{30pt}
\subfloat[]{\label{fig:diandeng1}\includegraphics[scale=0.3]{diandeng1.png}}\hspace{30pt}
\subfloat[]{\label{fig:diandeng2}\includegraphics[scale=0.3]{diandeng2.png}}\hspace{30pt}
\subfloat[]{\label{fig:diandeng3}\includegraphics[scale=0.4]{diandeng3.png}}
\caption{电灯元件绘制}
\end{figure}

\noindent
命令: circle\\
 指定圆的圆心或 [三点(3P)/两点(2P)/切点、切点、半径(T)]:
指定圆的半径或 [直径(D)]: 6\\
命令: divide\\
选择要定数等分的对象:\\
输入线段数目或 [块(B)]: 8\\
命令: line 指定第一点: node\\
于\\
指定下一点或 [放弃(U)]: node\\
于\\
指定下一点或 [放弃(U)]:\\
命令:  LINE 指定第一点: node\\
于\\
指定下一点或 [放弃(U)]: node\\
于\\
命令: line\\
指定第一点: node\\
于\\
指定下一点或 [放弃(U)]: $@10<0$\\
指定下一点或 [放弃(U)]:\\
命令: mirror\\
选择对象: 找到 1 个\\
选择对象:  指定镜像线的第一点: node\\
于\\
指定镜像线的第二点: node\\
于\\
要删除源对象吗?[是(Y)/否(N)] $<N>:$
\subsection{二极管元件}
二极元件绘制过程:先绘制等边三角形,再绘制截止线,最后绘制电线。具体尺寸和结果如图所示。

\noindent
\begin{figure}[htbp]
\centering
\subfloat[]{\label{fig:erjiguancicu}\includegraphics[scale=0.6]{erjiguancicu.png}}\hspace{30pt}
\subfloat[]{\label{fig:erjiguan}\includegraphics[scale=1]{erjiguan1.png}}
\caption{二极管元件绘制}
\end{figure}

\noindent
命令:polygon 输入侧面数$ <4>$:3
指定正多边形的中心点或 [边(E)]: e
指定边的第一个端点:0,0
指定边的第二个端点: $@6<270$
命令: line \\
指定第一点: mid\\
于\\
指定下一点或 [放弃(U)]: $@10<180$\\
指定下一点或 [放弃(U)]:\\
命令: line \\
指定第一点:end\\
于\\
指定下一点或 [放弃(U)]:end\\
于\\
指定下一点或 [放弃(U)]:$ @5<90$\\
指定下一点或 [闭合(C)/放弃(U)]:\\
命令:  LINE \\
指定第一点:end\\
于\\
指定下一点或 [放弃(U)]: $@5<-90$\\
指定下一点或 [放弃(U)]:\\
命令: line\\
指定第一点:end\\
指定下一点或 [放弃(U)]: $@10<0$\\
指定下一点或 [放弃(U)]:\\

\subsection{三极管元件}
三极管元件绘制过程:先绘一等边三角形,接下来将一这进行偏移,进行修剪得到内部结构,再用pline绘制箭头,最后绘制边接线。其尺寸及结果如图所示。

\noindent
\begin{figure}[htbp]
\centering
\subfloat[]{\label{fig:sanjiguancicu}\includegraphics[scale=0.4]{sanjiguancicu.png}}\hspace{30pt}
\subfloat[]{\label{fig:sanjiguanPNP}\includegraphics[scale=0.4]{sanjiguanPNP.png}}\hspace{30pt}
\subfloat[]{\label{fig:sanjiguanNPN}\includegraphics[scale=0.4]{sanjiguanNPN.png}}
\caption{三极管元件绘制}
\end{figure}

\noindent
命令: polygon \\
输入侧面数 $<3>$:\\
指定正多边形的中心点或 [边(E)]: e\\
指定边的第一个端点:0,0\\
指定边的第二个端点: $@6<-30$\\
命令: explode\\
选择对象: 找到 1 个\\
选择对象:\\
命令: offset\\
当前设置: 删除源=否  图层=源  OFFSETGAPTYPE=0
指定偏移距离或 [通过(T)/删除(E)/图层(L)] $<$通过$>$: 3\\
选择要偏移的对象,或 [退出(E)/放弃(U)] $<$退出$>$:\\
指定要偏移的那一侧上的点,或 [退出(E)/多个(M)/放弃(U)] $<$退出$>$:\\
选择要偏移的对象,或 [退出(E)/放弃(U)] $<$退出$>$:\\
命令: trim\\
当前设置:投影=UCS,边=无\\
选择剪切边...\\
选择对象或 $<$全部选择$>$:  找到 1 个
选择对象:
选择要修剪的对象,或按住 Shift 键选择要延伸的对象,或
[栏选(F)/窗交(C)/投影(P)/边(E)/删除(R)/放弃(U)]:\\
选择要修剪的对象,或按住 Shift 键选择要延伸的对象,或
[栏选(F)/窗交(C)/投影(P)/边(E)/删除(R)/放弃(U)]:\\
选择要修剪的对象,或按住 Shift 键选择要延伸的对象,或
[栏选(F)/窗交(C)/投影(P)/边(E)/删除(R)/放弃(U)]:\\
命令: erase 找到 1 个\\
命令: line 指定第一点: mid\\
于
指定下一点或 [放弃(U)]: $@10<180$\\
指定下一点或 [放弃(U)]:\\
命令: line\\
指定第一点: end\\
于\\
指定下一点或 [放弃(U)]:$ @10<90$\\
指定下一点或 [放弃(U)]:\\
命令: line\\
指定第一点:end\\
指定下一点或 [放弃(U)]:$ @10<-90$
指定下一点或 [放弃(U)]:\\
命令: PLINE
指定起点:end\\
当前线宽为 0.0000\\
指定下一个点或 [圆弧(A)/半宽(H)/长度(L)/放弃(U)/宽度(W)]: w\\
指定起点宽度$ <$0.0000$>$:
指定端点宽度 $<$0.0000$>$: 1.5
指定下一个点或 [圆弧(A)/半宽(H)/长度(L)/放弃(U)/宽度(W)]:$ @1.5<150$
指定下一点或 [圆弧(A)/闭合(C)/半宽(H)/长度(L)/放弃(U)/宽度(W)]:\\

\zhishi{绘图命令解析}
\subsection{mirror}
mirror命令用于创建选定对象的副本。参数[是(Y)/否(N)]表示是否要删除源对象,若选是则为删除源对象,否则为不删除,默认情况为否。
\subsection{block}
block命令用于从选定对象中创建一个块定义。所谓块,就是将一个或多个对象结合起来以创建一个对象集并形成一个结合对象。
\subsection{scale}
scale命令用于放大或缩小选定对象,使用后对象的比例保持不变。对于要放大或缩小的对象,在进行缩放时要指缩放时所需要基点和比例因子,当比例因子大于1时为放大选定对象,比例因子介于0到1之间时为缩小选定对象。scale命令中各个选项的含义如下:
\begin{itemize}
\item 复制(C):创建选定缩放对象的副本。
\item 参照(R):按参照长度和新长度缩放所选对象。
\end{itemize}
\subsection{divide}
divide命令用于创建沿对象的长度或周长等间隔排列的点对象或块。divide命令的选项含义如下:
\begin{itemize}
\item 块(B):沿选定对象等间距放置块。如果块具有可变属性,插入的块中将不包含这些属性。 
\begin{itemize}
\item 是:指定插入块的 X 轴方向与定数等分对象在等分点相切或对齐。
\item 否:按其法线方向对齐块。
\end{itemize} 
\end{itemize}
\subsection{explode}
explode命令用于选定对象的分解操作。分解即是将复杂对象分解为简单对象。该命令常用于块对象的操作,并且该操作不会更改块的定义内容。
\subsection{pline}
pline命令用于创建二维多段线,它是由直线段和圆弧段组成的单个对象。pline命令中各选项的含义:
\begin{itemize}
\item 圆弧(A):将圆弧段添加到多段线中。
\begin{itemize}
\item 角度(A):指定圆弧段的从起点开始的包含角。
\item 圆心(CE):指定圆弧段的圆心。
\item 关闭(CL):从指定的最后一点到起点绘制圆弧段,从而创建闭合的多段线。必须至少指定两个点才能使用该选项。
\item 方向(D):指定圆弧段的起始方向。
\item 半宽(H):指定从宽多段线线段的中心到其一边的宽度。
\item 直线(L):退出“圆弧”选项并返回初始 PLINE 命令提示。
\item 半径(R):指定圆弧段的半径。 
\item 第二点(S):指定三点圆弧的第二点和端点。
\item 放弃(U):删除最近一次添加到多段线上的圆弧段。 
\item 宽度(W):指定下一圆弧段的宽度。
\end{itemize}
\item 长度(L):在与上一线段相同的角度方向上绘制指定长度的直线段。如果上一线段是圆弧,将绘制与该圆弧段相切的新直线段。
\item 放弃(U):删除最近一次添加到多段线上的直线段。
\item 宽度(W):指定下一条直线段的宽度。
\end{itemize}
\endinput