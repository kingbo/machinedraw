\section{复杂样板图样}

{\bfseries 知识目标}
\begin{itemize}
\item 掌握圆弧连接的绘制方法
\item 掌握图样的流程
\end{itemize}

{\bfseries 技能目标}
\begin{itemize}
\item 能够完成具有圆弧连接的复杂图样
\item 能够按照规范完成复杂图样抄绘
\end{itemize}

通过对本项目中\ref{sec:gongzhi}、\ref{sec:falan}和\ref{sec:yangban}的学习,我已经掌握了能够完成本项目刚开始所展示的图\ref{fig:shangmu1}那样复杂的图样。现在就让我们一步一步来完成这个复杂的图样,用激动人心的成果来结束该项目并检验的学习成果。
\subsection{设置绘图环境}
首先运用\ref{sec:gongzhi}中所学的方法将图形界限设置为$420\times 297$,接下来按照\ref{sec:falan}中所学的图层管理方法,新建绘制图\ref{fig:shangmu1}所需的中心线图层、细实线图层和粗实线图层,设置方法具体参见\ref{sec:gongzhi}和\ref{sec:falan}所述内容,此处不再赘述。
\subsection{绘制图框和标题栏}
根据图幅尺寸绘制图样的的边框:
\begin{lstlisting}
%命令: RECTANG%
%指定第一个角点或 [倒角(C)/标高(E)/圆角(F)/厚度(T)/宽度(W)]: 0,0%
%指定另一个角点或 [面积(A)/尺寸(D)/旋转(R)]: 420,297%
%命令: RECTANG%
%指定第一个角点或 [倒角(C)/标高(E)/圆角(F)/厚度(T)/宽度(W)]: 25,10%
%指定另一个角点或 [面积(A)/尺寸(D)/旋转(R)]: 410,287%
\end{lstlisting}
接下来根据图\ref{fig:biaotilan1}所示的尺寸绘制学生用标题栏。

\begin{lstlisting}
%命令: line 指定第一点: 240,0%
%指定下一点或 [放弃(U)]:$ @40<90$%
%指定下一点或 [放弃(U)]:$ @180<0$%
%指定下一点或 [闭合(C)/放弃(U)]:%
%命令: OFFSET%
%当前设置: 删除源=否  图层=源  OFFSETGAPTYPE=0%
%指定偏移距离或 [通过(T)/删除(E)/图层(L)]$ <$通过$>$:  8%
%选择要偏移的对象,或 [退出(E)/放弃(U)]$ <$退出$>$:%
%指定要偏移的那一侧上的点,或 [退出(E)/多个(M)/放弃(U)]$ <$退出$>$:%
%选择要偏移的对象,或 [退出(E)/放弃(U)]$ <$退出$>$:%
%指定要偏移的那一侧上的点,或 [退出(E)/多个(M)/放弃(U)] $<$退出$>$:%
%选择要偏移的对象,或 [退出(E)/放弃(U)] $<$退出$>$:%
%指定要偏移的那一侧上的点,或 [退出(E)/多个(M)/放弃(U)] $<$退出$>$:%
%选择要偏移的对象,或 [退出(E)/放弃(U)] $<$退出$>$:%
%指定要偏移的那一侧上的点,或 [退出(E)/多个(M)/放弃(U)] $<$退出$>$:%
%选择要偏移的对象,或 [退出(E)/放弃(U)] $<$退出$>$:%
%命令: OFFSET%
%当前设置: 删除源=否  图层=源  OFFSETGAPTYPE=0%
%指定偏移距离或 [通过(T)/删除(E)/图层(L)] $<$8.0000$>$:  15%
%选择要偏移的对象,或 [退出(E)/放弃(U)] $<$退出$>$:%
%指定要偏移的那一侧上的点,或 [退出(E)/多个(M)/放弃(U)]$ <$退出$>$:%
%选择要偏移的对象,或 [退出(E)/放弃(U)] $<$退出$>$:%
%命令:  OFFSET%
%当前设置: 删除源=否  图层=源  OFFSETGAPTYPE=0%
%指定偏移距离或 [通过(T)/删除(E)/图层(L)]$ <$15.0000$>$:  30%
%选择要偏移的对象,或 [退出(E)/放弃(U)] $<$退出$>$:%
%指定要偏移的那一侧上的点,或 [退出(E)/多个(M)/放弃(U)]$ <$退出$>$:%
%选择要偏移的对象,或 [退出(E)/放弃(U)] $<$退出$>$:%
%命令:  OFFSET%
%当前设置: 删除源=否  图层=源  OFFSETGAPTYPE=0%
%指定偏移距离或 [通过(T)/删除(E)/图层(L)] $<$30.0000$>$:  15%
%选择要偏移的对象,或 [退出(E)/放弃(U)] $<$退出$>$:%
%指定要偏移的那一侧上的点,或 [退出(E)/多个(M)/放弃(U)]$ <$退出$>$:%
%选择要偏移的对象,或 [退出(E)/放弃(U)] $<$退出$>$:%
%命令:  OFFSET%
%当前设置: 删除源=否  图层=源  OFFSETGAPTYPE=0%
%指定偏移距离或 [通过(T)/删除(E)/图层(L)]$ <$15.0000$>$:  30%
%选择要偏移的对象,或 [退出(E)/放弃(U)] $<$退出$>$:%
%指定要偏移的那一侧上的点,或 [退出(E)/多个(M)/放弃(U)] $<$退出$>$:%
%选择要偏移的对象,或 [退出(E)/放弃(U)] $<$退出$>$:%
%命令:  OFFSET%
%当前设置: 删除源=否  图层=源  OFFSETGAPTYPE=0%
%指定偏移距离或 [通过(T)/删除(E)/图层(L)] $<$30.0000$>$:  45%
%选择要偏移的对象,或 [退出(E)/放弃(U)] $<$退出$>$:%
%指定要偏移的那一侧上的点,或 [退出(E)/多个(M)/放弃(U)]$ <$退出$>$:%
%选择要偏移的对象,或 [退出(E)/放弃(U)] $<$退出$>$:%
%命令:  OFFSET%
%当前设置: 删除源=否  图层=源  OFFSETGAPTYPE=0%
%指定偏移距离或 [通过(T)/删除(E)/图层(L)]$ <$45.0000$>$:  15%
%选择要偏移的对象,或 [退出(E)/放弃(U)] $<$退出$>$:%
%指定要偏移的那一侧上的点,或 [退出(E)/多个(M)/放弃(U)] $<$退出$>$:%
%选择要偏移的对象,或 [退出(E)/放弃(U)] $<$退出$>$:%
%命令: TRIM%
%当前设置:投影=UCS,边=无%
%选择剪切边...%
%选择对象或 $<$全部选择$>$:  找到 1 个%
%选择对象: 找到 1 个,总计 2 个%
%选择对象:%
%选择要修剪的对象,或按住 Shift 键选择要延伸的对象,或%
%[栏选(F)/窗交(C)/投影(P)/边(E)/删除(R)/放弃(U)]:%
%选择要修剪的对象,或按住 Shift 键选择要延伸的对象,或%
%[栏选(F)/窗交(C)/投影(P)/边(E)/删除(R)/放弃(U)]:%
%选择要修剪的对象,或按住 Shift 键选择要延伸的对象,或%
%[栏选(F)/窗交(C)/投影(P)/边(E)/删除(R)/放弃(U)]:%
%选择要修剪的对象,或按住 Shift 键选择要延伸的对象,或%
%[栏选(F)/窗交(C)/投影(P)/边(E)/删除(R)/放弃(U)]:%
%选择要修剪的对象,或按住 Shift 键选择要延伸的对象,或%
%[栏选(F)/窗交(C)/投影(P)/边(E)/删除(R)/放弃(U)]:%
%选择要修剪的对象,或按住 Shift 键选择要延伸的对象,或%
%[栏选(F)/窗交(C)/投影(P)/边(E)/删除(R)/放弃(U)]:%
%命令:  TRIM%
%当前设置:投影=UCS,边=无%
%选择剪切边...%
%选择对象或 $<$全部选择$>$:  找到 1 个%
%选择对象: 找到 1 个,总计 2 个%
%选择对象:%
%选择要修剪的对象,或按住 Shift 键选择要延伸的对象,或%
%[栏选(F)/窗交(C)/投影(P)/边(E)/删除(R)/放弃(U)]:%
%选择要修剪的对象,或按住 Shift 键选择要延伸的对象,或%
%[栏选(F)/窗交(C)/投影(P)/边(E)/删除(R)/放弃(U)]:%
%选择要修剪的对象,或按住 Shift 键选择要延伸的对象,或%
%[栏选(F)/窗交(C)/投影(P)/边(E)/删除(R)/放弃(U)]:%
%选择要修剪的对象,或按住 Shift 键选择要延伸的对象,或%
%[栏选(F)/窗交(C)/投影(P)/边(E)/删除(R)/放弃(U)]:%
%选择要修剪的对象,或按住 Shift 键选择要延伸的对象,或%
%[栏选(F)/窗交(C)/投影(P)/边(E)/删除(R)/放弃(U)]:%
\end{lstlisting}
\subsection{绘制已经线段图形}

\begin{lstlisting}
%命令: POINT%
%当前点模式:  PDMODE=0  PDSIZE=0.0000%
%指定点:%
%命令: XLINE %
%指定点或 [水平(H)/垂直(V)/角度(A)/二等分(B)/偏移(O)]: @109.5,90%
%指定通过点: @1$<$0%
%指定通过点: @1$<$90%
%指定通过点:%
%命令: CIRCLE %
%指定圆的圆心或 [三点(3P)/两点(2P)/切点、切点、半径(T)]: int 于%
%指定圆的半径或 [直径(D)]: 10%
%命令:  CIRCLE %
%指定圆的圆心或 [三点(3P)/两点(2P)/切点、切点、半径(T)]: int 于%
%指定圆的半径或 [直径(D)] $<$10.0000$>$: 18%
%命令: OFFSET%
%当前设置: 删除源=否  图层=源  OFFSETGAPTYPE=0%
%指定偏移距离或 [通过(T)/删除(E)/图层(L)] $<$通过$>$:  30%
%选择要偏移的对象,或 [退出(E)/放弃(U)] $<$退出$>$:%
%指定要偏移的那一侧上的点,或 [退出(E)/多个(M)/放弃(U)]$<$退出$>$:%
%选择要偏移的对象,或 [退出(E)/放弃(U)] $<$退出$>$:%
%命令:  OFFSET%
%当前设置: 删除源=否  图层=源  OFFSETGAPTYPE=0%
%指定偏移距离或 [通过(T)/删除(E)/图层(L)]$ <$30.0000$>$:  56%
%选择要偏移的对象,或 [退出(E)/放弃(U)] $<$退出$>$:%
%指定要偏移的那一侧上的点,或 [退出(E)/多个(M)/放弃(U)] $<$退出$>$:%
%选择要偏移的对象,或 [退出(E)/放弃(U)] $<$退出$>$:%
%命令: CIRCLE %
%指定圆的圆心或 [三点(3P)/两点(2P)/切点、切点、半径(T)]: int 于%
%指定圆的半径或 [直径(D)]$ <$20.0000$>$: 38%
%命令: offset%
%当前设置: 删除源=否  图层=源  OFFSETGAPTYPE=0%
%指定偏移距离或 [通过(T)/删除(E)/图层(L)] $<56.0000>$:  2%
%选择要偏移的对象,或 [退出(E)/放弃(U)]$ <$退出$>$:%
%指定要偏移的那一侧上的点,或 [退出(E)/多个(M)/放弃(U)]$<$退出$>$:%
%选择要偏移的对象,或 [退出(E)/放弃(U)] $<$退出$>$:%
%命令: TRIM%
%当前设置:投影=UCS,边=无%
%选择剪切边...%
%选择对象或 $<$全部选择$>$:  找到 1 个%
%选择对象:%
%选择要修剪的对象,或按住 Shift 键选择要延伸的对象,或%
%[栏选(F)/窗交(C)/投影(P)/边(E)/删除(R)/放弃(U)]:%
%选择要修剪的对象,或按住 Shift 键选择要延伸的对象,或%
%[栏选(F)/窗交(C)/投影(P)/边(E)/删除(R)/放弃(U)]:%
%选择要修剪的对象,或按住 Shift 键选择要延伸的对象,或%
%[栏选(F)/窗交(C)/投影(P)/边(E)/删除(R)/放弃(U)]:%
%选择要修剪的对象,或按住 Shift 键选择要延伸的对象,或%
%[栏选(F)/窗交(C)/投影(P)/边(E)/删除(R)/放弃(U)]:%
%选择要修剪的对象,或按住 Shift 键选择要延伸的对象,或%
%[栏选(F)/窗交(C)/投影(P)/边(E)/删除(R)/放弃(U)]:%
%命令: erase%
%选择对象: 找到 1 个%
%选择对象:%
%命令: OFFSET%
%当前设置: 删除源=否  图层=源  OFFSETGAPTYPE=0%
%指定偏移距离或 [通过(T)/删除(E)/图层(L)] $<56.0000>$:  203%
%选择要偏移的对象,或 [退出(E)/放弃(U)] $<$退出$>$:%
%指定要偏移的那一侧上的点,或 [退出(E)/多个(M)/放弃(U)]$ <$退出$>$:%
%选择要偏移的对象,或 [退出(E)/放弃(U)] $<$退出$>$:%
%命令:  OFFSET%
%当前设置: 删除源=否  图层=源  OFFSETGAPTYPE=0%
%指定偏移距离或 [通过(T)/删除(E)/图层(L)]$ <203.0000>$:  26%
%选择要偏移的对象,或 [退出(E)/放弃(U)] $<$退出$>$:%
%指定要偏移的那一侧上的点,或 [退出(E)/多个(M)/放弃(U)] $<$退出$>$:%
%选择要偏移的对象,或 [退出(E)/放弃(U)] $<$退出$>$:%
%命令: CIRCLE %
%指定圆的圆心或 [三点(3P)/两点(2P)/切点、切点、半径(T)]: int 于%
%指定圆的半径或 [直径(D)] $<38.0000>$: 12%
%命令:  CIRCLE %
%指定圆的圆心或 [三点(3P)/两点(2P)/切点、切点、半径(T)]: int 于%
%指定圆的半径或 [直径(D)] $<12.0000>$: 22%
%命令: offset%
%当前设置: 删除源=否  图层=源  OFFSETGAPTYPE=0%
%指定偏移距离或 [通过(T)/删除(E)/图层(L)] $<26.0000>$:  2%
%选择要偏移的对象,或 [退出(E)/放弃(U)]$ <$退出$>$:%
%指定要偏移的那一侧上的点,或 [退出(E)/多个(M)/放弃(U)]$<$退出$>$:%
%选择要偏移的对象,或 [退出(E)/放弃(U)] $<$退出$>$:%
%命令: TRIM%
%当前设置:投影=UCS,边=无%
%选择剪切边...%
%选择对象或 $<$全部选择$>$:  找到 1 个%
%选择对象:%
%选择要修剪的对象,或按住 Shift 键选择要延伸的对象,或%
%[栏选(F)/窗交(C)/投影(P)/边(E)/删除(R)/放弃(U)]:%
%选择要修剪的对象,或按住 Shift 键选择要延伸的对象,或%
%[栏选(F)/窗交(C)/投影(P)/边(E)/删除(R)/放弃(U)]:%
%选择要修剪的对象,或按住 Shift 键选择要延伸的对象,或%
%[栏选(F)/窗交(C)/投影(P)/边(E)/删除(R)/放弃(U)]:%
%选择要修剪的对象,或按住 Shift 键选择要延伸的对象,或%
%[栏选(F)/窗交(C)/投影(P)/边(E)/删除(R)/放弃(U)]:%
%选择要修剪的对象,或按住 Shift 键选择要延伸的对象,或%
%[栏选(F)/窗交(C)/投影(P)/边(E)/删除(R)/放弃(U)]:%
%命令: erase%
%选择对象: 找到 1 个%
%选择对象:%
%命令: OFFSET%
%当前设置: 删除源=否  图层=源  OFFSETGAPTYPE=0%
%指定偏移距离或 [通过(T)/删除(E)/图层(L)] $<26.0000>$:  45%
%选择要偏移的对象,或 [退出(E)/放弃(U)] $<$退出$>$:%
%指定要偏移的那一侧上的点,或 [退出(E)/多个(M)/放弃(U)] $<$退出$>$:%
%选择要偏移的对象,或 [退出(E)/放弃(U)] $<$退出$>$:%
%命令:  OFFSET%
%当前设置: 删除源=否  图层=源  OFFSETGAPTYPE=0%
%指定偏移距离或 [通过(T)/删除(E)/图层(L)]$ <45.0000>$:  38%
%选择要偏移的对象,或 [退出(E)/放弃(U)] $<$退出$>$:%
%指定要偏移的那一侧上的点,或 [退出(E)/多个(M)/放弃(U)]$ <$退出$>$:%
%选择要偏移的对象,或 [退出(E)/放弃(U)] $<$退出$>$:%
%命令: CIRCLE %
%指定圆的圆心或 [三点(3P)/两点(2P)/切点、切点、半径(T)]: int 于%
%指定圆的半径或 [直径(D)]$ <$22.0000$>$: 14%
%命令:  CIRCLE %
%指定圆的圆心或 [三点(3P)/两点(2P)/切点、切点、半径(T)]: int 于%
%指定圆的半径或 [直径(D)] $<14.0000>$: 5%
%命令: offset%
%当前设置: 删除源=否  图层=源  OFFSETGAPTYPE=0%
%指定偏移距离或 [通过(T)/删除(E)/图层(L)] $<38.0000>$:  52%
%选择要偏移的对象,或 [退出(E)/放弃(U)]$ <$退出$>$:%
%指定要偏移的那一侧上的点,或 [退出(E)/多个(M)/放弃(U)] $<$退出$>$:%
%选择要偏移的对象,或 [退出(E)/放弃(U)]$ <$退出$>$:%
%命令:  OFFSET%
%当前设置: 删除源=否  图层=源  OFFSETGAPTYPE=0%
%指定偏移距离或 [通过(T)/删除(E)/图层(L)] $<52.0000>$:  100%
%选择要偏移的对象,或 [退出(E)/放弃(U)] $<$退出$>$:%
%指定要偏移的那一侧上的点,或 [退出(E)/多个(M)/放弃(U)] $<$退出$>:$%
%选择要偏移的对象,或 [退出(E)/放弃(U)] $<$退出$>$:%
%命令: CIRCLE%
%指定圆的圆心或 [三点(3P)/两点(2P)/切点、切点、半径(T)]: int 于%
%指定圆的半径或 [直径(D)] <5.0000>: 5%
%命令: OFFSET%
%当前设置: 删除源=否  图层=源  OFFSETGAPTYPE=0%
%指定偏移距离或 [通过(T)/删除(E)/图层(L)] $<100.0000>$:  11%
%选择要偏移的对象,或 [退出(E)/放弃(U)]$ <$退出$>$:%
%指定要偏移的那一侧上的点,或 [退出(E)/多个(M)/放弃(U)]$ <$退出$>$:%
%选择要偏移的对象,或 [退出(E)/放弃(U)]$ <$退出$>$:%
%命令:  OFFSET%
%当前设置: 删除源=否  图层=源  OFFSETGAPTYPE=0%
%指定偏移距离或 [通过(T)/删除(E)/图层(L)] $<$11.0000$>$:  118%
%选择要偏移的对象,或 [退出(E)/放弃(U)] $<$退出$>$:%
%指定要偏移的那一侧上的点,或 [退出(E)/多个(M)/放弃(U)] $<$退出$>$:%
%选择要偏移的对象,或 [退出(E)/放弃(U)] $<$退出$>$:%
%命令: CIRCLE %
%指定圆的圆心或 [三点(3P)/两点(2P)/切点、切点、半径(T)]: int 于%
%指定圆的半径或 [直径(D)] $<5.0000>$: 5%
%命令: CIRCLE %
%指定圆的圆心或 [三点(3P)/两点(2P)/切点、切点、半径(T)]: int 于%
%指定圆的半径或 [直径(D)]$ <5.0000>$: 14%
%命令: OFFSET%
%当前设置: 删除源=否  图层=源  OFFSETGAPTYPE=0%
%指定偏移距离或 [通过(T)/删除(E)/图层(L)]$ <118.0000>$:  36%
%选择要偏移的对象,或 [退出(E)/放弃(U)]$ <$退出$>$:%
%指定要偏移的那一侧上的点,或 [退出(E)/多个(M)/放弃(U)]$ <$退出$>$:%
%选择要偏移的对象,或 [退出(E)/放弃(U)]$ <$退出$>$:%
%命令:  OFFSET%
%当前设置: 删除源=否  图层=源  OFFSETGAPTYPE=0%
%指定偏移距离或 [通过(T)/删除(E)/图层(L)] $<36.0000>$:  20%
%选择要偏移的对象,或 [退出(E)/放弃(U)] $<$退出$>$:%
%指定要偏移的那一侧上的点,或 [退出(E)/多个(M)/放弃(U)] $<$退出$>$:%
%选择要偏移的对象,或 [退出(E)/放弃(U)] $<$退出$>$:%
%命令: CIRCLE %
%指定圆的圆心或 [三点(3P)/两点(2P)/切点、切点、半径(T)]: int 于%
%指定圆的半径或 [直径(D)] $<$14.0000$>$: 14%
%命令:offset%
%当前设置: 删除源=否  图层=源  OFFSETGAPTYPE=0%
%指定偏移距离或 [通过(T)/删除(E)/图层(L)]$ <20.0000>$:  2%
%选择要偏移的对象,或 [退出(E)/放弃(U)] $<$退出$>$:%
%指定要偏移的那一侧上的点,或 [退出(E)/多个(M)/放弃(U)]$ <$退出$>$:%
%选择要偏移的对象,或 [退出(E)/放弃(U)]$ <$退出$>$:%
%命令: trim%
%当前设置:投影=UCS,边=无%
%选择剪切边...%
%选择对象或 $<$全部选择$>$:  找到 1 个%
%选择对象:%
%选择要修剪的对象,或按住 Shift 键选择要延伸的对象,或%
%[栏选(F)/窗交(C)/投影(P)/边(E)/删除(R)/放弃(U)]:%
%选择要修剪的对象,或按住 Shift 键选择要延伸的对象,或%
%[栏选(F)/窗交(C)/投影(P)/边(E)/删除(R)/放弃(U)]:%
%选择要修剪的对象,或按住 Shift 键选择要延伸的对象,或%
%[栏选(F)/窗交(C)/投影(P)/边(E)/删除(R)/放弃(U)]:%
%选择要修剪的对象,或按住 Shift 键选择要延伸的对象,或%
%[栏选(F)/窗交(C)/投影(P)/边(E)/删除(R)/放弃(U)]:%
%选择要修剪的对象,或按住 Shift 键选择要延伸的对象,或%
%[栏选(F)/窗交(C)/投影(P)/边(E)/删除(R)/放弃(U)]:%
%命令: erase%
%选择对象: 找到 1 个%
%选择对象:%
%命令:offset%
%当前设置: 删除源=否  图层=源  OFFSETGAPTYPE=0%
%指定偏移距离或 [通过(T)/删除(E)/图层(L)]$ <2.0000>$: %
%选择要偏移的对象,或 [退出(E)/放弃(U)] $<$退出$>$:%
%指定要偏移的那一侧上的点,或 [退出(E)/多个(M)/放弃(U)]$ <$退出$>$:%
%选择要偏移的对象,或 [退出(E)/放弃(U)]$ <$退出$>$:%
%命令: trim%
%当前设置:投影=UCS,边=无%
%选择剪切边...%
%选择对象或 $<$全部选择$>$:  找到 1 个%
%选择对象:%
%选择要修剪的对象,或按住 Shift 键选择要延伸的对象,或%
%[栏选(F)/窗交(C)/投影(P)/边(E)/删除(R)/放弃(U)]:%
%选择要修剪的对象,或按住 Shift 键选择要延伸的对象,或%
%[栏选(F)/窗交(C)/投影(P)/边(E)/删除(R)/放弃(U)]:%
%选择要修剪的对象,或按住 Shift 键选择要延伸的对象,或%
%[栏选(F)/窗交(C)/投影(P)/边(E)/删除(R)/放弃(U)]:%
%选择要修剪的对象,或按住 Shift 键选择要延伸的对象,或%
%[栏选(F)/窗交(C)/投影(P)/边(E)/删除(R)/放弃(U)]:%
%选择要修剪的对象,或按住 Shift 键选择要延伸的对象,或%
%[栏选(F)/窗交(C)/投影(P)/边(E)/删除(R)/放弃(U)]:%
%命令: erase%
%选择对象: 找到 1 个%
%选择对象:%
%命令:offset%
%当前设置: 删除源=否  图层=源  OFFSETGAPTYPE=0%
%指定偏移距离或 [通过(T)/删除(E)/图层(L)]$ <2.0000>$: %
%选择要偏移的对象,或 [退出(E)/放弃(U)] $<$退出$>$:%
%指定要偏移的那一侧上的点,或 [退出(E)/多个(M)/放弃(U)]$ <$退出$>$:%
%选择要偏移的对象,或 [退出(E)/放弃(U)]$ <$退出$>$:%
%命令: trim%
%当前设置:投影=UCS,边=无%
%选择剪切边...%
%选择对象或 $<$全部选择$>$:  找到 1 个%
%选择对象:%
%选择要修剪的对象,或按住 Shift 键选择要延伸的对象,或%
%[栏选(F)/窗交(C)/投影(P)/边(E)/删除(R)/放弃(U)]:%
%选择要修剪的对象,或按住 Shift 键选择要延伸的对象,或%
%[栏选(F)/窗交(C)/投影(P)/边(E)/删除(R)/放弃(U)]:%
%选择要修剪的对象,或按住 Shift 键选择要延伸的对象,或%
%[栏选(F)/窗交(C)/投影(P)/边(E)/删除(R)/放弃(U)]:%
%选择要修剪的对象,或按住 Shift 键选择要延伸的对象,或%
%[栏选(F)/窗交(C)/投影(P)/边(E)/删除(R)/放弃(U)]:%
%选择要修剪的对象,或按住 Shift 键选择要延伸的对象,或%
%[栏选(F)/窗交(C)/投影(P)/边(E)/删除(R)/放弃(U)]:%
%命令: erase%
%选择对象: 找到 1 个%
%选择对象:%
%命令:offset%
%当前设置: 删除源=否  图层=源  OFFSETGAPTYPE=0%
%指定偏移距离或 [通过(T)/删除(E)/图层(L)]$ <2.0000>$: %
%选择要偏移的对象,或 [退出(E)/放弃(U)] $<$退出$>$:%
%指定要偏移的那一侧上的点,或 [退出(E)/多个(M)/放弃(U)]$ <$退出$>$:%
%选择要偏移的对象,或 [退出(E)/放弃(U)]$ <$退出$>$:%
%命令: trim%
%当前设置:投影=UCS,边=无%
%选择剪切边...%
%选择对象或 $<$全部选择$>$:  找到 1 个%
%选择对象:%
%选择要修剪的对象,或按住 Shift 键选择要延伸的对象,或%
%[栏选(F)/窗交(C)/投影(P)/边(E)/删除(R)/放弃(U)]:%
%选择要修剪的对象,或按住 Shift 键选择要延伸的对象,或%
%[栏选(F)/窗交(C)/投影(P)/边(E)/删除(R)/放弃(U)]:%
%选择要修剪的对象,或按住 Shift 键选择要延伸的对象,或%
%[栏选(F)/窗交(C)/投影(P)/边(E)/删除(R)/放弃(U)]:%
%选择要修剪的对象,或按住 Shift 键选择要延伸的对象,或%
%[栏选(F)/窗交(C)/投影(P)/边(E)/删除(R)/放弃(U)]:%
%选择要修剪的对象,或按住 Shift 键选择要延伸的对象,或%
%[栏选(F)/窗交(C)/投影(P)/边(E)/删除(R)/放弃(U)]:%
%命令: erase%
%选择对象: 找到 1 个%
%选择对象:%
\end{lstlisting}
完成上述绘图过程后,其结果如图\ref{fig:fuzatuyang1}所示。
\begin{figure}[htbp]
\centering
\includegraphics[scale=0.7]{fuzatuyang1.pdf}
\caption{复杂样板图样已知线段}\label{fig:fuzatuyang1}
\end{figure}
\subsection{绘制中间线段图形}
\begin{lstlisting}
%命令: LINE%
%指定第一点: tan 到%
%指定下一点或 [放弃(U)]: $@50<0$%
%指定下一点或 [放弃(U)]:%
%命令: line 指定第一点: tan 到%
%指定下一点或 [放弃(U)]: $@-118<0$%
%指定下一点或 [放弃(U)]:%
%命令: LINE %
%指定第一点: tan 到%
%指定下一点或 [放弃(U)]: $@150<30$%
%指定下一点或 [放弃(U)]:%
%命令: LINE %
%指定第一点: tan 到%
%指定下一点或 [放弃(U)]:$ @30<90$%
%指定下一点或 [放弃(U)]:%
\end{lstlisting}

\subsection{绘制连接线段图形}
\begin{lstlisting}
%命令: fillet%
%当前设置: 模式 = 修剪,半径 = 0.0000%
%选择第一个对象或 [放弃(U)/多段线(P)/半径(R)/修剪(T)/多个(M)]: R %
%指定圆角半径$ <0.0000>$: 45%
%选择第一个对象或 [放弃(U)/多段线(P)/半径(R)/修剪(T)/多个(M)]:%
%选择第二个对象,或按住 Shift 键选择对象以应用角点或 [半径(R)]:%
%命令: fillet%
%当前设置: 模式 = 修剪,半径 = 45.0000%
%选择第一个对象或 [放弃(U)/多段线(P)/半径(R)/修剪(T)/多个(M)]: r%
% 指定圆角半径 $<45.0000>$: 20%
%选择第一个对象或 [放弃(U)/多段线(P)/半径(R)/修剪(T)/多个(M)]:%
%选择第二个对象,或按住 Shift 键选择对象以应用角点或 [半径(R)]:%
%命令: fillet%
%当前设置: 模式 = 修剪,半径 = 20.0000%
%选择第一个对象或 [放弃(U)/多段线(P)/半径(R)/修剪(T)/多个(M)]: r%
%指定圆角半径 $<20.0000>$: 10%
%选择第一个对象或 [放弃(U)/多段线(P)/半径(R)/修剪(T)/多个(M)]:%
%选择第二个对象,或按住 Shift 键选择对象以应用角点或 [半径(R)]:%
%命令: CIRCLE %
%指定圆的圆心或 [三点(3P)/两点(2P)/切点、切点、半径(T)]: t%
%指定对象与圆的第一个切点: tan 到%
%指定对象与圆的第二个切点:%
%指定圆的半径: 150%
%命令: CIRCLE %
%指定圆的圆心或 [三点(3P)/两点(2P)/切点、切点、半径(T)]: t%
%指定对象与圆的第一个切点:%
%指定对象与圆的第二个切点:%
%指定圆的半径 $<150.0000>$: 116%
%命令: trim%
%当前设置:投影=UCS,边=无%
%选择剪切边...%
%选择对象或$ <$全部选择$>$:  找到 1 个%
%选择对象: 找到 1 个,总计 2 个%
%选择对象: 找到 1 个,总计 3 个%
%选择对象: 找到 1 个,总计 4 个%
%选择对象: 找到 1 个,总计 5 个%
%选择对象: 找到 1 个,总计 6 个%
%选择对象:%
%选择要修剪的对象,或按住 Shift 键选择要延伸的对象,或%
%[栏选(F)/窗交(C)/投影(P)/边(E)/删除(R)/放弃(U)]:%
%选择要修剪的对象,或按住 Shift 键选择要延伸的对象,或%
%[栏选(F)/窗交(C)/投影(P)/边(E)/删除(R)/放弃(U)]:%
%选择要修剪的对象,或按住 Shift 键选择要延伸的对象,或%
%[栏选(F)/窗交(C)/投影(P)/边(E)/删除(R)/放弃(U)]:%
%选择要修剪的对象,或按住 Shift 键选择要延伸的对象,或%
%[栏选(F)/窗交(C)/投影(P)/边(E)/删除(R)/放弃(U)]:%
%命令: trim%
%当前设置:投影=UCS,边=无%
%选择剪切边...%
%选择对象或 $<$全部选择$>$:  找到 1 个%
%选择对象: 找到 1 个,总计 2 个%
%选择对象: 找到 1 个,总计 3 个%
%选择对象:%
%选择要修剪的对象,或按住 Shift 键选择要延伸的对象,或%
%[栏选(F)/窗交(C)/投影(P)/边(E)/删除(R)/放弃(U)]:%
%选择要修剪的对象,或按住 Shift 键选择要延伸的对象,或%
%[栏选(F)/窗交(C)/投影(P)/边(E)/删除(R)/放弃(U)]:%
%选择要修剪的对象,或按住 Shift 键选择要延伸的对象,或%
%[栏选(F)/窗交(C)/投影(P)/边(E)/删除(R)/放弃(U)]:%
%命令: trim%
%当前设置:投影=UCS,边=无%
%选择剪切边...%
%选择对象或$ <$全部选择$>$:  找到 1 个%
%选择对象: 找到 1 个,总计 2 个%
%选择对象:%
%选择要修剪的对象,或按住 Shift 键选择要延伸的对象,或%
%[栏选(F)/窗交(C)/投影(P)/边(E)/删除(R)/放弃(U)]:%
%选择要修剪的对象,或按住 Shift 键选择要延伸的对象,或%
%[栏选(F)/窗交(C)/投影(P)/边(E)/删除(R)/放弃(U)]:%

\end{lstlisting}
完成所有的绘图过程后,其结果如\ref{fig:fuzatuyang2}所示。
\begin{figure}[htbp]
\centering
\includegraphics[scale=0.7]{fuzatuyang2.pdf}
\caption{复杂样板图样结果}\label{fig:fuzatuyang2}
\end{figure}
\endinput
