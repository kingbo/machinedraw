\section{V型块三视图}
{\bfseries 知识目标}
\begin{itemize}
\item 掌握三视图形成
\item 掌握三视图绘图规律和对应关系
\item 掌握平面体三视图的规律
\end{itemize}

{\bfseries 技能目标}
\begin{itemize}
\item 能够应用三视图对应关系,使用AutoCAD绘制平面体的三视图
\end{itemize}

图\ref{fig:Vxingkuai}所示的V型块是广泛应用于工业生产中的机械零件的简化形式。V型块在工业生产中主要用于安放轴类零件,以便于找准和划出中心线。本任务以绘制V型块三视图为目标,实现让读者了解并掌握平面体三视图绘制方法和规律的目的,掌握三视图形成原理和对应关系,掌握应用AutoCAD进行三视图绘制的方法和技巧。通过完成该任务,读者最终能够实现应用三视图对应关系,掌握应用AutoCAD绘制平面体三视图的技能。
\noindent
\begin{figure}[htbp]
\centering
\begin{tikzpicture}
\draw[line width=0.7mm](0,0)--(30:50mm)coordinate(a1)--++(0,30mm)coordinate(a2);
\draw[line width=0.7mm](0,0)--(150:20mm)coordinate(a3)--++(0,30mm)coordinate(a4);
\draw[line width=0.7mm](0,0)--(0,30mm)--(a4);
\draw[line width=0.7mm](a2)--++(150:20mm)coordinate(a5);
\draw[line width=0.7mm](a2)--++(30:-10mm)coordinate(a6);
\draw[line width=0.7mm](a5)--++(30:-10mm)coordinate(a7);
\draw[line width=0.7mm](a6)--(a7);
\draw[line width=0.7mm](0,30mm)--++(30:10mm)coordinate(a8);
\draw[line width=0.7mm](a4)--++(30:10mm)coordinate(a9);
\draw[line width=0.7mm](a8)--(a9);
\draw[line width=0.7mm]($(0,0)!.4!(30:50mm)$)++(0,10mm)coordinate(b1)--++(30:10mm)coordinate(b2);
\draw[line width=0.7mm](b1)--++(0,10mm)coordinate(b3)(b2)--++(0,10mm)coordinate(b4);
\draw[line width=0.7mm](a9)--(b3)(a6)--(b4);
\draw[line width=0.7mm](b4)--++(150:20mm)coordinate(b5);
\draw[line width=0.7mm](b5)--(a7);
\draw[line width=0.7mm](b5)--++(0,-10mm)--(b2);
\draw[line width=0.4mm](0,0)--++(30:-10mm)coordinate(c1)(a3)--++(30:-10mm)coordinate(c2);
\draw[<->,line width=0.4mm](c1)++(30:2mm)--++(150:20mm)node[midway,sloped,above]{20};
\draw[line width=0.4mm](b1)--++(30:-10mm)coordinate(c3);
\draw[<->,line width=0.4mm](c3)++(30:2mm)--++(0,-10mm)node[midway,sloped,above]{10};
\draw[line width=0.4mm](b4)--++(30:10mm)coordinate(c4);
\draw[<->,line width=0.4mm](c4)++(30:-2mm)--++(0,-20mm)node[midway,sloped,above]{20};
\draw[line width=0.4mm](a1)--++(30:10mm)coordinate(c5)(a2)--++(30:10mm);
\draw[<->,line width=0.4mm](c5)--++(0,30mm)node[midway,sloped,above]{30};
\draw[line width=0.4mm](b3)--++(0,7mm)coordinate(c6)(b4)--++(0,7mm);
\draw[<->,line width=0.4mm](c6)++(0,-2mm)--++(30:10mm)node[midway,above]{10};
\draw[line width=0.4mm](a9)--++(0,7mm)coordinate(c7)(a7)--++(0,7mm);
\draw[<->,line width=0.4mm](c7)++(0,-2mm)--++(30:30mm)node[midway,above]{30};
\draw[line width=0.4mm](a4)--++(0,17mm)coordinate(c8)(a5)--++(0,17mm);
\draw[<->,line width=0.4mm](c8)++(0,-2mm)--++(30:50mm)node[midway,above]{50};
\end{tikzpicture}
\caption{V型块}\label{fig:Vxingkuai}
\end{figure}

\subsection{绘制长方体的三视图}
通过观察图\ref{fig:Vxingkuai}所示的V型块,我们可以看出该物体是由在一个长方体上面开出了一个V字型的沟槽。本着化繁为简,易于理解的目的,我们暂时先忽略V型块上的V字型沟槽的存在,将其简化为一个长方体,并完成长方体的三视图,然后在此基础上再完成V字型沟槽的三视图,二者合一即可形成最终的V型块三视图。在开始绘制长方体的三视图前,我们需要了解投影、投影法、正投影、视图和三视图等概念的含义。
\subsubsection{投影}
物体在阳光或灯光下所生的具有影子就称为投影。由于影子只显示了物体的外部轮廓,不能够体现物体内外细节,不具备工程实际价值。在工程实际中,采用不同的线型将物体的内外空间几何元素加以抽像,使之能够具体地反映物体的内外细节,从而形成比较完备的、实用的投影法。投影法主要分为中心投影法和平行投影法两类。
\subsubsection{平行投影法}
中心投影法的所有投影线都不是平行的,主要用于绘制效果比较逼真的建筑或产品立体图,故在此不作讲解,有兴趣的读者可查阅相关书籍深入学习。

图\ref{pingxingtouyin}所示即为平行投影法,从图中可以看出,平行投影法是将光源置于无穷远处,使所有的投影线之间为平行关系的一种投影方法。若投影线与投影面垂直称为正投影法;若投影线与投影面倾斜的称之为斜投影法。平行投影法中,正投影法主要应用于工程图样的绘,斜投影主要用于绘制立体图形。
\begin{figure}[htbp]
\centering
\subfloat[斜投影法]{\label{fig:xietouyinfa}
\begin{tikzpicture}
\draw(0,0)--(30mm,0)--++(30:30mm)--++(-30mm,0)--cycle;
\begin{scope}[xshift=10mm]
\draw[line width=0.4mm](0,0)++(30:10mm)coordinate(a1)--++(10mm,0) coordinate(a2)--++(30:10mm) coordinate(a3)--++(-10mm,0) coordinate(a4)--cycle;
\draw(a1)--++(75:30mm)coordinate(b1) (a2)--++(75:30mm)coordinate(b2) (a3)--++(75:30mm)coordinate(b3) (a4)--++(75:30mm)coordinate(b4);
\draw[line width=0.4mm]($(a1)!.8!(b1)$)--($(a2)!.8!(b2)$)--($(a3)!.8!(b3)$)--($(a4)!.8!(b4)$)--cycle;
\end{scope}
\end{tikzpicture}
}
\subfloat[正投影法]{\label{fig:zhentouyinfa}
\begin{tikzpicture}
\draw(0,0)--(30mm,0)--++(30:30mm)--++(-30mm,0)--cycle;
\begin{scope}[xshift=10mm]
\draw[line width=0.4mm](0,0)++(30:10mm)coordinate(a1)--++(10mm,0) coordinate(a2)--++(30:10mm) coordinate(a3)--++(-10mm,0) coordinate(a4)--cycle;
\draw(a1)--++(0,30mm)coordinate(b1) (a2)--++(0,30mm)coordinate(b2) (a3)--++(0,30mm)coordinate(b3) (a4)--++(0,30mm)coordinate(b4);
\draw[line width=0.4mm]($(a1)!.8!(b1)$)--($(a2)!.8!(b2)$)--($(a3)!.8!(b3)$)--($(a4)!.8!(b4)$)--cycle;
\end{scope}
\end{tikzpicture}
}
\caption{平行投影法}\label{pingxingtouyin}
\end{figure}
\begin{enumerate}
\item 投影的概念
\item 正投影的基本特性
\begin{enumerate}
\item 真实性

当直线、平面与投影面平行时,投影反映实长和实形
\item 积聚性

当直线、平面与投影垂直时,投影积聚成点和直线
\item 类似性

当直线、平面与投影面倾斜时,直线的投影仍是直线,但比实长短;平面的投影为一个与它既不全等也不相似的类似多边形。
\end{enumerate}
\end{enumerate}

\endinput