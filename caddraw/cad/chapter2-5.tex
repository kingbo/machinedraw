\subsection{图样绘制方法和步骤}
\subsubsection{CAD绘图流程}
\begin{enumerate}
\item 设置图层。根据绘图样所需线型创建图层并设置其颜色、线型、线宽等参数。
\item 画图框及标题栏。
\item 确定绘图比例,布置图形,使图形在图纸上的位置和大小适中,各图形间应留有适当空隙及标注尺寸的位置。
\item 绘制图样。先画图形的基准线、对称线、中心线及有关定位尺寸线,再根据定型尺寸画主要轮廓线,然后由大到小,由整体到局部,画出其他所有图线。
\item 标注尺寸。
\end{enumerate}
\subsubsection{手工绘图流程}
\begin{enumerate}
\item 画图前的准备工作
    \begin{enumerate}
    \item 阅读必要的参考资料,对所画图形的内容与要求进行了解。
    \item 准备好必要的制图工具,将铅笔与圆规内的铅削好。
    \item 固定图纸
    \end{enumerate}
\item 画底稿
\begin{enumerate}
\item 画图框及标题栏。
\item 确定绘图比例,布置图形,使图形在图纸上的位置和大小适中,各图形间应留有适当空隙及标注尺寸的位置。
\item 绘制图样。先画图形的基准线、对称线、中心线及有关定位尺寸线,再根据定型尺寸画主要轮廓线,然后由大到小,由整体到局部,画出其他所有图线。
\item 标注尺寸(将在后续任务中详细讲解)。
\end{enumerate}
\item 加深图线。按照先细后粗,先曲线后直线,先图形后尺寸,先图线后符号、文字的顺序,从上到下,从左到右进行。

\end{enumerate}

\endinput