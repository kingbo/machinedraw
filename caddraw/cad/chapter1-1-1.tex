\subsection{正投影法}
在图\ref{fig:tiaoyafabei}所示图样中,位于左边的图形称之为主视图。该视图清晰的表明杯零件为典型的回转体零件。位于右边的视图称之为左视图或者左全剖视图。它清楚地表达了杯零件的内部结构。

要理解什么是视图,我们需要先理解什么是投影。投影是物体在阳光或灯光下所产生的影子。由于影子只能够表现物体轮廓而不能够表现物体的内部结果,工程实际中将物体内外空间几何元素加以抽象,并用不同的线型进行表示,实现物体内外细节的表达,从而形成的比较完备的、实用的投影方法。投影法分为中心投影法和平行投影法两类。所有投影线都互不平行且汇聚于点的投影法称为中心投影法。中心投影法主要用于绘制效果比较逼真的建筑或产品立体图。图\ref{pingxingtouyin}所示的投影法是平行投影法,从中可以看出其所有的投影线都是相互平行的,其中投影线倾斜于投影面则为斜投影,投影线垂直于投影面则为正投影法。工程中将用正投影法绘制的物体图形称为视图。
\begin{figure}[htbp]
\centering
\subfloat[斜投影法]{\label{fig:xietouyinfa}
\begin{tikzpicture}
\draw(0,0)--(30mm,0)--++(30:30mm)--++(-30mm,0)--cycle;
\begin{scope}[xshift=10mm]
\draw[line width=0.4mm](0,0)++(30:10mm)coordinate(a1)--++(10mm,0) coordinate(a2)--++(30:10mm) coordinate(a3)--++(-10mm,0) coordinate(a4)--cycle;
\draw(a1)--++(75:30mm)coordinate(b1) (a2)--++(75:30mm)coordinate(b2) (a3)--++(75:30mm)coordinate(b3) (a4)--++(75:30mm)coordinate(b4);
\draw[line width=0.4mm]($(a1)!.8!(b1)$)--($(a2)!.8!(b2)$)--($(a3)!.8!(b3)$)--($(a4)!.8!(b4)$)--cycle;
\end{scope}
\end{tikzpicture}
}
\subfloat[正投影法]{\label{fig:zhentouyinfa}
\begin{tikzpicture}
\draw(0,0)--(30mm,0)--++(30:30mm)--++(-30mm,0)--cycle;
\begin{scope}[xshift=10mm]
\draw[line width=0.4mm](0,0)++(30:10mm)coordinate(a1)--++(10mm,0) coordinate(a2)--++(30:10mm) coordinate(a3)--++(-10mm,0) coordinate(a4)--cycle;
\draw(a1)--++(0,30mm)coordinate(b1) (a2)--++(0,30mm)coordinate(b2) (a3)--++(0,30mm)coordinate(b3) (a4)--++(0,30mm)coordinate(b4);
\draw[line width=0.4mm]($(a1)!.8!(b1)$)--($(a2)!.8!(b2)$)--($(a3)!.8!(b3)$)--($(a4)!.8!(b4)$)--cycle;
\end{scope}
\end{tikzpicture}
}
\caption{平行投影法}\label{pingxingtouyin}
\end{figure}
\endinput