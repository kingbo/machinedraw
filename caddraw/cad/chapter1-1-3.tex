\subsection{三视图投影规律}
从图\ref{fig:threeviewguanxi}中,我们可以得出:主视图反映物体的上下和左右关系,即反映物体的长和高;俯视图反映物体的左右和前后关系,即反映物体的长和宽;左视图反映物体的上下和前后,即反映物体的高和宽。因此,三视图的投影规律为:
\begin{itemize}
\item 主、俯视图长对正;
\item 主、左视图高平齐;
\item 俯、左视图宽相等。
\end{itemize}
\begin{figure}[htbp]
\begin{tikzpicture}
\begin{scope}[scale=1]
\draw[line width=0.4mm](0,0)coordinate(a1)--++ (20mm,0)coordinate(a2)--++ (0,20mm)coordinate(a3)--++ (-10mm,0)coordinate(a4)--++(0,-10mm)coordinate(a5)--++ (-10mm,0)coordinate(a6)--cycle;
\draw(a4)node[above]{\tiny 上};
\draw(a4)++(0,-20mm)node[below]{\tiny 下};
\draw(a6)node[left]{\tiny 左};
\draw(a2)++(0,10mm)node[right]{\tiny 右};
\draw[line width=0.4mm](a1)++(0,-25mm)coordinate(b1)--++(0,-10mm) coordinate(b2)--++(20mm,0)coordinate(b3)--++(0,10mm) coordinate(b4)--cycle;
\draw[line width=0.4mm](b1)++(10mm,0)node[above]{\tiny 后}--++(0,-10mm)node[below]{\tiny 前};
\draw(a1)--(b1)(a2)--(b4);
\draw[<->]($(a1)!.6!(b1)$)--($(a2)!.6!(b4)$)node[midway,above]{\tiny 长对正};
\draw[line width=0.4mm](a2)++(25mm,0)coordinate(c1)--++(10mm,0)coordinate(c2)--++(0,20mm) coordinate(c3)--++(-10mm,0)coordinate(c4)--cycle;
\draw[line width=0.4mm](c1)++(0,10mm)node[left]{\tiny后}--++(10mm,0)node[right]{\tiny前};
\draw(a2)--(c1)(a3)--(c4);
\draw[<->]($(a2)!.6!(c1)$)--($(a3)!.6!(c4)$)node[midway,above,sloped]{\tiny高平齐};
\draw(b3)--++(10mm,0)coordinate(d1);
\draw(b4)--++(10mm,0)coordinate(d2);
\draw[<->]($(b3)!.8!(d1)$)coordinate(d3)--($(b4)!.8!(d2)$)coordinate(d4);
\draw(c1)--++(0,-10mm)coordinate(d5);
\draw(c2)--++(0,-10mm)coordinate(d6);
\draw[<->]($(c1)!.8!(d5)$)coordinate(d7)--($(c2)!.8!(d6)$)coordinate(d8);
\draw($(d4)!.4!(d3)$)coordinate(d9)--($(d7)!.4!(d8)$)coordinate(d10);
\draw($(d9)!.5!(d10)$)--++(6mm,0)node[right]{\tiny宽相等};
\end{scope}
\end{tikzpicture}
\caption{三视图投影规律}\label{fig:threeviewguanxi}
\end{figure}
三视图的投影规律不仅适用于物体整体之间的投影,也适用于空间中的点、线面。同时它也是画图和读图的基础规则。
