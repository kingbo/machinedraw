
\subsection{绘制左视图特征图}
\begin{procedure}
\item 设置图层。

建立“中心线”和“实线”两个图层,分别用于管理中心线和实线两类图形,并将图层设置为“中心线”层。
\item 切换视图为左视图。

\item 绘制左视图中心线。

绘制两条相互垂直的构造线型中心作为绘图参考线
\begin{lstlisting}
|命令: XLINE|
|指定点或 [水平(H)/垂直(V)/角度(A)/二等分(B)/偏移(O)]: 0,52|
|指定通过点:$ @1<0$|
|指定通过点:$ @1<90$|
|指定通过点:|
\end{lstlisting}
偏移产生$\phi 9$圆的中心线
\begin{lstlisting}
|命令: OFFSET|
|当前设置: 删除源=否  图层=源  OFFSETGAPTYPE=0|
|指定偏移距离或 [通过(T)/删除(E)/图层(L)] $<$通过$>$:42|
|选择要偏移的对象,或 [退出(E)/放弃(U)] $<$退出$>$:|
|指定要偏移的那一侧上的点,或 [退出(E)/多个(M)/放弃(U)] $<$退出$>$:|
|选择要偏移的对象,或 [退出(E)/放弃(U)] $<$退出$>$:|
\end{lstlisting}
\item 将图层切换为实线图层,绘制忽略连接圆弧的特征图,其结果如图\ref{fig:duanguaitezhengtu}所示。
\begin{figure}[htbp]
\centering
\subfloat[]{\label{fig:duanguaitezhengtu}\includegraphics[scale=0.62]{duanguaitezhengtu.png}}\hspace{30pt}
\subfloat[]{\label{fig:duanguaitezhengtu1}\includegraphics[scale=0.75]{duanguaitezhengtu1.png}}
\caption{端盖左视特征图绘制过程}
\end{figure}
绘主特征图。
\begin{lstlisting}
|命令: line|
|指定第一个点: 0,64|
|指定下一点或 [放弃(U)]: @0,40|
|指定下一点或 [放弃(U)]: @10,0|
|指定下一点或 [闭合(C)/放弃(U)]: @0,-21|
|指定下一点或 [闭合(C)/放弃(U)]: @18,0|
|指定下一点或 [闭合(C)/放弃(U)]: @0,-10|
|指定下一点或 [闭合(C)/放弃(U)]: @-19,0|
|指定下一点或 [闭合(C)/放弃(U)]: @0,1|
|指定下一点或 [闭合(C)/放弃(U)]: @-4,0|
|指定下一点或 [闭合(C)/放弃(U)]: @0,-10|
|指定下一点或 [闭合(C)/放弃(U)]: c|
\end{lstlisting}
绘制$\phi 9$孔特征图。
\begin{lstlisting}
|命令: rectang|
|指定第一个角点或 [倒角(C)/标高(E)/圆角(F)/厚度(T)/宽度(W)]:|
|指定另一个角点或 [面积(A)/尺寸(D)/旋转(R)]: @10,4.5|
\end{lstlisting}
\item 绘制圆弧连接,其结果如\ref{fig:duanguaitezhengtu1}所示。

绘制圆弧外连接的通常使用圆角命令。启动圆角命令的方法有:
\begin{itemize}
\item 键盘输入FILLET\index{fillet}或F。
\item 【修改】$\rightarrow$【圆角】。
\item 【修改】$\triangleright$【圆角】图标\includegraphics[scale=0.6]{fillet.png}。
\end{itemize}
启动圆角命令后,要求选择对象。通常情况下,第一次启动圆角命令,其半径值为零。因此需要使用【半径(R)】选项先设置半径。
\begin{lstlisting}
|命令: fillet|
|当前设置: 模式 = 修剪,半径 = 0.0000|
|选择第一个对象或 [放弃(U)/多段线(P)/半径(R)/修剪(T)/多个(M)]: r| 
\end{lstlisting}
输入半圆角半径值。
\begin{lstlisting}
|指定圆角半径 $<$0.0000$>$: 3|
\end{lstlisting}
选择用于生成圆角的两边。
\begin{lstlisting}
|选择第一个对象或 [放弃(U)/多段线(P)/半径(R)/修剪(T)/多个(M)]:|
|选择第二个对象,或按住 Shift 键选择对象以应用角点或 [半径(R)]:|
\end{lstlisting}
用相同的方法生成其余两个圆角。由于圆角半径是一致的,因此不需要重复设置圆角半径,只需要直接选择用于圆角的边即可。
\begin{lstlisting}
|命令: FILLET|
|当前设置: 模式 = 修剪,半径 = 3.0000|
|选择第一个对象或 [放弃(U)/多段线(P)/半径(R)/修剪(T)/多个(M)]:|
|选择第二个对象,或按住 Shift 键选择对象以应用角点或 [半径(R)]:|
|命令: FILLET|
|当前设置: 模式 = 修剪,半径 = 3.0000|
|选择第一个对象或 [放弃(U)/多段线(P)/半径(R)/修剪(T)/多个(M)]:|
|选择第二个对象,或按住 Shift 键选择对象以应用角点或 [半径(R)]:
\end{lstlisting}
\item 面域特征图。
\begin{lstlisting}
|命令: region|
|选择对象: 指定对角点: 找到 12 个|
|选择对象:|
\end{lstlisting}
\end{procedure}
圆角命令的注意事项和技巧:
\begin{tips}
\item 如果圆角半径值为零,则不会生成圆角。
\item 圆角命令用于绘制三段圆弧内连接不是很方便,建立用使用圆或者圆弧命令,通过捕捉切点来完成。
\end{tips}

\endinput