\chapter{投影的基本知识}
\begin{itemize}
\item 教学目的:
\begin{enumerate}
\item 使学生掌握投影的基本概念
\item 使学生掌握三视图形成规律
\end{enumerate}
\item 教学内容:
\begin{enumerate}
\item 投影法
\item 物体的三视图
\end{enumerate}
\item 教学重点 、难点:
\begin{enumerate}
\item 三视图的形成规律
\end{enumerate}
\item 教学方法、手段:
\begin{enumerate}
\item 实例分析、多媒体
\end{enumerate}
\item 课时分配:
2课时
\item 作业:
\end{itemize}
\section{投影法}
\begin{enumerate}
\item 投影的概念
\item 正投影的基本特性
\begin{enumerate}
\item 真实性

当直线、平面与投影面平行时,投影反映实长和实形
\item 积聚性

当直线、平面与投影垂直时,投影积聚成点和直线
\item 类似性

当直线、平面与投影面倾斜时,直线的投影仍是直线,但比实长短;平面的投影为一个与它既不全等也不相似的类似多边形。
\end{enumerate}
\end{enumerate}
\section{物体的三视图}
\begin{enumerate}
\item 三视图的形成
\begin{enumerate}
\item 投影面的设置

直立在观察者正对面的的投影面叫做正投影,用$V$表示;
水平位置的投影面叫做水平投影影面,用$H$表示;
右侧的投影面叫做侧立投影面,用$W$表示。
\item 分面进行投影

从物体的前面向后看,在正面上得到的视图叫做主视图;
从物体的上面向下看,在水平面上得到的视图叫做俯视图;
从物体的左侧向右看,在侧面上得到的视图叫做左视图。
\item 投影面的展开摊平

$V$面保持不动,使$H$面绕$X$轴向下旋转$90\degree$与$V$面成一平面,让$W$面绕$Z$轴向右旋转$90\degree$也与$V$面成一平面。
\end{enumerate}
\item 三视图的投影规律

主俯长对正,主左高平齐,俯左宽相等。
\item 三视图的画法
\begin{enumerate}
\item 根据轴测图选主视方向
\item 画定位(基准)线及大体形状
\item 画细部结构形状
\item 完成三视图,检查加粗
\end{enumerate}
\end{enumerate}
\endinput