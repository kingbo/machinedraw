%%%%%%%%%%%%%%%%%%%%%%%%%%%%%%%%%%%%%%%%%%%%%%%%%%%
%%%%%%%%%%%%%%%教案头%%%%%%%%%%%%%%%%%%%%%%%%%%%%%%%
%%%%%%%%%%%%%%%%%%%%%%%%%%%%%%%%%%%%%%%%%%%%%%%%%%%
\mode <article>{

\begin{longtable}{|m{20mm}|m{20mm}|m{20mm}|m{20mm}|m{20mm}|m{28mm}|}
\caption*{\huge 教案头}\\
\hline
\endfirsthead
\multicolumn{6}{l}{(续表)}\\
\hline
\endhead
\hline
\multicolumn{6}{l}{\itshape 接下一页表格.......}\\ [2ex]
\endfoot
\hline
\endlastfoot
\centering{授课单元}&\multicolumn{3}{m{60mm}|}{\centering 水塔供水系统的PLC控制}&\centering{授课日期}&2014年06月9日 \\
\hline
\centering 授课地点 & \multicolumn{3}{m{60mm}|}{B4-209}&\centering 授课学时 & 2 \\
\hline
& \multicolumn{2}{m{40mm}|}{能力目标} & \multicolumn{2}{m{40mm}|}{知识目标}&素质目标 \\
\cline{2-6}
\centering 教学目标&\multicolumn{2}{m{40mm}|}{\begin{enumerate}
\item 能够运PLC顺序功能设计法实现塔供水系统的控制
\item 能够根据控制要求进行系统I/O分配
\end{enumerate}} &\multicolumn{2}{m{40mm}|}{\begin{enumerate}
\item 掌握PLC的顺序功能设计方法
\end{enumerate}} & {能够进行故障分析,以培养问题分析能力}\\
\hline
\centering 能力训练任务或案例 &\multicolumn{5}{m{108mm}|}{\begin{enumerate}
\item 编制水塔供水系统的I/o分配表
\item 设计水塔供水系统的PLC控制程序
\item PLC控制接线
\item PLC控制程序输入
\item PLC控制检验
\end{enumerate}}\\
\hline
\centering 教学重点 & \multicolumn{5}{m{108mm}|}{水截供水系统控制程序的设计}\\
\hline
\centering 教学难点与解决办法 &\multicolumn{5}{m{108mm}|}{难点:PLC顺序功能程序编写。解决方法:通过举例演示使学生了解并掌握PLC顺序功能设计法}\\
\hline
\centering 德育内容 &\multicolumn{5}{m{108mm}|}{无}\\
\hline
 &教材 & \multicolumn{4}{l|}{PLC技术}\\
\cline{2-6}& 教学资源 &\multicolumn{4}{m{88mm}|}{PPT}\\
\cline{2-6}\centering 使用的教学材料& 主要教学仪器设备和工具等 &\multicolumn{4}{m{88mm}|}{投影机、西门子PLC编程控制台、万用表、螺丝刀}\\
\cline{2-6}& 主要耗材 &\multicolumn{4}{m{88mm}|}{\qquad}\\
\hline
\centering 教学模式 &\multicolumn{2}{l|}{项目化}&\centering 教学手段 &\multicolumn{2}{l|}{项目化}\\
\hline
\centering 学生成果与过程考核方式 &\multicolumn{5}{m{108mm}|}{成果:PLC控制程序、控制程序I/O分配表、电动机控制接线。考核方式:对项目完成情况进行评分,作为该项目考核结果}
\end{longtable}
\clearpage

%%%%%%%%%%%%%%%%%%%%%%%%%%%%%%%%%%%%%%%%%%%%%%%%%%%%%
%%%%%%%%%%%%%%%教学实施过程%%%%%%%%%%%%%%%%%%%%%%%%%%%%
%%%%%%%%%%%%%%%%%%%%%%%%%%%%%%%%%%%%%%%%%%%%%%%%%%%%%

\begin{landscape}
\begin{longtable}{|m{10mm}|m{50mm}|m{50mm}|m{50mm}|m{15mm}|}
\caption*{\huge 教学组织与实施}\\
\hline
\endfirsthead
\multicolumn{5}{l}{\small 接上页}\\
\hline
\multicolumn{1}{|c|}{步骤}&\multicolumn{1}{c|}{教学内容}&\multicolumn{1}{c|}{教师活动}&\multicolumn{1}{c|}{学生活动}&\multicolumn{1}{c|}{时间}\\
\hline
\endhead

\multicolumn{5}{r}{\small 接下页}\\
\endfoot
\hline
\endlastfoot
\multicolumn{1}{|c|}{步骤}&\multicolumn{1}{c|}{教学内容}&\multicolumn{1}{c|}{教师活动}&\multicolumn{1}{c|}{学生活动}&\multicolumn{1}{c|}{时间}\\\hline
引入&\begin{enumerate}
\item 水塔供水控制系统
\end{enumerate} &\begin{enumerate}
\item 介绍项目情景
\item 讲述控制要求
\item 布置编制I/O分配表和程序流程图任务
\end{enumerate} &\begin{enumerate}
\item 学生记录控制要求
\item 学生领取项目任务
\end{enumerate} &5 \\\hline
设计&
\begin{enumerate}
\item 编制水塔供水控制I/O分配表和程序流程图
\end{enumerate} &\begin{enumerate}
\item 了解学生编制I/O分配表和程序流程图的情况
\end{enumerate} &\begin{enumerate}
\item 学生编制I/O分配表和程序流程图
\item 学生展示编制的I/O分配表和程序流程图
\end{enumerate} &10 \\\hline
设计&\begin{enumerate}
\item 设计水塔供水控制系统的PLC控制程序
\end{enumerate}
&\begin{enumerate}
\item 指导学生进行PLC控制程序设计
\end{enumerate} &\begin{enumerate}
\item 学生设计PLC程序
\end{enumerate} &20 \\\hline
操作&
\begin{enumerate}
\item PLC控制接线
\end{enumerate} &\begin{enumerate}
\item 教师指导学生进行PLC控制接线
\item 教师收集学生接线过程中存在的问题
\end{enumerate} &\begin{enumerate}
\item 学生根据设计进行PLC控制接线
\end{enumerate} &10 \\\hline
编程&
\begin{enumerate}
\item 编写电机控制PLC程序
\end{enumerate} &\begin{enumerate}
\item 指导学生编写PLC程序
\end{enumerate} &\begin{enumerate}
\item 学生根据设计编写PLC程序
\end{enumerate} &25 \\\hline
\centering 检验修改&\begin{enumerate}
\item 检验PLC控制程序的正确性
\end{enumerate}&\begin{enumerate}
\item 教师收集PLC控制程序存在的问题
\item 教师指导学生进行程序验证
\item 教师展示出错程序
\item 总结错误原因
\end{enumerate}&\begin{enumerate}
\item 学生进行PLC控制程序的控制验证,并记录结果
\item 学生讨论出错原因
\item 学生根据验证情况修改PLC控制程序
\end{enumerate}&15 \\\hline
\centering 本次课总结(评价)&\begin{enumerate}
\item 顺序功能设计法 
\end{enumerate}&讲解顺序功能设计法  &学生倾听记录 &5 \\\hline
\centering 学生学习笔记或工单等检查情况&\multicolumn{4}{m{165mm}|}{\quad}\\\hline
\centering 课后作业&\multicolumn{4}{m{165mm}|}{\begin{enumerate}
\item 
\end{enumerate}}\\\hline
\centering 教学体会&\multicolumn{4}{m{165mm}|}{\quad}\\
\end{longtable}

\end{landscape}
\clearpage

%%%%%%%%%%%%%%%%%%%%%%%%%%%%%%%%%%%%%%%%%%%%%%%%%%%%
%%%%%%%%%%%%%%%%%%%%板书设计%%%%%%%%%%%%%%%%%%%%%%%%%
%%%%%%%%%%%%%%%%%%%%%%%%%%%%%%%%%%%%%%%%%%%%%%%%%%%%

\begin{center}
{\huge 板书设计}
\end{center}
}

\begin{frame}[containsverbatim]{水塔供水系统的PLC控制}
\begin{block}{控制要求}
某居民住宅小区内生活水塔,高40米,由设在水塔附近的三台水泵为其供水。水泵电动机功率为33KW,额定电压380V。水塔正常水位变化2.15M,由安装在水箱内的上、下水位开关S1、S2进行控制。为反映各水泵工作是否正常,在每台水泵的压力出口处设置压力继电器SP1—SP2,将其常开触点作为PLC输入,检测出水压力是否正常。具体控制要求如下:  (1)两台电动机均为降压启动,以减小启动电流的冲击,启动时间为t1。 (2)电动机启动时间错开,上台电动机全压运行t2后,下一台台才能启动 (3)两台电动机均设置有过载保护  (4)两台水泵正常运行时采用一用一备,为防止备用泵长期闲置锈蚀,要求备用机组可用按钮任意切换。  (5)设手动/自动转换开关SAC。手动时,可由操作者分别启动每一台水泵,各水泵不进行联动;自动时,由上、下水位开关对水泵的起停自动控制,且启动时要联动。  (6)若运行中任一台水泵出现故障,备用机组立即投入运行。 
\end{block}
\end{frame}
\begin{frame}{}
\begin{block}{工作任务}
\begin{itemize}
\item 编制I/O配置表
\item 绘制顺序功能图
\item 设计PLC控制程序
\item 控制接线
\item 控制程序录入
\item 控制检查与修正
\end{itemize}
\end{block}
\end{frame}


\endinput
